
% \lettrine{C}ommuniquer des informations d'une machine à une autre est un besoin
% ancien. Une solution se developpe rapidement sous l'impulsion du projet Arpanet
% qui deviendra plus tard la base de l'internet. Initialement créé dans le but de
% connecter une machine à une autre, l'échelle évolue afin d'impliquer plusieurs
% machines communicant entre elles. Il s'agit alors d'un réseau de communication.



% Les machines d'un réseau sont adressables : Grâce à une métadonnée, un message
% émis par une machine trouve un chemin jusqu'à la machine ciblée. Par exemple,
% une adresse IP (\emph{Internet Protocol}) permet de retrouver une machine de
% manière transparente : Les infrastructures sous-jacentes sur l'itinéraire entre
% les machines, tels que les routeurs, sont inconnues. Un réseau construit sur de
% telles adresses logiques est appelé réseau superposé (\emph{network
%   overlay}). Un réseau superposé peut être construit sur un autre réseau
% superposé.

\begin{figure}
  \begin{center}
    \begin{tikzpicture}[scale=1.2]
\newcommand\X{55pt}
\newcommand\Y{55pt}


\draw[fill=white] (0*\X,0*\Y) +(5pt, 5pt) rectangle +(-5pt, -5pt);
\draw[fill=white] (1*\X,0*\Y) +(5pt, 5pt) rectangle +(-5pt, -5pt);
\draw[<->] (5+0*\X, 0*\Y) -- (-5+1*\X, 0*\Y);
\draw (0.5*\X,-5-1*\Y) node[anchor=north, align=center]{2 nœuds; \DARKBLUE{2} arcs};

\begin{scope}[shift={(1.5*\X,0*\Y)}]
\draw[fill=white] (0*\X,0*\Y) +(5pt, 5pt) rectangle +(-5pt, -5pt);
\draw[fill=white] (1*\X,0*\Y) +(5pt, 5pt) rectangle +(-5pt, -5pt);
\draw[fill=white] (1*\X,-1*\Y) +(5pt, 5pt) rectangle +(-5pt, -5pt);
\draw[<->] (5+0*\X, 0*\Y) -- (-5+1*\X, 0*\Y);
\draw[<->] ( 1*\X, -5+0*\Y) -- ( 1*\X, 5-1*\Y);
\draw[<->] (5+0*\X, -5+0*\Y) -- (-5+1*\X, 5-1*\Y);
\draw (0.5*\X,-5-1*\Y) node[anchor=north, align=center]{3 nœuds; \DARKBLUE{6} arcs};
\end{scope}

\begin{scope}[shift={(3*\X,0*\Y)}]
\draw[fill=white] (0*\X,0*\Y) +(5pt, 5pt) rectangle +(-5pt, -5pt);
\draw[fill=white] (1*\X,0*\Y) +(5pt, 5pt) rectangle +(-5pt, -5pt);
\draw[fill=white] (1*\X,-1*\Y) +(5pt, 5pt) rectangle +(-5pt, -5pt);
\draw[fill=white] (0*\X,-1*\Y) +(5pt, 5pt) rectangle +(-5pt, -5pt);
\draw[<->] (5+0*\X, 0*\Y) -- (-5+1*\X, 0*\Y);
\draw[<->] ( 1*\X, -5+0*\Y) -- ( 1*\X, 5-1*\Y);
\draw[<->] (5+0*\X, -5+0*\Y) -- (-5+1*\X, 5-1*\Y);
\draw[<->] (5+0*\X, 5-1*\Y) -- (-5+1*\X, -5+0*\Y);
\draw[<->] (5+0*\X,-1*\Y) -- (-5+1*\X, -1*\Y);
\draw[<->] ( 0*\X,5-1*\Y) -- ( 0*\X, -5+0*\Y);
\draw (0.5*\X,-5-1*\Y) node[anchor=north, align=center]{4 nœuds; \DARKBLUE{12} arcs};
\end{scope}

\begin{scope}[shift={(4.5*\X,0*\Y)}]
\draw[fill=white] (0.33*\X,0*\Y) +(5pt, 5pt) rectangle +(-5pt, -5pt);
\draw[fill=white] (0.66*\X,0*\Y) +(5pt, 5pt) rectangle +(-5pt, -5pt);
\draw[fill=white] (0*\X,-0.5*\Y) +(5pt, 5pt) rectangle +(-5pt, -5pt);
\draw[fill=white] (1*\X,-0.5*\Y) +(5pt, 5pt) rectangle +(-5pt, -5pt);
\draw[fill=white] (0.5*\X,-1*\Y) +(5pt, 5pt) rectangle +(-5pt, -5pt);

\draw[<->] (-5+0.33*\X, 0*\Y) -- (0*\X,5-0.5*\Y);
\draw[<->] ( 5+0.66*\X, 0*\Y) -- (1*\X,5-0.5*\Y);
\draw[<->] (0.33*\X, -5+0*\Y) -- (0.5*\X, 5-1*\Y);
\draw[<->] (0.66*\X, -5+0*\Y) -- (0.5*\X, 5-1*\Y);
\draw[<->] (5+0*\X, 5-0.5*\Y) -- (-5+0.66*\X, -5-0*\Y);
\draw[<->] (-5+1*\X, 5-0.5*\Y) -- (5+0.33*\X, -5-0*\Y);
\draw[<->] (0*\X, -5-0.5*\Y) -- (-5+0.5*\X, -1*\Y);
\draw[<->] (1*\X, -5-0.5*\Y) -- (5+0.5*\X, -1*\Y);
\draw[<->] (5+0.33*\X, 0*\Y) -- (-5+0.66*\X, 0*\Y);
\draw[<->] (5+0*\X, -0.5*\Y) -- (-5+1*\X, -0.5*\Y);
\draw (0.5*\X,-5-1*\Y) node[anchor=north, align=center]{5 nœuds; \DARKBLUE{20} arcs};
\end{scope}


\end{tikzpicture}
    \caption[Graphes complets]{\label{net:fig:completegraph}Graphes complets.}
  \end{center}
\end{figure}

\lettrine{L}'édition collaborative~\cite{ellis1991groupware} permet de répartir
le travail de rédaction d'un document selon les trois dimensions : le temps,
l'espace, et les activités~\cite{desanctis1987foundation,
  grudin1994computersupported, johansen1988groupware}. L'éditeur collaboratif en
constitue l'outil principal récemment popularisé grâce aux éditeurs web tels que
\emph{Google Docs}~\cite{googledocs} ou \emph{ShareLatex}~\cite{sharelatex}. Ces
éditeurs proposent à leurs utilisateurs de créer et modifier un document en
temps réel. Grâce à un simple lien, un utilisateur en mesure de partager ce
document à des amis ou des collègues. Ces derniers sont à leur tour capables de
lire et modifier le document en temps réel.

Les utilisateurs impliquées dans l'édition peuvent être éloignés les uns des
autres géographiquement. Par exemple, un japonais, un français et un brésilien
peuvent éditer un même document en temps réel depuis leur pays d'origine. Cette
répartition géographique necessite l'établissement de moyens de communication
entre éditeurs distants. De plus, le web favorise la dissémination d'idées. Il
n'est dès lors pas impossible qu'une document soit lu et rédigé par des milliers
de participants avant de tomber dans l'oubli.  Chaque modification doit parvenir
efficacement à tous les participants afin que le même document puisse être lu
partout. Pour établir ces communications, les éditeurs collaboratifs doivent
construire un réseau superposé.

Un réseau superposé (\emph{network overlay}) permet à un ensemble de machines de
communiquer entre elles au dessus d'un autre réseau tels que, par exemple,
l'internet ou un autre réseau superposé.  Toutes sortes de réseaux superposés
existent. Si chaque machine possède un canal de communication avec chacune des
autres machines, alors la topologie correspond à celle d'un graphe complet
(cf. figure~\ref{net:fig:completegraph}). La progression quadratique du nombre
d'arcs en fonction du nombre de nœuds empêche un tel système d'atteindre de
grandes envergures. Afin de résoudre ce problème, les machines ne possèdent
qu'une vue partielle du réseau. Cette vue partielle constitue le voisinage
direct d'un nœud. Le défi consiste alors à peupler ces vues avec des liens
logiques selon les critères souhaités.

Dans ce manuscrit, nous nous intéressons aux protocoles d'échantillonnage
aléatoire de pairs~\cite{jelasity2007gossip} dont l'objectif est de fournir à
chaque machine une vue partielle peuplée d'adresses logiques réparties selon une
loi uniforme. La topologie en résultant est proche de celle des graphes
aléatoires~\cite{erdos1959random}. Ces réseaux possèdent d'intéressantes
propriétés tels que la tolérance aux pannes ou un faible diamètre. Cette
première garantie que lorsqu'un nœud disparaît soudainement le graphe reste
connecté avec une forte probabilité, la seconde garantie que les messages se
propagent rapidement à tous les membres du réseau.

% Puisque les protocoles de dissémination de messages font un usage soutenu de ces
% vues partielles, améliorer le protocole d'échantillonnage revient à améliorer la
% dissémination. Le problème scientifique consiste à 

L'état de l'art des protocoles d'échantillonnage aléatoire de pairs se divise en
deux catégories selon les vues partielles fournies :
\begin{inparaenum}[(i)]
\item les approches fournissant des vues partielles de taille
  constante~\cite{eugster2003lightweight, leitao2007dependable,
    tolgyeski2009adaptive, voulgaris2005cyclon} qui doivent être configurées
  \emph{a priori}. Dans ce cas le développeur doit prévoir les dimensions des
  réseaux gérés par son application;
  %%Cela entraîne un  surdimensionnement des vues partielles.
\item les approches fournissant des vues partielles s'auto-ajustant pendant la
  vie du réseau~\cite{ganesh2001scamp, ganesh2003peer}. Malheureusement, ces
  approches ne sont pas adaptées aux réseaux dynamiques.
\end{inparaenum}

Les protocoles d'échantillonnages aléatoire de pairs sont au cœur de nombreuses
autres approches~\cite{folz2016cyclades, jelasity2009tman,
  krasikova2016distributed, voulgaris2013vicinity}. Entre autres, la propagation
de messages~\cite{birman1999bimodal, kermarrec2003probabilistic} d'un pair à
tous les autres (\emph{broadcast}) fait un usage intensif des vues
partielles. Un pair souhaitant propager un message choisit un ensemble de
voisins auquel il communique le message. A son tour, chaque pair recevant un tel
message en fait de même. Les messages parviennent à tous les pairs par
transitivité à la manière d'une épidémie.

Afin d'adapter le trafic généré par la transmission de messages, \textbf{comment
  adapter efficacement le voisinage de chaque éditeur au nombre fluctuant de
  collaborateurs ?}

Ce chapitre présente \SPRAY~\cite{nedelec2015spray}, un protocole dont les pairs
voient leur vue partielle s'ajuster gracieusement à la taille du réseau, sans
l'usage d'aucune connaissance globale. Le cycle de vie d'un pair, divisé entre
son entrée, sa vie, et son départ, est conçu pour conserver un nombre d'arcs
dans le système consistant avec la taille de ce dernier. Les protocoles
construits au dessus de \SPRAY peuvent bénéficier de cette auto-ajustement. Par
exemple, le trafic généré par un protocole de dissémination de messages peut
évoluer selon les dimensions du réseau.

Ce chapitre commence par introduire vocabulaire et notations. La
section~\ref{net:sec:stateoftheart} présente l'état de l'art des protocoles
d'échantillonnage de pairs en les divisant en deux catégories : ceux dont les
vues partielles sont fixes, et ceux dont les vues partielles s'adaptent au
réseau. La section~\ref{net:sec:spray} présente \SPRAY, un protocole appartenant
à la seconde catégorie. La section~\ref{net:sec:properties} présente et compare
les propriétés des protocoles d'échantillonnages. La
section~\ref{net:sec:usecase} présente l'effet de \SPRAY sur un protocole de
dissémination épidémique de messages. Le chapitre se conclut à la
section~\ref{net:sec:conclusion}.

%%% Local Variables:
%%% mode: latex
%%% TeX-master: "../../paper"
%%% End:
