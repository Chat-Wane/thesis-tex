

\lettrine{L}es éditeurs collaboratifs répartis~\cite{ellis1991groupware} suivent
le schéma de réplication optimiste~\cite{demers1987epidemic,
  johnson1975maintenance, ladin1992providing, saito2005optimistic}. Pour que
toutes les répliques d'un document convergent vers un état équivalent, toute
opération générée sur une réplique doit parvenir à toutes les autres répliques
afin d'être intégrée. Un mécanisme de diffusion fiable, décentralisé et
compatible avec le Web est requis.

Grâce à WebRTC~\cite{webrtc}, il est dorénavant possible d'établir des
connexions de navigateur à navigateur et par conséquent d'implémenter un tel
mécanisme de diffusion.
% La récente introduction de \emph{WebRTC}~\cite{webrtc}, un protocole permettant
% l'établissement de connexions d'un navigateur à l'autre, rend l'implémentation
% d'un mécanisme de diffusion possible.
% , permet de construire des éditeurs collaboratifs décentralisés fonctionnant
% directement dans les navigateurs Web.
Les mécanismes de diffusion se basent sur des protocoles d'échantillonnage de
pairs. Toutefois, déployer ces protocoles avec WebRTC n'est pas sans poser de
problèmes :
\begin{enumerate}[(i)]
\item WebRTC ne gère ni les adresses ni les routes ce qui rend l'établissement
  de connexion plus coûteux que sur un réseau IP et sujet aux défaillances.
\item Les navigateurs Web fonctionnent sur des outils informatiques aux
  capacités hétérogènes tels que les ordinateurs de bureau, les téléphones
  portables ou les tablettes tactiles. Le protocole doit donc réduire sa
  consommation en ressources au maximum.
\item Avec WebRTC, un simple lien HTTP %(\emph{HyperText Transfer Protocol})
  permet de partager l'accès à une session d'édition en temps réel. Cette
  simplicité d'accès expose l'éditeur aux explosions soudaines de popularité
  caractéristiques du Web.  Par exemple, un lien \og tweeté \fg par une
  célébrité sur un événement particulier peut attirer des milliers de
  participants en très peu de temps. Cette soudaine popularité retombe lorsque
  l'événement s'achève.  Le mécanisme de diffusion et le protocole duquel il
  dépend doivent s'adapter à ces montées fulgurantes en nombre de participants et
  maintenir leur qualité de service avant de revenir à leur configuration
  initiale lorsque le pic de popularité retombe.
\end{enumerate}


% \paragraph{Un même réseau.} La taille d'un réseau fluctue au cours du temps. Par
% exemple, lorsque les utilisateurs possèdent des habitudes identiques, il peut
% alors y avoir des heures creuses où le réseau ne compte que peu de membres. À
% l'inverse, lors d'événements particuliers, beaucoup d'entrées en très peu de
% temps peuvent avoir lieu. Le Web amplifie ces fluctuations. En effet, le Web
% favorise la diffusion d'idées. Par exemple, dans le cadre de l'édition
% collaborative, un simple lien HTTP permet le partage d'un document. Un lien
% \emph{tweeté} par une célébrité sur un événement particulier peut attirer des
% milliers de participants en très peu de temps. Cette soudaine popularité retombe
% lorsque l'événement s'achève.  Pour gérer ces cas, les approches fournissant des
% vues dont la taille est configurée \emph{a priori} doivent être pessimistes :
% les vues partielles sont surdimensionnées afin de supporter ces cas
% éphémères. Les protocoles qui dépendent, tel que la dissémination de messages,
% héritent de ce pessimisme qui les conduit à, par exemple, produire un trafic
% excessif lorsque le réseau est plus modeste que prévu.


% \paragraph{Plusieurs réseaux.} La taille des réseaux créés par une même
% application varie. Par exemple, un document partagé pour un événement de grande
% ampleur n'a pas la même audience qu'un document partagé avec quelques
% amis. Malgré tout, le développeur d'application doit prévoir large afin de
% supporter les grands groupes s'ils existent, au détriment des plus petits
% groupes d'utilisateurs.




Malheureusement, les protocoles d'échantillonnage actuels ne parviennent pas à
répondre à l'ensemble de ces problématiques. D'un côté,
\SCAMP~\cite{ganesh2001scamp, ganesh2003peer} propose une forme d'adaptation en
fournissant une vue dont la taille est $\ln (N) + k$, $N$ étant le nombre de
membres dans le réseau et $k$ une constante positive. Cependant, sa procédure
d'établissement de connexion impliquant des propagations aléatoires
systématiques n'est pas adapté au contexte WebRTC. D'une autre côté, les
protocoles plus utilisés tels que \CYCLON~\cite{voulgaris2005cyclon} fournissent
une vue dont la taille est constante et paramétrée lors du déploiement. De ce
fait, le développeur est obligé de surdimensionner la taille des vues afin de
gérer les possibles pics de popularités.



% L'état de l'art des protocoles d'échantillonnage aléatoire de pairs se divise en
% deux catégories selon les vues partielles fournies :
% \begin{inparaenum}[(i)]
% \item les approches fournissant des vues partielles de taille
%   constante~\cite{eugster2003lightweight, leitao2007dependable,
%     tolgyeski2009adaptive, voulgaris2005cyclon} qui doivent être configurées
%   \emph{a priori}. Dans ce cas le développeur doit prévoir les dimensions des
%   réseaux gérés par son application;
%   %%Cela entraîne un  surdimensionnement des vues partielles.
% \item les approches fournissant des vues partielles s'auto-ajustant pendant la
%   vie du réseau~\cite{ganesh2001scamp, ganesh2003peer}. Malheureusement, ces
%   approches ne sont pas adaptées aux réseaux dynamiques.
% \end{inparaenum}

% Les protocoles d'échantillonnages aléatoire de pairs sont au cœur de nombreuses
% autres approches~\cite{folz2016cyclades, jelasity2009tman,
%   krasikova2016distributed, voulgaris2013vicinity}. Entre autres, la propagation
% de messages~\cite{birman1999bimodal, kermarrec2003probabilistic} d'un pair à
% tous les autres (\emph{broadcast}) fait un usage intensif des vues
% partielles. Un pair souhaitant propager un message choisit un ensemble de
% voisins auquel il communique le message. A son tour, chaque pair recevant un tel
% message en fait de même. Les messages parviennent à tous les pairs par
% transitivité à la manière d'une épidémie.

% Afin d'adapter le trafic généré par la transmission de messages, \textbf{comment
% adapter efficacement le voisinage de chaque éditeur au nombre fluctuant de
 % collaborateurs ?}

Ce chapitre présente \SPRAY~\cite{nedelec2015spray}, un protocole dont les pairs
voient leur vue partielle s'ajuster à la taille du réseau, sans l'usage d'aucune
connaissance globale. Le cycle de vie d'un pair, divisé entre son entrée, sa vie
et son départ, est conçu pour conserver un nombre d'arcs dans le système en
rapport avec la taille de ce dernier. Les protocoles construits au dessus de
\SPRAY peuvent bénéficier de cet ajustement automatique. Ainsi, le trafic
généré par un protocole de dissémination de messages peut évoluer selon les
dimensions du réseau.

Ce chapitre commence par présenter l'état de l'art des protocoles
d'échantillonnage de pairs en les divisant en deux catégories : ceux dont les
vues partielles sont fixes, et ceux dont les vues partielles s'adaptent au
réseau. La section~\ref{net:sec:problem} définit le problème scientifique.  La
section~\ref{net:sec:spray} présente \SPRAY, un protocole appartenant à la
seconde catégorie. La section~\ref{net:sec:properties} présente et compare les
propriétés des protocoles d'échantillonnage. La section~\ref{net:sec:usecase}
présente l'effet de \SPRAY sur un protocole de dissémination épidémique de
messages. La section~\ref{net:sec:conclusion} conclut le chapitre.

%%% Local Variables:
%%% mode: latex
%%% TeX-master: "../../paper"
%%% End:
