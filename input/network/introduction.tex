
\lettrine{C}ommuniquer des informations d'une machine à une autre est un besoin
ancien. Une solution se developpe rapidement sous l'impulsion du projet Arpanet
qui deviendra plus tard la base de l'internet. Initialement créé dans le but de
connecter une machine à une autre, l'échelle évolue afin d'impliquer plusieurs
machines communicant entre elles. Il s'agit alors d'un réseau de communication.

Les machines d'un réseau sont adressables : Grâce à une métadonnée, un message
émis par une machine trouve un chemin jusqu'à la machine ciblée. Par exemple,
une adresse IP (\emph{Internet Protocol}) permet de retrouver une machine de
manière transparente : Les infrastructures sous-jacentes sur l'itinéraire entre
les machines, tels que les routeurs, sont inconnues. Un réseau construit sur de
telles adresses logiques est appelé réseau superposé (\emph{network
  overlay}). Un réseau superposé peut être construit sur un autre réseau
superposé.

\begin{figure}
  \begin{center}
    \begin{tikzpicture}[scale=1.2]
\newcommand\X{55pt}
\newcommand\Y{55pt}


\draw[fill=white] (0*\X,0*\Y) +(5pt, 5pt) rectangle +(-5pt, -5pt);
\draw[fill=white] (1*\X,0*\Y) +(5pt, 5pt) rectangle +(-5pt, -5pt);
\draw[<->] (5+0*\X, 0*\Y) -- (-5+1*\X, 0*\Y);
\draw (0.5*\X,-5-1*\Y) node[anchor=north, align=center]{2 peers; 2 arcs};

\begin{scope}[shift={(1.5*\X,0*\Y)}]
\draw[fill=white] (0*\X,0*\Y) +(5pt, 5pt) rectangle +(-5pt, -5pt);
\draw[fill=white] (1*\X,0*\Y) +(5pt, 5pt) rectangle +(-5pt, -5pt);
\draw[fill=white] (1*\X,-1*\Y) +(5pt, 5pt) rectangle +(-5pt, -5pt);
\draw[<->] (5+0*\X, 0*\Y) -- (-5+1*\X, 0*\Y);
\draw[<->] ( 1*\X, -5+0*\Y) -- ( 1*\X, 5-1*\Y);
\draw[<->] (5+0*\X, -5+0*\Y) -- (-5+1*\X, 5-1*\Y);
\draw (0.5*\X,-5-1*\Y) node[anchor=north, align=center]{3 peers; 6 arcs};
\end{scope}

\begin{scope}[shift={(3*\X,0*\Y)}]
\draw[fill=white] (0*\X,0*\Y) +(5pt, 5pt) rectangle +(-5pt, -5pt);
\draw[fill=white] (1*\X,0*\Y) +(5pt, 5pt) rectangle +(-5pt, -5pt);
\draw[fill=white] (1*\X,-1*\Y) +(5pt, 5pt) rectangle +(-5pt, -5pt);
\draw[fill=white] (0*\X,-1*\Y) +(5pt, 5pt) rectangle +(-5pt, -5pt);
\draw[<->] (5+0*\X, 0*\Y) -- (-5+1*\X, 0*\Y);
\draw[<->] ( 1*\X, -5+0*\Y) -- ( 1*\X, 5-1*\Y);
\draw[<->] (5+0*\X, -5+0*\Y) -- (-5+1*\X, 5-1*\Y);
\draw[<->] (5+0*\X, 5-1*\Y) -- (-5+1*\X, -5+0*\Y);
\draw[<->] (5+0*\X,-1*\Y) -- (-5+1*\X, -1*\Y);
\draw[<->] ( 0*\X,5-1*\Y) -- ( 0*\X, -5+0*\Y);
\draw (0.5*\X,-5-1*\Y) node[anchor=north, align=center]{4 peers; 12 arcs};
\end{scope}

\begin{scope}[shift={(4.5*\X,0*\Y)}]
\draw[fill=white] (0.33*\X,0*\Y) +(5pt, 5pt) rectangle +(-5pt, -5pt);
\draw[fill=white] (0.66*\X,0*\Y) +(5pt, 5pt) rectangle +(-5pt, -5pt);
\draw[fill=white] (0*\X,-0.5*\Y) +(5pt, 5pt) rectangle +(-5pt, -5pt);
\draw[fill=white] (1*\X,-0.5*\Y) +(5pt, 5pt) rectangle +(-5pt, -5pt);
\draw[fill=white] (0.5*\X,-1*\Y) +(5pt, 5pt) rectangle +(-5pt, -5pt);

\draw[<->] (-5+0.33*\X, 0*\Y) -- (0*\X,5-0.5*\Y);
\draw[<->] ( 5+0.66*\X, 0*\Y) -- (1*\X,5-0.5*\Y);
\draw[<->] (0.33*\X, -5+0*\Y) -- (0.5*\X, 5-1*\Y);
\draw[<->] (0.66*\X, -5+0*\Y) -- (0.5*\X, 5-1*\Y);
\draw[<->] (5+0*\X, 5-0.5*\Y) -- (-5+0.66*\X, -5-0*\Y);
\draw[<->] (-5+1*\X, 5-0.5*\Y) -- (5+0.33*\X, -5-0*\Y);
\draw[<->] (0*\X, -5-0.5*\Y) -- (-5+0.5*\X, -1*\Y);
\draw[<->] (1*\X, -5-0.5*\Y) -- (5+0.5*\X, -1*\Y);
\draw[<->] (5+0.33*\X, 0*\Y) -- (-5+0.66*\X, 0*\Y);
\draw[<->] (5+0*\X, -0.5*\Y) -- (-5+1*\X, -0.5*\Y);
\draw (0.5*\X,-5-1*\Y) node[anchor=north, align=center]{5 peers; 20 arcs};
\end{scope}


\end{tikzpicture}
    \caption[Graphes complets]{\label{net:fig:completegraph}Graphes complets.}
  \end{center}
\end{figure}

Toutes sortes de réseaux superposés existent. Par exemple, si chaque machine
possède un canal de communication avec chacunes des autres machines, alors la
topologie correspond à celle d'un graphe complet
(cf. figure~\ref{net:fig:completegraph}). La progression quadratique du nombre
d'arcs en fonction du nombre de nœuds empêche un tel système d'atteindre de
grandes envergures. Afin de résoudre ce problème, les machines ne possèdent
qu'une vue partielle du réseau. Cette vue partielle constitue le voisinage
direct d'un nœud. Le défi consiste alors à peupler ces vues avec des liens
logiques selon les critères souhaités.

Dans ce manuscrit, nous nous intéressons aux protocoles d'échantillonnage
aléatoire de pairs dont l'objectif est de fournir à chaque machine une vue
partielle peuplée d'adresses logiques réparties selon une loi uniforme. La
topologie en résultant est proche de celle des graphes aléatoires. Ces réseaux
possèdent d'interessantes propriétés tels que la tolérence aux pannes (lorsqu'un
nœud disparait soudainement le graphe reste connecté avec une forte
probabilité), ou un faible diamêtre.

% Puisque les protocoles de dissémination de messages font un usage soutenu de ces
% vues partielles, améliorer le protocole d'échantillonnage revient à améliorer la
% dissémination. Le problème scientifique consiste à 

L'état de l'art est divisé en deux catégories selon les vues partielles fournies :
\begin{inparaenum}
\item les approches fournissant des vues partielles de taille constante qui
  doivent être configurées \emph{a priori}. Dans ce cas le développeur doit
  prévoir les dimensions des réseaux gérés par son application. Cela entraine un
  surdimensionnement des vues partielles.
\item les approches fournissant des vues partielles s'auto-ajustant pendant la
  vie du réseau. Malheureusement, ces approches ne sont pas adaptées aux réseaux
  dynamiques.
\end{inparaenum}

Pour rappel, la question de recherche est : \textbf{Comment disséminer les
  messages de telle façon que le trafic s’adapte automatiquement aux
  fluctuations des réseaux ?}  Sachant que la dissémination fait un usage
intensif des vues partielles, aucunes des approches de l'état de l'art ne permet
une telle adaptation.

Ce chapitre présente \SPRAY, un protocole dont les pairs voient leur vue
partielle s'ajuster gracieusement à la taille du réseau. Le cycle de vie d'un
pair, divisé entre son entrée, sa vie, et son départ, est conçu pour conserver
un nombre d'arcs dans le système consistant avec la taille de ce dernier. Les
protocoles construits au dessus de \SPRAY peuvent bénéficier de cette
auto-ajustement. Par exemple, le trafic généré par un protocole de dissémination
de messages peut évoluer selon les dimensions du réseau.

Ce chapitre commence par introduire vocabulaire et notations. La
section~\ref{net:sec:stateoftheart} présente l'état de l'art des protocoles
d'échantillonnage de pairs en les divisant en deux catégories : Ceux dont les
vues partielles sont fixes, et ceux dont les vues partielles s'adaptent au
réseau. La section~\ref{net:sec:spray} présente \SPRAY, un protocole appartenant
à la seconde catégorie. La section~\ref{net:sec:properties} présente et compare
les propriétés des protocoles d'échantillonages. Le chapitre se conclut à la
section~\ref{net:sec:conclusion}.

%%% Local Variables:
%%% mode: latex
%%% TeX-master: "../../paper"
%%% End:
