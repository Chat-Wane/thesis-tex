
\begin{figure}
  \begin{center}
    \begin{tikzpicture}[scale=1.2]
\newcommand\X{55pt}
\newcommand\Y{55pt}


\draw[fill=white] (0*\X,0*\Y) +(5pt, 5pt) rectangle +(-5pt, -5pt);
\draw[fill=white] (1*\X,0*\Y) +(5pt, 5pt) rectangle +(-5pt, -5pt);
\draw[<->] (5+0*\X, 0*\Y) -- (-5+1*\X, 0*\Y);
\draw (0.5*\X,-5-1*\Y) node[anchor=north, align=center]{2 peers; 2 arcs};

\begin{scope}[shift={(1.5*\X,0*\Y)}]
\draw[fill=white] (0*\X,0*\Y) +(5pt, 5pt) rectangle +(-5pt, -5pt);
\draw[fill=white] (1*\X,0*\Y) +(5pt, 5pt) rectangle +(-5pt, -5pt);
\draw[fill=white] (1*\X,-1*\Y) +(5pt, 5pt) rectangle +(-5pt, -5pt);
\draw[<->] (5+0*\X, 0*\Y) -- (-5+1*\X, 0*\Y);
\draw[<->] ( 1*\X, -5+0*\Y) -- ( 1*\X, 5-1*\Y);
\draw[<->] (5+0*\X, -5+0*\Y) -- (-5+1*\X, 5-1*\Y);
\draw (0.5*\X,-5-1*\Y) node[anchor=north, align=center]{3 peers; 6 arcs};
\end{scope}

\begin{scope}[shift={(3*\X,0*\Y)}]
\draw[fill=white] (0*\X,0*\Y) +(5pt, 5pt) rectangle +(-5pt, -5pt);
\draw[fill=white] (1*\X,0*\Y) +(5pt, 5pt) rectangle +(-5pt, -5pt);
\draw[fill=white] (1*\X,-1*\Y) +(5pt, 5pt) rectangle +(-5pt, -5pt);
\draw[fill=white] (0*\X,-1*\Y) +(5pt, 5pt) rectangle +(-5pt, -5pt);
\draw[<->] (5+0*\X, 0*\Y) -- (-5+1*\X, 0*\Y);
\draw[<->] ( 1*\X, -5+0*\Y) -- ( 1*\X, 5-1*\Y);
\draw[<->] (5+0*\X, -5+0*\Y) -- (-5+1*\X, 5-1*\Y);
\draw[<->] (5+0*\X, 5-1*\Y) -- (-5+1*\X, -5+0*\Y);
\draw[<->] (5+0*\X,-1*\Y) -- (-5+1*\X, -1*\Y);
\draw[<->] ( 0*\X,5-1*\Y) -- ( 0*\X, -5+0*\Y);
\draw (0.5*\X,-5-1*\Y) node[anchor=north, align=center]{4 peers; 12 arcs};
\end{scope}

\begin{scope}[shift={(4.5*\X,0*\Y)}]
\draw[fill=white] (0.33*\X,0*\Y) +(5pt, 5pt) rectangle +(-5pt, -5pt);
\draw[fill=white] (0.66*\X,0*\Y) +(5pt, 5pt) rectangle +(-5pt, -5pt);
\draw[fill=white] (0*\X,-0.5*\Y) +(5pt, 5pt) rectangle +(-5pt, -5pt);
\draw[fill=white] (1*\X,-0.5*\Y) +(5pt, 5pt) rectangle +(-5pt, -5pt);
\draw[fill=white] (0.5*\X,-1*\Y) +(5pt, 5pt) rectangle +(-5pt, -5pt);

\draw[<->] (-5+0.33*\X, 0*\Y) -- (0*\X,5-0.5*\Y);
\draw[<->] ( 5+0.66*\X, 0*\Y) -- (1*\X,5-0.5*\Y);
\draw[<->] (0.33*\X, -5+0*\Y) -- (0.5*\X, 5-1*\Y);
\draw[<->] (0.66*\X, -5+0*\Y) -- (0.5*\X, 5-1*\Y);
\draw[<->] (5+0*\X, 5-0.5*\Y) -- (-5+0.66*\X, -5-0*\Y);
\draw[<->] (-5+1*\X, 5-0.5*\Y) -- (5+0.33*\X, -5-0*\Y);
\draw[<->] (0*\X, -5-0.5*\Y) -- (-5+0.5*\X, -1*\Y);
\draw[<->] (1*\X, -5-0.5*\Y) -- (5+0.5*\X, -1*\Y);
\draw[<->] (5+0.33*\X, 0*\Y) -- (-5+0.66*\X, 0*\Y);
\draw[<->] (5+0*\X, -0.5*\Y) -- (-5+1*\X, -0.5*\Y);
\draw (0.5*\X,-5-1*\Y) node[anchor=north, align=center]{5 peers; 20 arcs};
\end{scope}


\end{tikzpicture}
    \caption[Graphes complets]{\label{net:fig:completegraph}Graphes complets.}
  \end{center}
\end{figure}

% \lettrine{L}'édition collaborative~\cite{ellis1991groupware} permet de répartir
% le travail de rédaction d'un document selon les trois dimensions : le temps,
% l'espace, et les activités~\cite{desanctis1987foundation,
%   grudin1994computersupported, johansen1988groupware}. L'éditeur collaboratif en
% constitue l'outil principal récemment popularisé grâce aux éditeurs web tels que
% \emph{Google Docs}~\cite{googledocs} ou \emph{ShareLatex}~\cite{sharelatex}. Ces
% éditeurs proposent à leurs utilisateurs de créer et modifier un document en
% temps réel. Grâce à un simple lien, un utilisateur en mesure de partager ce
% document à des amis ou des collègues. Ces derniers sont à leur tour capables de
% lire et modifier le document en temps réel.

% Les utilisateurs impliquées dans l'édition peuvent être éloignés les uns des
% autres géographiquement. Par exemple, un japonais, un français et un brésilien
% peuvent éditer un même document en temps réel depuis leur pays d'origine. Cette
% répartition géographique necessite l'établissement de moyens de communication
% entre éditeurs distants. De plus, le web favorise la dissémination d'idées. Il
% n'est dès lors pas impossible qu'un document soit lu et rédigé par des milliers
% de participants avant de tomber dans l'oubli.  Chaque modification doit parvenir
% efficacement à tous les participants afin que le même document puisse être lu
% partout. Pour établir ces communications, les éditeurs collaboratifs doivent
% construire un réseau superposé (\emph{network overlay}).

\lettrine{L}es éditeurs collaboratifs répartis~\cite{ellis1991groupware} suivent
le schéma de réplication optimiste~\cite{demers1987epidemic,
  johnson1975maintenance, ladin1992providing, saito2005optimistic}. Pour que les
répliques d'un document convergent vers un état équivalent, toute opération
générée sur une réplique doit parvenir à toutes les autres répliques afin d'être
intégrée. Un mécanisme de diffusion efficace et fiable est requis.

La récente introduction de \emph{WebRTC}~\cite{webrtc}, un protocole permettant
l'établissement de connexions d'un navigateur à l'autre, permet de construire
des éditeurs collaboratifs décentralisés fonctionnant directement dans les
navigateurs web.  Les mécanismes de diffusion se basent sur des protocoles
d'échantillonnage de pairs. Toutefois, déployer ces protocoles avec WebRTC n'est
pas sans poser de problèmes :
\begin{itemize}
\item WebRTC ne gère ni les adresses ni les routes ce qui rend l'établissement
  des connexions plus coûteuses et sujettes aux défaillances que sur un réseau
  IP.
\item Les navigateurs web fonctionnent sur des outils informatiques aux
  capacités hétérogènes tels que les ordinateurs de bureau, les téléphones
  portables ou les tablettes tactiles. Le protocole doit donc réduire sa
  consommation de ressources au maximum.
\item Avec WebRTC, un simple lien HTTP %(\emph{HyperText Transfer Protocol})
  permet de partager l'accès à une session d'édition en temps réel. Cette
  simplicité d'accès expose l'éditeur aux explosions soudaines de popularité
  caractéristiques du web.
  %% TODO : insert concrete example
  Le mécanisme de diffusion et le protocole duquel il dépend doivent s'adapter à
  ces montés fulgurantes en nombre de participants et maintenir leur qualité de
  service avant de revenir à leur configuration initiale lorsque le pic de
  popularité retombe.
\end{itemize}

Malheureusement, les protocoles d'échantillonnage actuels ne parviennent pas à
répondre à l'ensemble de ces problématiques. D'un coté,
\SCAMP~\cite{ganesh2001scamp, ganesh2003peer} propose une forme d'adaptation en
fournissant une vue dont la taille est $\ln (N) + k$, avec $N$ étant le nombre
de membres dans le réseau et $k$ une constante positive. Cependant, son
processus d'établissement des connexions impliquant des propagations aléatoires
systématiques n'est pas adapté au contexte WebRTC. D'une autre coté, les
protocoles plus utilisés tels que \CYCLON~\cite{voulgaris2005cyclon} fournissent
une vue dont la taille est constante et paramétrée lors du
déploiement. Cependant, le développeur est obligé de surdimensionner la taille
des vues afin de gérer les possibles pics de popularités.



% L'état de l'art des protocoles d'échantillonnage aléatoire de pairs se divise en
% deux catégories selon les vues partielles fournies :
% \begin{inparaenum}[(i)]
% \item les approches fournissant des vues partielles de taille
%   constante~\cite{eugster2003lightweight, leitao2007dependable,
%     tolgyeski2009adaptive, voulgaris2005cyclon} qui doivent être configurées
%   \emph{a priori}. Dans ce cas le développeur doit prévoir les dimensions des
%   réseaux gérés par son application;
%   %%Cela entraîne un  surdimensionnement des vues partielles.
% \item les approches fournissant des vues partielles s'auto-ajustant pendant la
%   vie du réseau~\cite{ganesh2001scamp, ganesh2003peer}. Malheureusement, ces
%   approches ne sont pas adaptées aux réseaux dynamiques.
% \end{inparaenum}

% Les protocoles d'échantillonnages aléatoire de pairs sont au cœur de nombreuses
% autres approches~\cite{folz2016cyclades, jelasity2009tman,
%   krasikova2016distributed, voulgaris2013vicinity}. Entre autres, la propagation
% de messages~\cite{birman1999bimodal, kermarrec2003probabilistic} d'un pair à
% tous les autres (\emph{broadcast}) fait un usage intensif des vues
% partielles. Un pair souhaitant propager un message choisit un ensemble de
% voisins auquel il communique le message. A son tour, chaque pair recevant un tel
% message en fait de même. Les messages parviennent à tous les pairs par
% transitivité à la manière d'une épidémie.

% Afin d'adapter le trafic généré par la transmission de messages, \textbf{comment
% adapter efficacement le voisinage de chaque éditeur au nombre fluctuant de
 % collaborateurs ?}

Ce chapitre présente \SPRAY~\cite{nedelec2015spray}, un protocole dont les pairs
voient leur vue partielle s'ajuster à la taille du réseau, sans l'usage d'aucune
connaissance globale. Le cycle de vie d'un pair, divisé entre son entrée, sa
vie, et son départ, est conçu pour conserver un nombre d'arcs dans le système
consistant avec la taille de ce dernier. Les protocoles construits au dessus de
\SPRAY peuvent bénéficier de cette auto-ajustement. Par exemple, le trafic
généré par un protocole de dissémination de messages peut évoluer selon les
dimensions du réseau.

Ce chapitre commence par présenter l'état de l'art des protocoles
d'échantillonnage de pairs en les divisant en deux catégories : ceux dont les
vues partielles sont fixes, et ceux dont les vues partielles s'adaptent au
réseau. La section~\ref{net:sec:problem} définit le problème scientifique.
La section~\ref{net:sec:spray} présente
\SPRAY, un protocole appartenant à la seconde catégorie. La
section~\ref{net:sec:properties} présente et compare les propriétés des
protocoles d'échantillonnages. La section~\ref{net:sec:usecase} présente l'effet
de \SPRAY sur un protocole de dissémination épidémique de messages. Le chapitre
se conclut à la section~\ref{net:sec:conclusion}.

%%% Local Variables:
%%% mode: latex
%%% TeX-master: "../../paper"
%%% End:
