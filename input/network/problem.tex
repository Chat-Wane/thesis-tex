
\section{Définition du problème}
\label{net:sec:problem}


À la lumière des différentes limitations des approches composant l'état de
l'art, nous définissons le problème suivant :

\begin{problem}
  \label{net:problem:properties}
  Soit $t$ une unité de temps arbitraire, soit $\mathcal{N}^t$ l'ensemble des
  membres non-byzantins du réseau à un instant $t$ et soit $P_i^t$ la vue
  partielle du nœud $n_i \in \mathcal{N}^t$. Un protocole d'échantillonnage
  aléatoire de pairs efficace doit assurer les propriétés suivantes :
  \begin{enumerate}
  \item Taille des vues partielles : \hfill $\forall n_i \in \mathcal{N}^t$,
    $|P_i^t| \approx \mathcal{O}(\ln |\mathcal{N}^t|)$
  \item Établissement de connexion : \hfill $\mathcal{O}(1)$
  \end{enumerate}
\end{problem}
En d'autres termes, les vues partielles doivent s'adapter aux évolutions du
réseau et rester équilibrées. Les approches à taille fixe échouent à fournir
cette propriété.
Le nombre de voisins parcourus doit être borné par une constante. \SCAMP échoue
à fournir cette propriété puisque chaque connexion implique une dissémination
aléatoire à une distance non bornée.

%%% Local Variables:
%%% mode: latex
%%% TeX-master: "../../paper"
%%% End:
