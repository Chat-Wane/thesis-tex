
\section{Spray : un protocole d'échantillonnage adaptatif}
\label{net:sec:spray}

\SPRAY est un protocole d'échantillonnage de pairs adaptatif inspiré à
la fois de \SCAMP~\cite{ganesh2003peer} et
\CYCLON~\cite{voulgaris2005cyclon}. \SPRAY comprend trois parties représentant
le cycle de vie d'un nœud dans le réseau. Tout d'abord, le processus consistant
à rejoindre le réseau, qui injecte un nombre logarithmique d'arcs comparé à la
taille du réseau.  Ensuite, chaque pair exécute un processus periodique dont le
but est d'équilibrer les vues partielles en termes de taille et d'uniformité sur
les pairs références dans celles-ci. Le réseau converge rapidement vers une
topologie possédant des propriétés similaires à celles des graphes
aléatoires. Enfin, un nœud est capable de quitter le réseau à n'importe quel
moment sans en notifier le réseau (l'équivalent d'une défaillance) sans dégrader
les propriétés du reste du réseau.

L'obtention de cette propriété d'adaptivité repose essentiellement sur le fait
de conserver un nombre d'arcs cohérent durant tout le cycle de vie du réseau.
En effet, en opposition à \CYCLON, \SPRAY est toujours à la limite du nombre
optimal d'arcs. Comme \SPRAY n'ajoute jamais d'arcs après le processus d'entrée,
toute suppression d'arcs est définitive. De telles décisions ne sont donc pas
prises à la légère. Ainsi, dans un premier temps, \SPRAY ajoute des arcs lors de
l'entrée d'un nœud dans le réseau. Dans un second temps, le processus periodique
de mélange des arcs de \SPRAY préserve tous les arcs du réseau.  Dans un
troisième temps, le processus de sortie supprime précautionneusement quelques
arcs. Dans l'idéal, il s'agit du nombre d'arcs ajouté par le dernier nœud entré
dans le réseau.

Parfois, conserver un nombre d'arcs global constant force les processus de
mélange et de sortie à créer des doublons dans les vues partielles. Ainsi, une
vue partielle peut contenir plusieurs fois le même voisin. Toutefois, ces
doublons restent peu nombreux. De ce fait, ils n'ont pas d'impact notable sur la
connexité du réseau.

Ce chapitre est composé de 5 sections. Les
sections~\ref{net:sec:joining},~\ref{net:sec:shuffling},
et~\ref{net:sec:leaving} présentent respectivement le processus d'entrée, de
mélange, et de départ d'un nœud du réseau. La section~\ref{net:sec:properties}
présente les propriétés de \SPRAY mises en évidence grâce à des simulations. La
section~\ref{net:sec:merge} présente une problématique originale à \SPRAY : la
fusion de réseaux.

% \subsection{WebRTC}

% \begin{figure*}
% \centering
% \subfloat[Figure A][\label{fig:webrtcA}
% $p_1$ connects to $p_2$ using the signaling service.
% 1: $p_1$ pushes its offer ticket;
% 2: $p_2$ pulls the ticket;
% 3: $p_2$ pushes its response;
% 4: $p_1$: pulls the response and establishes a
% bidirectional connection with $p_2$. $p_3$ does the same with $p_2$.
% Figure~\ref{fig:webrtcB} depicts the resulting network.]{
%   
\begin{tikzpicture}[scale=1.1]

\newcommand\X{40pt};
\newcommand\Y{18pt};

\draw( 1.3*\X, 0); %% spacing
\draw(-1.3*\X, 0); %% spacing

\draw[fill=white,very thick](0*\X, 0*\Y) 
node{\emph{signaling service}} +(-40pt,-5pt) rectangle +(40pt,5pt);

\small
\draw[->,dashed, very thick](-5 -1*\X, 5-2*\Y) --
node[anchor=east]{1} (-20pt,-5pt);
\draw[->,dashed, very thick]( 5 -1*\X, 5-2*\Y) --
node[anchor=west]{4} (-10pt,-5pt);

\draw[->,dashed, very thick](-5pt,  5-3*\Y) --
node[anchor=east]{2}(-5pt,-5pt);
\draw[->,dashed, very thick](5pt , 5-3*\Y) --
node[anchor=west]{3} (5pt,-5pt);


\draw[fill=white, very thick]
(-1*\X,-2*\Y) node{$p_1$} +(-5pt,-5pt) rectangle +(5pt,5pt);
\draw[fill=white, very thick]
(0*\X, -3*\Y) node{$p_2$} +(-5pt,-5pt) rectangle +(5pt,5pt);
\draw[fill=white] (1*\X, -2*\Y) node{$p_3$} +(-5pt,-5pt) rectangle +(5pt,5pt);

\end{tikzpicture}

% \begin{tikzpicture}
% \matrix (m) [matrix of math nodes,row sep=4em,column sep=4em] {
% \node(ss)[draw]{signaling}; & \node(p3)[draw]{p3}; \\
% \node(p1)[draw]{p1}; & \node(p2)[draw]{p2}; \\
% };
% \path[->]
%   (p2) edge[dashed] node[fill=white]{1:emit} (ss)
%   (p3) edge[dashed] node[fill=white,bend left]{2:pull} (ss)
%   (p3) edge[dashed, bend right] node[fill=white]{3:accept} (ss)
%   (p2) edge[dashed,bend left] node[fill=white]{4:pull} (ss)
%   (p3) edge[<->,thick] node[fill=white,right]{5:connected} (p2);
% \end{tikzpicture}}
% \hspace{5pt}
% \subfloat[Figure B][\label{fig:webrtcB}
% $p_1$ connects to $p_3$ using $p_2$ as mediator.
% 1: $p_1$ sends its offer ticket to $p_2$;
% 2: $p_2$ forwards it to $p_3$ and registers $p_1$ as the emitter;
% 3: $p_3$ sends its response to $p_2$;
% 4: $p_2$ forwards it to the emitter $p_1$ which connects to $p_3$.]{
%   
\begin{tikzpicture}[scale=1.2]

\newcommand\X{40pt};
\newcommand\Y{18pt};

\draw(1.5*\X, 0); %% spacing
\draw(-1.5*\X, 0); %% spacing

\draw[fill=white](0*\X, 0*\Y)
node{\emph{signaling service}} +(-30pt,-5pt) rectangle +(30pt,5pt);

\small
\draw[<->, very thick](5-1*\X,-2*\Y)--
node[anchor=south]{1$\rightarrow$}
node[anchor=north]{$\leftarrow$4}(-5pt,-3*\Y);
\draw[<->, very thick](5pt,-3*\Y)--
node[anchor=south]{2$\rightarrow$}
node[anchor=north]{$\leftarrow$3}(-5+1*\X,-2*\Y);

\draw[fill=white, very thick]
(-1*\X,-2*\Y) node{$p_1$} +(-5pt,-5pt) rectangle +(5pt,5pt);
\draw[fill=white, very thick]
(0*\X, -3*\Y) node{$p_2$} +(-5pt,-5pt) rectangle +(5pt,5pt);
\draw[fill=white, very thick]
(1*\X, -2*\Y) node{$p_3$} +(-5pt,-5pt) rectangle +(5pt,5pt);

\end{tikzpicture}

% \begin{tikzpicture}
% \matrix (m) [matrix of math nodes,row sep=4em,column sep=4em] {
% \node(ss)[draw]{signaling}; & \node(p3)[draw]{p3}; \\
% \node(p1)[draw]{p1}; & \node(p2)[draw]{p2}; \\
% };
% \path[->]
%   (p1) edge[dashed,bend left] node[fill=white]{1:emit} (p2)
%   (p2) edge[dashed,bend left] node[fill=white,left]{2:emit/p1} (p3)
%   (p3) edge[dashed,bend left] node[fill=white,right]{3:accept/p1} (p2)
%   (p2) edge[dashed,bend left] node[fill=white]{4:accept} (p1)
%   (p1) edge[<->,thick] (p2)
% %  (p1) edge[<->,thick,bend left] (p3)
%   (p2) edge[<->,thick]  (p3);

% \end{tikzpicture}}
% \hspace{5pt}
% \subfloat[Figure C][\label{fig:webrtcC}
% The resulting network overlay: a fully connected network composed of
% 3 members.]{
%   
\begin{tikzpicture}[scale=1.1]

\newcommand\X{40pt};
\newcommand\Y{18pt};

\draw(1.3*\X, 0); %% spacing
\draw(-1.3*\X, 0); %% spacing

\draw[fill=white](0*\X, 0*\Y)
node{\emph{signaling service}} +(-40pt,-5pt) rectangle +(40pt,5pt);

\small
\draw[<->](5-1*\X,-2*\Y)--(-5pt,-3*\Y);
\draw[<->](5pt,-3*\Y)--(-5+1*\X,-2*\Y);
\draw[<->, very thick](5 - 1*\X, 2.5 -2*\Y)--(-5+1*\X, 2.5 -2*\Y);

\draw[fill=white]
(-1*\X,-2*\Y) node{$p_1$} +(-5pt,-5pt) rectangle +(5pt,5pt);
\draw[fill=white]
(0*\X, -3*\Y) node{$p_2$} +(-5pt,-5pt) rectangle +(5pt,5pt);
\draw[fill=white]
(1*\X, -2*\Y) node{$p_3$} +(-5pt,-5pt) rectangle +(5pt,5pt);

\end{tikzpicture}


% \begin{tikzpicture}
% \matrix (m) [matrix of math nodes,row sep=4em,column sep=4em] {
% \node(ss)[draw]{signaling}; & \node(p3)[draw]{p3}; \\
% \node(p1)[draw]{p1}; & \node(p2)[draw]{p2}; \\
% };
% \path[->]
%   (p1) edge[<->,thick] (p2)
%   (p1) edge[<->,thick] (p3)
%   (p2) edge[<->,thick]  (p3);
% \end{tikzpicture}}
% \caption{\label{fig:webrtc}Creating an overlay network on top of WebRTC.}
% \end{figure*}


% WebRTC permet la communication en temps réel entre navigateurs web et ce, même
% en présence de configurations réseaux complexes impliquant firewall, proxy, ou
% NAT (Network Address Translation). Toutefois, WebRTC ne gère ni l'adressage, ni
% le routage. Pour établir une connexion, les navigateurs s'échangent des offres
% et acquittements via un médiateur commun (e.g. mails, services dédiés de
% signalement, connections WebRTC connues, etc.). Dans la
% figure~\ref{fig:webrtcA}, $p_1$ souhaite se connecter à $p_2$. Par conséquence,
% $p_1$ envoie son ticket d'offre au service de signalement connu. Le pair $p_2$
% récupère l'offre, la poinçonne et la renvoie au service de signalement. Enfin,
% $p_1$ récupère le ticket poinçonné et établie une connexion bidirectionnelle
% avec $p_2$. De manière identique, $p_3$ établie une connexion avec $p_2$. Nous
% appellerons la procédure d'aller-retour des tickets un \emph{three-way
%   handshake}. Désormais, le pair $p_1$ est capable d'établir une connexion avec
% $p_3$ sans passer par l'intermédiaire du serveur. Pour cela, il utilise $p_2$
% comme médiateur. Toutefois, si le pair $p_2$ crash durant cette procédure, la
% connexion ne pourra s'effectuer correctement, et ce même si une route
% alternative existe (puisque WebRTC ne gère pas le routage).

% Utiliser les services de signalement et les connections WebRTC existantes permet
% de déployer facilement les protocoles d'échantillonnage aléatoire de
% pairs~\cite{jelasity2007gossip}. Ces derniers étant présent dans les navigateurs
% modernes disponibles sur les smartphones, les tablettes, etc. Dans ce contexte,
% il est impératif de conserver autant que possible un petit nombre de connections
% afin de réduire le trafic réseau et limiter la consommation de ressources.

%\subsubsection{Facilité d'accès}
%\subsubsection{Contraintes supplémentaires}



%%% Local Variables:
%%% mode: latex
%%% TeX-master: "../../paper"
%%% End:
