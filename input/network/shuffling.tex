
\section{Mélanger son voisinage}
\label{net:sec:shuffling}

Au contraire de \CYCLON, \SPRAY mélange des vues partielles dont les tailles
peuvent être différentes. Ce processus a pour but d'équilibrer les tailles de
vue partielles ainsi que de mélanger les adresses logiques à l'interieur de
celles-ci.  Ce processus possède la contrainte de conserver l'exact même nombre
d'arcs.

Les deux nœuds impliqués dans le mélange s'envoient l'un l'autre la moitié de
leur vue partielle. Après l'intégration de ces nouvelles références, la taille
de leur vue partielle tend vers la moyenne, et la somme globale en demeure
inchangée. Dans ce but, les vues partielles sont des multiensembles. Ainsi, si
un nœud réçoit un arc déjà connue, il le conserve en tant que doublon.  De cette
façon, le nombre d'arcs reste globalement constant.

Si les doublons ont un impact négatif sur les propriétés du réseau, la plupart
de ceux-ci disparaissent après le processus de mélange. En proportion, ils
deviennent négligeable dès lors que le réseau grandit.

\begin{algorithm}[h]
  
\small
\algrenewcommand{\algorithmiccomment}[1]{\hskip2em$\rhd$ #1}

\newcommand{\comment}[1]{$\rhd$ #1}

\algblockdefx[act]{act}{endAct}
  [0] {\textbf{ACTIVE THREAD:}}

\algsetblockdefx[pas]{pas}{endPas}
  {65535}{}
  [0] {\textbf{PASSIVE THREAD:}}


\newcommand{\LINEFOR}[2]{%
  \algorithmicfor\ {#1}\ \algorithmicdo\ {#2} %
  }

\newcommand{\LINEIFTHEN}[2]{%
  \algorithmicif\ {#1}\ \algorithmicthen\ {#2} %
  }

\newcommand{\INDSTATE}[1][1]{\State\hspace{\algorithmicindent}}

\begin{algorithmic}[1]
  \Statex
  \act
    \Function{loop}{ } \hfill \comment{Every $\Delta\,t$}
    \State $\mathcal{P} \leftarrow incrementAge(\mathcal{P})$;
    \State \textbf{let} $ \langle q,\, age \rangle \leftarrow getOldest(\mathcal{P})$;
    \State \textbf{let} $sample \leftarrow $ \label{line:samplesize}
    $getSample(\mathcal{P}\setminus\left\{\langle q, age\rangle\right\}, \left \lceil{|\mathcal{P}|\over{2}} \right \rceil-1) \uplus \left\{\langle p, 0 \rangle\right\}$;
    \State $sample \leftarrow replace(sample,\,q,\,p)$; \label{line:replace1}
    \State $sendTo(q,\, 'exchange',\, sample)$;
    \State \textbf{let} $sample'\leftarrow receiveFrom(q)$;
    \State $sample \leftarrow replace(sample,\,p,\,q)$;
    \State $\mathcal{P} \leftarrow (\mathcal{P} \setminus sample) \uplus
    sample'$;
    \EndFunction
  \endAct
  
  \pas
    \Function{onExchange}{$o,\, sample$} \hfill \comment{$o: origin$}
    \State \textbf{let} $sample' \leftarrow getSample(\mathcal{P} ,\, \left\lceil |\mathcal{P}|\over{2} \right\rceil )$;
    \State $sample' \leftarrow replace(sample',\,o,\,p);$ \label{line:replace2}
    \State $sendTo(o ,\, sample')$;
    \State $sample' \leftarrow replace(sample',\,p,\,o)$;
    \State $\mathcal{P} \leftarrow (\mathcal{P} \setminus sample') \uplus
    sample$; 
    \EndFunction
%%  \endPas
\end{algorithmic}

  \caption{\label{net:algo:shuffle}The cyclic protocol of \SPRAY.}
\end{algorithm}

L'algorithme~\ref{net:algo:shuffle} montre les instructions periodiquement
éxecutées par chacun des nœuds. Il est divisé en deux parties, à savoir le
processus actif qui est appelé régulièrement afin d'initier un mélange de vues,
et le processus passif qui réagit au message du processus actif. Les fonctions
qui ne sont pas explicitement définies sont les suivantes:
\begin{inparaenum}[]
\item $\textsc{incrementAge}(view)$ qui incrémente l'âge des éléments présent
  dans la vue $view$ et retourne la vue modifiée;
\item $\textsc{getOldest}(view)$ qui retourne le plus vieux des voisins présent
  dans la vue partielle $view$;
\item $\textsc{getSample}(view,\, size)$ qui retourne un échantillon aléatoire
  de la vue contenant $size$ éléments;
\item $\textsc{replace}(view,\,old,\,new)$ qui remplace les occurrences de $old$
  par $new$ dans la vue $view$. Si les doublons sont admis dans la vue
  partielle, les autoréférences restent hautement indésirables;
\item $\textsc{rand}()$ qui retourne un nombre flottant aléatoire compris entre
  $0$ et $1$.
\end{inparaenum}

Dans le processus actif, la fonction $loop$ est appelée tous les intervalles
$\Delta$ de temps. Tout d'abord, la fonction incrémente l'âge de chacun des arcs
présent dans la vue partielle $P$. Ensuite, le nœud le plus âgé $q$ est choisi
afin d'initier un mélange. Si le nœud $q$ ne peut être joint -- car parti ou
défaillant -- alors le nœud $p$ éxécute une fonction appropriée (cf.
§\ref{net:sec:leaving}) jusqu'à trouver un nœud avec lequel communiquer. Ainsi,
le vieillissement comme héritage de \CYCLON permet d'accélérer la suppression
des références obsolètes. Une fois que le nœud initiateur $p$ a trouvé un nœud
avec qui effectuer l'échange, il selectionne un échantillon de sa vue partielle
tout en excluant la référence à $q$ et en s'incluant lui-même. La taille de
l'échantillon correspond à la moitié de la taille de la vue partielle avec au
minimum $1$ nœud : sa propre référence (cf. ligne~\ref{net:line:samplesize}). 

Lorsque le nœud $q$ réçoit la demande d'échange, l'événement \textsc{onExchange}
est déclanché. La réponse de $q$ contient également la moitié de sa vue
partielle. Puisque il existe potentiellement des doublons dans les vues
partielles, les nœuds pourraient envoyer des références de leur vis-à-vis par
mégarde ce qui créerait des boucles ($p$ envoie des références de $q$ à
$q$). Les lignes~\ref{net:line:replace1} et~\ref{net:line:replace2} permettent de
remplacer les autoréférences dans l'échantillon afin d'éviter ce comportement
indésirable. Après réception des échantillons respectifs, les nœuds suppriment
l'échantillon qu'ils ont envoyé avant d'intégrer celui qu'ils ont réçu.

\begin{figure*}
  \centering
  \subfloat[Deux nœuds commencent à échanger leurs vues dans \SPRAY.]
  [Le nœud $n_6$ initie un mélange avec $n_1$ lui envoyant $\{n_6,\,n_9\}$.]
  {
\begin{tikzpicture}[scale=1.2]

  \newcommand\X{35pt};
  \newcommand\Y{15pt};

  \draw[->](5+0*\X, 0*\Y) -- (-5+ 2*\X, 0*\Y); %% 1 -> 2
  \draw[->] (-5+2*\X, 5pt) -- (5+\X, \Y);
  \draw[->](2*\X,5pt) -- (5+1*\X, 2*\Y); %% 2 -> 6
  \draw[->] (-5+2*\X, -5pt) -- (5+\X, -\Y);
  \draw[->] (-5+2*\X, -5pt) -- (5+\X, -2*\Y);

  \draw[->,very thick, color=darkblue](-5+\X,2*\Y) -- (0pt,5pt); %% 6 -> 1

  \draw[->](-5+\X, 1*\Y) -- ( 5pt,5pt);
  \draw[->](-5+\X, -1*\Y) -- ( 5pt,-5pt);
  \draw[->](-5+\X, -2*\Y) -- ( 5pt,-5pt);

  \draw[->](-5+\X, 5+2*\Y)to[out=120,in=30](0pt,5+2*\Y); %% 6 -> 7
  \draw[->](-5+\X, 5+2*\Y)to[out=120,in=30](-5-\Y ,5+2*\Y); %% 6 -> 8
  \draw[->](-5+\X, 5+2*\Y)to[out=120,in=30](-10-2*\Y,5+2*\Y); %% 6 -> 9

  \normalsize
  \draw[fill=white, very thick]
  (0*\X, 0*\Y) node{$n_1$} +(-5pt,-5pt) rectangle +(5pt,5pt);
  \draw[fill=white](2*\X, 0*\Y) node{$n_2$} +(-5pt,-5pt) rectangle +(5pt,5pt);

  \draw[fill=white,very thick, draw=darkblue]
  (1*\X,2*\Y) node{\DARKBLUE{$n_6$}} +(-5pt,-5pt) rectangle +(5pt,5pt);
  \draw[fill=white](1*\X,1*\Y) node{$n_5$} +(-5pt,-5pt) rectangle +(5pt,5pt);
  \draw[fill=white](1*\X,-1*\Y) node{$n_4$} +(-5pt,-5pt) rectangle +(5pt,5pt);
  \draw[fill=white](1*\X,-2*\Y) node{$n_3$} +(-5pt,-5pt) rectangle +(5pt,5pt);

  \draw[fill=white]( 0*\X,2*\Y)
  node{$n_7$} +(-5pt,-5pt) rectangle +(5pt,5pt);
  \draw[fill=white](-5+-\Y,2*\Y)node{$n_8$} +(-5pt,-5pt) rectangle +(5pt,5pt);
  \draw[fill=white, draw=darkblue](-10+-2*\Y,2*\Y)
  node{\DARKBLUE{$n_9$}} +(-5pt,-5pt) rectangle +(5pt,5pt);
  

\end{tikzpicture}}
  \hspace{40pt}
  \subfloat[Un nœud \SPRAY receptionne et répond à l'échange.]
  [Le nœud $n_1$ reçoit le message de $n_6$. En réponse, il envoie le 
  multiensemble $\{n_2\}$. Il ajoute $\{n_6,\,n_9\}$ à sa vue.]
  {
\begin{tikzpicture}[scale=1.15]

  \newcommand\X{35pt};
  \newcommand\Y{15pt};

  \draw[->](5+0*\X, 0*\Y) -- (-5+ 2*\X, 0*\Y); %% 1 -> 2
  \draw[->] (-5+2*\X, 5pt) -- (5+\X, \Y);
  \draw[->](2*\X,5pt) -- (5+1*\X, 2*\Y); %% 2 -> 6
  \draw[->] (-5+2*\X, -5pt) -- (5+\X, -\Y);
  \draw[->] (-5+2*\X, -5pt) -- (5+\X, -2*\Y);

  \draw[->,dashed, very thick, color=darkblue](0pt,5pt)--(-5+\X, 2*\Y); %% 1 -> 6

  \draw[->](-5+\X, 1*\Y) -- ( 5pt,5pt);
  \draw[->](-5+\X, -1*\Y) -- ( 5pt,-5pt);
  \draw[->](-5+\X, -2*\Y) -- ( 5pt,-5pt);

  \draw[->](-5+\X, 5+2*\Y)to[out=120,in=30](0pt,5+2*\Y); %% 6 -> 7
  \draw[->](-5+\X, 5+2*\Y)to[out=120,in=30](-5-\Y ,5+2*\Y); %% 6 -> 8
  
  \draw[->,dashed, very thick, color=darkblue](-5pt,5pt)--(-10-2*\Y,-5+2*\Y); %% 1 -> 9

  \normalsize
  \draw[fill=white, very thick]
  (0*\X, 0*\Y) node{$n_1$} +(-5pt,-5pt) rectangle +(5pt,5pt);
  \draw[fill=white, draw=darkblue](2*\X, 0*\Y)
  node{\DARKBLUE{$n_2$}} +(-5pt,-5pt) rectangle +(5pt,5pt);

  \draw[fill=white,very thick]
  (1*\X,2*\Y) node{$n_6$} +(-5pt,-5pt) rectangle +(5pt,5pt);
  \draw[fill=white](1*\X,1*\Y) node{$n_5$} +(-5pt,-5pt) rectangle +(5pt,5pt);
  \draw[fill=white](1*\X,-1*\Y) node{$n_4$} +(-5pt,-5pt) rectangle +(5pt,5pt);
  \draw[fill=white](1*\X,-2*\Y) node{$n_3$} +(-5pt,-5pt) rectangle +(5pt,5pt);

  \draw[fill=white]( 0*\X,2*\Y)
  node{$n_7$} +(-5pt,-5pt) rectangle +(5pt,5pt);
  \draw[fill=white](-5+-\Y,2*\Y)node{$n_8$} +(-5pt,-5pt) rectangle +(5pt,5pt);
  \draw[fill=white](-10+-2*\Y,2*\Y) node{$n_9$} +(-5pt,-5pt) rectangle +(5pt,5pt);
  

\end{tikzpicture}}
  \hspace{10pt}
  \subfloat[Le nœud initiateur réçoit la réponse et établit des connexions.]
  [Le nœud $n_6$ reçoit la réponse de $n_1$ et ajoute $\{n_2\}$ à sa vue
  partielle.]
  {
\begin{tikzpicture}[scale=1.2]

  \newcommand\X{35pt};
  \newcommand\Y{15pt};

  \draw[->] (-5+2*\X, 5pt) -- (5+\X, \Y);
  \draw[->,dashed, very thick, color=darkblue]
  (5+\X, 2*\Y)to[out=-20,in=110](2*\X, 5pt); %% 6 -> 2
  \draw[->](2*\X,5pt)to[out=160,in=-70](5+1*\X, 2*\Y); %% 2 -> 6
  \draw[->] (-5+2*\X, -5pt) -- (5+\X, -\Y);
  \draw[->] (-5+2*\X, -5pt) -- (5+\X, -2*\Y);

  \draw[->](0pt,5pt)--(-5+\X, 2*\Y); %% 1 -> 6

  \draw[->](-5+\X, 1*\Y) -- ( 5pt,5pt);
  \draw[->](-5+\X, -1*\Y) -- ( 5pt,-5pt);
  \draw[->](-5+\X, -2*\Y) -- ( 5pt,-5pt);

  \draw[->](-5+\X, 5+2*\Y)to[out=120,in=30](0pt,5+2*\Y); %% 6 -> 7
  \draw[->](-5+\X, 5+2*\Y)to[out=120,in=30](-5-\Y ,5+2*\Y); %% 6 -> 8
  
  \draw[->](-5pt,5pt)--(-10-2*\Y,-5+2*\Y); %% 1 -> 9

  \normalsize
  \draw[fill=white]
  (0*\X, 0*\Y) node{$n_1$} +(-5pt,-5pt) rectangle +(5pt,5pt);
  \draw[fill=white](2*\X, 0*\Y) node{$n_2$} +(-5pt,-5pt) rectangle +(5pt,5pt);

  \draw[fill=white,very thick]
  (1*\X,2*\Y) node{$n_6$} +(-5pt,-5pt) rectangle +(5pt,5pt);
  \draw[fill=white](1*\X,1*\Y) node{$n_5$} +(-5pt,-5pt) rectangle +(5pt,5pt);
  \draw[fill=white](1*\X,-1*\Y) node{$n_4$} +(-5pt,-5pt) rectangle +(5pt,5pt);
  \draw[fill=white](1*\X,-2*\Y) node{$n_3$} +(-5pt,-5pt) rectangle +(5pt,5pt);

  \draw[fill=white]( 0*\X,2*\Y)
  node{$n_7$} +(-5pt,-5pt) rectangle +(5pt,5pt);
  \draw[fill=white](-5+-\Y,2*\Y)node{$n_8$} +(-5pt,-5pt) rectangle +(5pt,5pt);
  \draw[fill=white](-10+-2*\Y,2*\Y) node{$n_9$} +(-5pt,-5pt) rectangle +(5pt,5pt);
  

\end{tikzpicture}}
  \caption{\label{net:fig:cyclicexample}Exemple du processus de mélange dans \SPRAY.}
\end{figure*}

La figure~\ref{net:fig:cyclicexample} décrit le méchanisme périodique de
\SPRAY. Ce scénario suit celui de la figure~\ref{net:fig:joiningexample} : le
nœud $n_1$ vient de rejoindre le réseau. Le nœud $n_6$ initie un échange avec
$n_1$ qui est ici le plus vieux nœud de sa vue partielle. $n_6$ choisit
$\left\lceil{|P_6|\div 2}\right\rceil = 1$ nœud aléatoirement parmi son
voisinnage. Dans ce cas, il choisit $n_9$ parmi $\{n_9,\,n_8,\,n_7\}$. Il envoie
à $n_1$ cet échantillon auquel est ajoutée sa propre adresse logique. En
réponse, le nœud $n_1$ choisit $\left\lceil{|P_1|\div 2}\right\rceil = 1$ nœud
de sa vue partielle. L'échantillon composé du seul voisin $n_2$ est
envoyé. Immédiatement après, $n_1$ peut supprimer l'échantillon envoyé et
intégrer celui réçu résultant en une vue partielle composée de $\{n_6,\, n_9\}$.
De la même façon, après réception de l'échantillon de $n_1$, $n_6$ supprime et
intègre les échantillons adéquats, résultant en une vue partielle composée de
$\{n_2,\,n_7,\,n_8\}$.

La procédure d'échange tend à réduire l'écart de taille des vues partielles. De
plus, elle disperse les références des nœuds afin de supprimer les groupes trop
denses dûs au processus d'entrée dans le réseau.

En ce qui concerne le temps de convergence de l'algorithme de mélange, il existe
une relation proche entre \SPRAY et le protocole d'aggrégation proactif
présenté dans~\cite{jelasity2004epidemic, montresor2004robust}. Celui-ci déclare
que, sous l'hypothèse d'un échantillonnage suffisamment aléatoire, la valeur
moyenne $\mu$ et la variance $\sigma^2$ a un cycle $i$ sont :
\begin{center}
  $\mu_i = {1\over{|\mathcal{N}|}} \sum\limits_{x \in \mathcal{N}} a_{i,\,x}$
  \hfill
  $\sigma^2_i = {1\over{|\mathcal{N}|-1}}\sum\limits_{x \in \mathcal{N}}
  (a_{i,\,x} - \mu_i)^2$
\end{center}
avec $a_{i,\,x}$ la valeur stockée par le nœud $n_x$ au cycle $i$. La variance
estimée doit converger en 0 au cours des cycles. En d'autres termes, les valeurs
se rapprochent les unes des autres au cours des cycles. Dans le cas de \SPRAY,
cette valeur $a_{i,\,x}$ est la taille de la vue partielle du nœud $n_x$ au
cycle $i$. En effet, chaque échange du nœud $n_1$ et $n_2$ correspond à une
aggrégation dont le résultat est :
$|P_1|\approx|P_2|\approx{(|P_1| + |P_2|) \div 2}$.  De plus, à chaque cycle,
chaque nœud est impliqué dans un protocole d'échange au moins une fois (celui
qu'il initie), et dans le meilleur des cas $1+Poisson(1)$ (celui qu'il initie
et, en moyenne, celui qu'il réçoit d'un autre nœud). Cette relation étant
établie, nous en déduisons que la taille des vues partielles des nœuds de \SPRAY
converge en temps exponentiel vers la moyenne globale. De plus, chaque cycle
réduit la variance du système à un taux compris entre ${1\div 2}$ et
$1\div ({2\sqrt{\text{e}}})$.

%%% Local Variables:
%%% mode: latex
%%% TeX-master: "../../paper"
%%% End:
