
\section{Spray : un protocole d'échantillonnage adaptatif}
\label{net:sec:spray}

\SPRAY est un protocole d'échantillonnage de pairs adaptatif inspiré à
la fois de \SCAMP~\cite{ganesh2003peer} et
\CYCLON~\cite{voulgaris2005cyclon}. \SPRAY comprend trois parties représentant
le cycle de vie d'un nœud dans le réseau. Tout d'abord, le processus consistant
à rejoindre le réseau, qui injecte un nombre logarithmique d'arcs comparé à la
taille du réseau.  Ensuite, chaque pair exécute un processus périodique dont le
but est d'équilibrer les vues partielles en termes de taille et d'uniformité sur
les pairs références dans celles-ci. Le réseau converge rapidement vers une
topologie possédant des propriétés similaires à celles des graphes
aléatoires. Enfin, un nœud est capable de quitter le réseau à n'importe quel
moment sans en notifier le réseau (l'équivalent d'une défaillance) sans dégrader
les propriétés du reste du réseau.

L'obtention de cette propriété d'adaptivité repose essentiellement sur le fait
de conserver un nombre d'arcs cohérent durant tout le cycle de vie du réseau.
En effet, en opposition à \CYCLON, \SPRAY est toujours à la limite du nombre
optimal d'arcs. Comme \SPRAY n'ajoute jamais d'arcs après le processus d'entrée,
toute suppression d'arcs est définitive. De telles décisions ne sont donc pas
prises à la légère. Ainsi, dans un premier temps, \SPRAY ajoute des arcs lors de
l'entrée d'un nœud dans le réseau. Dans un second temps, le processus périodique
de mélange des arcs de \SPRAY préserve tous les arcs du réseau.  Dans un
troisième temps, le processus de sortie supprime précautionneusement quelques
arcs. Dans l'idéal, il s'agit du nombre d'arcs ajoutés par le dernier nœud entré
dans le réseau.

Parfois, conserver un nombre d'arcs global constant force les processus de
mélange et de départ à créer des doublons dans les vues partielles. Ainsi, une
vue partielle peut contenir plusieurs fois le même voisin. Toutefois, ces
doublons restent peu nombreux. De ce fait, ils n'ont pas d'impact notable sur la
connexité du réseau.

Cette section se décompose en trois parties représentant le cycle de vie d'un
nœud : le processus d'entrée (cf. §\ref{net:subsec:joining}) , les mélanges à
répétition (cf. §\ref{net:subsec:shuffling}, la gestion des départs
(cf. §\ref{net:subsec:leaving}).

\subsection{Rejoindre le réseau}
\label{net:subsec:joining}

\begin{figure*}
  \centering
  \subfloat[Un nœud \SPRAY rejoint un réseau.]
  [Le nœud $n_1$ contacte $n_2$ afin de rejoindre le réseau. $n_1$ ajoute
  $n_2$ à sa vue partielle.]
  {
\begin{tikzpicture}[scale=1.2]

  \newcommand\X{35pt};
  \newcommand\Y{15pt};

  \draw(-0.75*\X, 0pt); %% positioning
  \draw( 2.75*\X, 0pt); %% positioning

  \scriptsize
  \draw[->,dashed,very thick, color=darkblue](5+0*\X, 0*\Y) -- 
  node[anchor=south]{(a)}(-5+ 2*\X, 0*\Y);
  \draw[->] (-5+2*\X, 5pt) -- (5+\X, \Y);
  \draw[->] (-5+2*\X, 5pt) --  (5+\X, 2*\Y);
  \draw[->] (-5+2*\X, -5pt) -- (5+\X, -\Y);
  \draw[->] (-5+2*\X, -5pt) -- (5+\X, -2*\Y);

  \normalsize
  \draw[fill=white, very thick, draw=darkblue]
  (0*\X, 0*\Y) node{\DARKBLUE{$n_1$}} +(-5pt,-5pt) rectangle +(5pt,5pt);
  \draw[fill=white, very thick]
  (2*\X, 0*\Y) node{$n_2$} +(-5pt,-5pt) rectangle +(5pt,5pt);

  \draw[fill=white](1*\X,2*\Y) node{$n_6$} +(-5pt,-5pt) rectangle +(5pt,5pt);
  \draw[fill=white](1*\X,1*\Y) node{$n_5$} +(-5pt,-5pt) rectangle +(5pt,5pt);
  \draw[fill=white](1*\X,-1*\Y) node{$n_4$} +(-5pt,-5pt) rectangle +(5pt,5pt);
  \draw[fill=white](1*\X,-2*\Y) node{$n_3$} +(-5pt,-5pt) rectangle +(5pt,5pt);
  
\end{tikzpicture}}
  \hspace{40pt}
  \subfloat[Le contact annonce le nouvel arrivant.]
  [L'événement \textsc{onSubs}($n_1$) est déclanché en $n_2$ qui
  dissémine l'annonce.]
  {
\begin{tikzpicture}[scale=1.2]

  \newcommand\X{35pt};
  \newcommand\Y{15pt};

  \draw(-0.75*\X, 0pt); %% positioning
  \draw( 2.75*\X, 0pt); %% positioning

  \scriptsize
  \draw[->](5+0*\X, 0*\Y) -- (-5+ 2*\X, 0*\Y);
  \draw[->, very thick, color=darkblue] (-5+2*\X, 5pt) -- (5+\X, \Y);
  \draw[->, very thick, color=darkblue] (-5+2*\X, 5pt) --
  node[anchor=south west]{(b)} (5+\X, 2*\Y);
  \draw[->, very thick, color=darkblue] (-5+2*\X, -5pt) -- (5+\X, -\Y);
  \draw[->, very thick, color=darkblue] (-5+2*\X, -5pt) --
  node[anchor=north west]{(b)}(5+\X, -2*\Y);

  \normalsize
  \draw[fill=white]
  (0*\X, 0*\Y) node{$n_1$} +(-5pt,-5pt) rectangle +(5pt,5pt);
  \draw[fill=white, very thick, draw=darkblue]
  (2*\X, 0*\Y) node{\DARKBLUE{$n_2$}} +(-5pt,-5pt) rectangle +(5pt,5pt);

  \draw[fill=white, very thick]
  (1*\X,2*\Y) node{$n_6$} +(-5pt,-5pt) rectangle +(5pt,5pt);
  \draw[fill=white, very thick]
  (1*\X,1*\Y) node{$n_5$} +(-5pt,-5pt) rectangle +(5pt,5pt);
  \draw[fill=white, very thick]
  (1*\X,-1*\Y) node{$n_4$} +(-5pt,-5pt) rectangle +(5pt,5pt);
  \draw[fill=white, very thick]
  (1*\X,-2*\Y) node{$n_3$} +(-5pt,-5pt) rectangle +(5pt,5pt);

\end{tikzpicture}}
  \hspace{10pt}
  \subfloat[Les voisins du contact ajoute le nouvel arrivant.]
  [L'événement \textsc{onFwdSubs}($n_1$) est déclanché en $n_{3-6}$.
  Chacun de ces nœuds ajoute $n_1$ dans sa vue partielle.]
  {
\begin{tikzpicture}[scale=1.2]

  \newcommand\X{35pt};
  \newcommand\Y{15pt};

  \draw(-0.75*\X, 0pt); %% positioning
  \draw( 2.75*\X, 0pt); %% positioning

  \scriptsize
  \draw[->](5+0*\X, 0*\Y) -- (-5+ 2*\X, 0*\Y);
  \draw[->] (-5+2*\X, 5pt) -- (5+\X, \Y);
  \draw[->] (-5+2*\X, 5pt) -- (5+\X, 2*\Y);
  \draw[->] (-5+2*\X, -5pt) -- (5+\X, -\Y);
  \draw[->] (-5+2*\X, -5pt) -- (5+\X, -2*\Y);

  \draw[->,dashed, very thick, color=darkblue](-5+\X, 2*\Y) --
  node[anchor=south east]{(c)} ( 5pt,5pt);
  \draw[->,dashed, very thick, color=darkblue](-5+\X, 1*\Y) -- ( 5pt,5pt);
  \draw[->,dashed, very thick, color=darkblue](-5+\X, -1*\Y) -- ( 5pt,-5pt);
  \draw[->,dashed, very thick, color=darkblue](-5+\X, -2*\Y) --
  node[anchor=north east]{(c)}( 5pt,-5pt);

  \normalsize
  \draw[fill=white, very thick]
  (0*\X, 0*\Y) node{$n_1$} +(-5pt,-5pt) rectangle +(5pt,5pt);
  \draw[fill=white]
  (2*\X, 0*\Y) node{$n_2$} +(-5pt,-5pt) rectangle +(5pt,5pt);

  \draw[fill=white, very thick, draw=darkblue]
  (1*\X,2*\Y) node{\DARKBLUE{$n_6$}} +(-5pt,-5pt) rectangle +(5pt,5pt);
  \draw[fill=white, very thick, draw=darkblue]
  (1*\X,1*\Y) node{\DARKBLUE{$n_5$}} +(-5pt,-5pt) rectangle +(5pt,5pt);
  \draw[fill=white, very thick, draw=darkblue]
  (1*\X,-1*\Y) node{\DARKBLUE{$n_4$}} +(-5pt,-5pt) rectangle +(5pt,5pt);
  \draw[fill=white, very thick, draw=darkblue]
  (1*\X,-2*\Y) node{\DARKBLUE{$n_3$}} +(-5pt,-5pt) rectangle +(5pt,5pt);
 

\end{tikzpicture}}
  \caption[Processus d'entrée dans \SPRAY]
  {\label{net:fig:joiningexample} Exemple de processus d'entrée dans
    le réseau de \SPRAY.}
\end{figure*}

Le processus d'entrée d'un nœud dans le réseau est, dans \SPRAY, la seule
manière d'introduire de nouveaux arcs. Afin de répondre à la première partie de
l'énoncé du problème~\ref{net:problem:properties}, ce nombre d'arcs doit
augmenter logarithmiquement comparé à la taille du réseau. Tout comme dans
\SCAMP, nous supposons que chacuns des nœuds respecte cette contrainte. Dès
lors, ces derniers utilisent cette supposition afin de propager l'adresse
logique du nouvel arrivant. L'algorithme~\ref{net:algo:joining} décrit la façon
dont le contact propage l'annonce à son voisinage où elle est directement
intégrée à leur vue partielle. En définitive, le nombre d'arcs dans le réseau
augmente de $1 + \ln |\mathcal{N}|$ et ce, seulement en utilisant des
interactions de voisin à voisin. Globalement, cela correspond à une augmentation
de l'ordre de $|\mathcal{N}| \ln |\mathcal{N}|$ arcs, similaire au seuil de
connexité exigé par les graphes aléatoires~\cite{erdos1959random}.

\begin{algorithm}[h]
  
\small
\algrenewcommand{\algorithmiccomment}[1]{\hskip2em$\rhd$ #1}

\newcommand{\comment}[1]{\hfill $\rhd$ #1}


\algblockdefx[initially]{initially}{endInitially}
  [0] {\textbf{INITIALLY:}} 

\algsetblockdefx[pas]{pas}{endPas}
  {65535}{}
  [0] {\textbf{EVENTS:}}

\newcommand{\LINEFOR}[2]{%
  \algorithmicfor\ {#1}\ \algorithmicdo\ {#2} %
  }

\newcommand{\LINEIFTHEN}[2]{%
  \algorithmicif\ {#1}\ \algorithmicthen\ {#2} %
  }

\newcommand{\INDSTATE}[1][1]{\State\hspace{\algorithmicindent}}

\begin{algorithmic}[1]
  \Statex
  \initially
    \State $P \leftarrow \varnothing$;
    \hfill \comment{the partial view is a multiset}
    \State $p$ ; \hfill \comment{identity of the local peer}
  \endInitially
  
  \pas
  \Function{onSubs}{$o$} \comment{$o: origin$}
    \If {($|P| = 0$)}
      \State \textsc{onFwdSub}($o$); \comment{special case: 2-nodes network.}
    \Else
      \State \DARKBLUE{\LINEFOR{\textbf{each} $\langle q,\,\_\, \rangle \in P$}
      {$\textsc{sendTo}(q$, 'fwdSubs', $o)$;}}
    \EndIf
    \EndFunction
    \Statex
    \Function{onFwdSubs}{$o$} \comment{$o: origin$}
    \State $P \leftarrow P \,\DARKBLUE{\uplus} \left\{\langle o,\, 0 \rangle\right\}$;
    \EndFunction
%%  \endPas
\end{algorithmic}

  \caption[Processus d'entrée de \SPRAY] {\label{net:algo:joining}Processus
    d'entrée de \SPRAY.}
\end{algorithm}

L'algorithme~\ref{net:algo:joining} présente les instructions de \SPRAY
concernant l'entrée d'un nouveau nœud dans le réseau. La vue partielle est un
multiensemble de couples $\langle n,\, age\rangle$ qui associe à chaque voisin
$n$ son âge dans la vue partielle $age$. Ce multiensemble permet de gérer les
doublons. Ces doublons sont généralement perçus négativement du fait de la
redondance d'information. Toutefois, ils nous sont nécessaire afin de ne pas en
perdre. L'âge joue le même rôle que dans \CYCLON, c'est à dire qu'il accélère la
suppression des nœuds avec lesquels il n'est plus possible de communiquer car
partis ou défaillant. L'événement \textsc{onSubs} est déclanché chaque fois qu'un
nœud rejoint le réseau via ce contact. Ce dernier distribue l'adresse logique du
nouvel arrivant à son voisinage, plusieurs fois aux doublons, et indépendament
de l'âge. L'événement \textsc{onFwdSubs} se déclanche lorsqu'un nœud reçoit la
redirection de l'adresse logique. Le nœud établit alors la connexion dont il
initialise l'âge à $0$.

La figure~\ref{net:fig:joiningexample} décrit un scénario où le pair $n_1$
contacte le pair $n_2$ afin de rejoindre le réseau composé de $\{n_2$, $n_3$,
$n_4$, $n_5$, $n_6\}$. Pour simplifier, la figure ne montre que les arcs
nouvellement introduits ainsi que les voisinages de $n_1$ et $n_2$. Le pair
$n_1$ ajoute directement $n_2$ dans sa vue partielle. Ce dernier redirige
l'adresse logique de $n_1$ à chacun de ses voisins.  Chacun de ces voisins
ajoute alors $n_1$ à leur vue partielle. Au total, \SPRAY établit 5
connexions. Le réseau en résultant est connexe.

Malheureusement, les vues partielles des derniers arrivant sont clairement
déséquilibrées comparé au reste du réseau. De ce fait, ils violent la première
condition de l'énoncé du problème~\ref{net:problem:properties}. Le processus de
mélange décrit dans la section suivante a pour objectif d'équilibrer les vues
partielles.

\subsection{Mélanger son voisinage}
\label{net:subsec:shuffling}

Au contraire de \CYCLON, \SPRAY mélange des vues partielles dont les tailles
peuvent être différentes. Ce processus a pour but d'équilibrer les tailles de
vue partielles ainsi que de mélanger les adresses logiques à l'interieur de
celles-ci.  Ce processus possède la contrainte de conserver l'exact même nombre
d'arcs.

Les deux nœuds impliqués dans le mélange s'envoient l'un l'autre la moitié de
leur vue partielle. Après l'intégration de ces nouvelles références, la taille
de leur vue partielle tend vers la moyenne, et la somme globale en demeure
inchangée. Dans ce but, les vues partielles sont des multiensembles. Ainsi, si
un nœud réçoit un arc déjà connue, il le conserve en tant que doublon.  De cette
façon, le nombre d'arcs reste globalement constant.

Si les doublons ont un impact négatif sur les propriétés du réseau, la plupart
de ceux-ci disparaissent après le processus de mélange. En proportion, ils
deviennent négligeable dès lors que le réseau grandit.

\begin{algorithm}[h]
  
\small
\algrenewcommand{\algorithmiccomment}[1]{\hskip2em$\rhd$ #1}

\newcommand{\comment}[1]{$\rhd$ #1}

\algblockdefx[act]{act}{endAct}
  [0] {\textbf{ACTIVE THREAD:}}

\algsetblockdefx[pas]{pas}{endPas}
  {65535}{}
  [0] {\textbf{PASSIVE THREAD:}}


\newcommand{\LINEFOR}[2]{%
  \algorithmicfor\ {#1}\ \algorithmicdo\ {#2} %
  }

\newcommand{\LINEIFTHEN}[2]{%
  \algorithmicif\ {#1}\ \algorithmicthen\ {#2} %
  }

\newcommand{\INDSTATE}[1][1]{\State\hspace{\algorithmicindent}}

\begin{algorithmic}[1]
  \Statex
  \act
    \Function{loop}{ } \hfill \comment{Every $\Delta\,t$}
    \State $P \leftarrow \textsc{incrementAge}(P)$;
    \State \textbf{let} $ \langle q,\, age \rangle \leftarrow \textsc{getOldest}(P)$;

    \State \textbf{let}
    $sample \leftarrow \textsc{getSample}(P \setminus\left\{\langle q,
      age\rangle\right\}, \DARKBLUE{\left \lceil{|P|\over{2}} \right \rceil-1}) \uplus
    \DARKBLUE{\left\{\langle p, 0 \rangle\right\}}$; \label{net:line:samplesize}

    \State $sample \leftarrow \DARKBLUE{\textsc{replace}}(sample,\,q,\,p)$;
    \label{net:line:replace1}
    \State $\textsc{sendTo}(q$, 'exchange', $sample)$;
    \State \textbf{let} $sample'\leftarrow \textsc{receiveFrom}(q)$;
    \State $sample \leftarrow \textsc{replace}(sample,\,p,\,q)$;
    \State $P \leftarrow (P\,\DARKBLUE{\setminus\, sample})\,\DARKBLUE{\uplus\, sample'}$;
    \EndFunction
  \endAct
  
  \pas
    \Function{onExchange}{$o,\, sample$} \hfill \comment{$o: origin$}
    \State \textbf{let}
    $sample' \leftarrow \textsc{getSample}(P ,\, \DARKBLUE{\left\lceil |P|\over{2}
    \right\rceil} )$;
    \State $sample' \leftarrow \textsc{replace}(sample',\,o,\,p);$
    \label{net:line:replace2}
    \State $\textsc{sendTo}(o$, $sample')$;
    \State $sample' \leftarrow \textsc{replace}(sample',\,p,\,o)$;
    \State $P \leftarrow (P \setminus sample') \uplus sample$; 
    \EndFunction
%%  \endPas
\end{algorithmic}

  \caption[Processus périodique de mélange de \SPRAY]
  {\label{net:algo:shuffle}Processus périodique de mélange de \SPRAY.}
\end{algorithm}

L'algorithme~\ref{net:algo:shuffle} montre les instructions periodiquement
éxecutées par chacun des nœuds. Il est divisé en deux parties, à savoir le
processus actif qui est appelé régulièrement afin d'initier un mélange de vues,
et le processus passif qui réagit au message du processus actif. Les fonctions
qui ne sont pas explicitement définies sont les suivantes:
\begin{inparaenum}[]
\item $\textsc{incrementAge}(view)$ qui incrémente l'âge des éléments présent
  dans la vue $view$ et retourne la vue modifiée;
\item $\textsc{getOldest}(view)$ qui retourne le plus vieux des voisins présent
  dans la vue partielle $view$;
\item $\textsc{getSample}(view,\, size)$ qui retourne un échantillon aléatoire
  de la vue contenant $size$ éléments;
\item $\textsc{replace}(view,\,old,\,new)$ qui remplace les occurrences de $old$
  par $new$ dans la vue $view$. Si les doublons sont admis dans la vue
  partielle, les autoréférences restent hautement indésirables;
\item $\textsc{rand}()$ qui retourne un nombre flottant aléatoire compris entre
  $0$ et $1$.
\end{inparaenum}

Dans le processus actif, la fonction $loop$ est appelée tous les intervalles
$\Delta$ de temps. Tout d'abord, la fonction incrémente l'âge de chacun des arcs
présent dans la vue partielle $P$. Ensuite, le nœud le plus âgé $q$ est choisi
afin d'initier un mélange. Si le nœud $q$ ne peut être joint -- car parti ou
défaillant -- alors le nœud $p$ éxécute une fonction appropriée (cf.
§\ref{net:subsec:leaving}) jusqu'à trouver un nœud avec lequel
communiquer. Ainsi, le vieillissement comme héritage de \CYCLON permet
d'accélérer la suppression des références obsolètes. Une fois que le nœud
initiateur $p$ a trouvé un nœud avec qui effectuer l'échange, il selectionne un
échantillon de sa vue partielle tout en excluant la référence à $q$ et en
s'incluant lui-même. La taille de l'échantillon correspond à la moitié de la
taille de la vue partielle avec au minimum $1$ nœud : sa propre référence
(cf. ligne~\ref{net:line:samplesize}).


Lorsque le nœud $q$ réçoit la demande d'échange, l'événement \textsc{onExchange}
est déclanché. La réponse de $q$ contient également la moitié de sa vue
partielle. Puisque il existe potentiellement des doublons dans les vues
partielles, les nœuds pourraient envoyer des références de leur vis-à-vis par
mégarde ce qui créerait des boucles ($p$ envoie des références de $q$ à
$q$). Les lignes~\ref{net:line:replace1} et~\ref{net:line:replace2} permettent de
remplacer les autoréférences dans l'échantillon afin d'éviter ce comportement
indésirable. Après réception des échantillons respectifs, les nœuds suppriment
l'échantillon qu'ils ont envoyé avant d'intégrer celui qu'ils ont réçu.

\begin{figure*}
  \centering
  \subfloat[Deux nœuds commencent à échanger leurs vues dans \SPRAY]
  [Le nœud $n_6$ initie un mélange avec $n_1$ lui envoyant $\{n_6,\,n_9\}$.]
  {
\begin{tikzpicture}[scale=1.2]

  \newcommand\X{35pt};
  \newcommand\Y{15pt};

  \draw[->](5+0*\X, 0*\Y) -- (-5+ 2*\X, 0*\Y); %% 1 -> 2
  \draw[->] (-5+2*\X, 5pt) -- (5+\X, \Y);
  \draw[->](2*\X,5pt) -- (5+1*\X, 2*\Y); %% 2 -> 6
  \draw[->] (-5+2*\X, -5pt) -- (5+\X, -\Y);
  \draw[->] (-5+2*\X, -5pt) -- (5+\X, -2*\Y);

  \draw[->,very thick, color=darkblue](-5+\X,2*\Y) -- (0pt,5pt); %% 6 -> 1

  \draw[->](-5+\X, 1*\Y) -- ( 5pt,5pt);
  \draw[->](-5+\X, -1*\Y) -- ( 5pt,-5pt);
  \draw[->](-5+\X, -2*\Y) -- ( 5pt,-5pt);

  \draw[->](-5+\X, 5+2*\Y)to[out=120,in=30](0pt,5+2*\Y); %% 6 -> 7
  \draw[->](-5+\X, 5+2*\Y)to[out=120,in=30](-5-\Y ,5+2*\Y); %% 6 -> 8
  \draw[->](-5+\X, 5+2*\Y)to[out=120,in=30](-10-2*\Y,5+2*\Y); %% 6 -> 9

  \normalsize
  \draw[fill=white, very thick]
  (0*\X, 0*\Y) node{$n_1$} +(-5pt,-5pt) rectangle +(5pt,5pt);
  \draw[fill=white](2*\X, 0*\Y) node{$n_2$} +(-5pt,-5pt) rectangle +(5pt,5pt);

  \draw[fill=white,very thick, draw=darkblue]
  (1*\X,2*\Y) node{\DARKBLUE{$n_6$}} +(-5pt,-5pt) rectangle +(5pt,5pt);
  \draw[fill=white](1*\X,1*\Y) node{$n_5$} +(-5pt,-5pt) rectangle +(5pt,5pt);
  \draw[fill=white](1*\X,-1*\Y) node{$n_4$} +(-5pt,-5pt) rectangle +(5pt,5pt);
  \draw[fill=white](1*\X,-2*\Y) node{$n_3$} +(-5pt,-5pt) rectangle +(5pt,5pt);

  \draw[fill=white]( 0*\X,2*\Y)
  node{$n_7$} +(-5pt,-5pt) rectangle +(5pt,5pt);
  \draw[fill=white](-5+-\Y,2*\Y)node{$n_8$} +(-5pt,-5pt) rectangle +(5pt,5pt);
  \draw[fill=white, draw=darkblue](-10+-2*\Y,2*\Y)
  node{\DARKBLUE{$n_9$}} +(-5pt,-5pt) rectangle +(5pt,5pt);
  

\end{tikzpicture}}
  \hspace{40pt}
  \subfloat[Un nœud \SPRAY réceptionne et répond à l'échange]
  [Le nœud $n_1$ reçoit le message de $n_6$. En réponse, il envoie le 
  multiensemble $\{n_2\}$. Il ajoute $\{n_6,\,n_9\}$ à sa vue.]
  {
\begin{tikzpicture}[scale=1.2]

  \newcommand\X{35pt};
  \newcommand\Y{15pt};

  \draw[->](5+0*\X, 0*\Y) -- (-5+ 2*\X, 0*\Y); %% 1 -> 2
  \draw[->] (-5+2*\X, 5pt) -- (5+\X, \Y);
  \draw[->](2*\X,5pt) -- (5+1*\X, 2*\Y); %% 2 -> 6
  \draw[->] (-5+2*\X, -5pt) -- (5+\X, -\Y);
  \draw[->] (-5+2*\X, -5pt) -- (5+\X, -2*\Y);

  \draw[->,dashed, very thick, color=darkblue](0pt,5pt)--(-5+\X, 2*\Y); %% 1 -> 6

  \draw[->](-5+\X, 1*\Y) -- ( 5pt,5pt);
  \draw[->](-5+\X, -1*\Y) -- ( 5pt,-5pt);
  \draw[->](-5+\X, -2*\Y) -- ( 5pt,-5pt);

  \draw[->](-5+\X, 5+2*\Y)to[out=120,in=30](0pt,5+2*\Y); %% 6 -> 7
  \draw[->](-5+\X, 5+2*\Y)to[out=120,in=30](-5-\Y ,5+2*\Y); %% 6 -> 8
  
  \draw[->,dashed, very thick, color=darkblue](-5pt,5pt)--(-10-2*\Y,-5+2*\Y); %% 1 -> 9

  \normalsize
  \draw[fill=white, very thick]
  (0*\X, 0*\Y) node{$n_1$} +(-5pt,-5pt) rectangle +(5pt,5pt);
  \draw[fill=white, draw=darkblue](2*\X, 0*\Y)
  node{\DARKBLUE{$n_2$}} +(-5pt,-5pt) rectangle +(5pt,5pt);

  \draw[fill=white,very thick]
  (1*\X,2*\Y) node{$n_6$} +(-5pt,-5pt) rectangle +(5pt,5pt);
  \draw[fill=white](1*\X,1*\Y) node{$n_5$} +(-5pt,-5pt) rectangle +(5pt,5pt);
  \draw[fill=white](1*\X,-1*\Y) node{$n_4$} +(-5pt,-5pt) rectangle +(5pt,5pt);
  \draw[fill=white](1*\X,-2*\Y) node{$n_3$} +(-5pt,-5pt) rectangle +(5pt,5pt);

  \draw[fill=white]( 0*\X,2*\Y)
  node{$n_7$} +(-5pt,-5pt) rectangle +(5pt,5pt);
  \draw[fill=white](-5+-\Y,2*\Y)node{$n_8$} +(-5pt,-5pt) rectangle +(5pt,5pt);
  \draw[fill=white](-10+-2*\Y,2*\Y) node{$n_9$} +(-5pt,-5pt) rectangle +(5pt,5pt);
  

\end{tikzpicture}}
  \hspace{10pt}
  \subfloat[Le nœud initiateur reçoit la réponse et établit des connexions]
  [Le nœud $n_6$ reçoit la réponse de $n_1$ et ajoute $\{n_2\}$ à sa vue
  partielle.]
  {
\begin{tikzpicture}[scale=1.2]

  \newcommand\X{35pt};
  \newcommand\Y{15pt};

  \draw[->] (-5+2*\X, 5pt) -- (5+\X, \Y);
  \draw[->,dashed, very thick, color=darkblue]
  (5+\X, 2*\Y)to[out=-20,in=110](2*\X, 5pt); %% 6 -> 2
  \draw[->](2*\X,5pt)to[out=160,in=-70](5+1*\X, 2*\Y); %% 2 -> 6
  \draw[->] (-5+2*\X, -5pt) -- (5+\X, -\Y);
  \draw[->] (-5+2*\X, -5pt) -- (5+\X, -2*\Y);

  \draw[->](0pt,5pt)--(-5+\X, 2*\Y); %% 1 -> 6

  \draw[->](-5+\X, 1*\Y) -- ( 5pt,5pt);
  \draw[->](-5+\X, -1*\Y) -- ( 5pt,-5pt);
  \draw[->](-5+\X, -2*\Y) -- ( 5pt,-5pt);

  \draw[->](-5+\X, 5+2*\Y)to[out=120,in=30](0pt,5+2*\Y); %% 6 -> 7
  \draw[->](-5+\X, 5+2*\Y)to[out=120,in=30](-5-\Y ,5+2*\Y); %% 6 -> 8
  
  \draw[->](-5pt,5pt)--(-10-2*\Y,-5+2*\Y); %% 1 -> 9

  \normalsize
  \draw[fill=white]
  (0*\X, 0*\Y) node{$n_1$} +(-5pt,-5pt) rectangle +(5pt,5pt);
  \draw[fill=white](2*\X, 0*\Y) node{$n_2$} +(-5pt,-5pt) rectangle +(5pt,5pt);

  \draw[fill=white,very thick]
  (1*\X,2*\Y) node{$n_6$} +(-5pt,-5pt) rectangle +(5pt,5pt);
  \draw[fill=white](1*\X,1*\Y) node{$n_5$} +(-5pt,-5pt) rectangle +(5pt,5pt);
  \draw[fill=white](1*\X,-1*\Y) node{$n_4$} +(-5pt,-5pt) rectangle +(5pt,5pt);
  \draw[fill=white](1*\X,-2*\Y) node{$n_3$} +(-5pt,-5pt) rectangle +(5pt,5pt);

  \draw[fill=white]( 0*\X,2*\Y)
  node{$n_7$} +(-5pt,-5pt) rectangle +(5pt,5pt);
  \draw[fill=white](-5+-\Y,2*\Y)node{$n_8$} +(-5pt,-5pt) rectangle +(5pt,5pt);
  \draw[fill=white](-10+-2*\Y,2*\Y) node{$n_9$} +(-5pt,-5pt) rectangle +(5pt,5pt);
  

\end{tikzpicture}}
  \caption[Processus de mélange périodique dans \SPRAY]
  {\label{net:fig:cyclicexample}Exemple du processus de mélange dans \SPRAY.}
\end{figure*}

La figure~\ref{net:fig:cyclicexample} décrit le méchanisme périodique de
\SPRAY. Ce scénario suit celui de la figure~\ref{net:fig:joiningexample} : le
nœud $n_1$ vient de rejoindre le réseau. Le nœud $n_6$ initie un échange avec
$n_1$ qui est ici le plus vieux nœud de sa vue partielle. $n_6$ choisit
$\left\lceil{|P_6|\div 2}\right\rceil = 1$ nœud aléatoirement parmi son
voisinnage. Dans ce cas, il choisit $n_9$ parmi $\{n_9,\,n_8,\,n_7\}$. Il envoie
à $n_1$ cet échantillon auquel est ajoutée sa propre adresse logique. En
réponse, le nœud $n_1$ choisit $\left\lceil{|P_1|\div 2}\right\rceil = 1$ nœud
de sa vue partielle. L'échantillon composé du seul voisin $n_2$ est
envoyé. Immédiatement après, $n_1$ peut supprimer l'échantillon envoyé et
intégrer celui réçu résultant en une vue partielle composée de $\{n_6,\, n_9\}$.
De la même façon, après réception de l'échantillon de $n_1$, $n_6$ supprime et
intègre les échantillons adéquats, résultant en une vue partielle composée de
$\{n_2,\,n_7,\,n_8\}$.

La procédure d'échange tend à réduire l'écart de taille des vues partielles. De
plus, elle disperse les références des nœuds afin de supprimer les groupes trop
denses dûs au processus d'entrée dans le réseau.

En ce qui concerne le temps de convergence de l'algorithme de mélange, il existe
une relation proche entre \SPRAY et le protocole d'aggrégation proactif
présenté dans~\cite{jelasity2004epidemic, montresor2004robust}. Celui-ci déclare
que, sous l'hypothèse d'un échantillonnage suffisamment aléatoire, la valeur
moyenne $\mu$ et la variance $\sigma^2$ a un cycle $i$ sont :
\begin{center}
  $\mu_i = {1\over{|\mathcal{N}|}} \sum\limits_{x \in \mathcal{N}} a_{i,\,x}$
  \hfill
  $\sigma^2_i = {1\over{|\mathcal{N}|-1}}\sum\limits_{x \in \mathcal{N}}
  (a_{i,\,x} - \mu_i)^2$
\end{center}
avec $a_{i,\,x}$ la valeur stockée par le nœud $n_x$ au cycle $i$. La variance
estimée doit converger en 0 au cours des cycles. En d'autres termes, les valeurs
se rapprochent les unes des autres au cours des cycles. Dans le cas de \SPRAY,
cette valeur $a_{i,\,x}$ est la taille de la vue partielle du nœud $n_x$ au
cycle $i$. En effet, chaque échange du nœud $n_1$ et $n_2$ correspond à une
aggrégation dont le résultat est :
$|P_1|\approx|P_2|\approx{(|P_1| + |P_2|) \div 2}$.  De plus, à chaque cycle,
chaque nœud est impliqué dans un protocole d'échange au moins une fois (celui
qu'il initie), et dans le meilleur des cas $1+Poisson(1)$ (celui qu'il initie
et, en moyenne, celui qu'il réçoit d'un autre nœud). Cette relation étant
établie, nous en déduisons que la taille des vues partielles des nœuds de \SPRAY
converge en temps exponentiel vers la moyenne globale. De plus, chaque cycle
réduit la variance du système à un taux compris entre ${1\div 2}$ et
$1\div ({2\sqrt{\text{e}}})$.

\subsection{Quitter le réseau}
\label{net:subsec:leaving}

Les nœuds utilisant \SPRAY peuvent quitter le réseau sans en informer qui que ce
soit. Ainsi, les départs et défaillances sont gérés de façon identique. Sans
réaction appropriée, la suppression d'arcs entrainerait l'effondrement du
réseau.

Lorsqu'un nœud rejoint un réseau, il y injecte $1+\ln(|\mathcal{N}|)$
arcs. Néanmoins, après plusieurs échanges de voisinage, sa vue partielle se
trouve remplie d'autres références. Ainsi, lorsqu'il quitte le réseau, il
entraine la suppression de $\ln(|\mathcal{N}|)$ arcs provenant de sa vue
partielle, et $\ln(|\mathcal{N}|)$ arcs provenant des nœuds l'ayant dans leur
vue partielle. Par conséquent, $2\ln(\mathcal{N}|)$ arcs sont supprimés au lieu
des $1+\ln(|\mathcal{N}|)$. Afin d'y remedier, chaque nœud qui détecte une
défaillance ou un départ peut rétablir une connexion avec l'un de ces voisins en
introduisant un doublon -- doublon qui disparaitra rapidement après quelques
mélanges. La probabilité de réétablir une connexion est
$1-1\div{|\mathcal{P}|}$. Puisque ${|\mathcal{P}|}\approx \ln(|\mathcal{N}|)$
nœuds avaient le nœud défaillant/parti dans leur vue partielle, il est probable
qu'ils recréent tous la connexion perdue, à l'exception d'un nœud. De ce fait,
lorsqu'un nœud quitte le réseau, il entraine la suppression d'un nombre d'arcs
correspondant approximativement au nombre d'arcs ajouté lors de la dernière
entrée.

\begin{algorithm}[h]
  
\small
\algrenewcommand{\algorithmiccomment}[1]{\hskip2em$\rhd$ #1}

\newcommand{\comment}[1]{$\rhd$ #1}

\newcommand{\LINEFOR}[2]{%
  \algorithmicfor\ {#1}\ \algorithmicdo\ {#2} %
  }

\newcommand{\LINEIFTHEN}[2]{%
  \algorithmicif\ {#1}\ \algorithmicthen\ {#2} %
  }

\newcommand{\INDSTATE}[1][1]{\State\hspace{\algorithmicindent}}

\begin{algorithmic}[1]
  \Function{onPeerDown}{$q$} \hfill \comment{$q$: nœud parti ou défaillant}  
  \State \textbf{let} $occ \leftarrow 0$;

  \For{\textbf{each} $\langle n,\,age\rangle \in P$}
  \hfill \comment{supprime et compte}
  \If {($n=q$)}
  \State $P \leftarrow P\setminus \{\langle n,\,age\rangle \}$;
  \State $occ \leftarrow occ + 1$;
  \EndIf
  \EndFor

  \For{$i\leftarrow 0$ \textbf{to} $occ$} 
  \hfill \comment{Duplique de manière probabiliste}
  \If{($rand()>\DARKBLUE{{1\over{|P|+occ}}}$)}
    \State \textbf{let} $\langle n,\,\_ \,\rangle \leftarrow
    P[\left\lfloor rand()*|P|\right\rfloor]$;
    \State $P \leftarrow P \uplus
    \left\{\langle n,\, 0\rangle\right\}$;
  \EndIf
  \EndFor
  
  \EndFunction
  \Statex
  \Function{onArcDown}{$q,\,age$}\hfill\comment{$q$: arrival of the arc down}  
  \State $P \leftarrow P \setminus \{\langle q,\,age\rangle \}$;
  \State \textbf{let} $\langle n,\,\_ \,\rangle \leftarrow
  P[\left\lfloor rand()*|P|\right\rfloor]$;
  \State $P \leftarrow P \uplus
  \left\{\langle n,\, 0\rangle\right\}$;
  \hfill \comment{Duplique systématiquement}
  \EndFunction

\end{algorithmic}

  \caption[Gestion des pannes et départs de \SPRAY]
  {\label{net:algo:leaving}Gestion des pannes et départs de \SPRAY.}
\end{algorithm}

L'algorithme~\ref{net:algo:leaving} montre la manière selon laquelle \SPRAY gère
les départs et défaillances. La fonction \textsc{onPeerDown} montre la réaction
de \SPRAY lorsque un nœud $q$ est detecté comme parti ou défaillant. Dans un
premier temps, la fonction compte les occurences dudit nœud dans la vue
partielle et les supprime. Dans un second temps, une boucle ajoute de manière
probabiliste des doublons de références à des nœuds déjà connus. La probabilité
dépend de la taille de la vue partielle avant suppression.

\begin{figure*}
  \centering
  \subfloat[Un nœud \SPRAY quitte le réseau]
  [Le nœud $n_1$ quitte le réseau.]
  {
\begin{tikzpicture}[scale=1.2]

  \newcommand\X{35pt};
  \newcommand\Y{15pt};

  \large
  \draw[->](-5+\X, 1*\Y) --node{$\times$} ( 5pt,5pt);
  \draw[->](-5+\X, -1*\Y) --node{$\times$} ( 5pt,-5pt);
  \draw[->](-5+\X, -2*\Y) --node{$\times$} ( 5pt,-5pt);

  \draw[->, color=darkblue](-5pt,5pt)--
  node{\DARKBLUE{$\times$}}(-10-2*\Y,-5+2*\Y); %% 1 -> 9
  \draw[->, color=darkblue](-5pt,5pt)--
  node{\DARKBLUE{$\times$}}(-5-1*\Y,-5+2*\Y); %% 1 ->8 
  \draw[->, color=darkblue](-5pt,5pt)--
  node{\DARKBLUE{$\times$}}(0pt,-5+2*\Y); %% 1 -> 7
  \draw[->, color=darkblue](-5pt,5pt)--
  node{\DARKBLUE{$\times$}}(-5+\X,-5+2*\Y); %% 1 -> 6

  \normalsize
  \draw[->](5+ 1*\X, 5+ 1*\Y)--(-5+2*\X, 2*\Y); %% 5 -> 14
  \draw[->](5+1*\X,  1*\Y)--(-5+2*\X, 1*\Y); %% 5 -> 13 
  
  \draw[->](5+\X, 5-\Y) -- (-5+2*\X,0pt); %% 4 -> 12
  \draw[->](5+\X, -\Y) -- (-5+2*\X, -\Y); %% 4 -> 11
  
  \draw[->](5+\X, -2*\Y) -- (-5+2*\X, -2*\Y);
  
  \small
  \draw[fill=white,very thick, draw=darkblue]
  (0*\X, 0*\Y) node{\DARKBLUE{$n_1$}} +(-5pt,-5pt) rectangle +(5pt,5pt);
  \draw[thick, color=darkblue] (-5pt,-5pt) -- (5pt,5pt);
  \draw[thick, color=darkblue] (-5pt, 5pt) -- (5pt,-5pt);
  
  \draw[fill=white]
  (1*\X,1*\Y) node{$n_5$} +(-5pt,-5pt) rectangle +(5pt,5pt);
  \draw[fill=white]
  (1*\X,-1*\Y) node{$n_4$} +(-5pt,-5pt) rectangle +(5pt,5pt);
  \draw[fill=white]
  (1*\X,-2*\Y) node{$n_3$} +(-5pt,-5pt) rectangle +(5pt,5pt);

  \draw[fill=white](\X,2*\Y) node{$n_6$} +(-5pt,-5pt) rectangle +(5pt,5pt);

  \draw[fill=white]( 0*\X,2*\Y)
  node{$n_7$} +(-5pt,-5pt) rectangle +(5pt,5pt);
  \draw[fill=white](-5+-\Y,2*\Y)node{$n_8$} +(-5pt,-5pt) rectangle +(5pt,5pt);
  \draw[fill=white](-10+-2*\Y,2*\Y) node{$n_9$} +(-5pt,-5pt) rectangle +(5pt,5pt);
  
  \draw[fill=white](2*\X,2*\Y)node{$n_{14}$} +(-5pt,-5pt) rectangle +(5pt,5pt);
  \draw[fill=white](2*\X,1*\Y)node{$n_{13}$} +(-5pt,-5pt) rectangle +(5pt,5pt);
  \draw[fill=white](2*\X,0*\Y)node{$n_{12}$} +(-5pt,-5pt) rectangle +(5pt,5pt);
  \draw[fill=white](2*\X,-1*\Y)node{$n_{11}$}+(-5pt,-5pt) rectangle +(5pt,5pt);
  \draw[fill=white](2*\X,-2*\Y)node{$n_{10}$}+(-5pt,-5pt) rectangle +(5pt,5pt);

\end{tikzpicture}}
  \hspace{40pt}
  \subfloat[Certains nœuds détectent le départ]
  [Les nœuds $n_{3-5}$ s'aperçoivent qu'ils ne peuvent plus communiquer avec
  $n_1$]
  {
\begin{tikzpicture}[scale=1.2]

  \newcommand\X{35pt};
  \newcommand\Y{15pt};

  \large
  \draw[->, very thick, color=darkblue](-5+\X, 1*\Y) --
  node{\DARKBLUE{$\times$}} ( 5pt,5pt);
  \draw[->, very thick, color=darkblue](-5+\X, -1*\Y) --
  node{\DARKBLUE{$\times$}} ( 5pt,-5pt);
  \draw[->, very thick, color=darkblue](-5+\X, -2*\Y) --
  node{\DARKBLUE{$\times$}} ( 5pt,-5pt);

  \normalsize

  \draw[->](5+ 1*\X, 5+ 1*\Y)--(-5+2*\X, 2*\Y); %% 5 -> 14
  \draw[->](  5+1*\X, 1*\Y)--(-5+2*\X, 1*\Y); %% 5 -> 13 (v)
  
  \draw[->](5+\X, 5-\Y) -- (-5+2*\X,0pt); %% 4 -> 12
  \draw[->](5+\X, -\Y) -- (-5+2*\X, -\Y); %% 4 -> 11
  
  \draw[->](5+\X, -2*\Y) -- (-5+2*\X, -2*\Y);
  
  \small
  \draw[fill=white]
  (0*\X, 0*\Y) node{$n_1$} +(-5pt,-5pt) rectangle +(5pt,5pt);
  \draw (-5pt,-5pt) -- (5pt,5pt);
  \draw (-5pt, 5pt) -- (5pt,-5pt);
  
  \draw[fill=white, very thick, draw=darkblue]
  (1*\X,1*\Y) node{\DARKBLUE{$n_5$}} +(-5pt,-5pt) rectangle +(5pt,5pt);
  \draw[fill=white, very thick, draw=darkblue]
  (1*\X,-1*\Y) node{\DARKBLUE{$n_4$}} +(-5pt,-5pt) rectangle +(5pt,5pt);
  \draw[fill=white, very thick, draw=darkblue]
  (1*\X,-2*\Y) node{\DARKBLUE{$n_3$}} +(-5pt,-5pt) rectangle +(5pt,5pt);

  \draw[fill=white](\X,2*\Y) node{$n_6$} +(-5pt,-5pt) rectangle +(5pt,5pt);

  \draw[fill=white]( 0*\X,2*\Y)
  node{$n_7$} +(-5pt,-5pt) rectangle +(5pt,5pt);
  \draw[fill=white](-5+-\Y,2*\Y)node{$n_8$} +(-5pt,-5pt) rectangle +(5pt,5pt);
  \draw[fill=white](-10+-2*\Y,2*\Y) node{$n_9$} +(-5pt,-5pt) rectangle +(5pt,5pt);
  
  \draw[fill=white](2*\X,2*\Y)node{$n_{14}$} +(-5pt,-5pt) rectangle +(5pt,5pt);
  \draw[fill=white](2*\X,1*\Y)node{$n_{13}$} +(-5pt,-5pt) rectangle +(5pt,5pt);
  \draw[fill=white](2*\X,0*\Y)node{$n_{12}$} +(-5pt,-5pt) rectangle +(5pt,5pt);
  \draw[fill=white](2*\X,-1*\Y)node{$n_{11}$}+(-5pt,-5pt) rectangle +(5pt,5pt);
  \draw[fill=white](2*\X,-2*\Y)node{$n_{10}$}+(-5pt,-5pt) rectangle +(5pt,5pt);

\end{tikzpicture}}
  \hspace{40pt}
  \subfloat[Certains nœuds dupliquent des arcs]
  [Les nœuds $n_3$ et $n_5$ établissent des arcs doublons dans leur
  vue partielle.]
  {
\begin{tikzpicture}[scale=1.2]

  \newcommand\X{35pt};
  \newcommand\Y{15pt};

  \draw[->](5+ 1*\X, 5+ 1*\Y)--(-5+2*\X, 2*\Y); %% 5 -> 14
  \draw[->](5+1*\X, 2.5+ 1*\Y)--(-5+2*\X, 2.5+ 1*\Y); %% 5 -> 13 (^)
  \draw[->,dashed, very thick, color=darkblue]
  (  5+1*\X,-2.5+1*\Y)--(-5+2*\X,-2.5+1*\Y); %% 5 -> 13 (v)
  
  \draw[->](5+\X, 5-\Y) -- (-5+2*\X,0pt); %% 4 -> 12
  \draw[->](5+\X, -\Y) -- (-5+2*\X, -\Y); %% 4 -> 11
  
  \draw[->](5+\X, 2.5-2*\Y) -- (-5+2*\X, 2.5-2*\Y);
  \draw[->,dashed, very thick, color=darkblue](5+\X, -2.5-2*\Y) -- (-5+2*\X , -2.5-2*\Y);
  
  \small
  \draw[fill=white]
  (0*\X, 0*\Y) node{$n_1$} +(-5pt,-5pt) rectangle +(5pt,5pt);
  \draw (-5pt,-5pt) -- (5pt,5pt);
  \draw (-5pt, 5pt) -- (5pt,-5pt);
  
  \draw[fill=white, very thick]
  (1*\X,1*\Y) node{$n_5$} +(-5pt,-5pt) rectangle +(5pt,5pt);
  \draw[fill=white, very thick]
  (1*\X,-1*\Y) node{$n_4$} +(-5pt,-5pt) rectangle +(5pt,5pt);
  \draw[fill=white, very thick]
  (1*\X,-2*\Y) node{$n_3$} +(-5pt,-5pt) rectangle +(5pt,5pt);

  \draw[fill=white](\X,2*\Y) node{$n_6$} +(-5pt,-5pt) rectangle +(5pt,5pt);

  \draw[fill=white]( 0*\X,2*\Y)
  node{$n_7$} +(-5pt,-5pt) rectangle +(5pt,5pt);
  \draw[fill=white](-5+-\Y,2*\Y)node{$n_8$} +(-5pt,-5pt) rectangle +(5pt,5pt);
  \draw[fill=white](-10+-2*\Y,2*\Y) node{$n_9$} +(-5pt,-5pt) rectangle +(5pt,5pt);
  
  \draw[fill=white](2*\X,2*\Y)node{$n_{14}$} +(-5pt,-5pt) rectangle +(5pt,5pt);
  \draw[fill=white](2*\X,1*\Y)node{$n_{13}$} +(-5pt,-5pt) rectangle +(5pt,5pt);
  \draw[fill=white](2*\X,0*\Y)node{$n_{12}$} +(-5pt,-5pt) rectangle +(5pt,5pt);
  \draw[fill=white](2*\X,-1*\Y)node{$n_{11}$}+(-5pt,-5pt) rectangle +(5pt,5pt);
  \draw[fill=white](2*\X,-2*\Y)node{$n_{10}$}+(-5pt,-5pt) rectangle +(5pt,5pt);

\end{tikzpicture}}
  \caption[Gestion des départs et défaillances dans
  \SPRAY]{\label{net:fig:leavingexample}Exemple de gestion de départs et
    défaillances dans \SPRAY.}
\end{figure*}

La figure~\ref{net:fig:leavingexample} montre un exemple de réaction lors du
départ d'un nœud $n_1$. Ce dernier emporte avec lui $7$ connexions rendues
inutilisables. Les nœuds $n_3$, $n_4$, et $n_5$ ont toujours une référence
obsolète vers $n_1$ dans leur vue partielle. Le nœud $n_5$ a
$1-{1\div{|P_5|}}={2\div{3}}$ chances de remplacer la connexion. Dans cet
exemple, il double sa référence à $n_{13}$. De la même manière, $n_3$ et $n_4$
ne parviennent pas à communiquer avec $n_1$ et agissent en conséquence. Seul
$n_3$ crée un doublon remplaçant. Au total, $5$ connexions ont été supprimées.

Nous observons que la préservation de la connexité n'est pas garantie --
seulement avec la forte probabilité induite par les graphes aléatoires. En
effet, si le nœud $n_1$ est le seul pont reliant deux groupes de nœuds, l'ajout
d'arcs est insuffisant.

L'algorithme~\ref{net:algo:leaving} montre aussi que \SPRAY distingue les nœuds
injoignables des arcs inutilisables. En effet, la fonction \textsc{onArcDown}
gère les connexions dont la création a échoué. Ces arcs sont systématiquement
remplacés par un doublon. De ce fait, le nombre d'arc reste bien constant. La
distinction \textsc{onPeerDown} avec \textsc{onArcDown} est necessaire car la
première doit supprimer une petite quantité d'arcs. Sans cette suppression, le
nombre d'arcs augmenterait de manière incontrollée au gré des entrées et sorties
de nœuds.


%%% Local Variables:
%%% mode: latex
%%% TeX-master: "../../paper"
%%% End:
