
\section{Conclusion}
\label{net:sec:conclusion}

\begin{algorithm}[h]
  
\small
\algrenewcommand{\algorithmiccomment}[1]{\hskip2em$\rhd$ #1}

\newcommand{\comment}[1]{\hfill $\rhd$ #1}

\newcommand{\LINEFOR}[2]{%
  \algorithmicfor\ {#1}\ \algorithmicdo\ {#2} %
  }

\newcommand{\LINEIFTHEN}[2]{%
  \algorithmicif\ {#1}\ \algorithmicthen\ {#2} %
  }

\newcommand{\INDSTATE}[1][1]{\State\hspace{\algorithmicindent}}

\begin{algorithmic}[1]
  \Function{broadcast}{$m$} \comment{$m$: \emph{message to send}}
  \For{($n \in P$)}
  \State \textsc{sendTo}($n$, 'broadcast', $m$);
  \EndFor
  \EndFunction

  \Statex

  \Function{onBroadcast}{$m$} \comment{$m$: \emph{received message}}
  \If {($\neg\textsc{alreadyReceived}(m)$)}
  \State \textsc{broadcast}($m$);
  \EndIf
  \EndFunction
\end{algorithmic}

  \caption{\label{net:algo:broadcast} Dissémination de messages.}
\end{algorithm}

Comme le montre l'algorithme~\ref{net:algo:broadcast}, la propagation d'un
message à tout le réseau se base essentiellement sur les vues partielles
maintenues par chaque nœud. La fonction \textsc{alreadyReceived} s'assure de la
terminaison de la dissémination, i.e., un message ne voyage pas indéfiniement
dans le réseau. Avec au minimum une transmission par nœud, le coût en
communication à chaque nœud se voit attribuer un facteur multiplicatif de
l'ordre de la taille de sa vue partielle. Grâce à \SPRAY, cette dernière évolue
automatiquement vers le logarithme de la taille globale du réseau. Le traffic
généré hérite de cet ajustement automatique.

\TODO{Moar.}
%% Ainsi, en total transparence pour le développeur d'application, 
