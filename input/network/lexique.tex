
\section{Lexique et notations}
\label{net:sec:lexique}

Le vocabulaire que nous employons dans le reste de ce chapitre est imprunté aux
réseaux et aux graphes indifférement. Afin d'éviter toute confusion, nous
décrivons nos abus de langage et leur sens associé.

\begin{itemize}
\item [\textbf{Nœud},] ou \emph{pair}, ou \emph{machine}. L'ensemble des nœuds
  d'un graphe est noté $\mathcal{N}$ et $n_i$ est l'identité du nœud $i$
  appartenant à $\mathcal{N}$.
\item [\textbf{Arc},] ou \emph{canal}, ou \emph{adresse logique}, ou
  \emph{connexion}, ou \emph{descripteur}. Si un arc est établi entre deux
  nœuds, alors ils peuvent communiquer directement. Si une serie d'arcs existe
  entre deux nœuds, alors ils peuvent communiquer par transitivité.
\item [\textbf{Vue partielle},] ou \emph{table de voisinage}. L'ensemble des
  arcs de la table du nœud $n_i$ est noté $P_i$. Il est possible que des arcs
  obsolètes existent dans cette table, i.e., le nœud $n_i$ pense pouvoir
  communiquer avec son vis-à-vis via un arc mais ce dernier n'existe plus.
\item [\textbf{Graphe aléatoire}] correspond au graphe aléatoire du modèle
  Erdös-Renyi~\cite{erdos1959random}. Dans ce modèle, chaque nœud possède un
  arc vers un autre nœud selon une certaine probabilité. Le graphe à de fortes
  chances d'être connexe -- tous les nœuds peuvent communiquer entre eux, soit
  directement, soit par transitivité -- si le nombre d'arcs suit une progression
  $|\mathcal{N}|\log |\mathcal{N}|$.
\item [\textbf{\TODO{Autre}}]
\end{itemize}

%%% Local Variables:
%%% mode: latex
%%% TeX-master: "../../paper.tex"
%%% End:
