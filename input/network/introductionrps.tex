
\chapter{Échantillonnage aléatoire de pairs}
\label{net:chap:rps}
\minitoc

\lettrine{L}e transfert de données d'une machine à une autre est un besoin
ancien. Il se developpe rapidement sous l'impulsion du projet Arpanet qui
deviendra plus tard la base de l'Internet. Initialement créé dans le but de
connecter une machine à une autre, l'échelle évolue afin d'impliquer plusieurs
machines communicant entre elles. Il s'agit alors d'un réseau de communication.

Les machines d'un réseau sont adressables, i.e., grâce à une métadonnée, il est
possible de retrouver la machine ciblé afin d'établir un canal de
communication. Par exemple, une adresse IP (\emph{Internet Protocol}) permet de
retrouver une machine de manière transparente : Les infrastructures
sous-jacentes sur l'itinéraire entre les machines, tels que les routeurs, ne
sont pas connues. Un réseau construit sur de telles adresses logiques est appelé
réseau superposé (\emph{network overlay}). Un réseau superposé peut être
construit sur un autre réseau superposé.

\begin{figure}
  \centering 
  \begin{tikzpicture}[scale=1.2]
\newcommand\X{55pt}
\newcommand\Y{55pt}


\draw[fill=white] (0*\X,0*\Y) +(5pt, 5pt) rectangle +(-5pt, -5pt);
\draw[fill=white] (1*\X,0*\Y) +(5pt, 5pt) rectangle +(-5pt, -5pt);
\draw[<->] (5+0*\X, 0*\Y) -- (-5+1*\X, 0*\Y);
\draw (0.5*\X,-5-1*\Y) node[anchor=north, align=center]{2 peers; 2 arcs};

\begin{scope}[shift={(1.5*\X,0*\Y)}]
\draw[fill=white] (0*\X,0*\Y) +(5pt, 5pt) rectangle +(-5pt, -5pt);
\draw[fill=white] (1*\X,0*\Y) +(5pt, 5pt) rectangle +(-5pt, -5pt);
\draw[fill=white] (1*\X,-1*\Y) +(5pt, 5pt) rectangle +(-5pt, -5pt);
\draw[<->] (5+0*\X, 0*\Y) -- (-5+1*\X, 0*\Y);
\draw[<->] ( 1*\X, -5+0*\Y) -- ( 1*\X, 5-1*\Y);
\draw[<->] (5+0*\X, -5+0*\Y) -- (-5+1*\X, 5-1*\Y);
\draw (0.5*\X,-5-1*\Y) node[anchor=north, align=center]{3 peers; 6 arcs};
\end{scope}

\begin{scope}[shift={(3*\X,0*\Y)}]
\draw[fill=white] (0*\X,0*\Y) +(5pt, 5pt) rectangle +(-5pt, -5pt);
\draw[fill=white] (1*\X,0*\Y) +(5pt, 5pt) rectangle +(-5pt, -5pt);
\draw[fill=white] (1*\X,-1*\Y) +(5pt, 5pt) rectangle +(-5pt, -5pt);
\draw[fill=white] (0*\X,-1*\Y) +(5pt, 5pt) rectangle +(-5pt, -5pt);
\draw[<->] (5+0*\X, 0*\Y) -- (-5+1*\X, 0*\Y);
\draw[<->] ( 1*\X, -5+0*\Y) -- ( 1*\X, 5-1*\Y);
\draw[<->] (5+0*\X, -5+0*\Y) -- (-5+1*\X, 5-1*\Y);
\draw[<->] (5+0*\X, 5-1*\Y) -- (-5+1*\X, -5+0*\Y);
\draw[<->] (5+0*\X,-1*\Y) -- (-5+1*\X, -1*\Y);
\draw[<->] ( 0*\X,5-1*\Y) -- ( 0*\X, -5+0*\Y);
\draw (0.5*\X,-5-1*\Y) node[anchor=north, align=center]{4 peers; 12 arcs};
\end{scope}

\begin{scope}[shift={(4.5*\X,0*\Y)}]
\draw[fill=white] (0.33*\X,0*\Y) +(5pt, 5pt) rectangle +(-5pt, -5pt);
\draw[fill=white] (0.66*\X,0*\Y) +(5pt, 5pt) rectangle +(-5pt, -5pt);
\draw[fill=white] (0*\X,-0.5*\Y) +(5pt, 5pt) rectangle +(-5pt, -5pt);
\draw[fill=white] (1*\X,-0.5*\Y) +(5pt, 5pt) rectangle +(-5pt, -5pt);
\draw[fill=white] (0.5*\X,-1*\Y) +(5pt, 5pt) rectangle +(-5pt, -5pt);

\draw[<->] (-5+0.33*\X, 0*\Y) -- (0*\X,5-0.5*\Y);
\draw[<->] ( 5+0.66*\X, 0*\Y) -- (1*\X,5-0.5*\Y);
\draw[<->] (0.33*\X, -5+0*\Y) -- (0.5*\X, 5-1*\Y);
\draw[<->] (0.66*\X, -5+0*\Y) -- (0.5*\X, 5-1*\Y);
\draw[<->] (5+0*\X, 5-0.5*\Y) -- (-5+0.66*\X, -5-0*\Y);
\draw[<->] (-5+1*\X, 5-0.5*\Y) -- (5+0.33*\X, -5-0*\Y);
\draw[<->] (0*\X, -5-0.5*\Y) -- (-5+0.5*\X, -1*\Y);
\draw[<->] (1*\X, -5-0.5*\Y) -- (5+0.5*\X, -1*\Y);
\draw[<->] (5+0.33*\X, 0*\Y) -- (-5+0.66*\X, 0*\Y);
\draw[<->] (5+0*\X, -0.5*\Y) -- (-5+1*\X, -0.5*\Y);
\draw (0.5*\X,-5-1*\Y) node[anchor=north, align=center]{5 peers; 20 arcs};
\end{scope}


\end{tikzpicture}
  \caption{\label{net:fig:completegraph}Graphes complets}
\end{figure}

Toutes sortes de réseaux superposés existent. Par exemple, si chaque machine
possède un canal de communication avec chacunes des autres machines, alors la
topologie correspond à celle d'un graphe complet
(cf. figure~\ref{net:fig:completegraph}). La progression quadratique du nombre
d'arcs en fonction du nombre de noeuds empêche un tel système d'atteindre de
grandes envergures. Afin de résoudre ce problème, les machines ne possèdent
qu'une vue partielle du réseau. Cette vue partielle constitue le voisinage
direct d'un noeud. Le défi consiste alors à peupler ces vues avec les liens
logiques selon les critères souhaités.

Dans ce manuscrit, nous nous intéressons aux protocoles d'échantillonnage
aléatoire de pairs dont l'objectif est de fournir à chaque machine une vue
partielle peuplée d'adresses logiques réparties selon une loi uniforme. La
topologie en résultant est proche de celle des graphes aléatoires. Ces graphes
possèdent d'interessantes propriétés tels que la tolérence aux pannes --
lorsqu'un noeud disparait soudainement le graphe reste connecté avec une forte
probabilité -- un faible diamêtre, etc.

Deux familles d'approches existent quant à la taille des vues partielles :
\begin{itemize}
\item [\textbf{Taille fixe :}] fixée lors de la configuration, la taille des
  vues partielles ne change pas au cours du temps, même si la taille du réseau
  est susceptible de fluctuer (cf. §\ref{net:sec:fixed}).
\item [\textbf{Taille variable :}] les machines lorsqu'elles rejoignent le
  réseau contribue à hauteur du logarithme de la taille du réseau
  (cf. §\ref{net:sec:variable}).
\end{itemize}

%%% Local Variables:
%%% mode: latex
%%% TeX-master: "../../paper"
%%% End:
