\section{Quitter le réseau}
\label{net:sec:leaving}

Les nœuds utilisant \SPRAY peuvent quitter le réseau sans en informer qui que ce
soit. Ainsi, les départs et défaillances sont gérés de façon identique. Sans
réaction appropriée, la suppression d'arcs entrainerait l'effondrement du
réseau.

Lorsqu'un nœud rejoint un réseau, il y injecte $1+\ln(|\mathcal{N}|)$
arcs. Néanmoins, après plusieurs échanges de voisinage, sa vue partielle se
trouve remplie d'autres références. Ainsi, lorsqu'il quitte le réseau, il
entraine la suppression de $\ln(|\mathcal{N}|)$ arcs provenant de sa vue
partielle, et $\ln(|\mathcal{N}|)$ arcs provenant des nœuds l'ayant dans leur
vue partielle. Par conséquent, $2\ln(\mathcal{N}|)$ arcs sont supprimés au lieu
des $1+\ln(|\mathcal{N}|)$. Afin d'y remedier, chaque nœud qui détecte une
défaillance ou un départ peut rétablir une connexion avec l'un de ces voisins en
introduisant un doublon -- doublon qui disparaitra rapidement après quelques
mélanges. La probabilité de réétablir une connexion est
$1-1\div{|\mathcal{P}|}$. Puisque ${|\mathcal{P}|}\approx \ln(|\mathcal{N}|)$
nœuds avaient le nœud défaillant/parti dans leur vue partielle, il est probable
qu'ils recréent tous la connexion perdue, à l'exception d'un nœud. De ce fait,
lorsqu'un nœud quitte le réseau, il entraine la suppression d'un nombre d'arcs
correspondant approximativement au nombre d'arcs ajouté lors de la dernière
entrée.

\begin{algorithm}[h]
  \input{./input/network/leavingalgo.tex}
  \caption{\label{net:algo:leaving}The crash/departure handler of \SPRAY.}
\end{algorithm}

L'algorithme~\ref{net:algo:leaving} montre la manière selon laquelle \SPRAY gère
les départs et défaillances. La fonction \textsc{onPeerDown} montre la réaction
de \SPRAY lorsque un nœud $q$ est detecté comme parti ou défaillant. Dans un
premier temps, la fonction compte les occurences dudit nœud dans la vue
partielle et les supprime. Dans un second temps, une boucle ajoute de manière
probabiliste des doublons de références à des nœuds déjà connus. La probabilité
dépend de la taille de la vue partielle avant suppression.

\begin{figure*}
  \centering
  \subfloat[Un nœud \SPRAY quitte le réseau.]
  [Le nœud $n_1$ quitte le réseau.]
  {\input{./input/network/leavingexampleA.tex}}
  \hspace{40pt}
  \subfloat[Certains nœuds détectent le départ.]
  [Les nœuds $n_{3-5}$ s'aperçoivent qu'ils ne peuvent plus communiquer avec
  $n_1$]
  {\input{./input/network/leavingexampleB.tex}}
  \hspace{40pt}
  \subfloat[Certains nœuds dupliquent des arcs.]
  [Les nœuds $n_3$ et $n_5$ établissent des arcs doublons dans leur
  vue partielle.]
  {\input{./input/network/leavingexampleC.tex}}
  \caption{\label{net:fig:leavingexample}Exemple de gestion de
    départs/défaillances dans \SPRAY.}
\end{figure*}

La figure~\ref{net:fig:leavingexample} montre un exemple de réaction lors du
départ d'un nœud $n_1$. Ce dernier emporte avec lui $7$ connexions rendues
inutilisables. Les nœuds $n_3$, $n_4$, et $n_5$ ont toujours une référence
obsolète vers $n_1$ dans leur vue partielle. Le nœud $n_5$ a
$1-{1\div{|P_5|}}={2\div{3}}$ chances de remplacer la connexion. Dans cet
exemple, il double sa référence à $n_{13}$. De la même manière, $n_3$ et $n_4$
ne parviennent pas à communiquer avec $n_1$ et agissent en conséquence. Seul
$n_3$ crée un doublon remplaçant. Au total, $5$ connexions ont été supprimées.

Nous observons que la préservation de la connexité n'est pas garantie --
seulement avec la forte probabilité induite par les graphes aléatoires. En
effet, si le nœud $n_1$ est le seul pont reliant deux groupes de nœuds, l'ajout
d'arcs est insuffisant.

L'algorithme~\ref{net:algo:leaving} montre aussi que \SPRAY distingue les nœuds
injoignables des arcs inutilisables. En effet, la fonction \textsc{onArcDown}
gère les connexions dont la création a échoué. Ces arcs sont systématiquement
remplacés par un doublon. De ce fait, le nombre d'arc reste bien constant. La
distinction \textsc{onPeerDown} avec \textsc{onArcDown} est necessaire car la
première doit supprimer une petite quantité d'arcs. Sans cette suppression, le
nombre d'arcs augmenterait de manière incontrollée au gré des entrées et sorties
de nœuds.

%%% Local Variables:
%%% mode: latex
%%% TeX-master: "../../paper"
%%% End:
