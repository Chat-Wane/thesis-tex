
\section{Conclusion}
\label{net:sec:conclusion}

Dans ce chapitre, nous avons présenté \SPRAY, un protocole d'échantillonnage
aléatoire de pairs s'auto-ajustant par rapport aux variations de la taille du
réseau. Chaque nœud d'un réseau de taille $|\mathcal{N}|$ se voit attribuer une
vue partielle du réseau de taille $\ln |\mathcal{N}|$ avec laquelle il peut
communiquer. La topologie résultant des mélanges de voisinnages possède des
propriétés proches des graphes aléatoires. Entre autres, le réseau est tolérent
aux pannes et l'information se propage très rapidement. Bien que les connexions
ne s'établissent que de voisin à voisin, le réseau converge vers une topologie
dont les propriétés sont stables en très peu de temps.

\begin{itemize}
\item [\textbf{QR A.}] \textbf{Comment disséminer les messages de telle façon
    que le trafic s'adapte automatiquement aux fluctuations des réseaux?}
\end{itemize}


% \begin{algorithm}[h]
%   \input{input/network/broadcastalgo.tex}
%   \caption{\label{net:algo:broadcast} Dissémination de messages.}
% \end{algorithm}

Comme le montre l'algorithme~\ref{net:algo:broadcast}, la propagation d'un
message à tout le réseau se base essentiellement sur les vues partielles
maintenues par chaque nœud. La fonction \textsc{alreadyReceived} s'assure de la
terminaison de la dissémination, i.e., un message ne voyage pas indéfiniement
dans le réseau. Avec au minimum une transmission par nœud, le coût en
communication à chaque nœud se voit attribuer un facteur multiplicatif de
l'ordre de la taille de sa vue partielle. Grâce à \SPRAY, cette dernière évolue
automatiquement vers le logarithme de la taille globale du réseau. Le trafic
généré hérite de cet ajustement automatique.

Un développeur peut utiliser \SPRAY afin que le trafic généré suive les
variations du réseau, et ce, sans configuration préalable de sa part. Le
chapitre~\ref{editor:chap:crate} montre un exemple d'utilisation de \SPRAY dans
le contexte de l'édition collaborative temps réel. Dans ce cadre, les sessions
d'édition peuvent être de taille très différentes en fonction de la popularité
du document et au cours du temps. 

