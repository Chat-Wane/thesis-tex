
\section{Conclusion}
\label{net:sec:conclusion}

Dans ce chapitre, nous avons présenté \SPRAY, un protocole d'échantillonnage
aléatoire de pairs s'ajustant automatiquement aux fluctuations de la taille du
réseau. Chaque nœud d'un réseau de taille $|\mathcal{N}|$ se voit attribuer une
vue partielle du réseau de taille $\ln |\mathcal{N}|$ avec laquelle il peut
communiquer. Périodiquement, les nœuds échangent une partie de leur vue
partielle avec leur voisin le plus âgé. Bien que l'identité des nœuds ne se
propagent que de voisin à voisin, le réseau converge vers une topologie dont les
propriétés sont stables en très peu de temps. La topologie résultant de ces
mélanges de voisinages possède des propriétés proches des graphes
aléatoires. Par exemple, le réseau est tolérant aux pannes et les messages se
propagent efficacement.

Un développeur peut utiliser \SPRAY afin que le trafic généré suive les
variations du réseau, et ce, sans configuration préalable de sa part. Le
chapitre~\ref{editor:chap:crate} montre un exemple d'utilisation de \SPRAY dans
le contexte de l'édition collaborative temps réel dans les navigateurs Web.  Ce
contexte est propice à son utilisation car l'établissement de connexion d'un
navigateur à l'autre s'avère coûteux. Par conséquent, conserver un nombre
logarithmique de connexions s'établissant précautionneusement de proche en
proche constitue un avantage. De plus, le Web favorisant les échanges et la
propagation d'idées, un éditeur Web doit être capable de faire face aux pics
soudains de popularité.

