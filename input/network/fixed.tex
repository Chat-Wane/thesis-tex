

\section{Taille fixe}

Les protocoles d'échantillonnage aléatoire de pairs peuplant des vues dont la
taille est constante possèdent un socle commun à savoir l'échanges périodiques
de voisinages.

\begin{itemize}
\item [\textbf{\CYCLON :}] est un protocole où chaque nœud possède une vue
  partielle dont chaque adresse logique est associée à un âge. Régulièrement,
  les nœuds initient un mélange (\emph{shuffling}) avec leur voisin le plus
  âgé. Si la communication est possible, le processus est amorcé et l'âge de
  tous les voisins est incrémenté, sinon l'adresse est supprimée. Cela permet de
  supprimer les nœuds considérés partis.

  Lors de ce mélange, le nœud envoit un nombre prédéfinit de ses adresses
  logiques au nœud choisit, en prenant soin de remplacer l'adresse de son
  homologue par la sienne propre. Les adresses sont choisies aléatoirement. À la
  reception, son vis-à-vis choisit un nombre d'adresses équivalent et les lui
  envoie. Les deux nœuds intègrent alors l'ensemble qu'ils ont réçu. Lorsque
  l'ensemble obtenu est trop grand pour la taille de vue partielle prédéfinie,
  des adresses sont supprimées en conservant de préférence les adresses réçues
  et en supprimant les duplicats.
  
  \begin{figure}
    \centering
    \subfloat[Choix aléatoire des pairs dans \CYCLON.]
    [\label{net:fig:cyclonexampleA} L'initiateur du mélange choisit d'envoyer $n_5$ et lui-même à son plus vieux voisin $n_2$. En retour, il lui propose $n_6$ et $n_7$.]
    {\begin{tikzpicture}[scale=1.2]
\newcommand\X{45pt}
\newcommand\Y{20pt}


\draw[fill=white, draw=darkblue]( 0*\X, -2*\Y)
node{\DARKBLUE{$n_1$}} +(5pt, 5pt) rectangle +(-5pt,-5pt);
\draw[fill=white]( 1*\X, -2*\Y) node{$n_2$} +(5pt, 5pt) rectangle +(-5pt,-5pt);


\draw[fill=white](-1*\X, 0*\Y) node{$n_3$} +(5pt, 5pt) rectangle +(-5pt,-5pt);
\draw[fill=white](-1*\X, -1*\Y) node{$n_4$} +(5pt, 5pt) rectangle +(-5pt,-5pt);
\draw[fill=white, draw=darkblue](-1*\X, -3*\Y)
node{\DARKBLUE{$n_5$}} +(5pt, 5pt) rectangle +(-5pt,-5pt);

\draw[fill=white, draw=darkblue]( 2*\X, -0*\Y)
node{\DARKBLUE{$n_6$}} +(5pt, 5pt) rectangle +(-5pt,-5pt);
\draw[fill=white, draw=darkblue]( 2*\X, -1*\Y)
 node{\DARKBLUE{$n_7$}} +(5pt, 5pt) rectangle +(-5pt,-5pt);
\draw[fill=white]( 2*\X, -3*\Y) node{$n_8$} +(5pt, 5pt) rectangle +(-5pt,-5pt);
\draw[fill=white]( 2*\X, -4*\Y) node{$n_9$} +(5pt, 5pt) rectangle +(-5pt,-5pt);

\draw[->] (-5+0*\X, -2*\Y) -- (5-1*\X, 0*\Y);
\draw[->] (-5+0*\X, -2*\Y) -- (5-1*\X, -1*\Y);
\draw[->] (-5+0*\X, -2*\Y) -- (5-1*\X, -3*\Y);

\small
\draw[->] (5+0*\X, -2*\Y) -- node[anchor=south]{oldest} (-5+1*\X, -2*\Y);

\draw[->] (5+1*\X, -2*\Y) -- (-5+2*\X, 0*\Y);
\draw[->] (5+1*\X, -2*\Y) -- (-5+2*\X, -1*\Y);
\draw[->] (5+1*\X, -2*\Y) -- (-5+2*\X, -3*\Y);
\draw[->] (5+1*\X, -2*\Y) -- (-5+2*\X, -4*\Y);
\end{tikzpicture}}
    \hspace{35pt}
    \subfloat[Établissement des connexions après le mélange dans \CYCLON.]
    [\label{net:fig:cyclonexampleB} Les arcs sont échangés de part et d'autre.]
    {\begin{tikzpicture}[scale=1.2]
\newcommand\X{45pt}
\newcommand\Y{20pt}


\draw[fill=white]( 0*\X, -2*\Y) node{$n_1$} +(5pt, 5pt) rectangle +(-5pt,-5pt);
\draw[fill=white]( 1*\X, -2*\Y) node{$n_2$} +(5pt, 5pt) rectangle +(-5pt,-5pt);


\draw[fill=white](-1*\X, 0*\Y) node{$n_3$} +(5pt, 5pt) rectangle +(-5pt,-5pt);
\draw[fill=white](-1*\X, -1*\Y) node{$n_4$} +(5pt, 5pt) rectangle +(-5pt,-5pt);
\draw[fill=white](-1*\X, -3*\Y) node{$n_5$} +(5pt, 5pt) rectangle +(-5pt,-5pt);

\draw[fill=white]( 2*\X, -0*\Y) node{$n_6$} +(5pt, 5pt) rectangle +(-5pt,-5pt);
\draw[fill=white]( 2*\X, -1*\Y) node{$n_7$} +(5pt, 5pt) rectangle +(-5pt,-5pt);
\draw[fill=white]( 2*\X, -3*\Y) node{$n_8$} +(5pt, 5pt) rectangle +(-5pt,-5pt);
\draw[fill=white]( 2*\X, -4*\Y) node{$n_9$} +(5pt, 5pt) rectangle +(-5pt,-5pt);

\draw[->] (-5+0*\X, -2*\Y) -- (5-1*\X, 0*\Y);
\draw[->] (-5+0*\X, -2*\Y) -- (5-1*\X, -1*\Y);
% \draw[->] (-5+0*\X, -2*\Y) -- (5-1*\X, -3*\Y);
\draw[->, color=darkblue] (5+0*\X, 5-2*\Y) -- (-5+2*\X, 0*\Y);
\draw[->, color=darkblue] (5+0*\X, 5-2*\Y) -- (-5+2*\X, -1*\Y);


\draw[<-, darkblue] (5+0*\X, -2*\Y) -- (-5+1*\X, -2*\Y);

\draw[->, color=darkblue] (-5+1*\X, -5-2*\Y) -- (5-1*\X, -3*\Y);
% \draw[->] (5+1*\X, -2*\Y) -- (-5+2*\X, 0*\Y);
% \draw[->] (5+1*\X, -2*\Y) -- (-5+2*\X, -1*\Y);
\draw[->] (5+1*\X, -2*\Y) -- (-5+2*\X, -3*\Y);
\draw[->] (5+1*\X, -2*\Y) -- (-5+2*\X, -4*\Y);
\end{tikzpicture}}
    \caption{\label{net:fig:cyclonexample} Exemple de mélange dans \CYCLON. Pour
      améliorer la lisibilité, seuls les vues partielles du $n_1$ et $n_2$ sont
      explicitées.}
  \end{figure}
  
  La figure~\ref{net:fig:cyclonexample} décrit un exemple de mélange initié par
  le nœud $n_1$. Dans cet exemple, les vues partielles sont configurées pour
  accueillir 4 arcs et en échanger 2 pendant les mélanges. La
  figure~\ref{net:fig:cyclonexampleA} montre que $n_1$ choisit son plus vieux
  voisin afin d'initier l'échange, à savoir $n_2$. Il incorpore dans l'échange
  sa propre identité ainsi que celle d'un arc connu aléatoirement. Le nœud $n_2$
  réçoit la demande de mélange et choisit 2 nœuds aléatoirement, ici $n_6$ et
  $n_7$ qu'il envoit à son tour. La figure~\ref{net:fig:cyclonexampleB} montre
  que les nœuds participants au mélange ont supprimé les arcs qu'ils ont envoyé
  et ajouté ceux nouvellement reçu. En particulier, l'inversion de l'arc ayant
  permi le mélange garantit que le graphe reste connexe.

\item [\textbf{Newscast :}]
\item [\textbf{Lpbcast :}]
\item [\textbf{HyParView :}]
\end{itemize}

Afin d'adapter la taille de la vue partielle à la taille du réseau, il serait
tentant d'utiliser un estimateur.
