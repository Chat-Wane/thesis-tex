

\section{Taille fixe}

Les protocoles d'échantillonnage aléatoire de pairs peuplant des vues dont la
taille est constante possèdent un socle commun à savoir l'échanges périodiques
de voisinages.

\begin{itemize}
\item [\textbf{\CYCLON :}] est un protocole où chaque noeud possède une vue
  partielle dont chaque adresse logique est associée à un âge. Régulièrement,
  les noeuds initient un mélange (\emph{shuffling}) avec leur voisin le plus
  âgé. Si la communication est possible, le processus est amorcé et l'âge de
  tous les voisins est incrémenté, sinon l'adresse est supprimée. Cela permet de
  supprimer les noeuds considérés partis.

  Lors de ce mélange, le noeud envoit un nombre prédéfinit de ses adresses
  logiques au noeud choisit, en prenant soin de remplacer l'adresse de son
  homologue par la sienne propre. Les adresses sont choisies aléatoirement. À la
  reception, son vis-à-vis choisit un nombre d'adresses équivalent et les lui
  envoie. Les deux noeuds intègrent alors l'ensemble qu'ils ont réçu. Lorsque
  l'ensemble obtenu est trop grand pour la taille de vue partielle prédéfinie,
  des adresses sont supprimées en conservant de préférence les adresses réçues.
\item [\textbf{Newscast :}]
\item [\textbf{Lpbcast :}]
\item [\textbf{HyParView :}]
\end{itemize}

Afin d'adapter la taille de la vue partielle à la taille du réseau, il serait
tentant d'utiliser un estimateur.
