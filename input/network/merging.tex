
\section{Fusion de réseaux}
\label{net:sec:merging}

La fusion de plusieurs réseaux n'est jamais abordée dans la littérature
concernant l'échantillonnage aléatoire de pairs, et pour cause : le processus
des approches à taille fixe s'en trouverai inchangé. (\TODO{Think about two
  networks with differents partial view size merging}) Dans le cadre des
approches à taille variable, le sujet devient intéressant car la fusion des
réseaux doit résulter en un réseau unique, lui-même possédant les mêmes
propriétés que ceux dont il est issue. Entre autre, le nombre d'arcs doit suivre
le seuil $\Theta (|\mathcal{N}|.\ln |\mathcal{N}|)$ arcs.

Lors de ce processus, nous supposons qu'au moins un des nœuds appartenant à l'un
réseau contacte l'autre réseau afin d'initier la fusion. Grâce à la connexion
qui en résulte, les réseaux sont à même de fusionner.

La première solution qui vient à l'esprit est la suivante : chacun des nœuds du
premier réseau utilise le contact afin de rejoindre le second
réseau. Malheureusement, cette solution est extrêmement lente -- puisque la
majorité des nœuds ignorent encore qui est le contact -- et susceptible
d'échouer -- le contact est un point unique de défaillance.

Un seconde solution consiste simplement à laisser le méchanisme de mélange faire
son office. Petit à petit, d'autres ponts entre les réseaux vont se former
jusqu'à ce que les deux réseaux soient indifférenciés. Malheureusement, les arcs
ne suivent pas l'augmentation relative au réseau. Le problème est le suivant :
\TODO{Moar.}


