
\section{Fusion de réseaux}
\label{net:sec:merging}

La fusion de réseaux consiste à obtenir un réseau unique comme l'union des
membres de plusieurs réseaux. Le réseau obtenu doit hériter des propriétés de
ses parents.  Lors de ce processus, nous supposons qu'au moins un des nœuds
appartenant à l'un des réseau contacte l'autre réseau afin d'initier la
fusion. Grâce à la connexion qui en résulte, les réseaux sont à même de
communiquer, et donc de fusionner.

Les approches à taille fixe sont triviales à étendre : les mélanges périodiques
suffisent à garantir un réseau connexe. Si toutefois les vues partielles sont
configurées avec des tailles différentes, il suffit alors de prendre la taille
maximum des deux. Par exemple, un nœud avec une vue partielle de 5 voisins verra
sa vue augmenter à $7$ voisins après un échange avec le nœud dont la vue
partielle est peuplée de $7$ références.

\SPRAY est une approche dont les vues partielles évoluent automatiquement en
réaction aux entrées et sorties du réseau. En particulier, les vues partielles
suivent une progression logarithmiques comparée à la taille du réseau. La fusion
de réseaux \SPRAY doit résulter en un réseau \SPRAY garantissant de même.

La première solution qui vient à l'esprit est la suivante : chacun des nœuds du
premier réseau utilise le contact afin de rejoindre le second réseau, comme un
sablier dont les grains passent un tube étroit pour rejoindre l'autre bulbe sous
l'effet de la gravité. Malheureusement, cette solution est extrêmement lente --
puisque la majorité des nœuds ignorent encore qui est le contact -- et
susceptible d'échouer -- puisque le contact est un point unique de défaillance.

Un seconde solution consiste simplement, à l'instar des approches à taille fixe,
à laisser le méchanisme de mélange faire son office. Petit à petit, d'autres
ponts entre les réseaux vont se former jusqu'à ce que les deux réseaux soient
indifférenciés. Malheureusement, les arcs ne suivent pas l'augmentation relative
au réseau.

\begin{problem}
  Soit $\mathcal{N}_1,\, \mathcal{N}_2,\, \ldots ,\, \mathcal{N}_k$ des réseaux
  de taille arbitraire. On a :
\begin{equation}
  \sum\limits_{i \in \mathbb{N}_{<k}} |\mathcal{N}_i|\ln (|\mathcal{N}_i|) < (\sum\limits_{i \in \mathbb{N}_{<k}} |\mathcal{N}_i|)\ln{(\sum\limits_{i \in \mathbb{N}_{<k}} |\mathcal{N}_i|)}
\end{equation}
Comment adapter les nombres d'arcs effectif (à gauche) pour qu'il atteigne le
nombre d'arcs requis (à droite)?
\end{problem}

Pour répondre à ce problème, chacun des nœuds appartenant aux réseaux impliqués
dans la fusion doit être capable de
\begin{inparaenum}[(i)]
\item détecter lorsqu'un nouveau réseau fusionne avec celui dans lequel il se
  trouve (cf. §\ref{net:subsec:detection}),
\item détecter lorsqu'il a glané suffisamment d'informations pour procéder à la
  fusion (cf. §\ref{net:subsec:activation}),
\item ajuster sa vue partielle en conséquence (cf. §\ref{net:subsec:merging}).
\end{inparaenum}


\subsection{Détection de fusion}
\label{net:subsec:detection}

\subsection{Activation de fusion}
\label{net:subsec:activation}

\subsection{Ajustement de vue partielle}
\label{net:subsec:merging}

\subsection{Validation}
