
\section{État de l'art}
\label{net:sec:stateoftheart}
Deux familles d'approches existent quant à la taille des vues partielles :
\begin{inparaenum}[(i)]
\item La famille d'approches dont la taille est fixée lors de la
  configuration. Ainsi la taille des vues partielles ne change pas au cours du
  temps, même si la taille du réseau est susceptible de fluctuer
  (cf. §\ref{net:subsec:fixed}).
\item La famille dont les machines, lorsqu'elles rejoignent le réseau,
  contribuent à hauteur du logarithme de la taille du réseau
  (cf. §\ref{net:subsec:variable}).
\end{inparaenum}

\subsection{Taille fixe}
\label{net:subsec:fixed}

Les protocoles d'échantillonnage aléatoire de pairs peuplant des vues dont la
taille est constante possèdent un socle commun à savoir l'échanges périodiques
de voisinages.

\paragraph{\CYCLON~\cite{voulgaris2005cyclon} :} est un protocole où chaque nœud
possède une vue partielle dont chaque adresse logique est associée à un
âge. Régulièrement, les nœuds initient un mélange (\emph{shuffling}) avec leur
voisin le plus âgé. Si la communication est possible, le processus est amorcé et
l'âge de tous les voisins est incrémenté, sinon l'adresse est supprimée. Cela
permet de supprimer les nœuds considérés comme partis.

\noindent Lors de ce mélange, le nœud envoit un nombre prédéfinit de ses
adresses logiques au nœud choisit, en prenant soin de remplacer l'adresse de son
homologue par la sienne propre. Les adresses sont choisies aléatoirement. À la
reception, son vis-à-vis choisit un nombre d'adresses équivalent et les lui
envoie. Les deux nœuds intègrent alors l'ensemble qu'ils ont réçu. Lorsque
l'ensemble obtenu est trop grand pour la taille de vue partielle prédéfinie, des
adresses sont supprimées en conservant de préférence les adresses réçues et en
supprimant les duplicats.
  
\begin{figure}
  \centering
  \subfloat[Choix aléatoire des pairs dans \CYCLON.]
  [\label{net:fig:cyclonexampleA} L'initiateur du mélange choisit d'envoyer $n_5$ et lui-même à son plus vieux voisin $n_2$. En retour, il lui propose $n_6$ et $n_7$.]
  {\begin{tikzpicture}[scale=1.2]
\newcommand\X{45pt}
\newcommand\Y{20pt}


\draw[fill=white, draw=darkblue]( 0*\X, -2*\Y)
node{\DARKBLUE{$n_1$}} +(5pt, 5pt) rectangle +(-5pt,-5pt);
\draw[fill=white]( 1*\X, -2*\Y) node{$n_2$} +(5pt, 5pt) rectangle +(-5pt,-5pt);


\draw[fill=white](-1*\X, 0*\Y) node{$n_3$} +(5pt, 5pt) rectangle +(-5pt,-5pt);
\draw[fill=white](-1*\X, -1*\Y) node{$n_4$} +(5pt, 5pt) rectangle +(-5pt,-5pt);
\draw[fill=white, draw=darkblue](-1*\X, -3*\Y)
node{\DARKBLUE{$n_5$}} +(5pt, 5pt) rectangle +(-5pt,-5pt);

\draw[fill=white, draw=darkblue]( 2*\X, -0*\Y)
node{\DARKBLUE{$n_6$}} +(5pt, 5pt) rectangle +(-5pt,-5pt);
\draw[fill=white, draw=darkblue]( 2*\X, -1*\Y)
 node{\DARKBLUE{$n_7$}} +(5pt, 5pt) rectangle +(-5pt,-5pt);
\draw[fill=white]( 2*\X, -3*\Y) node{$n_8$} +(5pt, 5pt) rectangle +(-5pt,-5pt);
\draw[fill=white]( 2*\X, -4*\Y) node{$n_9$} +(5pt, 5pt) rectangle +(-5pt,-5pt);

\draw[->] (-5+0*\X, -2*\Y) -- (5-1*\X, 0*\Y);
\draw[->] (-5+0*\X, -2*\Y) -- (5-1*\X, -1*\Y);
\draw[->] (-5+0*\X, -2*\Y) -- (5-1*\X, -3*\Y);

\small
\draw[->] (5+0*\X, -2*\Y) -- node[anchor=south]{ancien} (-5+1*\X, -2*\Y);

\draw[->] (5+1*\X, -2*\Y) -- (-5+2*\X, 0*\Y);
\draw[->] (5+1*\X, -2*\Y) -- (-5+2*\X, -1*\Y);
\draw[->] (5+1*\X, -2*\Y) -- (-5+2*\X, -3*\Y);
\draw[->] (5+1*\X, -2*\Y) -- (-5+2*\X, -4*\Y);
\end{tikzpicture}}
  \hspace{35pt}
  \subfloat[Établissement des connexions après le mélange dans \CYCLON.]
  [\label{net:fig:cyclonexampleB} Les arcs sont échangés de part et d'autre.]
  {\begin{tikzpicture}[scale=1.2]
\newcommand\X{45pt}
\newcommand\Y{20pt}


\draw[fill=white]( 0*\X, -2*\Y) node{$n_1$} +(5pt, 5pt) rectangle +(-5pt,-5pt);
\draw[fill=white]( 1*\X, -2*\Y) node{$n_2$} +(5pt, 5pt) rectangle +(-5pt,-5pt);


\draw[fill=white](-1*\X, 0*\Y) node{$n_3$} +(5pt, 5pt) rectangle +(-5pt,-5pt);
\draw[fill=white](-1*\X, -1*\Y) node{$n_4$} +(5pt, 5pt) rectangle +(-5pt,-5pt);
\draw[fill=white](-1*\X, -3*\Y) node{$n_5$} +(5pt, 5pt) rectangle +(-5pt,-5pt);

\draw[fill=white]( 2*\X, -0*\Y) node{$n_6$} +(5pt, 5pt) rectangle +(-5pt,-5pt);
\draw[fill=white]( 2*\X, -1*\Y) node{$n_7$} +(5pt, 5pt) rectangle +(-5pt,-5pt);
\draw[fill=white]( 2*\X, -3*\Y) node{$n_8$} +(5pt, 5pt) rectangle +(-5pt,-5pt);
\draw[fill=white]( 2*\X, -4*\Y) node{$n_9$} +(5pt, 5pt) rectangle +(-5pt,-5pt);

\draw[->] (-5+0*\X, -2*\Y) -- (5-1*\X, 0*\Y);
\draw[->] (-5+0*\X, -2*\Y) -- (5-1*\X, -1*\Y);
% \draw[->] (-5+0*\X, -2*\Y) -- (5-1*\X, -3*\Y);
\draw[->, color=darkblue] (5+0*\X, 5-2*\Y) -- (-5+2*\X, 0*\Y);
\draw[->, color=darkblue] (5+0*\X, 5-2*\Y) -- (-5+2*\X, -1*\Y);


\draw[<-, darkblue] (5+0*\X, -2*\Y) -- (-5+1*\X, -2*\Y);

\draw[->, color=darkblue] (-5+1*\X, -5-2*\Y) -- (5-1*\X, -3*\Y);
% \draw[->] (5+1*\X, -2*\Y) -- (-5+2*\X, 0*\Y);
% \draw[->] (5+1*\X, -2*\Y) -- (-5+2*\X, -1*\Y);
\draw[->] (5+1*\X, -2*\Y) -- (-5+2*\X, -3*\Y);
\draw[->] (5+1*\X, -2*\Y) -- (-5+2*\X, -4*\Y);
\end{tikzpicture}}
  \caption{\label{net:fig:cyclonexample} Exemple de mélange dans \CYCLON. Pour
    améliorer la lisibilité, seuls les vues partielles du $n_1$ et $n_2$ sont
    explicitées.}
\end{figure}
  
\noindent La figure~\ref{net:fig:cyclonexample} décrit un exemple de mélange
initié par le nœud $n_1$. Dans cet exemple, les vues partielles sont configurées
pour accueillir 4 arcs et en échanger 2 pendant les mélanges. La
figure~\ref{net:fig:cyclonexampleA} montre que $n_1$ choisit son plus vieux
voisin afin d'initier l'échange, à savoir $n_2$. Il incorpore dans l'échange sa
propre identité ainsi que celle d'un arc connu aléatoirement. Le nœud $n_2$
réçoit la demande de mélange et choisit 2 nœuds aléatoirement, ici $n_6$ et
$n_7$ qu'il envoit à son tour. La figure~\ref{net:fig:cyclonexampleB} montre que
les nœuds participants au mélange ont supprimé les arcs qu'ils ont envoyé et
ajouté ceux nouvellement reçu. En particulier, l'inversion de l'arc ayant permi
le mélange garantit que le graphe reste connexe.

\paragraph{Newscast~\cite{tolgyeski2009adaptive} :} est un protocole où les vues
partielles associent à chaque adresse logique une estampille
(\emph{timestamp}). Périodiquement, les nœuds de Newscast effectuent un mélange
périodique. Pour cela, ils choisissent l'un de leurs voisins aléatoirement et
envoient la totalité de leur vue partielle à laquelle est ajoutée leur propre
adresse associée à une estampille à jour. Lors de la récéption, leur homologue
en fait de même. Tout deux fusionnent leur table de voisinnage de telle sorte
que seules les entrées les plus récentes selon l'estampille sont conservées.

\noindent Un nœud qui quitte le réseau n'effectue plus ce mélange périodique et
ainsi n'annonce plus sa présence au reste du réseau. Petit à petit, les vues
partielles le contenant se débarassent de l'adresse logique menant à lui car
obsolète comparativement à celles nouvellement réçues.

\noindent Les pertes de messages dûes à la non-fiabilité des communications
mettent en dangers la distribution des arcs. Par exemple, si un message sur deux
est perdu lorsqu'un certain nœuds tente de communiquer, il perd tout autant
d'occasions de s'annoncer au réseau. Par conséquent, il encourt le risque d'être
évincé de certaines vues partielles, voir de disparaître complètement. En
d'autres termes, le nœud est moins bien connecté que les autres à cause de ces
messages perdus. Newscast propose de compenser cela. Ainsi, si un nœud perd un
message sur deux, il emet deux fois plus. La distribution des arcs s'en trouve
ajustée.

\begin{figure}
  \centering
  \subfloat[Initiation du mélange dans Newscast.]
  [\label{net:fig:newscastexampleA} L'initiateur du mélange $n_2$ aléatoirement et lui envoit sa vue à laquelle il s'ajoute avec une estampille à jour. À la reception, $n_2$ en fait de même.]
  {\begin{tikzpicture}[scale=1.2]
\newcommand\X{45pt}
\newcommand\Y{20pt}


\draw[fill=white, draw=darkblue]( 0*\X, -2*\Y)
node{\DARKBLUE{$n_1$}} +(5pt, 5pt) rectangle +(-5pt,-5pt);
\draw[fill=white, draw=darkblue]( 1*\X, -2*\Y)
node{\DARKBLUE{$n_2$}} +(5pt, 5pt) rectangle +(-5pt,-5pt);


\draw[fill=white, draw=darkblue](-1*\X, 0*\Y)
node{\DARKBLUE{$n_3$}} +(5pt, 5pt) rectangle +(-5pt,-5pt);
\draw[fill=white, draw=darkblue](-1*\X, -1*\Y)
node{\DARKBLUE{$n_4$}} +(5pt, 5pt) rectangle +(-5pt,-5pt);
\draw[fill=white, draw=darkblue](-1*\X, -3*\Y)
node{\DARKBLUE{$n_5$}} +(5pt, 5pt) rectangle +(-5pt,-5pt);

\draw[fill=white, draw=darkblue]( 2*\X, -0*\Y)
node{\DARKBLUE{$n_6$}} +(5pt, 5pt) rectangle +(-5pt,-5pt);
\draw[fill=white, draw=darkblue]( 2*\X, -1*\Y)
 node{\DARKBLUE{$n_7$}} +(5pt, 5pt) rectangle +(-5pt,-5pt);
\draw[fill=white, draw=darkblue]( 2*\X, -3*\Y)
node{\DARKBLUE{$n_8$}} +(5pt, 5pt) rectangle +(-5pt,-5pt);
\draw[fill=white, draw=darkblue]( 2*\X, -4*\Y)
node{\DARKBLUE{$n_9$}} +(5pt, 5pt) rectangle +(-5pt,-5pt);

\scriptsize

\draw[->] (-5+0*\X, -2*\Y) -- node[anchor=west]{13:12} (5-1*\X, 0*\Y);
\draw[->] (-5+0*\X, -2*\Y) -- node[anchor=north east]{13:15} (5-1*\X, -1*\Y);
\draw[->] (-5+0*\X, -2*\Y) -- node[anchor=north west]{13:16} (5-1*\X, -3*\Y);

\small

\draw[->] (5+0*\X, -2*\Y) -- node[anchor=south]{aléatoire} (-5+1*\X, -2*\Y);

\scriptsize

\draw[->] (5+1*\X, -2*\Y) -- node[anchor=south east]{13:13} (-5+2*\X, 0*\Y);
\draw[->] (5+1*\X, -2*\Y) -- node[anchor=north west]{13:13} (-5+2*\X, -1*\Y);
\draw[->] (5+1*\X, -2*\Y) -- node[anchor=south west]{13:14} (-5+2*\X, -3*\Y);
\draw[->] (5+1*\X, -2*\Y) -- node[anchor=north east]{13:17} (-5+2*\X, -4*\Y);
\end{tikzpicture}}
  \hspace{35pt}
  \subfloat[Établissement des connexions après le mélange dans Newscast.]
  [\label{net:fig:newscastexampleB} Les arcs sont échangés de part et d'autre en conservant les estampilles les plus récentes.]
  {\begin{tikzpicture}[scale=1.2]
\newcommand\X{45pt}
\newcommand\Y{20pt}


\draw[fill=white]( 0*\X, -2*\Y) node{$n_1$} +(5pt, 5pt) rectangle +(-5pt,-5pt);
\draw[fill=white]( 1*\X, -2*\Y) node{$n_2$} +(5pt, 5pt) rectangle +(-5pt,-5pt);


\draw[fill=white](-1*\X, 0*\Y) node{$n_3$} +(5pt, 5pt) rectangle +(-5pt,-5pt);
\draw[fill=white](-1*\X, -1*\Y) node{$n_4$} +(5pt, 5pt) rectangle +(-5pt,-5pt);
\draw[fill=white](-1*\X, -3*\Y) node{$n_5$} +(5pt, 5pt) rectangle +(-5pt,-5pt);

\draw[fill=white]( 2*\X, -0*\Y) node{$n_6$} +(5pt, 5pt) rectangle +(-5pt,-5pt);
\draw[fill=white]( 2*\X, -1*\Y) node{$n_7$} +(5pt, 5pt) rectangle +(-5pt,-5pt);
\draw[fill=white]( 2*\X, -3*\Y) node{$n_8$} +(5pt, 5pt) rectangle +(-5pt,-5pt);
\draw[fill=white]( 2*\X, -4*\Y) node{$n_9$} +(5pt, 5pt) rectangle +(-5pt,-5pt);

\scriptsize

% \draw[->] (-5+0*\X, -2*\Y) -- (5-1*\X, 0*\Y);
\draw[->] (-5+0*\X, -2*\Y) -- (5-1*\X, -1*\Y)
node[anchor=west]{\ 13:15};
\draw[->] (-5+0*\X, -2*\Y) -- (5-1*\X, -3*\Y);



\draw[->, color=darkblue] (5+0*\X, -5 -2*\Y) -- (-5+2*\X, -4*\Y)
node[anchor=east]{13:17\ };

\draw[<->, color=darkblue] (5+0*\X, -2*\Y) --
node[anchor=south]{13:18}(-5+1*\X, -2*\Y);

\draw[->, color=darkblue] (-5+1*\X, -5-2*\Y) -- (5-1*\X, -3*\Y)
node[anchor=west]{\ 13:16};
% \draw[->] (5+1*\X, -2*\Y) -- (-5+2*\X, 0*\Y);
% \draw[->] (5+1*\X, -2*\Y) -- (-5+2*\X, -1*\Y);
% \draw[->] (5+1*\X, -2*\Y) -- (-5+2*\X, -3*\Y);
\draw[->] (5+1*\X, -2*\Y) -- (-5+2*\X, -4*\Y);
\end{tikzpicture}}
  \caption{\label{net:fig:newscastexample} Exemple de mélange dans
    Newscast. Ici, les estampilles sont simplement formatées hh:mm.}
\end{figure}

\noindent La figure~\ref{net:fig:newscastexample} montre un exemple de mélange
avec le protocole Newscast. Là encore, la taille des vues partielles est fixée à
4 entrées. Le pair $n_1$ initie le mélange avec un voisin aléatoire $n_2$. Il
envoit l'intégralité de sa vue à laquelle il s'ajoute avec l'estampille actuelle
13:18. Le nœud $n_2$ le reçoit et en fait de même. Les deux nœuds $n_1$ et $n_2$
conservent les quatres arcs les plus récents qu'ils possèdent, à savoir les arcs
vers leur homologue, $n_4$, $n_5$ et $n_9$. La
figure~\ref{net:fig:newscastexampleB} nous permet de remarquer que le réseau
semble collapser. En particulier, certains pairs ne sont plus référencés ni par
$n_1$, ni par $n_2$. Toutefois, il ne s'agit là que de la représentation locale
au mélange. Globalement, la topologie résultante est proche de celle des graphes
aléatoires.

\paragraph{Lpbcast~\cite{eugster2003lightweight} :} est le diminutif de
\emph{lightweight probabilistic broadcast}. C'est un protocole dont les vues
partielles ne comprennent que des adresses logiques. C'est lorsqu'un pair
diffuse un message au réseau que les vues sont mises à jour. Chaque message
comporte un ensemble borné d'identités de nœuds ayant quitté le réseau, et un
ensemble borné d'identités de nœuds échantillonnés au cours du cheminement du
message. Si aucun message n'a été réçu ni envoyé dans un certain intervalle de
temps, une diffusion est automatiquement générée afin d'activer le méchanisme de
mélange.

\noindent Lors de la réception de tels messages, un nœud peuple sa propre vue
partielle des nouveaux éléments présent dans le message avant de la réajuster à
la taille maximale autorisée en supprimant aléatoirement des arcs.

\paragraph{HyParView~\cite{leitao2007dependable} :} est un protocole possédant
deux vues partielles. La première est active et est utilisée lors de l'émission
de messages au réseau. La seconde est passive et sert à remplacer les arcs
obsolètes de la vue active, e.g., lors d'une défaillance ou d'un départ
volontaire.  La vue active est considérablement plus petite que la vue passive
et les arcs compris dans cette première sont vérifiés à chaque message émit.

\noindent Lors d'un mélange, les arcs contenus dans les deux vues sont sujets à
échange dans le but de supprimer les nœuds défaillants de toutes les vues
passives. Similairement à \CYCLON, les arcs privilégiés sont ceux nouvellement
reçus.


\subsection{Taille variable}
\label{net:subsec:variable}

La famille de protocoles d'échantillonnage aléatoire de pairs peuplant des vues
dont la taille n'est pas définie au préalable a pour seul représentant :

\paragraph{\SCAMP~\cite{ganesh2001scamp, ganesh2003peer} :} est l'acronyme de
\emph{SCAlable Membership Protocol}. Il s'agit d'un protocole d'échantillonnage
associant une réaction à certains évenements. En particulier, l'entrée dans le
réseau doit permettre d'augmenter la taille moyenne des vues partielles. Cette
augmentation est logarithmique et correspond au seuil précis
$\Theta (|\mathcal{N}|\log |\mathcal{N}|)$ des graphes aléatoires. Le départ
d'un nœud doit entrainer une équivalente diminution.

\noindent Les pairs utilisant \SCAMP maintiennent deux vues partielles
correspondants aux arcs entrant et aux arcs sortant.

\noindent Lorsque un nœud rejoins le réseau, il contacte un membre qu'il ajoute
directement à sa vue partielle. Ce contact annonce ensuite le nouveau venu au
réseau. Pour ce faire, il crée autant de messages d'annonce qu'il possède de
voisins et les émets. Chaque nœud recevant ce type de messages est libre
d'accepter l'annonce en ajoutant alors l'identité du nouveau venu à sa vue
partielle, ou la refuser, auquel cas il réexpedie le message à l'un de ces
voisins. Afin de répartir équitablement les annonces, chaque nœud $n_i$ a une
probabilité inversement proportionnelle à la taille de sa vue partielle
d'accepter l'annonce : $1/(|P_i|+1)$. Cette répartition des arcs seules permet
de créer une topologie proche de celle des graphes aléatoires. Cependant, la vue
partielle du nouveau pair est faiblement peuplée au contraire des pairs plus
vieux. Cela met en danger la robustesse du réseau face aux défaillances.

\begin{figure}
  \centering
  
\begin{tikzpicture}[scale=1.3]
  
  \draw[fill=white, draw=darkblue] (0pt, 0pt)
  node{\DARKBLUE{$n_1$}} +(-5pt,-5pt) rectangle +(5pt,5pt);

  \begin{scope}[shift={(75pt,0pt)}]
  \draw[fill=white] (-25pt, 0pt) node{$n_2$} +(-5pt,-5pt) rectangle +(5pt,5pt);
  \draw[fill=white] (0pt,  25pt) node{$n_3$} +(-5pt,-5pt) rectangle +(5pt,5pt);
  \draw[fill=white] ( 25pt, 0pt) node{$n_4$} +(-5pt,-5pt) rectangle +(5pt,5pt);
  \draw[fill=white] (0pt, -25pt) node{$n_5$} +(-5pt,-5pt) rectangle +(5pt,5pt);


  \draw[->] (-20pt, 5pt) -- (-5pt, 20pt); 
  \draw[->] (-20pt,-5pt) -- (-5pt,-20pt); 
  \draw[->] ( 5pt,-20pt) -- (20pt, -5pt); 
  \draw[->] (20pt,  0pt) -- (-20pt, 0pt); 
  \draw[->] (-5pt, 30pt) -- (-30pt, 5pt); 

  \draw[->, color=darkblue] (-70pt, 0pt) -- (-30pt, 0pt); 
  \draw[->, color=darkblue] (-5pt, 30pt) -- (-70pt, 5pt); 
  \draw[->, color=darkblue] (30pt, -5pt)to[out=-85,in=-95](-70pt,-5pt);

  \scriptsize
  \draw (-25pt,-5pt) node[align=left,anchor=north east]
  {dissémine\\$|P_2|$ copies};
  \draw (5pt,-30pt) node[align=left,anchor=north west]
  {$1/2$ chance d'accepter,\\refuse, fait suivre à $n_4$};
  \draw (5pt, 30pt) node[align=left,anchor=south west]
  {$1/2$ chance d'accepter,\\accepte, crée $n_3 \rightarrow n_1$};
  \draw (30pt, 0pt) node[align=left,anchor=west]
  {$1/2$ chance d'accepter,\\accepte, crée $n_4 \rightarrow n_1$};
  \end{scope}
  
%%  \small
%%  \begin{scope}[shift={(130pt,0pt)}]
%%    \draw[fill=white](5pt, 0pt)node{$p_x$}+(-5pt,-5pt) rectangle +(5pt,5pt);
%%    \draw (10pt,0pt) node[anchor=west]{Peer $p_x$};
%%    \draw[->](0pt, -12pt)--(10pt, -12pt) node[anchor=west]{Established link};
%%    \draw[->, densely dashed](0pt, -19pt)--(10pt, -19pt)
%%    node[anchor=west]{New link};
%%    
%%  \end{scope}
  
\end{tikzpicture}
  \caption{\label{net:fig:scampexample} Exemple de pair rejoignant le réseau
    dans \SCAMP.}
\end{figure}

\noindent La figure~\ref{net:fig:scampexample} montre l'entrée d'un nœud $n_1$
dans un petit réseau \SCAMP composé des nœuds $n_2$, $n_3$, $n_4$ et $n_5$. Dans
cet exemple, le nœud $n_1$ contacte $n_2$ et l'ajoute directement à sa vue
partielle. $n_2$, possédant 2 voisins, copie l'adresse logique de $n_1$ 2 fois
et les dissémine à $n_3$ et $n_5$. Lorsque $n_3$ réçoit le message, il possède
une chance sur deux d'accepter l'annonce ce qu'il fait : une connexion est créée
entre $n_3$ et $n_1$. Le nœud $n_5$ procède de la même façon. Toutefois, il
n'accepte pas l'annonce et retransmet le message à son voisin $n_4$. Ce dernier
l'accepte et l'arc allant de $n_4$ à $n_1$ est créé. Au total, 3 arcs ont été
créés à l'arrivée du pair $n_1$.


\noindent Lorsqu'un pair souhaite quitter le réseau,
\begin{inparaenum}[(i)]
\item il œuvre à préserver la connectivité de ce dernier et
\item il emporte avec lui une quantité d'arcs correspondant à l'entrée du
  dernier pair dans le réseau.
\end{inparaenum}
Pour ce faire, le nœud va agir comme pont entre les nœuds de sa vue partielle
entrante et ceux de sa vue partielle sortante. Ainsi, il permettra de resouder
la maille du réseau qu'il aurait défait en partant. Ce méchanisme est
généralement applicable à tous les protocoles d'échantillonnage mais il requière
de maintenir deux vues partielles -- entrante et sortante -- et des connexions
bidirectionnelles. Il requière aussi un travail supplémentaire de la part du
nœud sortant impossible à garantir lors de défaillances : le nœud devrat servir
d'intermédiaire à la connexion de tous les voisins de sa vue partielle entrante
(sauf un) vers autant de voisins appartenant à sa vue partielle sortante. Ainsi,
le nombre d'arcs supprimé correspond en moyenne à celui injecté à l'entrée d'un
nœud dans le réseau. Ce méchanisme améliore considérablement le maintient de la
connexité du réseau mais ne représente pas une solution fiable pour autant. En
effet, certains nœuds peuvent se trouver sans voisins dans l'une de leur vue.
  

\begin{figure}
  \centering \subfloat[Départ d'un pair dans \SCAMP.]
  [\label{net:fig:scampexampleB} Le nœud $n_1$ souhaite quitter le réseau. Il
  en informe ses vues partielles afin d'effectuer un pont entre eux.]
  {
\begin{tikzpicture}[scale=1.2]

  \newcommand\X{55pt};
  \newcommand\Y{15pt};

  \draw[<-](5+0*\X, -2*\Y)--(-5+1*\X, 0*\Y);
  \draw[<-](5+0*\X, -2*\Y)--(-5+1*\X, -1*\Y);
  \draw[<-](5+0*\X, -2*\Y)--(-5+1*\X, -3*\Y);

  \draw[->](-5+0*\X, -2*\Y)--(5-1*\X, 0*\Y);
  \draw[->](-5+0*\X, -2*\Y)--(5-1*\X, -1*\Y);
  \draw[->](-5+0*\X, -2*\Y)--(5-1*\X, -3*\Y);
  \draw[->](-5+0*\X, -2*\Y)--(5-1*\X, -4*\Y);

  \small
  \draw[fill=white,very thick, draw=darkblue]
  (0*\X, -2*\Y) node{\DARKBLUE{$n_1$}} +(-5pt,-5pt) rectangle +(5pt,5pt);
  \draw[thick, color=darkblue] (-5pt,-5-2*\Y) -- (5pt,5-2*\Y);
  \draw[thick, color=darkblue] (-5pt, 5-2*\Y) -- (5pt,-5-2*\Y);
  
  \draw[fill=white, draw=darkblue]
  (1*\X,0*\Y) node{\DARKBLUE{$n_2$}} +(-5pt,-5pt) rectangle +(5pt,5pt);
  \draw[fill=white, draw=darkblue]
  (1*\X,-1*\Y) node{\DARKBLUE{$n_3$}} +(-5pt,-5pt) rectangle +(5pt,5pt);
  \draw[fill=white]
  (1*\X,-3*\Y) node{$n_4$} +(-5pt,-5pt) rectangle +(5pt,5pt);

  \draw[fill=white, draw=darkblue]
  (-1*\X,0*\Y) node{\DARKBLUE{$n_5$}} +(-5pt,-5pt) rectangle +(5pt,5pt);
  \draw[fill=white, draw=darkblue]
  (-1*\X,-1*\Y) node{\DARKBLUE{$n_6$}} +(-5pt,-5pt) rectangle +(5pt,5pt);
  \draw[fill=white]
  (-1*\X,-3*\Y) node{$n_7$} +(-5pt,-5pt) rectangle +(5pt,5pt);
  \draw[fill=white]
  (-1*\X,-4*\Y) node{$n_8$} +(-5pt,-5pt) rectangle +(5pt,5pt);

\end{tikzpicture}}
  \hspace{45pt}
  \subfloat[Établissement de connexions entre vues partielles dans \SCAMP.]
  [\label{net:fig:scampexampleC} Le nœud $n_1$ notifie aux nœuds $n_2$ et
  $n_3$ qu'ils doivent ajouter $n_5$ et $n_6$ dans leur vue partielle sortante
  respective.]  {
\begin{tikzpicture}[scale=1.2]

  \newcommand\X{55pt};
  \newcommand\Y{15pt};

  % \draw[<-](5+0*\X, -2*\Y)--(-5+1*\X, 0*\Y);
  % \draw[<-](5+0*\X, -2*\Y)--(-5+1*\X, -1*\Y);
  % \draw[<-](5+0*\X, -2*\Y)--(-5+1*\X, -3*\Y);

  % \draw[->](-5+0*\X, -2*\Y)--(5-1*\X, 0*\Y);
  % \draw[->](-5+0*\X, -2*\Y)--(5-1*\X, -1*\Y);
  % \draw[->](-5+0*\X, -2*\Y)--(5-1*\X, -3*\Y);
  % \draw[->](-5+0*\X, -2*\Y)--(5-1*\X, -4*\Y);

  \draw[->, very thick, color=darkblue] (-5+1*\X, 0*\Y) -- (5-1*\X, 0*\Y);
  \draw[->, very thick, color=darkblue] (-5+1*\X, -1*\Y) -- (5-1*\X, -1*\Y);

  \small
  \draw[fill=white,very thick]
  (0*\X, -2*\Y) node{$n_1$} +(-5pt,-5pt) rectangle +(5pt,5pt);
  \draw[thick] (-5pt,-5-2*\Y) -- (5pt,5-2*\Y);
  \draw[thick] (-5pt, 5-2*\Y) -- (5pt,-5-2*\Y);
  
  \draw[fill=white]
  (1*\X,0*\Y) node{$n_2$} +(-5pt,-5pt) rectangle +(5pt,5pt);
  \draw[fill=white]
  (1*\X,-1*\Y) node{$n_3$} +(-5pt,-5pt) rectangle +(5pt,5pt);
  \draw[fill=white]
  (1*\X,-3*\Y) node{$n_4$} +(-5pt,-5pt) rectangle +(5pt,5pt);

  \draw[fill=white]
  (-1*\X,0*\Y) node{$n_5$} +(-5pt,-5pt) rectangle +(5pt,5pt);
  \draw[fill=white]
  (-1*\X,-1*\Y) node{$n_6$} +(-5pt,-5pt) rectangle +(5pt,5pt);
  \draw[fill=white]
  (-1*\X,-3*\Y) node{$n_7$} +(-5pt,-5pt) rectangle +(5pt,5pt);
  \draw[fill=white]
  (-1*\X,-4*\Y) node{$n_8$} +(-5pt,-5pt) rectangle +(5pt,5pt);

\end{tikzpicture}}
  \caption{\label{net:fig:scampexample2} Exemple de sortie de réseau dans
    \SCAMP. Seules les vues partielles entrante et sortante de $n_1$ sont
    explicitées.}
\end{figure}

\noindent La figure~\ref{net:fig:scampexample2} présente le mécanisme activé lors du
départ du nœud $n_1$ utilisant \SCAMP. $n_1$ fournit à chaque nœud de sa vue
partielle entrante l'adresse logique d'un nœud de sa vue partielle
sortante. Ainsi, le pair $n_2$ ajoute $n_5$ à sa vue partielle sortante, $n_3$
ajoute $n_6$ à sa vue partielle sortante, $n_4$ n'ajoute rien. La
figure~\ref{net:fig:scampexampleC} montre les arcs après le départ de
$n_1$. Entre autre, nous observons que si $n_4$ ne possède pas d'autres voisins
dans sa vue partielle sortante, ou si $n_7$ ou $n_8$ ne possèdent pas d'autres
voisins dans leur vue partielle entrante, alors le réseau n'est plus connexe.


%%% Local Variables:
%%% mode: latex
%%% TeX-master: "../../paper"
%%% End:
