
\section{Motivations}
\label{net:sec:motivations}

Les approches des §\ref{net:sec:fixed} et §\ref{net:sec:variable} présentent des
limitations. Les approches appartenant à la famille dont les vues partielles
sont constantes ne sont pas adaptées aux contextes où :
\begin{itemize}
\item La taille du réseau fluctue au cours du temps. Par exemple, lorsque les
  utilisateurs possèdent des habitudes identiques, il peut alors y avoir des
  heures creuses où le réseau ne compte que peu de membres. À l'inverse, lors
  d'événements particuliers, beaucoup d'entrées en très peu de temps peuvent
  avoir lieu.  Les approches de taille fixe ne peuvent configurer la taille de
  leurs vues partielles au cours du temps. C'est pourquoi les vues partielles
  sont généralement surdimensionnées afin de pourvoir aux besoins des plus
  grands réseaux estimés. Lorsque ces réseaux sont plus modestes aux périodes
  creuses, les nœuds sont surconnectés.
\item La taille des réseaux d'une même application varie. Par exemple, l'ampleur
  du groupe collaboratif d'un projet comme \emph{linux} n'est pas comparable à
  celle d'un projet personnel partagé avec quelques amis. Malgré tout, le
  développeur d'application doit prévoir large afin de supporter les grands
  groupes s'ils existent, au détriment des plus petits groupes d'utilisateurs.
\end{itemize}

L'approche réactive qu'est \SCAMP possède quant à elle la capacité d'ajuster les
vues partielles de ces membres à la taille du réseau auquel ils
appartiennent. Malheureusement, cette approche fait face à deux problèmes :
\begin{itemize}
\item Les annonces sont systématiquement propagées à plusieurs voisins de
  distance. \TODO{Moar.}
\item Le protocole est très peu dynamique au cours du temps. \TODO{Moar.}
\end{itemize}



%%% Local Variables:
%%% mode: latex
%%% TeX-master: "../../paper"
%%% End:
