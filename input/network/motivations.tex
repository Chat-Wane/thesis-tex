
\section{Motivations}
\label{net:sec:motivations}

Les approches des §\ref{net:subsec:fixed} et §\ref{net:subsec:variable} présentent des
limitations. Les approches appartenant à la famille dont les vues partielles
sont constantes ne sont pas adaptées aux contextes où :

\paragraph{Un même réseau.} La taille du réseau fluctue au cours du temps. Par
exemple, lorsque les utilisateurs possèdent des habitudes identiques, il peut
alors y avoir des heures creuses où le réseau ne compte que peu de membres. À
l'inverse, lors d'événements particuliers, beaucoup d'entrées en très peu de
temps peuvent avoir lieu.  Les approches de taille fixe ne peuvent configurer la
taille de leurs vues partielles au cours du temps. C'est pourquoi les vues
partielles sont généralement surdimensionnées afin de pourvoir aux besoins des
plus grands réseaux estimés. Lorsque ces réseaux sont plus modestes aux périodes
creuses, les nœuds sont surconnectés.

\paragraph{Plusieurs réseaux.} La taille des réseaux d'une même application
varie. Par exemple, l'ampleur du groupe collaboratif d'un projet comme
\emph{Linux} n'est pas comparable à celle d'un projet personnel partagé avec
quelques amis. Malgré tout, le développeur d'application doit prévoir large afin
de supporter les grands groupes s'ils existent, au détriment des plus petits
groupes d'utilisateurs.


%\TODO{Maybe talk about estimators\ldots}

L'approche réactive qu'est \SCAMP possède quant à elle la capacité d'ajuster les
vues partielles de ces membres à la taille du réseau auquel ils
appartiennent. Malheureusement, cette approche fait face à deux problèmes :

\paragraph{Propagation des arcs.} Les annonces sont systématiquement propagées à
plusieurs voisins de distance. Ce méchanisme participe à la construction rapide
d'une topologie proche de celle des graphes aléatoires. Toutefois, entre
l'origine et la destination de l'annonce, toute perte de message empêche
l'établissement de la connexion. Cela s'avère particulièrement problématique
lorsque l'établissement d'une connexion nécéssite un aller-retour comme c'est le
cas dans le contexte WebRTC (cf. §\ref{editor:subsec:webrtc}). En effet, non
seulement le message ne doit pas se perdre, mais les nœuds et arcs composants le
chemin de l'annonce doivent rester disponibles.

\paragraph{Dynamisme.} Le protocole est très peu dynamique au cours du temps. Si
la phase d'entrée dans le réseau permet d'obtenir un graphe dont les arcs sont
relativement bien répartis, \SCAMP doit renouveller les vues partielles afin de
mieux intégrer les nouveaux arrivants en peuplant leurs vues partielles et de
décharger les nœuds plus anciens. Ainsi, le méchanisme de bail (\emph{lease})
est introduit et consiste à rejoindre à nouveau le réseau après un certains
temps. La vue partielle sortante est conservée tandis que la vue partielle
entrante est réinitialisée avant d'être naturellement remplie lors de l'entrée
dans le réseau. Malheureusement, ce méchanisme tend à faire augmenter le nombre
d'arcs dans le réseau de manière non bornée. De plus, rejoindre à nouveau le
réseau confronte au premier problème susmentionné.


À la lumière des différentes limitations de ces approches, nous définissons le
problème suivant :

\begin{problem}
  \label{net:problem:properties}
  Soit $t$ une unité de temps arbitraire, soit $\mathcal{N}^t$ l'ensemble des
  membres du réseau à un instant $t$ et soit $P_i^t$ la vue partielle du nœud
  $n_i \in \mathcal{N}^t$. Un protocole d'échantillonnage aléatoire de pairs
  efficace doit assurer les propriétés suivantes :
  \begin{enumerate}
  \item Taille des vues partielles : \hfill $\forall n_i \in \mathcal{N}^t$,
    $|P_i^t| \approx \Theta(\ln |\mathcal{N}^t|)$
  \item Établissement de connexion : \hfill $\mathcal{O}(1)$
%  \item \TODO{Convergence}
  \end{enumerate}
\end{problem}
En d'autres termes, les vues partielles doivent s'adapter aux variations du
réseau et rester équilibrées. Les approches à taille fixe échouent à fournir
cette propriété. Le temps d'établissement d'une connexion doit être
borné. Puisque notre modèle ne considère pas la latence, ce temps est mesuré en
nombre de voisins parcourus. \SCAMP échoue à fournir cette propriété puisque
chaque connexion implique une dissémination aléatoire à une distance non bornée.

%%% Local Variables:
%%% mode: latex
%%% TeX-master: "../../paper"
%%% End:
