

\section{Taille variable}

Les protocoles d'échantillonnage aléatoire de pairs peuplant des vues dont la
taille n'est pas définie au préalable a pour seul représentant :

\begin{itemize}
\item [\textbf{\SCAMP~\cite{ganesh2001scamp, ganesh2003peer} :}] est l'acronyme
  de \emph{SCAlable Membership Protocol}. Il s'agit d'un protocole
  d'échantillonnage associant une réaction à certains évenements. En
  particulier, l'entrée dans le réseau doit permettre d'augmenter la taille
  moyenne des vues partielles. Cette augmentation est logarithmique et
  correspond au seuil précis $\Theta (|\mathcal{N}|\log |\mathcal{N}|)$ des
  graphes aléatoires. Le départ d'un nœud doit entrainer une équivalente
  diminution.

  Les pairs utilisant \SCAMP maintiennent deux vues partielles correspondants
  aux arcs entrant et aux arcs sortant.

  Lorsque un nœud rejoins le réseau, il contacte un membre qu'il ajoute
  directement à sa vue partielle. Ce contact annonce ensuite le nouveau venu au
  réseau. Pour ce faire, il crée autant de messages d'annonce qu'il possède de
  voisins et les émets. Chaque nœud recevant ce type de messages est libre
  d'accepter l'annonce en ajoutant alors l'identité du nouveau venu à sa vue
  partielle, ou la refuser, auquel cas il réexpedie le message à l'un de ces
  voisins. Afin de répartir équitablement les annonces, chaque nœud $n_i$ a une
  probabilité inversement proportionnelle à la taille de sa vue partielle
  d'accepter l'annonce : $1/(|P_i|+1)$. Cette répartition des arcs seules permet
  de créer une topologie proche de celle des graphes aléatoires. Cependant, la
  vue partielle du nouveau pair est faiblement peuplée au contraire des pairs
  plus vieux.

  Lorsqu'un pair souhaite quitter le réseau, il œuvre à préserver la
  connectivité du réseau. 
\end{itemize}


%%% Local Variables:
%%% mode: latex
%%% TeX-master: "../../paper"
%%% End:
