
\section{Propriétés}
\label{net:sec:properties}

\SPRAY est un protocole d'échantillonnage aléatoire de pair adaptatif. Cette
section a pour objectif d'examiner ses propriétés empiriquement au travers une
série de simulations. Les métriques auxquelles nous nous intéressons sont
communément employées dans l'étude de performances de ce type de protocoles ou
en étude des graphes. Ceux-ci incluent le coefficient d'agglomération
(\emph{clustering coefficient}), la taille moyenne du plus court chemin, la
distribution des connexions entrantes, l'évolution du nombre d'arcs, les
composantes connexes, et la robustesse. À cela nous ajoutons une analyse sur les
doublons d'arcs, spécificité de \SPRAY.

% \TODO{Review this §.}
% \CYCLON constitue l'approche à taille fixe avec laquelle nous comparerons les
% résultats en lien avec l'adaptabilité. Les résultats attendus sont :
% \begin{inparaenum}
% \item des propriétés identiques pour les deux approches lorsque \CYCLON est
%   configuré de manière optimale,
% \item une meilleure efficacité de \SPRAY lorsque les vues partielles de \CYCLON
%   sont trop grande par rapport à la taille du réseau,
% \item une plus grande robustesse lorsque les vues partielles de \CYCLON sont
%   trop petites par rapport à la taille du réseau.
% \item Enfin, contrairement à \CYCLON, \SPRAY possède des doublons dans ses vues
%   partielles. Toutefois, ce nombre de doublons est supposé négligeable.
% \end{inparaenum}
% \SCAMP constitue l'approche avec laquelle nous comparerons \SPRAY lorsque nous
% examinerons les échecs dans l'établissement des connexions. Nous nous attendons
% à ce que \SCAMP échoue à maintenir un réseau connecté, contrairement à \SPRAY.

Les expérimentations ont été effectuées sur le simulateur
\PEERSIM~\cite{montresor2009peersim}. Le code relatif aux protocoles
d'échantillonnages \CYCLON, \SCAMP, et \SPRAY sont disponibles sur la
plate-forme Github à l'adresse : \url{http://github.com/justayak/peersim-spray}.

\subsection{Coefficient d'agglomération}
\label{net:subsec:clustering}

\begin{figure*}
  \centering
  \subfloat[Coefficient d'agglomération de \CYCLON.]
  [Coefficient d'agglomération de \CYCLON.]
  {\includegraphics[width=0.47\textwidth]{img/spray/cycloncluster.eps}}
  \hspace{10pt}
  \subfloat[Coefficient d'agglomération de \SPRAY.]
  [Coefficient d'agglomération de \SPRAY.]
  {\includegraphics[width=0.47\textwidth]{img/spray/spraycluster.eps}}
  \caption{\label{net:fig:clustering}L'axe des abscisses marque le temps écoulé
    en nombre de cycles. L'axe des ordonnées note le coefficient d'agglomération
    sur une échelle logarithmique de base 10.}
\end{figure*}

Le coefficient d'agglomération désigne une mesure de groupement des nœuds. En
d'autres termes, si un petit groupe possède un grand nombre d'arcs les reliants,
alors la mesure grimpe. Ainsi, les graphes complets affichent un coefficient
maximal. À l'inverse, les graphes aléatoires affichent un coefficient très bas
car la répartition des arcs ainsi que leur nombre ne favorise pas l'apparition
de groupes.

Si le comportement de cette métrique fait consensus, il existe malgré tout
plusieurs formules pour la calculer. Dans cette simulation, nous employons le
coefficient local moyen $\overline{C}$~\cite{watts1998collective} effectuant la
moyenne du coefficient $C_x$ de chaque nœud $n_x$ :
\begin{center}
  $\overline{C} = {1\over |\mathcal{N}|}\sum\limits_{x\in\mathcal{N}}C_x$ \hfill
  avec
  $C_x = {|\{ (n_y,\,n_z) \in P_x, n_y \in P_z \vee n_z \in P_y \}|\over
    |P_x|.(|P_x|-1)}$
\end{center}

\begin{itemize}
\item [\textbf{Objectif :}] Montrer que \SPRAY converge rapidement vers une
  topologie ne comportant pas de groupes fortement connexes.
\item [\textbf{Description :}] Les simulations ciblent des réseaux composés
  respectivement de 0.1k, 1k, 10k, et 100k nœuds. Le représentant des approches
  à taille fixe est \CYCLON. Ce dernier est configuré de manière optimale pour
  1k pairs ($\ln(1000)\approx 7$ voisins). De plus, les nœuds utilisant \CYCLON
  échangent 3 de leurs 7 voisins à chaque mélange.
\item [\textbf{Résultat :}] La figure~\ref{net:fig:clustering} montre que
  \CYCLON démarre avec un coefficient d'agglomération plus bas que
  \SPRAY. Malgré cela, \SPRAY parvient à converger plus rapidement que
  \CYCLON. De plus, quand le nombre de nœuds augmente dans le réseau, le temps
  de convergence de \CYCLON en souffre fortement. À l'opposé, \SPRAY converge
  très rapidement quel que soit la taille du réseau. La
  figure~\ref{net:fig:clustering} montre aussi que les deux approches convergent
  vers un petit coefficient caractéristique des graphes aléatoires. Néanmoins,
  \CYCLON et \SPRAY n'atteignent pas les mêmes valeurs après convergence. À
  l'exception du cas où \CYCLON est configuré de manière optimale, les valeurs
  obtenues par \SPRAY sont soit au dessous -- lorsque les vues de \CYCLON sont
  trop grandes -- ou au dessus -- lorsque les vues de \CYCLON sont trop
  petites. Globalement, \SPRAY
  \begin{inparaenum}[(i)]
  \item converge vers un coefficient d'agglomération stable
  \item reflètant les besoins dûs à la taille du réseau.
  \end{inparaenum}
%%  Cela a une influence sur l'équilibrage des charges et la robustesse par
%%  rapport aux allées et venues de pairs.    
\item [\textbf{Explication :}] \CYCLON commence avec un coefficient
  d'agglomération plus faible car lorsqu'un nœud rejoint le réseau, il est
  annoncé au reste du réseau via une dissémination aléatoire. Ainsi, le réseau
  de départ est déjà légèrement équilibré au moment où la simulation commence. À
  l'opposé, un nouvel arrivant \SPRAY n'annonce son entrée qu'au voisinage de
  son contact. De ce fait, le réseau est fortement déséquilibré au départ de
  l'expérience quelle que soit taille du réseau. Malgré cela, \CYCLON ne
  converge pas aussi vite que \SPRAY vers un coefficient stable. En effet, la
  taille fixe de sa vue partielle ainsi que le nombre d'entrées de chaque
  mélange contraint la qualité des échanges.  Le coefficient d'agglomération
  mesure la connexité du voisinage de chaque nœud. Plus particulièrement, cela
  mesure à quel point le réseau est proche d'un graphe complet. Cela dépend donc
  des tailles de vue partielle qui, pour \CYCLON, sont fixées à la
  configuration.  Ainsi lorsque le nombre de nœuds est multiplié par 10, le
  coefficient s'en trouve divisé par 10. En revanche, les nœuds utilisant \SPRAY
  ont une taille de vue partielle reflétant automatiquement la taille du réseau.
  Ainsi, quand le réseau contient 1k nœuds, les vues partielles s'adaptent à
  cette taille. Par conséquent, \SPRAY est très légèrement en dessous de \CYCLON
  dans ce scénario car la taille moyenne de vue partielle est de 7.4 pour ce
  premier contre 7 pour ce second. En étendant ce raisonnement aux autres
  tailles de réseau, cela explique pourquoi \SPRAY converge vers une valeur plus
  basse lorsque les vues partielles de \CYCLON sont trop grandes (0.1k nœuds),
  et vers une valeur plus haute lorsque les vues partielles de \CYCLON sont trop
  petites (10k et 100k nœuds).
\end{itemize}

\subsection{Plus court chemin moyen}
\label{net:subsec:shortestpath}

\begin{figure}
  \centering
  \includegraphics[width=.8\textwidth]{img/spray/avgpath.eps}
  \caption{\label{net:fig:shortestpath} La moyenne des plus courts chemins dans
    \SPRAY et \CYCLON. L'axe des abscisses montre le nombre de nœuds appartenant
    aux réseaux sur une échelle logarithmique de base 10. L'axe des ordonnées
    montre la moyenne des plus courts chemins.}
\end{figure}

La moyenne des plus courts chemins permet de mesurer à quel point les nœuds sont
proches les un des autres. Plus précisément, il s'agit de compter le nombre
minimum de nœuds intermédiaires entre deux nœuds distincts. Ainsi, même s'il
existe plusieurs chemins pour aller d'un nœud $n_i$ à un nœud $n_j$, seul nous
importe le plus court.  Les graphes aléatoires possèdent un plus court chemin
moyen très petit grandissant en $\ln|\mathcal{N}|\over\ln\ln|\mathcal{N}|$
lorsque le nombre d'arcs suit $|\mathcal{N}|\ln|\mathcal{N}|$.

Cette métrique permet de cerner l'efficacité de la diffusion d'informations dans
le réseau. Par exemple, si le plus court chemin d'un nœud $n_i$ à un nœud $n_j$
est de 3, cela signifie qu'un message émanant de $n_i$ parviendra en seulement 3
intermédiaires à $n_j$.

\begin{itemize}
\item [\textbf{Objectif :}] Montrer que le caractère adaptatif de \SPRAY permet
  un meilleur passage à l'échelle du plus court chemin moyen.
\item [\textbf{Description :}] Les mesures sont effectuées sur un sous-ensemble
  de nœuds choisis aléatoirement parmis les membres du réseau. Le procédé est
  répété 100 fois afin d'éviter tout effets de bord dû à l'indéterminisme des
  protocoles d'échantillonnage. Les expérimentations concernent \CYCLON
  configuré de façon optimale pour différentes tailles de réseau. Le \CYCLON
  avec une taille de vue partielle de 7 cible environ 1.1k nœuds. Le \CYCLON
  avec une taille de vue partielle de 9 cible environ 8.1k nœuds. Le \CYCLON
  avec une taille de vue partielle de 11 cible environ 60k nœuds. Lors de toutes
  ces simulations, les mesures sont effectuées après convergence. Celles-ci sont
  faites lorsque la taille du réseau atteint 0.1k, 0.5k, 1k, 5k, 10k, 50k, et
  100k nœuds.
\item [\textbf{Résultat :}] La figure~\ref{net:fig:shortestpath} montre que
  \CYCLON et \SPRAY ont tous deux un chemin moyen relativement petit. Ainsi, les
  informations peuvent être disséminées à tous les membres du réseau très
  rapidement. La figure~\ref{net:fig:shortestpath} montre aussi que chaque
  exécution de \CYCLON prise séparément peut être divisée en trois phases.  Tout
  d'abord, la phase où les vues partielles de \CYCLON sont trop grandes
  dissémine l'information plus rapidement que \SPRAY. Lors les vues partielles
  sont optimales, \CYCLON et \SPRAY montrent des résultats similaires. Enfin,
  \SPRAY montre une meilleure efficacité lorsque les vues partielles de \CYCLON
  sont trop petites. Malgré tout, \SPRAY passe mieux à l'échelle que \CYCLON
  l'inclinaison de cette première est inférieure à n'importe quelle
  configuration de cette dernière.
\item [\textbf{Explication :}] Les mesures sont toutes effectuées après
  convergence, lorsque les réseaux possèdent une topologie proche des graphes
  aléatoires.  Dans de tels graphes, la taille du plus court chemin moyen reste
  petit.  La seconde observation concerne chacune des configurations de \CYCLON
  comparée à \SPRAY. Une taille de vue partielle surestimée pour \CYCLON est
  meilleure en terme de connexité entre nœuds et résulte dans de plus courts
  chemins en moyenne. En revanche, dès que cette taille devient insuffisante
  pour le réseau, \SPRAY devient plus efficace car il possède de plus grandes
  vues partielles. Puisque \SPRAY suit toujours la taille optimale de vue
  partielle, il passe mieux à l'échelle en terme de taille réseau.
\end{itemize}

\subsection{Distribution des arcs entrants}
\label{net:subsec:inview}

\begin{figure}
  \centering
  \includegraphics[width=.8\textwidth]{img/spray/histo.eps}
  \caption{\label{net:fig:inview} Distribution du nombre d'arcs entrants à
    chaque nœud dans \CYCLON et \SPRAY. L'axe des ordonnées montre la proportion
    de nœuds ayant une vue entrante de la taille correspondante sur l'axe des
    abscisses. Les figure du haut et du bas sont respectivement dédiées aux
    exécutions concernant \CYCLON et \SPRAY.}
\end{figure}

La distribution des arcs entrants correspond au nombre de fois qu'un nœud est
référencé dans une vue partielle. À cet égard, la distribution est différentes :
Lorsque la taille des vues partielles est controllée, les vues entrantes sont
aléatoirement remplies par d'autres nœuds au cours de l'exécution du
protocole. Certains nœuds sont donc moins représentés que d'autres.

Au même titre que la distribution des vues partielles, la distribution des arcs
sortants permet de déduire des faits sur la robustesse du réseau, l'équilibre de
la charge etc. \TODO{Moar.} 

\begin{itemize}
\item [\textbf{Objectif :}] Montrer que la distribution des arcs entrants dans
  \SPRAY suit gracieusement l'évolution du réseau.
\item [\textbf{Description :}] Dans cette expérimentation, \CYCLON est configuré
  avec une vue partielle contenant 7 voisins (optimal pour 1.1k nœuds). Les
  mesures sont effectuées après convergence sur des réseaux comprenant : 0.1k,
  1k, 100k, 500k nœuds.
\item [\textbf{Résultat :}] Le haut et le bas de la figure~\ref{net:fig:inview}
  montrent la distribution des connexions entrantes pour \CYCLON et \SPRAY,
  respectivement. Nous observons que \CYCLON possède une distribution identique
  quelle que soit la taille du réseau. La distribution d'un réseau de 0.1k nœuds
  est identique à celle de 500k nœuds, avec un fort pic sur la valeur moyenne de
  7 voisins. À l'opposé, la distribution des connexions entrantes de \SPRAY suit
  la taille du réseau. La figure~\ref{net:fig:inview} montre que les nœuds sont
  très concentrés autour des valeurs moyennes. Par exemple, lors de l'exécution
  du protocole \SPRAY avec 500k nœuds, la moyenne est de 13.37 avec 88\% des
  nœuds compris entre 12 et 14 voisins inclus. Entre autre, cela signifie que la
  charge est très équilibrée parmi les nœuds. Puisque chaque nœud est aussi
  important que son voisin en terme de connexité, le réseau est robuste
  vis-à-vis des défaillances.
\item [\textbf{Explication :}] Une fois configuré, \CYCLON doit gérer tous les
  réseaux, quelle que soit leur taille, avec une vue partielle dont la taille
  est constante. Proportionnellement, le nombre de fois qu'un nœuds est
  référencé dans les vues partielles ne change pas comparé à la taille
  réseau. En effet, le nombre d'arcs qu'un nœud apporte lorsqu'il se connecte au
  réseau constitue autant d'arcs le ciblant après quelques protocoles
  d'échanges. Puisque la taille de la vue partielle est constante, le degré des
  connexions entrantes reste stable. En revanche, dans \SPRAY, chaque nœuds
  rejoignant le réseau augmente le nombre d'arcs dans le réseau. Ainsi, le degré
  de connexions entrantes de chaque nœud grossit reflétant les variations du
  réseau. Par conséquent, la distribution de \SPRAY se décale lentement vers de
  plus hautes valeurs lorsque la taille du réseau augmente. \SPRAY ne possède
  pas de pics sur les valeurs moyennes car celles-ci sont des valeurs qui
  tombent entre deux entiers. Par exemple, si la taille moyenne des vues
  partielles est 6.5, cela signifie que la moitié de celle-ci ont une taille de
  6, et l'autre moitié une taille de 7. De tels réseaux sont robustes aux
  défaillances car aucun nœud n'est plus important que son voisin en terme de
  connexité. Si un nœud particulier quitte ou défaillit, les nœuds le
  référençant possèdent d'autres voisins avec qui communiquer afin de continuer
  à disséminer les informations.
\end{itemize}

\subsection{Évolution du nombre d'arcs}
\label{net:subsec:churn}

\begin{figure*}
  \centering
  \subfloat[Figure A][\label{net:fig:churnA}The x-axis denotes the
  elapsed time in cycles. The upper graph y-axis shows the number of total
  connections in the overlay while the lower graph y-axis shows the variance
  $\sigma^2$ of the partial view sizes in the network.]{
    \includegraphics[width=0.47\textwidth]{img/spray/churn.eps}}
  \hspace{10pt}
  \subfloat[Figure B][\label{net:fig:churnB}The x-axis denotes the
  elapsed time in cycles. The y-axis denotes the average partial view size.]{
    \includegraphics[width=0.47\textwidth]{img/spray/avgpv.eps}}
  \caption{\label{net:fig:churn}\CYCLON (partial view size configured to 9)
    and \SPRAY in a dynamic network. 2.5k peers join the network at cycles $0$,
    $10$, $20$, and $30$. Then 5k peers leave at cycle $40$. Finally 2.5k peers
    join at cycles $60$ and $70$. The final network contains 10k members.}
\end{figure*}

\begin{asparadesc}
\item [Objectif:] Montrer comment l'influence de l'adaptabilité sur un réseau
  dont la taille varie au court du temps.
\item [Description:] Cette expérimentation se concentre sur un réseau dynamique
  dont les membres rejoignent et quittent le système lorsqu'ils le
  souhaitent. Les exécutions concernent les protocoles d'échantillonnage \CYCLON
  et \SPRAY. La configuration de \CYCLON cible un réseau de taille 8.1k
  pairs. Ainsi, les vues partielles sont trop grandes lors de cette simulation
  qui implique au maximum 1k pairs. Lors de la première moitié de la simulation,
  4 groupes successifs comportant 250 pairs rejoignent le réseau. L'intervalle
  de temps entre ces entrées groupées est de 10 cycles. Ainsi, le réseau
  initialement vide contient 1k pairs au bout de 40 cycles. Ensuite, la moitié
  des membres du réseau le quitte (500 pairs). Enfin, 2 groupes de 250 pairs
  rejoignent le réseau à nouveau faisant passer la taille de celui-ci à 1k
  pairs. Les mesures concernent
  \begin{inparaenum}[(i)]
  \item le nombre de connexions dans le réseau au cours des cycles,
  \item la variance de la taille des vues partielles au cours des cycles,
  \item la taille moyenne des vues partielles au cours du temps.
  \end{inparaenum}
\item [Résultat:] La figure~\ref{net:fig:churn} montre les résultats de cette
  expérimentation. L'axe des abscisses représente les cycles (i.e. une mesure
  arbitraire de temps). La partie haute de la figure~\ref{net:fig:churnA}
  montre le nombre de connections établies dans le réseau (échelle
  $\times 10^3$) tandis que la partie basse de la figure montre la variance de
  taille des vues partielles parmis les pairs. En ce qui concerne \SPRAY, nous
  observons qu'à chaque groupe rejoignant le réseau, le nombre de connections
  augmente afin de refléter cette augmentation. Ces observations concordent avec
  les mesures sur la variance. En effet, à chaque groupe rejoignant le réseau,
  la variance augmente soudainement. Ensuite, elle décroît exponentiellement et
  converge vers 0 en moins de 10 cycles. La variance est plus élevée lorsque la
  taille du réseau est basse. Par exemple, le premier groupe de 250 pairs
  conduit à la plus forte hausse de variance. Au $40^{ème}$ cycle, la moitié des
  pairs quittent le réseau. Approximativement la moitié des connections sont
  supprimées mais n'influent pas sur la variance. Les 10 cycles suivant montre
  une très faible diminution du nombre d'arcs. Enfin, de nouveaux groupes de
  pairs sont réintroduits dans le réseau et mène au même conclusions que
  précédemment. \CYCLON montre le même genre de comportement. Néanmoins, le
  nombre d'arcs est invariablement supérieurs à celui de \SPRAY (entre 1000 et
  2500 connexions supplémentaires). La figure~\ref{net:fig:churnB} montre la
  taille moyenne des vues partielles de \SPRAY et de \CYCLON. Comme prévu,
  \CYCLON converge immédiatement vers la taille de vue partielle planifiée lors
  de sa configuration (9 voisins). À l'opposé, la taille de vue partielle
  moyenne de \SPRAY augmente logarithmiquement tandis que la taille du réseau
  augmente. Lorsque les départs de pairs surviennent au cycle 40, \CYCLON
  supprime les liens morts tout en remplissant les vues partielles afin
  d'atteindre la taille de vue partielle de sa configuration initiale. \SPRAY
  supprime seulement les arcs nécessaire afin de refléter la nouvelle taille
  réseau. À la fin de la simulation, les pairs utilisant \SPRAY ont une vue
  partielle contenant 6.6 voisins en moyenne (rappel: $\ln(1000)\approx 6.9$).
\item [Explication:] Puisque les vues partielles de \SPRAY s'adaptent à la
  taille du réseau, le nombre de connexions augmente lorsque des membres
  s'ajoute au réseau. Les pics de variance correspondent à la partie du
  protocole où les pairs rejoignent le réseau. Cette disparité provient du fait
  que les nouveaux arrivant commencent avec une petite vue partielle. Les pics
  sont plus petits lorsque le réseau est plus grand. En effet, les pairs étant
  présent avant que le nouveau groupe rejoigne le réseau ont eu quelques cycles
  d'échanges afin d'équilibrer leur vue partielle. Par conséquent, le poids des
  nouveaux arrivant est proportionnellement plus faible. Le départ des 500 pairs
  au cycle 40 ne perturbe pas la variance car ils sont effectués de manière
  aléatoire. Ainsi, aucun pair ne souffre plus qu'un autre des départs. La très
  petite décroissance du nombre d'arcs après les départs est due au fait que les
  pairs restant réalisent que les pairs sont injoignables seulement après
  quelques cycles. 
\end{asparadesc}

\subsection{Robustesse}

\begin{figure}
  \centering
  \includegraphics[width=.8\textwidth]{img/spray/resilience.eps}
  \caption{\label{fig:spray:resilience}Robustness of \CYCLON and \SPRAY to
    massive failures. The x-axis denotes the percentage of peers removed at once
    in a network containing 10k members. The y-axis denotes the number of
    components over the current network size (after the removals). The
    measurements concern the weak and strong components which basically means
    the number clusters in undirected or directed graph respectively.}
\end{figure}

\begin{asparadesc}
\item [Objectif:] Montrer que \SPRAY and \CYCLON sont tous deux robustes aux
  défaillances massives.
\item [Description:] Compter le nombre de composantes fortement connexes dans un
  réseau permet d'estimer la surface atteignable par les protocoles de
  dissémination d'informations. Par exemple, il y a deux composantes fortement
  connexes dans un réseau dont une partie peut atteindre l'autre sans que
  l'inverse soit vrai. Compter le nombre de composantes faiblement connexes dans
  une réseau permet d'estimer le moment où celui-ci n'est plus réparable. Un
  réseau est réparables si après quelques échanges de vues partielles, les
  informations peuvent être disséminées à tous les participants encore dans le
  réseau. \CYCLON est configuré pour obtenir des vues partielles contenant 9
  voisins. Le réseau compte 10k membres. Nous effectuons les suppressions de
  pairs après convergence des approches. Nous supprimons les pairs en une fois,
  allant de 25\% à 95\% par paliers de 5\% ce qui représente 16 exécutions par
  approche. La dernière mesure est effectuée à 99\% de suppressions. Les mesures
  sont effectuées immédiatement après les suppressions. Ces dernières concernent
  un pourcentage de pairs aléatoirement choisis parmi les 10k pairs.
\item [Résultat:] La figure~\ref{fig:spray:resilience} montre le ratio de
  composantes fortement/faiblement connexes du réseau après les suppressions de
  pairs. Tout d'abord, la figure montre que les deux protocoles
  d'échantillonnage de pairs, \SPRAY et \CYCLON, ne souffrent de comportement
  défaillant qu'après un haut pourcentage de suppressions, \CYCLON étant
  légèrement meilleur dans ce cas. La figure~\ref{fig:spray:resilience} montre
  que la dissémination d'informations (les composantes fortement connexes)
  commencent se dégrader lentement dès 45\%, et plus rapidement à
  70\%. Heureusement, la figure~\ref{fig:spray:resilience} nous montre aussi que
  ces approches sont capables de se réparer de ce clustering jusqu'à de haut
  taux de suppressions. En effet, les composantes faiblement connexes ne
  commencent à augmenter qu'à partir de 70\%, ce qui signifie que des parties du
  réseau sont complètement disjointes et donc, hors de toute possibilité de
  réparation.
\item [Explication:] Les protocoles d'échantillonnage de pairs \CYCLON et \SPRAY
  affichent de très similaires résultats car \CYCLON est configuré pour une
  taille de réseau de 10k pairs, tandis que \SPRAY s'ajuste automatiquement à
  cette taille de réseau. Par conséquent, leur nombre d'arcs est très
  proche. Ici, \CYCLON possède légèrement plus de connexions (car \SPRAY possède
  une part d'aléatoire) et aucun doublon dans les vues partielles (\SPRAY
  possède un faible taux de doublons,
  cf. Figure~\ref{fig:spray:duplicates}). Afin de mettre en dangers la
  dissémination d'information, il est nécessaire de supprimer un fort taux de
  pairs. En effet, chaque pair revêt grossièrement la même importance que son
  voisin. Ainsi, supprimer des pairs aléatoirement parmi ceux-ci n'affectent pas
  énormément le réseau en entier. Le réseau est capable de se réparer car les
  protocoles d'échantillonnage ne dépendent que d'arcs unidirectionnels pour
  fonctionner. Ainsi, s'il subsiste un lien entre deux composantes fortement
  connexes, les échanges de vues partielles permettent de réparer cela.
\end{asparadesc}

\subsection{Doublons}

\begin{figure}
  \centering
  \includegraphics[width=.8\textwidth]{img/spray/duplicates.eps}
  \caption{\label{fig:duplicates}Duplicates in networks of different size: the
    $\log_{10}$-scaled x-axis denotes the network size while y-axis denotes the
    proportion of peers without any duplicates in their partial view.}
\end{figure}



% \begin{figure}
%   \centering \includegraphics[width=.8\textwidth]{img/spray/degen.eps}
%   \caption{\label{fig:degeneration}\CYCLON, \SCAMP, and \SPRAY in network
%     subject to failures in the connection establishments. The x-axis denotes
%     the elapsed time in cycles ($10^3$-scaled). The y-axis of the top figure
%     denotes the global number of arcs ($10^3$-scaled). The y-axis of the bottom
%     figure denotes the ratio of weak components over the current network size.}
% \end{figure}

%%% Local Variables:
%%% mode: latex
%%% TeX-master: "../../paper"
%%% End:
