
\section{État de l'art}

Les protocoles d'échantillonnage aléatoire de pairs permettent à chaque pair
d'obtenir une vue partielle $\mathcal{P}$ des membres du réseau $\mathcal{N}$.
Ils remplissent ces vues avec les références de pairs choisies aléatoirement
parmi $\mathcal{N}$ en suivant une distribution uniforme et en utilisant
seulement des connaissances locales. Leur but est de converger vers un réseau
dont certaines propriétés sont semblables à celles des graphes
aléatoires~\cite{erdos1959random}. En particulier, ils assurent de manière
efficace la connectivité du réseau, la robustesse, la dissémination des
informations etc. Un large éventail de protocoles basés sur le \emph{gossip}
utilisent ce genre d'échantillonnage aléatoire (par exemple, la gestion de
topologies~\cite{jelasity2009tman, dabek2004vivaldi}).

\subsection{Voisinages à taille fixe}

Parmi ces protocoles d'échantillonnage aléatoire, la plupart utilise une vue
partielle dont la taille est fixée~\cite{tolgyeski2009adaptive,
  eugster2003lightweight, voulgaris2005cyclon}. Ainsi, ils doivent connaître
\emph{a priori} la taille maximum du réseau afin de se paramétrer. Changer ces
paramètres après la création du réseau n'est pas sans risques. Ainsi, en
utilisant de telles approches, il est possible de maintenir 7 connections
actives dans le navigateur quand bien même 4 étaient suffisante. L'instant
d'après, ces approches continuent de maintenir ces 7 connections alors que la
croissance du réseau fait qu'il en requière 10. De ce fait, les vues partiels
sont souvent surdimensionnées ce qui implique un coût supplémentaire.

\subsection{Éstimateurs de taile réseau}

Il est possible d'utiliser un estimateur de taille réseau afin d'introduire de
l'adaptativité dans ces protocoles. Ces estimateurs utilisent soit
\begin{itemize}
\item des techniques d'échantillonnage~\cite{mane05network, ganesh2007peer,
    kostoulas2007active} qui analysent un sous ensemble du réseau
  et en déduisent la taille du réseau en utilisant des fonctions probabilistes,
\item des techniques d'esquisse~\cite{flajolet2008hyperloglog,
    baquero2012extrema} qui utilisent des \emph{hash} afin de compresser un
  grand nombre d'informations et en déduire la taille du réseau grâce aux
  collisions,
\item des techniques de moyennes~\cite{jelasity2004epidemic, blasa2011symmetric}
  qui utilisent des fonctions d'agrégation convergeant aux cours des échanges
  vers une valeur dépendant de la taille du réseau.
\end{itemize}
Malheureusement, bien que ces techniques puissent être très précises dans leur
estimation, elles impliquent un coût en trafic non-négligeable.

\subsection{Voisinages à taille variable}

Le seul représentant d'une approche adaptative-par-conception se nomme
\SCAMP~\cite{ganesh2001scamp, ganesh2003peer}.  En effet, la taille des vues
partielles de ses pairs est variable et grandit de manière logarithmique comparé
à la taille du réseau. Néanmoins, \SCAMP souffre d'autres défauts. En
particulier, il dissémine systématiquement les connexions de manière aléatoire
dans le réseau. Ainsi, le pair à l'origine de l'établissement de la connexion
peut se trouver à plusieurs sauts du pair acceptant d'établir la connexion. Dans
WebRTC, il est nécessaire de faire un aller-retour pour établir une
connexion. Cela a impact fort sur la probabilité d'échec de \SCAMP. En effet,
soit $P_f$ la probabilité qu'un élément dans le chemin de dissémination (soit un
pair, soit une connexion) crash/parte pendant un saut du \emph{three way
  handshake}, sans possibilité de rétablissement. Soit $P_E$ la probabilité
qu'une connexion ne puisse être menée à bien. Sans le \emph{three-way
  handshake}, $P_E$ est:
\begin{equation}
  P_{E,\,1way}^{Scamp}= 1 - (1-P_f)^{k+1}
\end{equation}
Cela correspond à la probabilité de chaque élément (arc ou pair) dans le chemin
de taille $k+1$ reste actif pendant qu'il achemine le message (dans tous les
autres cas, ils sont autorisés à partir ou crash). Dans le contexte WebRTC, le
ticket d'offre doit être renvoyé à l'émetteur. Conséquemment, les éléments dans
le chemin de dissémination n'ont pas le droit de partir/crash avant que le
ticket leur soit retourné. Nous obtenons donc:
\begin{align} P_{E,\,3way}^{Scamp} &=1 - ((1-P_f)^{2(k+1)} (1-P_f)^{2k}
                                     \ldots (1-P_f)^2) \nonumber \\
                                   &=1-(1-P_f)^{k^2+3k+2}
\end{align}
En d'autres termes, le premier arc et pair choisis doivent rester actif durant
$2k+2$ sauts, le second $2k$ sauts etc. La classe de complexité d'échec de
\SCAMP augmente. Cela entraîne une forte diminution du nombre de connections au
cours du temps, mettant en danger la connectivité du réseau.

%%% Local Variables:
%%% mode: latex
%%% TeX-master: "../../paper"
%%% End:
