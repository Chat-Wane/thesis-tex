
\section{Introduction}

WebRTC permet la communication en temps réel entre navigateurs web et ce, même
en présence de configurations réseaux complexes impliquant firewall, proxy, ou
NAT (Network Address Translation). Toutefois, WebRTC ne gère ni l'adressage, ni
le routage. Pour établir une connexion, les navigateurs s'échangent des offres
et acquittements via un médiateur commun (e.g. mails, services dédiés de
signalement, connections WebRTC connues, etc.). Dans la
figure~\ref{fig:webrtcA}, $p_1$ souhaite se connecter à $p_2$. Par conséquence,
$p_1$ envoie son ticket d'offre au service de signalement connu. Le pair $p_2$
récupère l'offre, la poinçonne et la renvoie au service de signalement. Enfin,
$p_1$ récupère le ticket poinçonné et établie une connexion bidirectionnelle
avec $p_2$. De manière identique, $p_3$ établie une connexion avec $p_2$. Nous
appellerons la procédure d'aller-retour des tickets un \emph{three-way
  handshake}. Désormais, le pair $p_1$ est capable d'établir une connexion avec
$p_3$ sans passer par l'intermédiaire du serveur. Pour cela, il utilise $p_2$
comme médiateur. Toutefois, si le pair $p_2$ crash durant cette procédure, la
connexion ne pourra s'effectuer correctement, et ce même si une route
alternative existe (puisque WebRTC ne gère pas le routage).

Utiliser les services de signalement et les connections WebRTC existantes permet
de déployer facilement les protocoles d'échantillonnage aléatoire de
pairs~\cite{jelasity2007gossip}. Ces derniers étant présent dans les navigateurs
modernes disponibles sur les smartphones, les tablettes, etc. Dans ce contexte,
il est impératif de conserver autant que possible un petit nombre de connections
afin de réduire le trafic réseau et limiter la consommation de ressources.

\subsection{Champs d'applications}

\subsubsection{Dissémination d'informations}
\subsubsection{Gestion de topologies réseau}

\subsection{WebRTC}

\subsubsection{Facilité d'accès}
\subsubsection{Contraintes supplémentaires}


%%% Local Variables:
%%% mode: latex
%%% TeX-master: "../../paper"
%%% End:
