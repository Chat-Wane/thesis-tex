
\lettrine{G}oogle Docs a rendu l'accès à l'édition collaborative temps réel
aisé. Des millions d'utilisateurs partagent et rédigent leurs documents
directement dans leur navigateur web. Toutefois, Google sert d'intermédiaire
dans ces sessions d'édition grâce à un serveur central, ce qui soulève des
problèmes en termes de confidentialité, de censure, et d'intelligence
économique. À cela s'ajoutent des problèmes de passage à l'échelle, notamment en
ce qui concerne le nombre d'utilisateurs. Bien que les petits groupes d'auteurs
soient la cible principale de ce genre d'application, certains événements de
plus ample dimension tels que les cours en ligne ouverts et massifs
(\emph{MOOC}) nécessite le support des groupes plus larges. Google Docs
supportent les groupes de grande taille. En revanche, seuls les 50 premiers
utilisateurs ont le droit d'écrire ensemble, les suivants voient leurs accès
limités à la simple lecture du document. Même si seulement un petit ensemble
parmi les millions d'utilisateurs sont réellement en train d'écrire, n'importe
quel participant de la session d'édition devrait pouvoir lire et écrire
lorsqu'il le souhaite.

Les éditeurs temps réel décentralisés n'ont pas besoin de serveurs
intérmédiaires et par là même résolvent les problèmes liés à la
confidentialité. Toutefois les problèmes de passage à l'échelle
demeurent. Résoudre ces problèmes revient à trouver un bon compromis entre les
complexités en communication, en space et en temps. Par dessus tout, obtenir une
complexité en communication sous-linéaire comparée au nombre de participants est
crucial afin de gérer les groupes de grande taille.

Afin d'augmenter la réactivité aux changements et la disponibilité des
documents, les éditeurs temps réel utilisent la réplication optimiste de
séquences -- les documents étant simplement des séquences de caractères. En tant
que tel, chaque utilisateur héberge une copie locale d'un document et effectue
ses modifications directement dessus. Les changements sont propagés aux autres
répliques où ils sont intégrés. Le système est correct si, et seulement si, les
répliques intégrant un même ensemble d'opérations convergent vers un état
équivalent, i.e., les utilisateurs lisent un même document.

Les algorithmes décentralisés appartenant aux transformés opérationnels (OT)
transportent un vecteur d'état ou de contexte dans le but de détecter les
opérations concurrentes. Malheureusement, ces vecteurs croissent linéairement en
comparaison du nombre de membres ayant jamais participé à l'édition du
document. Ainsi, ces approches sont efficaces pour les petits groupes
d'utilisateurs mais ne sont pas adaptées aux groupes de plus large dimension,
plus dynamiques, et sujet aux aléas du réseau -- notamment sur la latence.

Les structures de données répliqués sans résolution de conflits (\emph{CRDTs}),
contrairement aux approches OT, ne payent pas le prix de la detection de
concurrence entre les opérations. Toutefois, ils transportent des identifiants
uniques et immuables pour chaque opération propagée sur le réseau. La taille de
ces identifiants impacte directement le trafic généré, et donc le passage à
l'échelle de l'approche. Deux classes de CRDTs conçues pour les séquences
existent :

\begin{itemize}
\item Les CRDTs tels que WOOT transportent un identifiant de taille constante --
ce qui est optimal. Toutefois, les éléments supprimés sont simplement cachés à
l'utilisateur et impactent négativement les performances du système. Pour s'en
débarasser, l'exécution d'un protocole de ramasse-miète réparti est
nécessaire. Malheureusement, ceux-ci ne passent pas à l'échelle.

\item Les CRDTs tels que Logoot ne nécessitent pas de pierres tombales mais
transportent des identifiants dont la taille varie à la génération. En fonction
de la stratégie d'allocation, ces identifiants peuvent croitre linérairement
comparé au nombre d'insertions effectuées sur le document. Équilibrer la
structure revient à éxecuter un protocol de consensus qui ne passe pas à
l'échelle.
\end{itemize}

\TODO{Meow state problem ?}

La section~\ref{repl:sec:stateoftheart} présente l'état de l'art. Elle introduit
le schema de réplication pessimiste avant de détailler le schema qui lui est
préféré : la réplication optimiste. La section présente ensuite les différentes
approches pour la réplication de séquences. En particulier, le fonctionnement
des approches proposant des opérations commutatives est étudié. La
section~\ref{repl:sec:motivation} expose les défaillances de l'état de l'art. La
section~\ref{repl:sec:proposal} présente \LSEQ, une approche basée sur les
opérations commutative dont la complexité est bornée supérieurement par un
polylogarithme sur le nombre d'insertions effectuées dans le document. La
section~\ref{repl:sec:validation} valide \LSEQ au travers de simulations. La
section~\ref{repl:sec:conclusion} conclut ce chapitre.


%%% Local Variables:
%%% mode: plain-tex
%%% TeX-master: "../../paper"
%%% End:
