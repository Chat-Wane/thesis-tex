
\lettrine{L}es éditeurs web tels que \emph{Google Docs}~\cite{googledocs} ou
\emph{ShareLaTeX}\cite{sharelatex} ont fortement contribué à l'adoption de
l'édition collaborative par le public~\cite{mogan2010impact}. Des millions
d'utilisateurs partagent et rédigent leurs documents en temps réel directement
dans leur navigateur web. Cependant un serveur central appartenant à un
fournisseur de services sert d'intermédiaire à l'édition. Cela soulève des
problèmes en termes de confidentialité, de censure, et d'intelligence
économique~\cite{cherrueau2016composer, gellman2013us, pearson2011toward}. À
cela s'ajoutent des problèmes de passage à l'échelle, notamment en ce qui
concerne le nombre d'utilisateurs. Bien que les petits groupes de collaborateurs
soient la cible principale de ce genre d'application, certains événements de
plus ample dimension tels que les cours en ligne ouverts et massifs
(\emph{MOOC})~\cite{breslow2013studying} nécessitent de supporter de plus larges
groupes. Google Docs gère les groupes de grande taille. En revanche, seuls les
$50$ premiers collaborateurs ont le droit d'écrire ensemble, les suivants voient
leurs accès limités à la simple lecture du document. Selon nous, même si
seulement un petit ensemble parmi les millions de collaborateurs est réellement
en train d'écrire, n'importe quel participant de la session d'édition devrait
pouvoir lire et écrire lorsqu'il le souhaite.

Les éditeurs décentralisés autorisant l'édition en temps réel n'ont pas besoin
de serveurs intermédiaires. Par conséquent, ils résolvent les problèmes liés à
la confidentialité. Toutefois les problèmes de passage à l'échelle
demeurent. Résoudre ces problèmes revient à trouver un bon compromis entre les
complexités en communication, en espace et en temps. Par dessus tout, obtenir
une complexité en communication sous-linéaire comparée au nombre de participants
est crucial pour gérer les groupes de grande taille.

Afin d'augmenter la réactivité aux changements et la disponibilité des
documents, les éditeurs temps réel utilisent la réplication
optimiste~\cite{demers1987epidemic, ladin1992providing, saito2005optimistic,
sun1998achieving} de séquences -- les documents étant simplement des séquences
de caractères. En tant que tel, chaque éditeur héberge une copie locale d'un
document et effectue ses modifications directement dessus. Les changements sont
propagés aux autres répliques où ils sont intégrés. Le système est correct si,
et seulement si, les répliques intégrant un même ensemble d'opérations
convergent vers un état équivalent, i.e., les utilisateurs lisent un même
document~\cite{bailis2013eventual, shapiro2011conflict}.

Les algorithmes décentralisés appartenant aux transformées opérationnelles
(OT)~\cite{sun1998operational, sun2009contextbased} transportent un vecteur
d'état ou de contexte dans le but de détecter les opérations
concurrentes. Malheureusement, ces vecteurs croissent linéairement en
comparaison du nombre de membres ayant jamais participé à l'édition du
document. Ainsi, ces approches sont efficaces pour les petits groupes
d'utilisateurs mais ne sont pas adaptées aux groupes de plus large dimension,
plus dynamiques, et sujet aux aléas du réseau -- notamment sur la latence.

Les structures de données répliqués sans résolution de conflits
(\emph{CRDTs})~\cite{shapiro2011comprehensive, shapiro2011conflict,
burckhardt2014replicated}, contrairement aux approches OT, ne payent pas le prix
de la détection de concurrence entre les opérations. Toutefois, ils transportent
des identifiants uniques et immuables pour chaque opération propagée sur le
réseau. La taille de ces identifiants a un impact direct sur le trafic généré.
Deux classes de CRDTs conçues pour les séquences existent :

\paragraph{Pierre tombales~\cite{ahmed2011evaluating, attiya2016specification,
conway2014language, grishchenko2010deep, oster2006data, roh2011replicated,
weiss2007wooki, wu2010partial, yu2012stringwise}.} Les CRDTs tels que
WOOT~\cite{oster2006data} transportent un identifiant de taille constante -- ce
qui est optimal. Toutefois, les éléments supprimés sont simplement cachés à
l'utilisateur et impactent négativement les performances du système. Pour s'en
débarrasser, l'exécution d'un protocole de ramasse-miettes
réparti~\cite{abdullahi1998garbage} est nécessaire. Malheureusement, ceux-ci ne
passent pas à l'échelle.

\paragraph{Identifiants de taille variable~\cite{andre2013supporting,
 preguica2009commutative, weiss2009logoot}.} Les CRDTs tels que
Logoot~\cite{weiss2009logoot} ne nécessitent pas de pierres tombales mais
transportent des identifiants dont la taille varie à la génération. En fonction
de la stratégie d'allocation, ces identifiants peuvent croître linéairement
comparé au nombre d'insertions effectuées sur le document. Équilibrer la
structure revient à exécuter un protocole de consensus qui ne passe pas à
l'échelle~\cite{mostefaoui2015signature}.

\textbf{Afin d'éviter tout protocole additionnel de relocalisation des
identifiants, comment allouer ces identifiants pour que leur taille soit
directement sous-linéaire?} Ce chapitre présente
\LSEQ~\cite{nedelec2013concurrency, nedelec2013lseq}, une fonction d'allocation
d'identifiants dont la taille croît de manière polylogarithmique par rapport au
nombre d'insertions dans la séquence. À ce titre, elle permet d'éviter
l'utilisation de protocole de relocalisation. Par conséquent, cette structure
passe à l'échelle.

La section~\ref{repl:sec:stateoftheart} présente l'état de l'art. Elle introduit
le schéma de réplication pessimiste avant de détailler le schéma qui lui est
préféré : La réplication optimiste. La section présente ensuite les différentes
approches pour la réplication de séquences. En particulier, le fonctionnement
des approches proposant des opérations commutatives est étudié. La
section~\ref{repl:sec:motivations} expose les défaillances de l'état de
l'art. La section~\ref{repl:sec:proposal} présente \LSEQ, une approche basée sur
les opérations commutatives dont la complexité est sous-linéaire. La
section~\ref{repl:sec:complexity} s'attache à démontrer ces bornes ainsi que les
conditions sous lesquelles celles-ci s'appliquent. La
section~\ref{repl:sec:validation} valide \LSEQ au travers de simulations. La
section~\ref{repl:sec:conclusion} conclut ce chapitre.


%%% Local Variables:
%%% mode: plain-tex
%%% TeX-master: "../../paper"
%%% End:
