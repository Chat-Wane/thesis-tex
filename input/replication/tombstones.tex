

\section{Pierres tombales}

Les structures de données sans résolution de conflits
(cf. chapitre~\ref{repl:chap:crdts}) pour séquences évitent le coûteux processus
de detection de concurrence. Les approches de cette famille utilisant des
pierres tombales~\cite{ahmed2011evaluating, conway2014language,
  grishchenko2010deep, oster2006data, preguica2009commutative,
  roh2011replicated, weiss2007wooki, wu2010partial, yu2012stringwise} se
caractérisent par la manière dont les suppressions sont traitées. En effet, lors
de l'opération de suppression, ces approches marquent simplement les éléments
supprimés afin de les cacher à l'utilisateur. Bien que supprimées, ces pierres
tombales existent toujours dans la structure représentant la séquence et
continuent d'impacter sur les performances.

\begin{itemize}
\item [\textbf{WOOT~\cite{oster2006data} :}] le premier représentant historique
  des CRDTs pour séquences suivit par deux extensions
  \textbf{WOOTO~\cite{weiss2007wooki}} et
  \textbf{WOOTH~\cite{ahmed2011evaluating}}. Dans cette approche chaque
  identifiant référence les identifiants voisins à l'insertion. Rassemblés, les
  identifiants peuvent être ordonnés grâce à un diagramme de Hasse
  (\REF). Toutefois, cet ordonnancement requière des deux bornes adjacentes
  qu'elles soient déjà intégrées et toujours présentes, d'où les suppressions
  réelles impossibles.

  \begin{figure}
    \centering
    
\begin{tikzpicture}[scale=1.1]

\newcommand\X{ 40pt}
\newcommand\Y{ 30pt}

\draw[fill=white](0 * \X, 0 * \Y) node{$\vdash$}+(-5pt,-5pt)rectangle+(5pt,5pt);
\draw[fill=white](7 * \X, 0 * \Y) node{$\dashv$}+(-5pt,-5pt)rectangle+(5pt,5pt);

\draw[fill=white](1 * \X, 1 * \Y) node{\textbf{Q}}+(-5pt,-5pt)rectangle+(5pt,5pt);
\draw[fill=white](2 * \X, 1 * \Y) node{\textbf{W}}+(-5pt,-5pt)rectangle+(5pt,5pt);

\draw[fill=white](1 * \X, 0 * \Y) node{\textbf{A}}+(-5pt,-5pt)rectangle+(5pt,5pt);
\draw[fill=white, draw=darkblue](2 * \X, 0 * \Y)
node{\DARKBLUE{\textbf{Z}}}+(-5pt,-5pt)rectangle+(5pt,5pt);
\draw[fill=white, draw=blue](3 * \X, 0 * \Y)
node{\BLUE{\textbf{E}}}+(-5pt,-5pt)rectangle+(5pt,5pt);
\draw[fill=white](4 * \X, 0 * \Y) node{\textbf{R}}+(-5pt,-5pt)rectangle+(5pt,5pt);
\draw[fill=white](5 * \X, 0 * \Y) node{\textbf{T}}+(-5pt,-5pt)rectangle+(5pt,5pt);
\draw[fill=white](6 * \X, 0 * \Y) node{\textbf{Y}}+(-5pt,-5pt)rectangle+(5pt,5pt);

\draw[thick](-5+1*\X, -5+0*\Y)--(5+1*\X, 5+0*\Y);
\draw[thick](-5+1*\X, 5+0*\Y)--(5+1*\X, -5+0*\Y);

\draw[thick, color=darkblue](-5+2*\X, -5+0*\Y)--(5+2*\X, 5+0*\Y);
\draw[thick, color=darkblue](-5+2*\X, 5+0*\Y)--(5+2*\X, -5+0*\Y);

\draw[->](1*\X,-5+0*\Y)to[out=-45,in=-135](7*\X, -5+0*\Y);
\draw[->](2*\X,-5+0*\Y)to[out=-45,in=-135](7*\X, -5+0*\Y);
\draw[->](3*\X,-5+0*\Y)to[out=-45,in=-135](7*\X, -5+0*\Y);
\draw[->](4*\X,-5+0*\Y)to[out=-45,in=-135](7*\X, -5+0*\Y);
\draw[->](5*\X,-5+0*\Y)to[out=-45,in=-135](7*\X, -5+0*\Y);
\draw[->](6*\X,-5+0*\Y)to[out=-45,in=-135](7*\X, -5+0*\Y);

\draw[<-](5+0*\X, 0*\Y)--(-5+1*\X, 0*\Y);
\draw[<-, color=darkblue](5+1*\X, 0*\Y)--(-5+2*\X, 0*\Y);
\draw[<-, color=blue](5+2*\X, 0*\Y)--(-5+3*\X, 0*\Y);
\draw[<-](5+3*\X, 0*\Y)--(-5+4*\X, 0*\Y);
\draw[<-](5+4*\X, 0*\Y)--(-5+5*\X, 0*\Y);
\draw[<-](5+5*\X, 0*\Y)--(-5+6*\X, 0*\Y);

\draw[->](1*\X,5+1*\Y)to[out=50,in=130](3*\X, 5+0*\Y);
%\draw[->](2*\X,5+1*\Y)to[out=45,in=135](3*\X, 5+0*\Y);
\draw[->](5+2*\X, 1*\Y)--(3*\X, 5+0*\Y);

\draw[<-](0*\X, 5+0*\Y)--(-5+1*\X, 1*\Y);
\draw[<-](5+1*\X, 1*\Y)--(-5+2*\X, 1*\Y);


\end{tikzpicture}
    \caption{\label{repl:fig:wootexample}Le diagramme de Hasse du modèle WOOT
      représentant la séquence 'QWERTY'. Bien que supprimé, le caractère Z est
      indispensable au bon ordonnancement de la séquence.}
  \end{figure}

  La figure~\ref{repl:fig:wootexample} illustre la nécessité de conserver les
  pierres tombales. Elle montre le diagramme de Hasse généré lors du scenario
  suivant : Tout d'abord, un utilisateur écrit 'AZERTY'. Ensuite, les deux
  premiers caractères sont supprimés afin d'être remplacés par les caractères
  'QW'. La séquence finale est 'QWERTY'. Toutefois, les identifiants ne sont pas
  modifiables, et l'identifiant du caractère 'E' référence l'identifiant de 'Z',
  lui-même référençant l'identifiant de 'A'. Par conséquent, supprimer
  complètement les identifiants de 'A' et/ou de 'Z' revient à rendre
  l'identifiant de 'E' non positionnable, et tout ceux qui en dépendent par
  transitivité.

\item [\textbf{Causal tree~\cite{grishchenko2010deep} :}] caractérise
  explicitement les relations causales grâce à une réprésentation sous forme
  d'arbre. Ainsi, chaque opération est accompagnée de l'identifiant de la
  dernière opération observée. En parcourant l'arbre et en appliquant les
  opérations, la séquence peut être retrouvée. Toutefois, les identifiants sont
  des horloges vectorielles (\emph{vector clock}) dont la taille est prohibitive
  (\REF). De plus, il est nécessaire de conserver tous les noeuds de cet arbre
  causal au cas où une opération y ferait référence.
\item [\textbf{Partial persistent sequence~\cite{wu2010partial} :}] definit les
  identifiants dans l'ensemble des nombres rationnels auxquel est ajoutée une
  limite quant à leur précision. Hélas, cette limite contraint la taille du
  document maximale. \TODO{Moar?}. \TODO{Maybe falls in the other category.}
\item [\textbf{Replicated growable array~\cite{roh2011replicated} :}] représente
  la séquence sous forme de liste supportant les opérations concurrentes. Une
  table de hachage apporte un accès rapide aux éléments grâce à leurs
  identifiants. Les éléments incluent une référence au voisin qu'ils précèdent
  lors de leur insertion. Toutefois, pour ne jamais briser la chaine ainsi
  construite, les éléments supprimés restes présents indéfinitivement mais
  cachés de l'utilisateur. \TODO{Double check.}
\item [\textbf{String-wise~\cite{yu2012stringwise} :}] cible principalement les
  chaines de caractères pouvant être subdivisées lors d'opérations. Les
  identifiants référencent alors les chaines adjacentes à l'insertion ainsi que
  les autres éléments de la chaines si subdivision il y a. De la même manière
  que pour les approches précédentes, les références rendent les suppressions
  réelles impossibles.
\item [\textbf{DiCE~\cite{conway2014language} :}] concentre principalement ses
  efforts sur les garanties de \TODO{\emph{confluence}} de la séquence. Chaque
  identifiant référence le voisin qu'il précède à l'insertion. L'ordre des
  éléments est alors fonction de ces relations de positionnement relatif, et de
  causalité.
\end{itemize}

Bien que l'accent soit mit sur l'impossibilité de réellement supprimer les
éléments de séquences répliquées, toutes ces approches s'avèrent pratiques
lorsque l'historique des opérations doit être conservé. Par exemple, dans le
cadre de l'encyclopédie \emph{Wikipédia}, conserver toutes les modifications
éffectuées permet de recouvrer une version vierge de vandalisme; dans le cadre
de \emph{Git}, il permet de recouvrer une version du code potentiellement sans
erreurs etc. Malheureusement, ces structures grandissent au moins linéairement
comparativement au nombre d'opérations effectuées sur la séquence. Cela devient
paradoxalement problématique, par exemple, lors de vandalisme où même le contenu
indésirable est conservé indéfinitivement. Survient alors le frustrant constat
d'avoir à stocker un fichier dont le poid ne reflète pas le contenu visible. De
plus, les éléments cachés s'accumulent et dégradent \TODO{continuellement} les
performances du système.

Lorsque conserver l'historique ne constitue pas une contrainte, purger la
structure de données des éléments cachés constitue une solution potentielle aux
dégradations de performances :

\begin{itemize}
\item [\textbf{ramasse-miètes :}] permet de nettoyer une structure de données en
  vidant de la mémoire les objets qui ne sont plus accessibles par le
  programme. Le contexte \textbf{réparti} étend le cadre du programme en
  considérant le local et le distant. Ainsi, supprimer réellement un élément de
  la séquence revient à s'interroger : \emph{Est-ce que
    \begin{inparaenum}[(i)]
    \item toutes les répliques ont supprimé l'élément et
    \item est-ce que tous les éléments référençant l'élément supprimé ont été
      intégré localement?
    \end{inparaenum}} Cela va sans dire qu'il est difficile d'apporter une
  réponse à ces deux questions. D'autant plus lorsque les possesseurs de
  réplique ne sont pas perpétuellement joignables.
\item [\textbf{core-nebula :}] propose de contraindre la topologie réseau afin
  de rendre les \TODO{consensus} possibles. \TODO{moar}
\end{itemize}

La section suivante présente une famille de structures pour séquences dont le
bon fonctionnemement ne requière pas de pierres tombales (\TODO{Treedoc\ldots}).

%%% Local Variables:
%%% mode: latex
%%% TeX-master: "../../paper"
%%% End:
