\section{Réplication pessimiste}
\label{repl:sec:pessimistic}

L'objectif de la réplication pessimiste est simple. Il consiste à donner
l'illusion que la donnée manipulée par les utilisateurs est unique,
indépendemment du nombre de répliques réel. Grâce à la réplication pessimiste,
il est facile de raisonner sur les données car leurs spécifications sont proches
de celles proposées dans un \TODO{contexte sans réplications}.

Toutefois, chacune des modifications effectuée doit être validée. L'autorité
décisionnelle diffère en fonction des approches :
\begin{itemize}
\item [\textbf{autorité centrale~\cite{alsberg1976principle} :}] l'un des
  serveurs est désigné responsable d'une donnée. Ceux qui souhaitent modifier la
  donnée sont alors dans l'obligation de demander l'accès exclusif pendant la
  mise en place de cette modification. Dans l'intervalle, les autres répliques
  ne peuvent soumettre de modifications. Enfin, lorsque la modification est
  achevée, la main est rendue au serveur qui peut autoriser d'autres
  modifications. C'est le méchanisme de vérouillage (\emph{lock}).
\item [\textbf{quorum~\cite{gifford1979weighted} :}] les modifications sont
  soumises à un vote où un certains nombre de serveurs \TODO{doivent approuver
    ou non}. \TODO{Serialisation}.
\end{itemize}

La réplication pessimiste est possible lorsque le nombre de répliques est connu,
plutôt petit, et souvent accessible. Ces contraintes sont notamment
satisfaisable dans le \TODO{Nuage}. Les services proposés possèdent d'avantage
d'assurances quant aux résultats de la manipulation des données. Toutefois, de
telles guaranties ne sont pas toujours nécéssaire et leur coût élévée n'est
alors plus justifié. \TODO{La section suivante est.}

%%% Local Variables:
%%% mode: latex
%%% TeX-master: "../../paper"
%%% End:
