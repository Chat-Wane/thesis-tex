
\footnotesize
\algrenewcommand{\algorithmiccomment}[1]{\hskip2em$\rhd$ #1}

\newcommand{\comment}[1]{$\rhd$ #1}


\algblockdefx[initially]{initially}{endInitially}
[0] {\textbf{INITIALLY:}} 

\algblockdefx[local]{local}{endLocal}
[0] {\textbf{LOCAL UPDATE:}}

\algsetblockdefx[received]{received}{endReceived}
{65535}{}
[0] {\textbf{RECEIVED UPDATE:}}

\newcommand{\LINEFOR}[2]{%
  \algorithmicfor\ {#1}\ \algorithmicdo\ {#2} %
}

\newcommand{\LINEIFTHEN}[2]{%
  \algorithmicif\ {#1}\ \algorithmicthen\ {#2} %
}

\newcommand{\INDSTATE}[1][1]{\State\hspace{\algorithmicindent}}

\begin{algorithmic}[1]
  \Statex
  \initially
  \State $I \leftarrow \varnothing$; \hfill \comment{CRDT pour compteurs}
    \State $n$; \hfill \comment{identité du serveur}
    \State $c \leftarrow 0$; \hfill \comment{compteur local}
  \endInitially
  
  \local
    \Function{inc}{\ } \hfill \comment{incrémentation}
    \State $c \leftarrow c+1$;
    \State $I \leftarrow I \cup \langle n,\, c \rangle$;
    \State \textsc{broadcast}('inc', \DARKBLUE{$\langle n,\, c \rangle$});
    \EndFunction
    \Function{query}{\ }{$\,\rightarrow \mathbb{N}$} \hfill \comment{lecture}
    \State \Return $|I|$
    \EndFunction
  \endLocal

  \State
  \State
  \State
  
  \received
    \Function{onInc}{$id$}
    \State $I \leftarrow I  \DARKBLUE{\,\cup\, id}$;
    \EndFunction
\end{algorithmic}
