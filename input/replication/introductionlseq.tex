
\chapter{LSeq : une fonction d'allocation polylogarithmique}
\label{repl:chap:lseq}
\minitoc

\lettrine{L}Seq est une fonction d'allocation d'identifiants de taille variable
pour les structures de données sans résolution de conflits dédiées aux
séquences. \LSEQ est composée de trois éléments dont les comportements, pris
individuellement, n'arrivent pas à pourvoir des identifiants de taille
satisfaisante. En revanche, utilisés simultanément, les défaillances des uns
sont comblées par les forces des autres. En résulte une allocation
d'identifiants dont la borne supérieur sur la taille est polylogarithmique
comparé à la taille du document. L'idée générale de \LSEQ consiste à accepter la
perte de niveaux des chemins composants ses identifiants pour peu que les
allocations suivantes réparent les dommages.

Le premier composant est un structure d'arbre dont l'arité augmente avec sa
profondeur (cf. §\ref{repl:sec:exponentialtree}). L'intuition étant que si la
profondeur de l'arbre augmente, c'est que le document à grandit suffisamment
pour nécessiter un plus large champs d'identifiants. Mais au lieu d'ouvrir un
espace aussi large qu'auparavant, on l'agrandit afin que le document puisse se
contenter de ce champs d'identifiants plus longuement.

Le second composant comprend deux sous-fonctions d'allocation conçues pour gérer
des comportements d'édition opposés (cf. §\ref{repl:sec:suballocation}). Ainsi,
l'une est appropriée lorsque l'édition est principalement faite de gauche à
droite tandis que l'autre cible l'édition de droite à gauche. Elles seront
utilisées ensemble afin de pouvoir gérer la plupart des cas d'édition.

Le troisième composant est une fonction assignant à chaque profondeur de l'arbre
une sous-fonction d'allocation parmi celles disponibles
(cf. §\ref{repl:sec:allocationchoice}). Cette fonction de choix doit fournir des
réponses similaires quelle que soit la réplique.

%%% Local Variables:
%%% mode: latex
%%% TeX-master: "../../paper"
%%% End:
