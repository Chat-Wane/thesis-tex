
\chapter{Schémas de réplication}
\label{repl:chap:replication}

\minitoc

\lettrine{L}a maintenance de répliques sur des machines distantes les unes des
autres est un problème ancien. Dès 1975, les bases de données répliquées font
leur apparition~\cite{johnson1975maintenance} afin de résoudre
\begin{inparaenum}[(i)]
\item les problèmes de défaillances~\cite{alsberg1976principle}, i.e., le
  serveur possédant les données étant inaccessible, le client peut contacter
  serveur alternatif connu pour posséder les mêmes données afin de satisfaire sa
  requête;
\item les problèmes de rapidité d'accès, i.e., le client peut contacter un
  serveur dont la latence est la plus faible afin de satisfaire plus rapidement
  sa requête.
\end{inparaenum}

Hélas, avec la réplication, la synchronisation de répliques devient le coeur du
problème. Puisque la communication entre serveurs n'est pas instantanée, les
modifications effectuées sur les données prennent du temps à parvenir aux
répliques. Cela implique des problèmes de
\begin{inparaenum}[(i)]
\item fraîcheur de données -- \emph{est-ce que la donnée que j'obtiens est la
    plus à jour?} -- et de
\item modifications concurrentes -- \emph{avec des modifications effectuées sur
    une même données, au même moment, par deux serveurs distants dont les
    résultats sont différents. Dois-je conserver les deux modifications, ou
    dois-je en privilégier une, ou dois-je employer une autre stratégie?}
\end{inparaenum}


Hélas, d'après le théorème CAP~\cite{gilbert2002brewer} il est impossible de de
répliquer sur un grand nombre de serveurs tout en garantissant à la fois :
\begin{itemize}
\item [\textbf{Cohérence :}] un contrat entre le développeur et la structure qui
  spécifie comment cette dernière se comporte suivant les opérations effectuées
  et leur ordonancement.
\item [\textbf{Disponibilité :}] le ratio entre le temps effectif durant lequel
  l'utilisateur accède à un service et le temps durant lequel il souhaite y
  accèder.
\item [\textbf{Tolérance aux pannes :}] les défaillances de serveurs
  n'entrainent pas une panne générale du système.
\end{itemize}

Face à ce constat, deux grandes familles de réplication : la réplication
pessimiste cherche à donner l'illusion d'une donnée unique lorsque la
réplication optimiste autorise à ses répliques de légères divergences
temporaires (cf. §\ref{repl:sec:schemas}). Cette dernière passant plus volontier
à l'échelle, nous nous intéresserons à deux familles d'approches y appartenant,
à savoir les transformés opérationnels dont les opérations sont modifiées à
l'intégration afin d'adapter l'opération au contexte d'exécution, et les
structures de données sans résolution de conflits dont les opérations commutent
(cf. §\ref{repl:sec:otorcrdts}). Finalement, la section~\ref{repl:sec:sequences}
s'intéresse plus particulièrement au type séquence, le plus proche d'un
document.

%%% Local Variables:
%%% mode: latex
%%% TeX-master: "../../paper"
%%% End:
