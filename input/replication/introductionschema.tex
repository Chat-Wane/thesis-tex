
\chapter{Schémas de réplication}
\label{repl:chap:replication}

\minitoc

\lettrine{L}a maintenance de répliques sur des machines distantes les unes des
autres est un problème ancien. Dès 1975, les bases de données répliquées font
leur apparition~\cite{johnson1975maintenance} afin de résoudre
\begin{inparaenum}[(i)]
\item les problèmes de défaillances~\cite{alsberg1976principle}, i.e., le
  serveur possédant les données étant inaccessible, le client peut contacter
  serveur alternatif connu pour posséder les mêmes données afin de satisfaire sa
  requête;
\item les problèmes de rapidité d'accès, i.e., le client peut contacter un
  serveur dont la latence est la plus faible afin de satisfaire plus rapidement
  sa requête.
\end{inparaenum}

Hélas, avec la réplication, la synchronisation de répliques devient le coeur du
problème. Puisque la communication entre serveurs n'est pas instantanée, les
modifications effectuées sur les données prennent du temps à parvenir aux
répliques. Cela implique des problèmes de
\begin{inparaenum}[(i)]
\item fraîcheur de données -- \emph{est-ce que la donnée que j'obtiens est la
    plus à jour?} -- et de
\item modifications concurrentes -- \emph{avec des modifications effectuées sur
    une même données, au même moment, par deux serveurs distants dont les
    résultats sont différents. Dois-je conserver les deux modifications, ou
    dois-je en privilégier une, ou dois-je employer une autre stratégie?}
\end{inparaenum}

D'après le bien connu théorème CAP~\cite{gilbert2002brewer} (\emph{Consistency,
  Availability, Partition tolerence}), il est impossible de \TODO{passer à
  l'échelle} tout en garantissant à la fois \TODO{
\begin{itemize}
\item la cohérence : un contrat entre le développeur et la structure qui
  spécifie comment cette dernière se comporte suivant les opérations effectuées
  et leur ordonancement. Les contraintes imposées à la structure peuvent être
  plus ou moins importantes selon les besoins.
\item la disponibilité : le ratio entre le temps effectif durant lequel
  l'utilisateur accède à un service et le temps durant lequel il souhaite y
  accèder. Dans le meilleur cas, le service est toujours disponible. \TODO{La
    plupart des services \emph{Cloud} proposent de 99 à 100\% (exclus) de
    disponibilité.}
\item la tolérance aux pannes : \TODO{les défaillances n'entrainent pas de defauts
  dans les propriétés susmentionnées}.
\end{itemize}
}

Face à ce constat, deux familles de réplication existent proposant différents
compromis. La section~\ref{repl:sec:pessimistic} décrit succintement les
approches dîtes pessimistes dont le critère de cohérence est fort mais ne
pouvant s'appliquer à grande échelle. À l'opposé, la
section~\ref{repl:sec:optimistic} décrit la réplication optimiste dont le
critère de cohérence est plus faible mais pouvant atteindre une plus grande
échelle.

%%% Local Variables:
%%% mode: latex
%%% TeX-master: "../../paper"
%%% End:
