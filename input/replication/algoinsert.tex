
\footnotesize
\algrenewcommand{\algorithmiccomment}[1]{\hskip2em$\rhd$ #1}

\newcommand{\comment}[1]{$\rhd$ #1}

\newcommand{\LINEFOR}[2]{%
  \algorithmicfor\ {#1}\ \algorithmicdo\ {#2} %
}

\newcommand{\LINEIFTHEN}[2]{%
  \algorithmicif\ {#1}\ \algorithmicthen\ {#2} %
}

\newcommand{\LINEIFTHENELSE}[3]{%
  \algorithmicif\ {#1}\ \algorithmicthen\ {#2} \algorithmicelse\ {#3} %
}

\newcommand{\LET}[2]{%
  \State \textbf{let}\ $#1 \leftarrow #2$%
}

\newcommand{\PIPE}[0]{%
  \,|\,%
}

\newcommand{\INDSTATE}[1][1]{\State\hspace{\algorithmicindent}}

\begin{algorithmic}[1]
  \Function{subInsert}
  {$t \in \mathcal{T},\, indexes \in \mathbb{N}^+,\, i\in\mathcal{I}$}
  {$\, \rightarrow \mathcal{T}$}
  \LET {[index \PIPE indexes]}{indexes};
  \LET {[path \PIPE i]}{i};
  \LET {\langle \_,\, \_,\, count,\, children\rangle}{t};
  \If {$(index<0)$}
  \State \TODO{TODO} 
  \Else
  \EndIf
  \EndFunction

  \Statex

  \Function{insert}
  {$t \in \mathcal{T},\, i \in \mathcal{I}$}
  {$\, \rightarrow \mathcal{T}$}
  \State \Return $\textsc{subInsert}(t,\,\textsc{getIndexesOf}(t,\, i),\,i)$;
  \EndFunction

\end{algorithmic}
