
\chapter{Structure de données sans résolution de conflits}
\label{repl:chap:crdts}

\minitoc

\lettrine{L}es structures de données sans résolution de
conflits~\cite{shapiro2011comprehensive} (CRDTs) appartiennent au schéma de
réplication optimiste. \TODO{Ils tirent leur nom de}.  Il en existe deux
familles équivalentes mais proposant un compromis différent :
\begin{itemize}
\item [\textbf{basée sur l'état :}] lors d'une opération, l'état local change et
  est envoyé en totalité aux autres répliques qui fusionnent alors l'état réçu
  et leur état propre. L'envoit d'un état est honéreux et doit être effectué
  avec parcimonie. En revanche, puisqu'il est autonome
  (\TODO{\emph{self-contained}}), il ne requière aucune garantie sur les moyens
  de diffusion.
\item [\textbf{basée sur les opérations :}] lors d'une opération, son résultat
  seul est envoyé aux autres répliques où il est intégré. Les résultats sont
  envoyées les uns après les autres aux cours des opérations ce qui est beaucoup
  moins coûteux que l'état complet. En revanche, cela requière une diffusion
  fiable, i.e., toutes les opérations doivent être inéluctablement reçues par
  toutes les répliques.
\end{itemize}

Dans le reste de ce manuscrit de thèse, nous nous intéresserons plus en détail à
cette seconde famille de structure répliquée. La
section~\ref{crdts:sec:properties} présente les propriétés de ces structures de
données répliquées. La section~\ref{crdts:sec:compteur} décrit les structures
possibles pour le compteur. La section~\ref{crdts:sec:set} décrit les structures
possibles pour les ensembles. Les problèmes de composition de ces structures
sont exposés en section~\ref{crdts:sec:composition}. La
section~\ref{crdts:sec:conclusion} conclue ce chapitre.


%%% Local Variables:
%%% mode: latex
%%% TeX-master: "../../paper"
%%% End:
