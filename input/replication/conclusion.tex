
\section{Conclusion}
\label{repl:sec:conclusion}

Dans ce chapitre, nous avons présenté \LSEQ, un fonction d'allocation
d'identifiants pour les structures de données répliquées sans résolution de
conflits conçues pour les séquences et n'utilisant pas de pierres tombales.

L'utilisation d'un arbre exponentiel en tant que structure permet d'améliorer
l'allocation sur des comportements d'édition au prix d'un pire cas plus
onéreux. Ce pire cas est rendu difficile à atteindre grâce à l'utilisation de
deux sous-fonctions d'allocation conçues pour les comportements d'édition
monotone. Si malgré tout un utilisateur malintentionné tente d'obtenir de tels
identifiants, la différence entre les identifiants attendus et les identifiants
incriminés est telle qu'il est facile de confondre le coupable, et donc, de
prendre des mesures à son encontre. La complexité des identifiants \LSEQ est
bornée par un polylogarithme comparativement au nombre d'insertions effectuées
sur la séquence. À ce titre, elle constitue un candidat idéal pour l'édition
collaborative en temps réel.

Tout comme les approches basées sur le maintient de répliques distantes, \LSEQ a
besoin d'un moyen de communication entre les répliques. En effet, les
identifiants générés et supprimés par \LSEQ doivent parvenir à toutes les
répliques pour que celles-ci convergent vers un état équivalent.  Le chapitre
suivant présente un protocole construisant un réseau superposé : chaque serveur
détenant une réplique connaît un ensemble de serveurs détenant également une
telle réplique et peut communiquer avec ceux-ci. La spécificité du protocole
présenté est qu'il maintient ces ensembles de telle sorte que le cardinal de
chaque ensemble croît et décroît logarithmiquement par rapport à la taille du
réseau, et ce, sans connaissance globale. Grâce à ce protocole, un éditeur
collaboratif est en mesure de disséminer efficacement les changements effectués
sur le document à tous les autres éditeurs collaboratifs impliqués dans la
rédaction (cf. §\ref{editor:chap:crate}).

% Le chapitre~\ref{editor:chap:crate} décrit un éditeur de texte collaboratif
% temps réel. Dans ce contexte, chaque identifiant généré doit être envoyé au
% reste des participants à la session d'édition. Maintenir une croissance des
% identifiants sous-linéaire permet de passer à l'échelle. Par conséquent, la
% structure ne nécessite pas de relocalisation d'identifiants s'avérant hors de
% prix.

