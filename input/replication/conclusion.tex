
\section{Conclusion}
\label{repl:sec:conclusion}

Dans ce chapitre, nous avons présenté \LSEQ, un fonction d'allocation
d'identifiants pour les structures de données sans résolution de conflits
conçues pour les séquences. 

\begin{itemize}
\item [\textbf{QR B.}] \textbf{Afin d'éviter tout protocole additionnel de
    relocalisation des identifiants, comment allouer ces identifiants pour que
    leur taille soit directement sous-linéaire.}
\end{itemize}

L'utilisation d'un arbre exponentiel en tant que structure permet d'améliorer
l'allocation sur des comportements d'édition au prix d'un pire cas plus
honéreux. Ce pire cas est rendu difficile à atteindre grâce à l'utilisation de
deux sous-fonctions d'allocation conçues pour les comportement d'édition
monotone. Si malgré tout un utilisateur malintentionné tente d'obtenir de tels
identifiants, la différence entre les identifiants attendus et les identifiants
incriminés est telle qu'il est facile de confondre le coupable, et de prendre
les mesures correspondantes. La complexité des identifiants \LSEQ est bornée par
un polylogarithme comparativement au nombre d'insertions effectuées sur la
séquence.

Le chapitre~\ref{editor:chap:crate} décrit un éditeur de texte collaboratif
temps réel. Dans ce contexte, chaque identifiant généré doit être envoyé au
reste des participants à la session d'édition. Maintenir une croissance des
identifiants sous-linéaire permet de passer à l'échelle. Par conséquent, la
structure ne nécessite pas de relocalisation d'identifiants s'avérant hors de
prix.

