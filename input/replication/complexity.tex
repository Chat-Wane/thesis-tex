
\section{Analyse en complexité}
\label{repl:sec:complexity}

Dans l'édition de texte, la plupart des comportements d'édition peuvent
empiriquement être résumés à la composition de deux comportements basiques.
\begin{inparaenum}
\item Le comportement d'édition aléatoire où l'auteur insert de nouveaux
  éléments à ce qui semble être des positions aléatoires dans la séquence. Par
  exemple, ce comportement peut apparaitre lors de corrections orthographiques,
  e.g. l'auteur écrit \texttt{QWETY} et réalise qu'il lui manque le
  \texttt{R}. Il l'ajoute donc dans un second temps;
\item Le comportement d'édition monotone où l'auteur insert de nouveaux éléments
  les uns à la suite des autres (avant ou après l'élément le plus récemment
  inséré). Par exemple, lorsqu'un auteur écrit \texttt{QWERTY}, il commence
  généralement par le caractère \texttt{Q} jusqu'à arriver au caractère
  \texttt{Y}. À l'opposé, les insertions dans un historique se font en tête pour
  des raisons pratiques. Ces comportements sont respectivements notés l'édition
  de droite à gauche, et l'édition de gauche à droite.
\end{inparaenum}

Cette section concentre l'analyse en complexité de \LSEQ sur les comportements
d'édition susmentionnés auxquels est ajouté une analyse en pire cas. Il est
important de noter que le comportement d'édition monotone constitue un cas
défavorable puisqu'il tend à déséquilibrer l'arbre stockant la séquence
répliquée. Aussitôt que les participants commencent à éditer à différents
endroits -- ce qui est plus proche d'un cas d'utilisation réel -- l'arbre
commence à se rééquilibrer et les opérations deviennent plus efficaces
proportionnellement. L'analyse ne fait pas mention de cas moyen puisqu'il
nécessite de connaître la distribution des positions d'insertions effectuées par
des être humains ce qui représente une tâche complexe.

\subsection{Complexité spatiale}

La complexité spatiale d'un seul identifiant est différenciée de la complexité
spatiale de l'arbre stockant la séquence. La première est importante puisqu'elle
influence directement la complexité en communication. En effet, chaque
identifiant est propagé à tous les possesseurs de réplique; La seconde est
importante puisqu'elle représente le mémoire allouée par chaque participant.

La section~\ref{repl:sec:proposal} spécifie que chaque nouvelle concaténation
dans le chemin de l'identifiant requière un bit additionnel pour l'encoder. Cela
représente au total $\mathcal{O}(e^2)$ bits pour encoder le chemin, où $e$ est
la profondeur du chemin dans l'arbre. Heureusement, cette profondeur est bornée
selon le comportement d'édition remplissant l'arbre exponentiel. 


\begin{figure}
  \begin{center}
    \subfloat[Arbre \LSEQ rempli par des éditions aléatoires.]
    [\label{repl:fig:randomlseqtree}Arbre \LSEQ rempli par des éditions aléatoires.]
    {
\begin{tikzpicture}[scale=1.2]

\newcommand\X{30pt}
\newcommand\Y{30pt}

\end{tikzpicture}}
    \hspace{10pt}
    \subfloat[Arbre \LSEQ rempli par des éditions monotones.]
    [\label{repl:fig:monotoniclseqtree}Arbre \LSEQ rempli par des éditions monotones.]
    {\input{input/replication/figmonotoniclseqtree.tex}}
    \hspace{10pt}
    \subfloat[Arbre \LSEQ rempli dans le pire cas.]
    [\label{repl:fig:worstlseqtree}Arbre \LSEQ rempli dans le pire cas.]
    {\input{input/replication/figworstlseqtree.tex}}
  \end{center}
\end{figure}


\paragraph{Comportement d'édition aléatoire. } Il remplit l'arbre à des
positions aléatoires. En conséquence, l'arbre reste équilibré
(cf. figure~\ref{repl:fig:randomlseqtree}). Étant exponentiel, l'arbre peut
stocker $\textstyle\sum\nolimits_{i=1}^{k}{2^{(i^2-i)/2}}$ éléments, où $k$ est
la profondeur de l'arbre. Ainsi, la borne supérieur sur la profondeur des
chemins est $\mathcal{O}(\sqrt{\log I})$ concaténations, où $I$ est le nombre
d'insertions. Puisque les chemins nécessitent $\mathcal{O}(e^2)$ bits pour être
encodés, les chemins ont une complexité spatiale optimale de
$\mathcal{O}(\log I)$. Ce résultat s'applique à toutes les approches utilisant
des identifiants de taille variable~\cite{preguica2009commutative,
  weiss2009logoot}. Toutefois, puisque \LSEQ utilise une structure d'arbre
factorisant les parties communes des identifiants, la complexité spatiale totale
n'est pas la somme des identifiants, mais seulement $\mathcal{O}(I\log I)$.

\paragraph{Comportement d'édition monotone.} Il remplit seulement une branche de
l'arbre (cf. figure~\ref{repl:fig:monotoniclseqtree}). Pourtant, comme l'arité
maximale d'un nœud augmente lorsque sa profondeur augmente, la croissance de
l'arbre ralentit au cours des insertions. Dans ce cas, l'arbre stocke jusqu'à
$2^{e+1}-1$ éléments, où $e$ est la profondeur de l'arbre. Ainsi, le nombre de
concaténation composant le chemin est $\mathcal{O}(\log I)$, où $I$ est le
nombre d'insertions. La représentation binaire des chemins grandissant de façon
quadratique, la complexité spatiale des identifiants est
$\mathcal{O}((\log I)^2)$. Globalement, la complexité spatiale de l'arbre reste
la même, à savoir $\mathcal{O}(I \log I)$.

\paragraph{Pire cas.} Chaque insertion augmente la profondeur de l'arbre
(cf. figure~\ref{repl:fig:worstlseqtree}). Ainsi, après $I$ insertions, l'arbre
comprend $e$ éléments, où $e$ est la profondeur de l'arbre. Chaque nouveau
chemin possède tous les autres chemins. Ainsi, la complexité spatiale des
identifiants est $\mathcal{O}(I^2)$ et la complexité spatiale de l'arbre est
elle aussi $\mathcal{O}(I^2)$.

\begin{table}
  \begin{center}
    \caption{\label{repl:table:lseqspace}
      Bornes supérieures de la complexité spatiale de \LSEQ, Logoot, et Treedoc. Où
      $I$ est le nombre d'insertions effectuées sur la séquence.}
    
\begin{tabularx}{0.9\textwidth}{@{}Xcccc@{}}
  \toprule
  \textsc{Editing behavior} & \multicolumn{2}{c}{\textsc{LSeq Space}} & \multicolumn{2}{c}{\textsc{Logoot} / \textsc{Treedoc Space}} \\ \cmidrule{2-3} \cmidrule{4-5} 
                   & \textsc{Identifier} & \textsc{Sequence} & \textsc{Identifier} & \textsc{Sequence} \\ \midrule
  Random editing & $\mathcal{O}(\log I)$ & $\mathcal{O}(I\log I)$ & $\mathcal{O}(\log I)$ & $\mathcal{O}(I)$\\
  Monotonic editing & $\mathcal{O}((\log I)^2)$ & $\mathcal{O}(I \log I)$ & $\mathcal{O}(I)$ & $\mathcal{O}(I)$ \\
  Worst case & $\mathcal{O}(I^2)$ & $\mathcal{O}(I^2)$ & $\mathcal{O}(I)$ & $\mathcal{O}(I)$  \\ \bottomrule
\end{tabularx}

  \end{center}
\end{table}

\paragraph{Résumé.} Le tableau~\ref{repl:table:lseqspace} résume les complexités
spatiales de \LSEQ. En particulier, la croissance des identifiants est attendue
entre un logarithme et un polylogarithme. La croissance est donc bien
sous-linéaire comparé au nombre d'insertions effectuées sur la séquence. Pour
atteindre cette amélioration, \LSEQ sacrifie son pire cas en le rendant
quadratique. Toutefois, ce pire cas est rendu difficile à obtenir. En effet, les
deux sous-fonctions d'allocation aux buts antagonistes résolvent leurs
limitations respectives. De plus, si un utilisateur malintentionné tente de
produire ce pire cas, la différence entre la croissance attendue et la
croissance observée serait telle que le coupable serait aisément
identifiable. Des mesures pourraient alors être prises. Le
tableau~\ref{repl:table:lseqspace} montre aussi que, comparé à l'état de l'art,
\LSEQ améliore grandement la taille des identifiants mais voit sa taille globale
légèrement dégradée, i.e., de $\mathcal{O}(I)$ à $\mathcal{O}(I\log I)$. Le
compromis reste avantageux car les communications héritent de l'amélioration sur
les identifiants.

\subsection{Complexité temporelle}

L'analyse en complexité temporelle fournit des indices sur l'évolution des
performances de chaque opération au cours des insertions. Ces opérations sont
découpées entre leur exécution locale et leur exécution distante. Tout comme
pour la complexité spatiale, l'analyse se porte sur les trois comportements
d'édition suivant : aléatoire, monotone, et pire cas.

\paragraph{Insertion locale.} Cette opération consiste simplement à construire
un nouveau chemin en fonction des chemins adjacents. Par conséquent, la
complexité dépend de la taille de ces chemins qui grandissent en fonction du
comportement d'édition. Les comportements aléatoire, monotone, et pire cas
conduisent respectivement à une croissance de l'arbre en
$\mathcal{O}(\sqrt{\log I})$, $\mathcal{O}(\log I)$, et $\mathcal{O}(I)$. La
complexité temporelle de la partie locale de l'insertion suit donc ces
même complexités.

\paragraph{Suppression locale.} L'opération consiste à propager l'identifiant à
supprimer à tous les participants. L'opération se fait donc en temps constant
$\mathcal{O}(1)$ quel que soit le comportement d'édition.

\paragraph{Insertion et suppression distante.} Ces deux opérations comportent
les mêmes instructions pour intégrer le résultat reçu de l'opération locale. Par
conséquent, leur complexité temporelle est identique. Encore une fois, les
comportements d'édition aléatoire, monotone, et pire cas conduisent
respectivement à une croissance de l'arbre exponentiel de l'ordre de
$\mathcal{O}(\sqrt{\log I})$, $\mathcal{O}(\log I)$, et $\mathcal{O}(I)$.  Les
fils de chaque nœud de l'arbre sont ordonnés selon leur chemin. Ainsi, une
recherche dichotomique récursive permet d'atteindre la feuille de l'arbre
recherchée. La suppression distante supprime les nœuds qu'elle traverse si ce
nœud ne possède qu'un seul élément parmi ses sous-arbres. La complexité
temporelle de la recherche dichotomique dépend du niveau $l$ à laquelle elle est
effectuée : $\mathcal{O}(\log 2^l)$. Les recherches à répétition conduisent à
une borne supérieure en
$\mathcal{O}(\textstyle\sum\nolimits_{i=1}^{e}(\log 2^i))$ où $e$e est la
profondeur de l'arbre. En remplaçant la variable $e$, les bornes supérieures sur
la complexité temporelle des opérations deviennent
$\mathcal{O}(\log I + \sqrt{\log I})$ pour le comportement aléatoire,
$\mathcal{O}((\log I)^2 + \log I)$ pour le comportement monotone, et
$\mathcal{O}(I)$ dans le pire cas. Pour ce dernier, cela signifie qu'une
comparaison par élément est nécessaire afin d'atteindre la feuille la plus
profonde.

\begin{table}
  \begin{center}
    \caption{\label{repl:table:lseqtime}
      Bornes supérieures de la complexité temporelle de \LSEQ. Où
      $I$ est le nombre d'insertions effectuées sur la séquence.}
    
\begin{tabularx}{0.75\textwidth}{@{}Xcccc@{}}
  \toprule
  \textsc{Editing Behavior} & \multicolumn{4}{c}{\textsc{Time}} \\
    & \multicolumn{2}{c}{\textsc{Local}} & \multicolumn{2}{c}{\textsc{Remote}} \\ \cmidrule(r){2-3} \cmidrule(l){4-5}
  & \textsc{ins} & \textsc{del} & \multicolumn{2}{c}{\textsc{ins} / \textsc{del}}  \\ \midrule
  Random editing & $\mathcal{O}(\sqrt{\log I})$ & $\mathcal{O}(1)$ & \multicolumn{2}{c}{$\mathcal{O}(\log I + \sqrt{\log I})$} \\
  Monotonic editing & $\mathcal{O}(\log I)$ & $\mathcal{O}(1)$ & \multicolumn{2}{c}{$\mathcal{O}((\log I)^2 +\log I)$} \\
  Worst case & $\mathcal{O}(I)$ & $\mathcal{O}(1)$ & \multicolumn{2}{c}{$\mathcal{O}(I)$} \\ \bottomrule
\end{tabularx}


  \end{center}
\end{table}


\paragraph{Interface arbre -- séquence.} La structure de données répliquée est
un arbre mais l'objet manipulé par l'utilisateur est une séquence. Il est donc
nécessaire de fournir les fonctionnalités d'accès élémentaires afin de faire le
lien entre l'arbre et la séquence.  La fonction \textsc{get} permet d'obtenir
l'identifiant à une position donnée dans la séquence.

% \subsubsection{get}
% \label{repl:subsec:get}

% \begin{wrapfigure}{r}{0.6\textwidth}
%   \vspace{-35pt} %% (ugly)
%   \begin{minipage}[t]{0.6\textwidth}
%     \begin{algorithm}[H]
%       \input{input/replication/algoget.tex}
%       \caption{\label{repl:algo:get} Get.}
%     \end{algorithm}
%   \end{minipage}
%   \vspace{-15pt}
% \end{wrapfigure}

% L'algorithme~\ref{repl:algo:get} présente les instructions de la fonction
% \textsc{get} permettant d'obtenir l'identifiant présent à une position donnée
% dans la séquence. Celle-ci est nécéssaire lorsque un élément est localement
% inséré ou supprimé. En particulier, elle permet de retrouver les identifiants
% des éléments adjacents à la position d'insertion. Ces derniers seront ensuite
% utilisés afin d'allouer un nouvel identifiant unique.

Afin de retrouver l'identifiant à la position $index$ de la séquence, l'arbre
est parcouru en largeur, de gauche à droite ou de droite à gauche selon la borne
la plus proche à chaque niveau.  Puisque chacun des noeuds sauvegarde le nombre
total de fils qu'il possède, il n'est pas nécessaire de parcourir l'ensemble des
feuilles de l'arbre précédant/suivant la position recherchée. Toutefois, selon
la fonction de parcours utilisée pour retrouver la position et la séquence
d'édition ayant rempli la structure, les performances peuvent varier.

\begin{figure}
  \centering
  \input{input/replication/figgetexample.tex}
  \caption{\label{repl:fig:getexample} Exemple de parcours effectué lors de la
    recherche de l'identifiant en position 5 correspondant au caractère
    \texttt{Y}.}
\end{figure}

\TODO{Rework.}
La figure~\ref{repl:fig:getexample} montre un exemple de recherche d'identifiant
dans un arbre représentant la séquence \texttt{QWERTY}. La recherche concerne la
position $5$ correspondant au caractère \texttt{Y}. L'arbre est parcouru en
partant du fils gauche de la racine. Ce fils sait que ses sous-branches
contiennent 3 caractères. L'algorithme en déduit que le prochain fils de la
racine est à la position 4. En examinant ce second fils, l'algorithme déduit que
l'identifiant recherché est inclue dans l'un de ses fils. Ainsi, il continue
l'exploration dans ce sous-arbre. Lorsqu'il examine le noeud contenant
\texttt{Y}, il sait que celui-ci est en position $5$. Par conséquent, il
retourne le chemin qu'il a empreinté pour y parvenir, à savoir [5.4]. Nous
pouvons observer dans cet exemple que si la recherche était parti de la droite,
elle n'aurait eu à parcourir qu'un seul élément intermédiaire (contenant le
caractère \texttt{T}), et aurait donc été plus efficace.

\begin{table}
  \begin{center}
    \caption{\label{repl:table:lseqlookup} Bornes supérieures de la complexité
      temporelle de l'opération \textsc{get} de \LSEQ. Où $I$ est le nombre
      d'insertions effectuées sur la séquence.}
    
\begin{tabularx}{0.6\textwidth}{@{}Xc@{}}
  \toprule
  \textsc{Editing behavior} & \textsc{Time} \\
  & \ \ \ \ \ \ \ \ \ \textsc{lookup} \ \ \ \ \ \ \ \ \ \\ \midrule
  Random editing & $\mathcal{O}(2^{\sqrt{\log I}})$ \\
  Monotonic / Worst case editing & $\mathcal{O}(I)$ \\ \bottomrule
%%  Worst case & $\mathcal{O}(I)$ \\ \bottomrule
\end{tabularx}


  \end{center}
\end{table}

La complexité temporelle de cet algorithme dépend de la séquence d'édition ayant
permit de remplir l'arbre. Ainsi, lorsque l'arbre est rempli aléatoirement, sa
profondeur est faible et un grand nombre d'éléments peuvent être passés. En
revanche, lorsque le comportement d'édition ayant rempli l'arbre est monotone,
une branche de l'arbre seulement est fortement remplie. L'algorithme doit la
parcourir afin de trouver l'identifiant à la position désirée. La complexité est
alors linéaire et correspond au pire cas. Cela s'avère problématique puisque
celle-ci est appelée deux fois par insertion locale. Heureusement, lorsque le
comportement d'édition est monotone, il est facile de se souvenir des bornes
d'insertion : l'une ne change jamais, et l'autre correspond à l'identifiant
nouvellement inséré. Si le comportement d'édition change, l'arbre commence à se
rééquilibrer et les performances se rapprochent de celles de l'édition
aléatoire.


% \subsubsection{indexOf}
% \label{repl:subsec:indexof}

% \begin{wrapfigure}{r}{0.6\textwidth}
%   \vspace{-35pt} %% (ugly)
%   \begin{minipage}[t]{0.6\textwidth}
%     \begin{algorithm}[H]
%       \input{input/replication/algoindexof.tex}
%       \caption{\label{repl:algo:indexof} indexOf.}
%     \end{algorithm}
%   \end{minipage}
%   \vspace{-15pt}
% \end{wrapfigure}

% L'algorithme~\ref{repl:algo:indexof} présente les instructions de la fonction
% \textsc{indexOf}. Celle-ci sert essentiellement à placer les éléments réçus dans
% la séquence perçue par l'utilisateur.  Celle-ci est divisée en trois :
% \begin{inparaenum}[(i)]
% \item La fonction \textsc{getIndexesOf} retourne les indices successifs dans les
%   listes triées constituants les fils des nœuds de l'arbre. Celle-ci utilise la
%   fonction \textsc{binaryIndexOf} possédant un complexité temporelle
%   logarithmique comparée à la taille du tableau trié parcouru.
% \item La fonction \textsc{getSum} parcours l'arbre afin d'en déduire l'indice de
%   séquence correspondant. Afin de ne pas avoir à parcourir l'entièreté des
%   branches, les noeuds sauvegardent le nombre de sous-branches qu'ils possèdent
%   (similairement à l'algorithme~\ref{repl:algo:get}).  La
%   ligne~\ref{repl:line:optimization} définit la façon dont l'arbre est
%   parcouru. Un optimisation consiste à parcourir l'arbre à partir de la borne la
%   plus proche de l'indice à la profondeur concernée.
% \item La fonction \textsc{indexOf} est composée des deux fonctions
%   susmentionnées.
% \end{inparaenum}

% \TODO{Figure explaining the operation.}

% La complexité temporelle de cette fonction est proche de celle de la fonction
% \textsc{get} car basée sur le même principe. La seule différence est le surcoût
% logarithmique lié à la recherche dichotomique à chaque niveau de l'arbre. Encore
% une fois, les complexités linéaires ne sont pas problématiques car ces
% opérations peuvent être effectuées en arrière-plan. De plus, l'utilisateur
% possède une vue partielle du document. Si les mises à jour ne tombent pas dans
% cette fenêtre, la fonction peut arrêter son éxecution.


% \subsection{Modification de la structure}

% La structure d'arbre représentant le document réparti se modifie au rythme des
% insertions et suppressions effectuées par les utilisateurs, qu'ils soient
% distants ou non. 

% \subsubsection{insert}


% \begin{wrapfigure}{r}{0.6\textwidth}
%   \vspace{-35pt} %% (ugly)
%   \begin{minipage}[t]{0.6\textwidth}
%     \begin{algorithm}[H]
%       
\footnotesize
\algrenewcommand{\algorithmiccomment}[1]{\hskip2em$\rhd$ #1}

\newcommand{\comment}[1]{$\rhd$ #1}

\newcommand{\LINEFOR}[2]{%
  \algorithmicfor\ {#1}\ \algorithmicdo\ {#2} %
}

\newcommand{\LINEIFTHEN}[2]{%
  \algorithmicif\ {#1}\ \algorithmicthen\ {#2} %
}

\newcommand{\LINEIFTHENELSE}[3]{%
  \algorithmicif\ {#1}\ \algorithmicthen\ {#2} \algorithmicelse\ {#3} %
}

\newcommand{\LET}[2]{%
  \State \textbf{let}\ $#1 \leftarrow #2$%
}

\newcommand{\PIPE}[0]{%
  \,|\,%
}

\newcommand{\INDSTATE}[1][1]{\State\hspace{\algorithmicindent}}

\begin{algorithmic}[1]
  \Function{subInsert}
  {$t \in \mathcal{T},\, indexes \in \mathbb{N}^+,\, i\in\mathcal{I}$}
  {$\, \rightarrow \mathcal{T}$}
  \LET {[index \PIPE indexes]}{indexes};
  \LET {[path \PIPE i]}{i};
  \LET {\langle \_,\, \_,\, count,\, children\rangle}{t};
  \If {$(index<0)$}
  \State \TODO{TODO} 
  \Else
  \EndIf
  \EndFunction

  \Statex

  \Function{insert}
  {$t \in \mathcal{T},\, i \in \mathcal{I}$}
  {$\, \rightarrow \mathcal{T}$}
  \State \Return $\textsc{subInsert}(t,\,\textsc{getIndexesOf}(t,\, i),\,i)$;
  \EndFunction

\end{algorithmic}

%       \caption{\label{repl:algo:insert} Insert.}
%     \end{algorithm}
%   \end{minipage}
%   \vspace{-15pt}
% \end{wrapfigure}

% L'algorithme~\ref{repl:algo:insert} présente les instructions de la fonction
% d'insertion.


% \subsubsection{delete}

%%% Local Variables:
%%% mode: latex
%%% TeX-master: "../../paper"
%%% End:
