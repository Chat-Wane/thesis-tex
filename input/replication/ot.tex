
\section{Transformés opérationnels}
\label{repl:sec:ot}

Les approches à transformés opérationnels~\cite{sun1998operational,
  sun2009contextbased} (OT) sont les plus anciennes et s'appliquent à un large
champs d'applications tels que l'édition de texte, l'édition d'images etc. Dans
le cadre de l'édition de texte, en plus des usuelles opérations d'insertion et
de suppression, OT founit des opérations ciblant les chaînes de caractères
telles que le déplacement, le couper -- coller, etc. Toutefois, l'analyse de
correction nécessite d'examiner chaque paire d'opérations ainsi que leurs
paramètres. En conséquence, lors de l'écriture du papier
\cite{imine2003proving}, peu d'approches étaient réellement correctes. De plus,
cette classe d'approches peut être divisé en deux sous-classes: les approches
centralisées et les approches décentralisées. Les premières réarrangent les
opérations sur un serveur central~\cite{nichols1995high} afin de faciliter la
convergence. Toutefois, la topologie elle-même implique un point individuel de
défaillance, des problèmes de confidentialité, des problèmes d'intelligence
économique, des problèmes de censure, et enfin, de passage à l'échelle. Les
approches décentralisées~\cite{sun2009contextbased}, quant à elle, nécessitent
un vecteur de version afin d'identifier les contextes de génération des
opérations reçues. Elles transforment les arguments de l'opération reçue par
rapport aux opérations concurrentes dans le but d'exécuter de manière cohérente
l'opération sans avoir à défaire et réexecuter ces opérations. De ce fait, bien
que l'exécution locale d'une opération soit très efficace, l'exécution des
opérations reçues est très coûteuse en cas de concurrence. Ainsi, confiné aux
environnements maîtrisés, OT reste efficace~\cite{mehdi2014merging}.

\begin{figure}
  \centering
  
\begin{tikzpicture}[scale=1.2]

  \newcommand\X{30pt};
  \newcommand\Y{30pt};
  
  \draw[->](0pt,   0pt)--(10*\X,   0pt);
  \draw[->](0pt, -1*\Y)--(10*\X, -1*\Y);
  \draw[->](0pt, -2*\Y)--(10*\X, -2*\Y);
  
  \draw[fill=black](0pt, 0pt) node[anchor=east]{copie 1 }circle(2pt);
  \draw[fill=black](0pt, -1*\Y) node[anchor=east]{copie 2 }circle(2pt);
  \draw[fill=black](0pt, -2*\Y) node[anchor=east]{copie 3 }circle(2pt);

  \draw(\X,2pt)--node[anchor=south]{[RTY]}( \X,   -2pt);
  \draw(\X,2 -1*\Y)--node[anchor=south]{[RTY]}(\X,-2 -1*\Y);
  \draw(\X,2 -2*\Y)--node[anchor=south]{[RTY]}(\X,-2 -2*\Y);
  \small
  \draw(3* \X,2pt)--node[anchor=north]{$insert(QWE,\,0)$}(3 * \X,   -2pt);
  \draw(3* \X,2 -2*\Y)--node[anchor=north]{$delete(0,\,3)$}(3 * \X,-2 -2*\Y);
  \normalsize

  \draw(3* \X,2pt)--node[anchor=south]{[QWERTY]}(3 * \X,   -2pt);
%  \draw(2* \X,2 -1*\Y)--node[anchor=south]{[ ]}(2* \X,-2 -1*\Y)
  \draw(3* \X,2 -2*\Y)--node[anchor=south]{[ ]}( 3 * \X,-2 -2*\Y);

  \draw[->, dashed] (5*\X, 0pt) -- (7*\X, -1*\Y);
  \draw[->, dashed] (5*\X, 0pt) -- (7*\X, -2*\Y);

  \small
  \draw[->, dashed] (5*\X, -2*\Y) -- (7*\X,  0*\Y)
  node[anchor=south]{$delete(3,\,3)$};
  \normalsize
  \draw[->, dashed] (5*\X, -2*\Y) -- (7*\X, -1*\Y);

  \draw(9*\X, 2 -0*\Y)--node[anchor=south]{[QWE]}(9*\X,-2 -0*\Y);
  \draw(9*\X, 2 -1*\Y)--node[anchor=south]{[QWE]}(9*\X,-2 -1*\Y);
  \draw(9*\X, 2 -2*\Y)--node[anchor=south]{[QWE]}(9*\X,-2 -2*\Y);


%%  \draw(9*\X, 2 -0*\Y)--node[anchor=south]{[QWERTY]}(9*\X,-2 -0*\Y);
%%  \draw(9*\X, 2 -1*\Y)--node[anchor=south]{[QWERTY]}(9*\X,-2 -1*\Y);
%%  \draw(9*\X, 2 -2*\Y)--node[anchor=south]{[QWERTY]}(9*\X,-2 -2*\Y);


%%  \draw[fill=white, very thick]
%%  (0*\X, 0*\Y) node{$p_1$} +(-5pt,-5pt) rectangle +(5pt,5pt);
%%  \draw[->](-5+\X, 5+2*\Y)to[out=120,in=30](0pt,5+2*\Y); %% 6 -> 7
\end{tikzpicture}
  \caption{\label{seq:fig:otexample}Exemple de transformé opérationnel
    garantissant la convergence lors d'opérations concurrentes. L'opération de
    suppression des 3 premiers caractères sur la copie 3 (RTY) est transformée
    afin de supprimer les 3 caractères à l'index 3 sur les autres copies.}
\end{figure}

La figure~\ref{seq:fig:otexample} illustre le principe de fonctionnement des
approches basées sur les transformés opérationnels. Dans ce scenario, les copies
sont toutes initialisées avec la séquence 'RTY'. Ensuite, tandis que la copie 1
insère les 3 caractères 'QWE' en tête de la séquence pour obtenir 'QWERTY', la
copie 3 supprime ses trois caractères pour obtenir la séquence vide. Avant la
dissémination de ces opérations, les copies ne sont pas identiques. Les copies
1, 2, et 3 ont respectivement les séquences [QWERTY], [RTY], []. Lorsque la
copie 1 réçoit l'opération de suppression, elle détecte que cette dernière est
en concurrence avec des opérations déjà intégrées, en l'occurence,
$insert(QWE,\,0)$. Son objectif est alors de déterminer l'impact que cette
opération concurrente a eu sur l'opération réçue afin d'en adapter les
arguments. Ici, l'insertion a décalé la séquence RTY de 3 positions vers la
droite. Par conséquent, La suppression, de part sa position, est elle aussi
décalée de 3 positions vers la droite. La résultat de la transformation est
$delete(3,\,3)$.  La copie 3, lorsqu'elle réçoit l'opération d'insertion,
detecte elle aussi que cette dernière est concurrente. Toutefois, la
transformation est sans effet sur ses arguments. La copie 2, selon l'ordre de
réception, se comporte comme la copie 1 ou la copie 3. À terme, les trois copies
convergent vers une séquence identique 'QWE'.
