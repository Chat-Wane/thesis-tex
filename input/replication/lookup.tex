\clearpage 

\section{indeOf}

\begin{wrapfigure}{r}{0.6\textwidth}
  \vspace{-35pt} %% (ugly)
  \begin{minipage}[t]{0.6\textwidth}
    \begin{algorithm}[H]
      
\footnotesize
\algrenewcommand{\algorithmiccomment}[1]{\hskip2em$\rhd$ #1}

\newcommand{\comment}[1]{$\rhd$ #1}

\newcommand{\LINEFOR}[2]{%
  \algorithmicfor\ {#1}\ \algorithmicdo\ {#2} %
}

\newcommand{\LINEIFTHEN}[2]{%
  \algorithmicif\ {#1}\ \algorithmicthen\ {#2} %
}

\newcommand{\INDSTATE}[1][1]{\State\hspace{\algorithmicindent}}

\begin{algorithmic}[1]
  \Function{getIndexesOf}{$t \in \mathcal{T}$, $i \in \mathcal{I}$}{$\, \rightarrow \mathbb{N}^+$}
  \State \textbf{let} $[hd\, |\, tl] \leftarrow i$;
  \State \textbf{let} $\langle \_,\, \_,\, children \rangle \leftarrow t$;
  \State \textbf{let} $index \leftarrow \DARKBLUE{\textsc{binaryIndexOf}}(children,\, hd)$;
  \If{$(index < 0 \wedge |tl| > 0)$}
  \State \Return $[index\, |\, \textsc{getIndexOf}(children[index],\,tl)]$;
  \Else
  \State \Return $[index]$;
  \EndIf
  \EndFunction

  \Statex

  \Function{getSum}{$t \in \mathcal{T}$, $indexes \in \mathbb{N}^+$}{$\, \rightarrow \mathbb{N}$}
  \State \textbf{let} $[hd\, |\, tl] \leftarrow indexes$;
  \State \textbf{let} $\langle \_,\, \_,\, children \rangle \leftarrow t$;
  \State \textbf{let} $sum \leftarrow 0$;
  \State \LINEIFTHEN{$(|tl|>0)$}{$sum \leftarrow \textsc{getSum}(children[hd],\,tl)$;}
  \For{\DARKBLUE{$i$ \textbf{from} 0 \textbf{to} $(hd-1)$}}\label{repl:line:optimization}
  \State \textbf{let} $\langle elem,\,count,\,\_ \rangle \leftarrow children[i]$;
  \State \LINEIFTHEN{$(elem)$}{$sum = sum + 1$;}
  \State $sum = sum + count$;
  \EndFor
  \State \Return $sum$;
  \EndFunction

  \Statex

  \Function{\DARKBLUE{getIndex}}{\DARKBLUE{$t \in \mathcal{T}$, $i \in \mathcal{I}$}}{\DARKBLUE{$\, \rightarrow \mathbb{N}$}}
  \State \textbf{let} $indexes \leftarrow \textsc{getIndexesOf}(t,\,i)$;
  \State \LINEIFTHEN{$(indexes[indexes-1]<0)$}{\Return $-1$;}
  \State \textbf{let} $\langle elem,\, \_,\,\_ \rangle \leftarrow t$;
  \State \textbf{let} $index \leftarrow \textsc{getSum}(t,\,indexes)$;
  \State \LINEIFTHEN{$(elem)$}{$index = index + 1$;}
  \State \Return $index$;
  \EndFunction
\end{algorithmic}

      \caption{\label{repl:algo:indexof} indexOf.}
    \end{algorithm}
  \end{minipage}
  \vspace{-15pt}
\end{wrapfigure}

L'algorithme~\ref{repl:algo:indexof} présente le fonctionnement de la fonction
destinée à retourner l'indice dans la séquence d'un identifiant. Elle est
primordiale afin d'éffectuer la liaison entre la structure de donnée répliquée
et la vision accordée à l'utilisateur. Celle-ci est divisée en trois :
\begin{inparaenum}[(i)]
\item La fonction \textsc{getIndexesOf} retourne les indices successifs dans les
  listes triées constituants les fils des nœuds de l'arbre. Celle-ci utilise la
  fonction \textsc{binaryIndexOf} possédant un compléxité temporelle
  logarithmique comparée à la taille du tableau trié parcouru.
\item La fonction \textsc{getSum} parcours l'arbre afin d'en déduire l'indice de
  séquence correspondant. Afin de ne pas avoir à parcourir l'entièreté des
  branches, les noeuds sauvegardent le nombre de sous-branches qu'ils possèdent.
  La ligne~\ref{repl:line:optimization} définit la façon dont l'arbre est
  parcouru. Un optimisation consiste à parcourir l'arbre à partir de la borne la
  plus proche de l'indice à la profondeur concernée.
\item La fonction \textsc{indexOf} est composées des deux fonctions
  susmentionnées.
\end{inparaenum}

\clearpage

\section{get}

\begin{wrapfigure}{r}{0.6\textwidth}
  \vspace{-35pt} %% (ugly)
  \begin{minipage}[t]{0.6\textwidth}
    \begin{algorithm}[H]
      
\footnotesize
\algrenewcommand{\algorithmiccomment}[1]{\hskip2em$\rhd$ #1}

\newcommand{\comment}[1]{$\rhd$ #1}

\newcommand{\LINEFOR}[2]{%
  \algorithmicfor\ {#1}\ \algorithmicdo\ {#2} %
}

\newcommand{\LINEIFTHEN}[2]{%
  \algorithmicif\ {#1}\ \algorithmicthen\ {#2} %
}

\newcommand{\INDSTATE}[1][1]{\State\hspace{\algorithmicindent}}

\begin{algorithmic}[1]
  \Function{leftmostBranch}{$t \in \mathcal{T}$}{$\, \rightarrow \mathcal{I}$}
  \State \textbf{let} $\langle \_,\, path,\, \_\, children \rangle \leftarrow t$;
  \If {$(|children|=0)$}
  \State \Return t
  \Else
  \State \Return []
  \EndIf
  \EndFunction

  \Function{sumFromLeft}{$t \in \mathcal{T}$, $index \in \mathbb{N}$, $id \in \mathcal{I}$}{$\, \rightarrow \mathcal{I}$}
  \If{$(index = 0)$}
  \State \Return $[id\, |\, \textsc{leftmostBranch}(t)]$;
  \Else
  \State \textbf{let} $i \leftarrow 0$;
  \State \textbf{let} $\langle \_,\, \_,\, children \rangle \leftarrow t$;
  \State \textbf{let} $\langle elem,\, count,\, \_ \leftarrow children[i]$;
  \While {$(index+children)$}
  \State meow
  \EndWhile
  \EndIf
  \EndFunction

  \Statex

  \Function{get}{$index \in \mathbb{N}$}{$\, \rightarrow \mathcal{I}$}
  \State fefef
  \EndFunction

\end{algorithmic}

      \caption{\label{repl:algo:get} get.}
    \end{algorithm}
  \end{minipage}
  \vspace{-15pt}
\end{wrapfigure}
