
\section{Exemples pédagogiques}

Cette section présente des exemples provenant du monde réel où la nécessitée
d'un ordre, un espace limité, et la difficulté de réordonnencement force à
attribuer des emplacements non-contigus.

\begin{itemize}
\item Les boites aux lettres du Laboratoire Informatique de Nantes-Atlantique
  sont ordonnées par ordre alphabetique. Cependant, les potentiels arrivées et
  départs de personnels force à conserver un espace entre les boites afin de ne
  pas décaler l'ensemble des boites suivantes.
\item Les premières versions de Basic permettent de sauter à une ligne grâce à
  l'instruction \textsc{GoTo}. Toutefois, les lignes sont assignées de manières
  fixes à une instruction. Il est courrant d'ajouter un espace de lignes libre
  entre chaque instruction au cas où le besoin s'en ferait sentir dans le futur.
\end{itemize}
