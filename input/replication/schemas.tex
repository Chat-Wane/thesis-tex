
\section{Pessimisme ou optimisme?}
\label{repl:sec:schemas}

L'objectif de la réplication pessimiste est simple. Il consiste à donner
l'illusion que la donnée manipulée par les utilisateurs est unique,
indépendemment du nombre de répliques réel. Grâce à la réplication pessimiste,
il est facile de raisonner sur les données car leurs spécifications sont proches
de celles proposées dans un \TODO{contexte sans réplications}.

Toutefois, chacune des modifications effectuée doit être validée. L'autorité
décisionnelle diffère en fonction des approches :
\begin{itemize}
\item [\textbf{autorité centrale~\cite{alsberg1976principle} :}] l'un des
  serveurs est désigné responsable d'une donnée. Ceux qui souhaitent modifier la
  donnée sont alors dans l'obligation de demander l'accès exclusif pendant la
  mise en place de cette modification. Dans l'intervalle, les autres répliques
  ne peuvent soumettre de modifications. Enfin, lorsque la modification est
  achevée, la main est rendue au serveur qui peut autoriser d'autres
  modifications. C'est le méchanisme de vérouillage (\emph{lock}).
\item [\textbf{quorum~\cite{gifford1979weighted} :}] les modifications sont
  soumises à un vote où un certains nombre de serveurs \TODO{doivent approuver
    ou non}. \TODO{Serialisation}.
\end{itemize}

La réplication pessimiste est possible lorsque le nombre de répliques est connu,
plutôt petit, et souvent accessible. Ces contraintes sont notamment
satisfaisable dans le \TODO{Nuage}. Les services proposés possèdent d'avantage
d'assurances quant aux résultats de la manipulation des données. Toutefois, de
telles guaranties ne sont pas toujours nécéssaire et leur coût élévée n'est
alors plus justifié. \TODO{La section suivante est.}

%% optimistic

En 1987, Demers et al. décrivent une base de données répliquée sur plusieurs
centaines de machines pouvant communiquer entre elles au travers de materiels
aux capacités hétérogènes~\cite{demers1987epidemic}. \TODO{moar.}

La réplication optimiste~\cite{johnson1975maintenance, saito2005optimistic} est
un paradigme de réplication consistant à appliquer les modifications directement
sur la réplique locale.  Ainsi, les données sont toujours disponibles et
réactives aux changements effectués. Ensuite, les modifications sont disséminées
aux autres possesseurs de cette donnée partagée où elles sont appliquées. Au
contraire des approches pessimistes, les approches optimistes ne vérouillent pas
l'accès aux données lors de modifications. En revanche, le critère de cohérence
assuré est plus faible. En particulier, les répliques ont l'autorisation d'avoir
des états temporairement divergeant entre eux :

\begin{itemize}
\item [\textbf{Cohérence à terme :}] lorsque toutes les modifications ont été
  reçues et appliquées par toutes les répliques, celle-ci possèdent un état
  équivalent. Puisque \emph{toutes} les modifications constituent un ensemble
  peu réaliste dans le cadre d'une éxecution réelle, un définition plus précise
  porte sur un sous-ensembles de ces modifications :
\item [\textbf{Cohérence forte à terme~\cite{shapiro2011conflict} :}] les
  répliques ayant réçu et appliqué les même modifications possèdent un état
  équivalent.
\end{itemize}

Ces critères de cohérence posent de nombreux problèmes par leur manque
d'expressivité. En particulier, l'état équivalent vers lequel les répliques
convergent n'est pas spécifié. Celui-ci peut n'avoir aucun lien avec l'éxecution
souhaitée par l'utilisateur. Par exemple, un ensemble dont l'ajout d'éléments
n'a aucun effet converge vers l'ensemble vide. Bien que d'une complète
inutilité, il maintient la cohérence forte à terme.

Récemment, de nombreux efforts ont été fournis afin de proposer des bases de
données avec cohérence à terme (\REF), ainsi que des langages sur lesquels
raisonner (\REF).

\begin{figure}
  \centering
  
\begin{tikzpicture}[scale=1.2]

  \newcommand\X{30pt};
  \newcommand\Y{30pt};
  
  \draw[->](0pt,   0pt)--(10*\X,   0pt);
  \draw[->](0pt, -1*\Y)--(10*\X, -1*\Y);
  \draw[->](0pt, -2*\Y)--(10*\X, -2*\Y);
  
  \draw[fill=black](0pt, 0pt) node[anchor=east]{réplique 1 }circle(2pt);
  \draw[fill=black](0pt, -1*\Y) node[anchor=east]{réplique 2 }circle(2pt);
  \draw[fill=black](0pt, -2*\Y) node[anchor=east]{réplique 3 }circle(2pt);

  \draw(\X,2pt)--node[anchor=south]{[ ]}( \X,   -2pt);
  \draw(\X,2 -1*\Y)--node[anchor=south]{[ ]}(\X,-2 -1*\Y);
  \draw(\X,2 -2*\Y)--node[anchor=south]{[ ]}(\X,-2 -2*\Y);

  \draw(2* \X,2pt)--node[anchor=south]{[QWE]}(2* \X,   -2pt);
%  \draw(2* \X,2 -1*\Y)--node[anchor=south]{[ ]}(2* \X,-2 -1*\Y);
%  \draw(2* \X,2 -2*\Y)--node[anchor=south]{[ ]}(2* \X,-2 -2*\Y);

  \draw[->, dashed] (2*\X, 0pt) -- (8*\X, -1*\Y);
  \draw[->, dashed] (2*\X, 0pt) -- (3*\X, -2*\Y);

  \draw(4*\X, 2 -0*\Y)--node[anchor=south]{[QWE]}(4*\X,-2 -0*\Y);
  \draw(4*\X, 2 -1*\Y)--node[anchor=south]{[ ]}(4*\X,-2 -1*\Y);
  \draw(4*\X, 2 -2*\Y)--node[anchor=south]{[QWE]}(4*\X,-2 -2*\Y);


  \draw(6*\X, 2 -2*\Y)--node[anchor=north]{[QWERTY]}(6*\X,-2 -2*\Y);


  \draw[->, dashed] (6*\X, -2*\Y)--(7*\X, -0*\Y);
  \draw[->, dashed] (6*\X, -2*\Y)--(7*\X, -1*\Y);

  \draw(9*\X, 2 -0*\Y)--node[anchor=south]{[\BLUE{QWERTY}]}(9*\X,-2 -0*\Y);
  \draw(9*\X, 2 -1*\Y)--node[anchor=south]{[\BLUE{QWERTY}]}(9*\X,-2 -1*\Y);
  \draw(9*\X, 2 -2*\Y)--node[anchor=south]{[\BLUE{QWERTY}]}(9*\X,-2 -2*\Y);


%%  \draw[fill=white, very thick]
%%  (0*\X, 0*\Y) node{$p_1$} +(-5pt,-5pt) rectangle +(5pt,5pt);
%%  \draw[->](-5+\X, 5+2*\Y)to[out=120,in=30](0pt,5+2*\Y); %% 6 -> 7
\end{tikzpicture}
  \caption{\label{repl:fig:optimisticexample} Exemple d'éxecution d'un protocole
    de réplication optimiste dont les répliques convergent vers la séquence
    'QWERTY'.}
\end{figure}

La figure~\ref{repl:fig:optimisticexample} présente un cas de séquence
répliquée.  Il existe trois copies d'une séquence initialement vide. La première
copie insère 'QWE' et en dissémine l'information. La troisième copie reçoit
l'opération et l'applique localement. Cette copie insère 'RTY' à la suite de
'QWE' afin d'obtenir 'QWERTY' et envoie l'information aux deux autres
copies. Quel que soit l'ordre de réception, le protocole garanti que les copies
convergent vers un état identique, ici, la séquence 'QWERTY'.


%%% Local Variables:
%%% mode: latex
%%% TeX-master: "../../paper"
%%% End:
