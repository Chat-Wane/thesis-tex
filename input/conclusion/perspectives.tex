
\section{Perspectives}

\subsection{O'Browser, Where Art Thou?}

\subsection{Fusion de réseaux}

\label{net:sec:merging}

La fusion de réseaux consiste à obtenir un réseau unique comme l'union des
membres de plusieurs réseaux. Le réseau obtenu doit hériter des propriétés de
ses parents.  Lors de ce processus, nous supposons qu'au moins un des nœuds
appartenant à l'un des réseau contacte l'autre réseau afin d'initier la
fusion. Grâce à la connexion qui en résulte, les réseaux sont à même de
communiquer, et donc de fusionner.

Les approches à taille fixe sont triviales à étendre : les mélanges périodiques
suffisent à garantir un réseau connexe. Si toutefois les vues partielles sont
configurées avec des tailles différentes, il suffit alors de prendre la taille
maximum des deux. Par exemple, un nœud avec une vue partielle de 5 voisins verra
sa vue augmenter à $7$ voisins après un échange avec le nœud dont la vue
partielle est peuplée de $7$ références.

\SPRAY est une approche dont les vues partielles évoluent automatiquement en
réaction aux entrées et sorties du réseau. En particulier, les vues partielles
suivent une progression logarithmiques comparée à la taille du réseau. La fusion
de réseaux \SPRAY doit résulter en un réseau \SPRAY garantissant de même.

La première solution qui vient à l'esprit est la suivante : chacun des nœuds du
premier réseau utilise le contact afin de rejoindre le second réseau, comme un
sablier dont les grains passent un tube étroit pour rejoindre l'autre bulbe sous
l'effet de la gravité. Malheureusement, cette solution est extrêmement lente --
puisque la majorité des nœuds ignorent encore qui est le contact -- et
susceptible d'échouer -- puisque le contact est un point unique de défaillance.

Un seconde solution consiste simplement, à l'instar des approches à taille fixe,
à laisser le méchanisme de mélange faire son office. Petit à petit, d'autres
ponts entre les réseaux vont se former jusqu'à ce que les deux réseaux soient
indifférenciés. Malheureusement, les arcs ne suivent pas l'augmentation relative
au réseau.

\begin{problem}
  Soit $\mathcal{N}_1,\, \mathcal{N}_2,\, \ldots ,\, \mathcal{N}_k$ des réseaux
  de taille arbitraire. On a :
\begin{equation}
  \sum\limits_{i \in \mathbb{N}_{<k}} |\mathcal{N}_i|\ln (|\mathcal{N}_i|) < (\sum\limits_{i \in \mathbb{N}_{<k}} |\mathcal{N}_i|)\ln{(\sum\limits_{i \in \mathbb{N}_{<k}} |\mathcal{N}_i|)}
\end{equation}
Comment adapter les nombres d'arcs effectif (à gauche) pour qu'il atteigne le
nombre d'arcs requis (à droite)?
\end{problem}


% Pour répondre à ce problème, chacun des nœuds appartenant aux réseaux impliqués
% dans la fusion doit être capable de
% \begin{inparaenum}[(i)]
% \item détecter lorsqu'un nouveau réseau fusionne avec celui dans lequel il se
%   trouve (cf. §\ref{net:subsec:detection}),
% \item détecter lorsqu'il a glané suffisamment d'informations pour procéder à la
%   fusion (cf. §\ref{net:subsec:activation}),
% \item ajuster sa vue partielle en conséquence (cf. §\ref{net:subsec:merging}).
% \end{inparaenum}

\subsection{Table de hachage répartie}

Une table de hachage répartie (DHT) est un système permettant de retrouver une
information rapidement dans un réseau de machines. Pour ce faire, l'information
recherchée possède une clé dans laquelle figure des données suffisantes à
l'exploration efficace du réseau. Le réseau est généralement organisé selon un
critère de proximité de clé et constitue un réseau superposé. 

Parmi les représentants des DHT on nomme Chord~\cite{stoica2001chord},
CAN~\cite{ratnasamy2001scalable}, Pastry~\cite{rowstron2001pastry},
Tapestry~\cite{zhao2006tapestry}, Kademlia~\cite{maymounkov2002kademlia}.
Toutes ces approches fournissent un vue partielle dont la taille est
logarithmique. Toutefois, tout comme pour les protocoles d'échantillonnage
aléatoire de pairs dont la vue partielle est fixe, tel que \CYCLON, la taille de
leur vue est configurée \emph{a priori} afin que le système puisse accueillir un
ensemble suffisamment large de clés.

\begin{figure}
  \begin{center}
    \input{input/conclusion/dhtexample.tex}
    \caption{\label{conclu:fig:dhtexample}Table de hachage répartie suivant le
      principe de Chord.}
  \end{center}
\end{figure}

Comme le montre la figure~\ref{conclu:fig:dhtexample}, l'idée est alors de
placer les systèmes classiques de DHT -- ici Chord -- afin qu'ils s'accordent
avec la vue partielle fournie par \SPRAY. Par exemple, si un nœud \SPRAY possède
3 voisins, le réseau superposé en charge de la DHT possède lui aussi 3
voisins. Toutefois, ces derniers sont choisis selon la distance à laquelle ils
se trouvent. Un premier voisin serait celui qui est le plus proche, le second
voisin celui qui se trouve à la moitié de la distance maximum, et le troisième
se trouve à un tier de la distance maximum. Un tel système permet de diriger les
messages ciblant un nœud particulier très efficacement : de l'ordre de
$\log\log N$ redirections, où $N$ est le nombre de membres du réseau. Grâce à
\SPRAY, cette propriété pourrait rester vraie quelle que soit la taille du
réseau.

\subsection{Compromis causalité et concurrence}

\subsection{Wiki réparti temps réel}

Les Wiki sont des espaces de collaborations permettant à plusieurs participants
d'éditer des pages à tour de rôle, i.e., l'édition n'est pas effectuée en temps
réel. Le succès de l'encyclopédie Wikipédia n'est plus à démontrer. Hélas, le
modèle même d'édition provoque l'apparition de conflits.


%%% Local Variables:
%%% mode: latex
%%% TeX-master: "../../paper"
%%% End:
