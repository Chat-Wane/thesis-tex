
\section{Résumé des contributions}
\label{conclu:sec:summary}

Dans ce manuscrit, notre objectif a été de fournir une solution complète en
réponse aux problèmes de passage à l'échelle et de facilité d'accès aux éditeurs
collaboratifs temps réel. Nos contributions s'étalent sur les trois points
suivants : 

\paragraph{Communication.} Les réseaux dont la topologie possède des similitudes
avec les graphes aléatoires sont au cœur de nombreuses approches
décentralisées. L'état de l'art se divise en deux catégories caractérisées par
la vue partielle quelles fournissent :
\begin{inparaenum}[(i)]
\item Les approches dont les vues partielles sont de taille constante définie
  lors de la configuration sont les plus nombreuses. Malheureusement, le
  développeur d'application doit prévoir les dimensions du réseau géré par son
  application. Cela n'est pas toujours possible et conduit à un
  surdimmensionnement des vues partielles. 
\item L'approche dont les vue partielle sont de taille variable ne supporte pas
  bien la dynamicité des réseaux. 
\end{inparaenum}
Nous proposons une approche, nommée \SPRAY, débarassant le développeur de ces
considérations. La taille des vues partielles suit grâcieusement.


\paragraph{Structure de données répliquée.}

\paragraph{Éditeur décentralisé dans le navigateur.}

%%% Local Variables:
%%% mode: latex
%%% TeX-master: "../../paper"
%%% End:
