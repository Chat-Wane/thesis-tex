
\section{Résumé des contributions}
\label{conclu:sec:summary}

% Dans ce manuscrit, notre objectif a été de fournir une solution complète en
% réponse aux problèmes de passage à l'échelle et de facilité d'accès des éditeurs
% collaboratifs temps réel. Nos contributions s'étalent sur les trois points
% suivants :

Dans ce manuscrit, notre objectif a été de répondre à la question de faisabilité
de l'édition collaborative temps réel dans les navigateurs Web, sans
l'intervention de tiers et sans limites quant aux dimensions du système. Nos
contributions s'étalent sur les trois points suivants :

\paragraph{Structure de données répliquée.} La réplication optimiste permet
d'améliorer l'accessiblité, la réactivité et la tolérance aux pannes de la
donnée partagée. Pour la séquence, l'état de l'art fait tout d'abord mention des
approches à transformées opérationnelles. Cependant, ces approches se voient
restreintes aux contextes avec peu de répliques et peu de latence. Plus
récemment, les structures dont les opérations commutent ont fait leur
apparition. Hélas, soit l'espace nécéssité croît continuellement, même en cas de
suppressions d'éléments; soit celles-ci utilisent des identifiants dont la
taille croît linéairement. Ces approches requièrent toutes un mécanisme
additionnel afin de recouvrer de bonnes performances. Malheureusement, ces
mécanismes sont inutilisables à large échelle. Nous proposons \LSEQ, une
stratégie d'allocation d'identifiants dont les identifiants croissent de manière
polylogarithmique par rapport au nombre d'insertions effectuées dans la
séquence. De plus, les éléments supprimés sont réellement détruit. Ainsi, cette
approche reste performante à large échelle et ne nécessite aucun mécanisme
additionnel.  Nous fournissons l'analyse en complexité -- temporelle et spatiale
-- et en validons les résultats grâce à des expérimentations. \LSEQ fait l'objet
d'une implémentation JavaScript qui est utilisée dans un éditeur collaboratif
temps réel décentralisé accessible dans les navigateurs Web.

\paragraph{Communication.} Les protocoles d'échantillonnage de pairs constituent
le cœur de nombreux systèmes répartis. Ces protocoles proposent à chaque pair
une vue partielle du réseau. Les approches fournissant des échantillons
aléatoires de pairs convergent vers des topologies proches des graphes
aléatoires. Elles garantissent certaines propriétés désirables dans notre
contexte telle que la robustesse face à la dynamicité du réseau.  L'état de
l'art se divise en deux catégories caractérisées par les vues partielles
qu'elles fournissent :
\begin{inparaenum}[(i)]
\item Les approches dont les vues partielles sont de taille constante définie
  lors de la configuration sont les plus nombreuses. Malheureusement, le
  développeur doit prévoir les dimensions des réseaux gérées par son
  application. Cela n'est pas toujours possible et conduit à un
  surdimmensionnement des vues partielles.
\item Les approches dont les vues partielles sont de taille variable ne
  supportent pas bien les contraintes imposées par le processus complexe
  d'établissement de connexion dans les navigateurs Web.
\end{inparaenum}
Nous proposons une approche, nommée \SPRAY, débarrassant le développeur de ces
considérations. La taille des vues partielles suit la taille du réseau. Grâce à
elle, les vues partielles sont toujours de taille logarithmique par rapport à la
taille du réseau actuel. L'établissement de connexion d'un voisin à l'autre
rend l'approche fiable même dans un réseau dynamique. En dépit de ces connexions
de proche en proche, la convergence vers un état aux propriétés stables est
extrêmement rapide. Ces propriétés ont été validées au travers de
simulations. \SPRAY a ensuite été implémenté en JavaScript afin d'être utilisé
dans un éditeur collaboratif temps réel décentralisé accessible dans les
navigateurs Web.

\paragraph{Éditeur décentralisé dans le navigateur.} Google Docs et Etherpad ont
rendu l'accès à l'édition collaborative aisé pour des millions
d'utilisateurs. Toutefois, la centralisation de ces services pose des problèmes
de confidentialité, de passage à l'échelle et de tolérance aux
défaillances. Nous proposons \CRATE, un éditeur collaboratif temps réel
décentralisé tournant directement dans les navigateurs Web. Étant décentralisé,
un document appartient seulement à ceux qui l'éditent, ce qui règle les
problèmes de confidentialité. Utilisant \SPRAY et \LSEQ, \CRATE parvient à
adapter le trafic généré à la session d'édition. Ce trafic croît
logarithmiquement par rapport à la taille de la session d'édition. Ce trafic
croît polylogarithmiquement par rapport au nombre d'insertions dans le
document. Ce résultat est validé par des expérimentations impliquant jusqu'à 601
navigateurs écrivant un document artificiel de plusieurs millions de caractères.

%%% Local Variables:
%%% mode: latex
%%% TeX-master: "../../paper"
%%% End:
