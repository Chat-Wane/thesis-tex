
\section{Résumé des contributions}
\label{conclu:sec:summary}

Dans ce manuscrit, notre objectif a été de fournir une solution complète en
réponse aux problèmes de passage à l'échelle et de facilité d'accès des éditeurs
collaboratifs temps réel. Nos contributions s'étalent sur les trois points
suivants :

\paragraph{Communication.} Les réseaux dont la topologie possède des similitudes
avec les graphes aléatoires sont au cœur de nombreuses approches
décentralisées. L'état de l'art se divise en deux catégories caractérisées par
les vues partielles qu'elles fournissent :
\begin{inparaenum}[(i)]
\item Les approches dont les vues partielles sont de taille constante définie
  lors de la configuration sont les plus nombreuses. Malheureusement, le
  développeur doit prévoir les dimensions des réseaux gérées par son
  application. Cela n'est pas toujours possible et conduit à un
  surdimmensionnement des vues partielles.
\item Les approches dont les vues partielles sont de taille variable ne
  supportent pas bien les réseaux dynamiques.
\end{inparaenum}
Nous proposons une approche, nommée \SPRAY, débarassant le développeur de ces
considérations. La taille des vues partielles suit gracieusement la taille du
réseau. Grâce à elle, les vues partielles sont toujours de taille logarithmique
par rapport à la taille du réseau actuel. L'établissement des connexions d'un
voisin à l'autre rend l'approche fiable même dans un réseau dynamique. En dépit
de ces connexions de proche en proche, la convergence vers un état aux
propriétés stables est extrêmement rapide. Ces propriétés ont été validées au
travers de simulations. \SPRAY a ensuite été implémenté en Javascript afin
d'être utilisé dans un éditeur collaboratif temps réel décentralisé accessible
dans les navigateurs web.

\paragraph{Structure de données répliquée.} La réplication optimiste permet
d'améliorer l'accessiblité, la réactivité, et la tolérence aux pannes de la
donnée partagée. Pour la séquence, l'état de l'art fait tout d'abord mention des
approches à transformés opérationnels. Hélas, ces approches se voient
restreintes aux contextes avec peu de répliques et peu de latences. Plus
récemment, les structures dont les opérations commutent ont fait leur
apparition. Hélas, soit celles-ci croissent continuellement, même en cas de
suppressions d'éléments; soit celles-ci utilisent des identifiants dont la
taille croît linéairement. Ces approches requièrent toutes un méchanisme
additionnel afin recouvrer de bonnes performances. Malheureusement, ces
méchanismes sont inutilisables à large dimensions. Nous proposons \LSEQ, une
stratégie d'allocation d'identifiants dont les identifiants croissent de manière
polylogarithmique comparé au nombre d'insertions effectuées sur la séquence. De
plus, les éléments supprimés sont réellement détruit. Ainsi, cette approche
reste performante à large échelle et ne nécessite aucun méchanisme additionnel.
Nous fournissons l'analyse en complexité -- temporelle et spatiale -- et en
validons les résultats grâce à des expérimentations. Là encore, \LSEQ fait
l'objet d'une implémentation Javascript qui est utilisée dans un éditeur
collaboratif temps réel décentralisé accessible dans les navigateurs web.


\paragraph{Éditeur décentralisé dans le navigateur.} Google Docs et Etherpad ont
rendu l'accès à l'édition collaborative aisée pour des millions
d'utilisateurs. Toutefois, la centralisation de ces services pose des problèmes
de confidentialité, de passage à l'échelle, et de tolérence aux pannes. Nous
proposons \CRATE, un éditeur collaboratif temps réel décentralisé tournant
directement dans les navigateurs web. Étant décentralisé, un document appartient
seulement à ceux qui l'éditent, ce qui règle les problèmes de
confidentialité. Utilisant \SPRAY et \LSEQ, \CRATE parvient à adapter le trafic
généré à la session d'édition. Ce trafic croît logarithmiquement par rapport à
la taille de la session d'édition. Ce trafic croît polylogarithmiquement par
rapport au nombre d'insertions dans le document. Ce résultat est validé par des
expérimentations impliquant jusqu'à 601 navigateurs écrivant un document
artificiel de plusieurs millions de caractères.

%%% Local Variables:
%%% mode: latex
%%% TeX-master: "../../paper"
%%% End:
