
\section{Contexte}

\lettrine{L}'édition collaborative concerne toutes les activités effectuées en
groupe dans le but de produire un document~\cite{ellis1989concurrency,
  johansen1988groupware}. Grâce à un effort collectif, les rédacteurs
bénéficient de multiples points de vue et deviennent plus impliqués dans
l'écriture~\cite{noel2004empirical}. Les documents en résultant sont de
meilleure qualité. Ainsi l'encyclopédie Wikipédia compte des millions d'articles
rédigés par des millions d'auteurs et sa version anglaise possède une fiabilité
comparable à celle de l'Encyclopædia Britannica~\cite{giles2005internet}.

Les éditeurs collaboratifs de texte sont simplement des outils facilitant
l'écriture de documents par plusieurs auteurs. Ils présentent un texte dont le
contenu est modifiable par un groupe de collaborateurs. Chaque modification est
transmise à tous les collaborateurs qui peuvent alors les voir et écrire en
conséquence. Grâce à ces outils, la tâche d'écriture peut être répartie selon le
temps et l'espace~\cite{desanctis1987foundation, grudin1994computersupported,
  johansen1988groupware}. Ainsi, il n'est plus impératif de se retrouver dans la
même pièce ni même de se retrouver au même moment pour participer à la rédaction
du document.

Si le premier éditeur collaboratif date de 1968~\cite{engelbart1968research},
leur adoption massive par le public n'est que récente
et émane principalement des éditeurs Web temps réel~\cite{mogan2010impact,
  perkel2014scientific} tels que Google Docs~\cite{googledocs} ou
ShareLaTeX~\cite{sharelatex}. Grâce au Web, il est possible de créer et d'éditer
facilement un document depuis un ordinateur de bureau, un téléphone portable ou
une tablette tactile. Un simple lien permet de le partager avec des amis ou des
collègues. L'édition collaborative devient si aisée qu'elle est à la portée de
tous.  Cependant, de nos jours, un petit groupe d'entreprises détient une
large portion du Web. Les applications Web populaires tels que les éditeurs
collaboratifs sont maintenues par quelques grandes sociétés. 
Cette organisation éminemment centralisée, où quelques serveurs sont en charge
d'un nombre titanesque de clients, pose des problèmes d'ordre éthique sur la
confidentialité, la propriété et la censure~\cite{cherrueau2016composer,
  gellman2013us, pearson2011toward}; et des problèmes d'ordre technique sur le
passage à l'échelle et la résilience aux pannes.

Pour ces raisons, la redécentralisation du Web, et plus généralement de
l'internet, a reçu beaucoup d'attention ces derniers temps~\cite{benet2014ipfs,
  maelstrom, mansour2016demonstration, wood2014ethereum}.  La décentralisation
constituerait un progrès vers une solution aux problèmes liés à l'éthique.  Par
exemple, la propriété des documents pourrait être rendue à ceux qui les
rédigent. La décentralisation constituerait une solution partielle aux problèmes
d'ordre technique. Par exemple, elle permettrait d'alléger l'infrastructure des
services Web. Chaque client devenant également un serveur, le poids de l'édition
s'en trouverait réparti parmi les participants au lieu du
Nuage~\cite{mell2011national}. 

La facilité d'accès apportée par le Web au service de la décentralisation
inspire également de nouveaux usages de plus grande ampleur et plus
dynamiques. Par exemple, un document n'est plus cantonné aux petits groupes : le
Web entier peut participer à son élaboration, donner ses avis, participer à sa
rédaction. Lors d'un cours de formation en ligne
(\emph{MOOC})~\cite{breslow2013studying}, 
la prise de notes peut rassembler des milliers de
personnes à son commencement. Ensuite, nombre d'étudiants quittent le cours par
manque d'intérêt. Enfin, beaucoup d'étudiants réintègrent le groupe d'édition
lorsque l'examen final approche.  Par conséquent, les éditeurs collaboratifs
doivent supporter des groupes dont les dimensions ont largement augmenté, et
dont le nombre de collaborateurs fluctue à chaque instant.

Se pose donc la question de la faisabilité d'une telle application :
\textbf{L'édition collaborative temps réel est-elle possible sur le Web, sans
  l'intervention d'un tiers et sans limites quant aux dimensions du système ?}

%%% Local Variables:
%%% mode: latex
%%% TeX-master: "../../paper"
%%% End:
