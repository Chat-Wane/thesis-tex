
\section{Contexte}

\lettrine{L}'émergence de l'outil textuel comme relai de la parole a permis de
nombreuses avancées sur le plan scientifique, juridique, culturel
etc. Récemment, l'informatique a étendu l'édition textuelle à plusieurs auteurs
en facilitant les interactions distantes. Si l'objectif de collaboration est
demeuré inchangé depuis les premiers éditeurs~\cite{engelbart1968research}, les
dimensions du système, quant à elles, ont crû.

En effet, l'adoption des éditeurs collaboratifs de
textes~\cite{ellis1991groupware} n'a eu de cesse d'augmenter ces dernières
décénies.
% Tout le monde écrit, et tout le monde souhaite partager ses écrits.
Quelle que soit l'heure, quel que soit le lieu, un auteur peut rédiger son
document numérique avec ses collaborateurs dont les expertises combinées en
amélioreront la qualité. Ainsi, l'encyclopédie \emph{Wikipédia} compte des
millions d'articles rédigés par des millions d'auteurs et sa version anglaise
possède une fiabilité combarable à celle de l'\emph{Encyclopædia
  Britannica}~\cite{giles2005internet}.

Améliorant à la fois efficacité et implication des
utilisateurs~\cite{noel2004empirical}, les éditeurs collaboratifs sont d'une
indéniable utilité. Toutefois, les outils proposés actuellement ne sont pas sans
défaut. Les éditeurs basés sur serveur central, tel que \emph{Google
  Docs}~\cite{googledocs}, posent des problèmes de
confidentialité~\cite{gellman2013us}, de passage à l'échelle, et de tolérence
aux pannes.
% L'inquiètude du public vis-à-vis des fournisseurs de services s'est par
%ailleurs avérée légitime lors des fuites sur
%\emph{PRISM}~\cite{gellman2013us}.
Afin de se réconcilier avec le principe de l'internet sans autorité centrale
opérant de manière complètement répartie, les éditeurs doivent se débarasser de
ces serveurs. Grâce à cela, les problèmes de confidentialité s'en trouvent
résolus. Toutefois, les problèmes de passage à l'échelle subsitent auxquels
s'ajoutent les difficultés d'utilisation autrefois résolues par le
Nuage~\cite{mell2011national}.


%%% Local Variables:
%%% mode: latex
%%% TeX-master: "../../paper"
%%% End:
