
\section{Contexte}

\lettrine{L}'édition collaborative concerne toutes les activités effectuées en
groupe dans le but de produire un document~\cite{ellis1989concurrency,
  johansen1988groupware}. Grâce à un effort collectif, les rédacteurs
bénéficient de multiples points de vues et deviennent plus impliqués dans
l'écriture~\cite{noel2004empirical}. Les documents en résultant sont de
meilleure qualité. Ainsi l'encyclopédie \emph{Wikipédia} compte des millions
d'articles rédigés par des millions d'auteurs et sa version anglaise possède une
fiabilité comparable à celle de l'\emph{Encyclopædia
  Britannica}~\cite{giles2005internet}.

Les éditeurs collaboratifs de textes sont simplement des outils facilitant
l'écriture de documents par plusieurs auteurs. Ils présentent un texte dont le
contenu est modifiable par un groupe de collaborateurs. Chaque modification est
transmise à tous les collaborateurs qui peuvent alors les voir et écrire en
conséquence. Grâce à ces outils, la tâche d'écriture peut être répartie selon le
temps et l'espace~\cite{desanctis1987foundation, grudin1994computersupported,
  johansen1988groupware}. Ainsi, il n'est plus impératif de se retrouver dans la
même pièce ni même de se retrouver au même moment pour participer à la rédaction
du document.

Depuis la présentation du premier éditeur de ce type par Engelbart en
1968~\cite{engelbart1968research}, les éditeurs collaboratifs ont
fleuri. Toutefois, leur adoption massive par le public n'est que récente et
provient principalement des éditeurs web~\cite{mogan2010impact,
  perkel2014scientific} tels que \emph{Google Docs}~\cite{googledocs} ou
\emph{ShareLaTeX}~\cite{sharelatex}. Grâce au web, il est possible de créer et
d'éditer facilement un document depuis un ordinateur, un portable, ou une
tablette. Un simple lien permet de le partager avec des amis ou des
collègues. L'édition collaborative devient si aisée qu'elle est à la portée de
tous. Cette facilité d'accès ouvre à de nouveaux usages. Par exemple, un
document n'est plus cantonné aux petits groupes : le web entier peut participer
à son élaboration, donner ses avis, participer à sa rédaction. Lors d'un cours
formation en ligne (\emph{MOOC})~\cite{breslow2013studying}, la prise de notes
peut rassembler des milliers de personnes à son commencement. Ensuite, nombre
d'étudiants quittent le cours par manque d'interêt. Enfin, beaucoup d'étudiants
réintègrent le groupe d'édition lorsque l'examen final approche.  Par
conséquent, les éditeurs collaboratifs doivent supporter des groupes dont les
dimensions ont largement augmenté, et dont le nombre de collaborateurs fluctue à
chaque instant. Le web est-il prêt à accueillir ce genre d'applications?

%Jusqu'alors, l'architecture des applications web était éminement centralisé,
%avec un serveur

%% Initialement pensé en terme decentralisé, le web a été pris à son compte par les
%% grandes fournisseurs de services ; .   . fef oizbfio z

%% Jusqu'alors, le web comme support à la collaboration a été utilisé uniquement de
%% manière centralisée, avec un server ... .  . . .

%Actuellement, une large portion du web est détenue par quelques acteurs.

%La majeur partie du web est détenue par quelques groupes 

De nos jours, un petit nombre d'entreprises détient une large portion du web. En
résulte une forte centralisation des données et de puissance de calcul. Cette
organisation éminemment centralisée, ou quelques serveurs sont en charge d'un
nombre titanesque de clients, pose les problèmes suivant :

 
% L'utilisation du web comme support à la collaboration n'est pas sans défaut. Son
% organisation éminemment centralisée, avec un serveur en charge de l'ensemble des
% clients, pose les problèmes suivant :

\paragraph{Confidentialité :} Chaque lettre tapée transite par un serveur
appartenant à un fournisseur de services. Ce dernier est en mesure de lire le
document, d'en vendre ou d'en concéder le contenu à des tiers tels que des
entreprises publicitaires ou des entités du
gouvernement~\cite{cherrueau2016composer, gellman2013us, pearson2011toward};

\paragraph{Censure :} Bien souvent, la propriété du document doit être accordée
au fournisseur de service. À tout moment, le document peut être supprimé, au bon
vouloir du fournisseur, si son contenu est jugé contradictoire aux termes du
service;

\paragraph{Passage à l'échelle :} Un serveur est en charge du document. Si le
nombre d'utilisateurs augmente et que l'infrastructure mise en place n'est pas
adaptée, le serveur risque de ne pouvoir le supporter. Les utilisateurs
souffrent alors d'une baisse de qualité de service. Par exemple, leur document
peut devenir temporairement inaccessible;

\paragraph{Point unique de défaillance :} Puisqu'un seul serveur est en charge
du document, si celui-ci tombe en panne, l'accès aux documents qu'il héberge est
compromis~\cite{demers1987epidemic}.

La redécentralisation du web, et plus généralement de l'internet, a reçu
beaucoup d'attention ces derniers temps~\cite{benet2014ipfs, maelstrom,
  mansour2016demonstration, wood2014ethereum}. La décentralisation permettrait
de résoudre les problèmes liés à la confidentialité, à la censure, et aux
défaillances. Les documents appartiendraient réellement à ceux qui les
rédigent. La décentralisation permettrait aussi d'alléger l'infrastructure des
services web. Chaque client devenant également un serveur, le poids de l'édition
s'en trouverait réparti parmi les participants au lieu du
Nuage~\cite{mell2011national}.


\textbf{L'édition collaborative temps réel est-elle possible sur le web, sans
  l'intervention d'un tiers et sans limites quant aux dimensions du système?}

%%% Local Variables:
%%% mode: latex
%%% TeX-master: "../../paper"
%%% End:
