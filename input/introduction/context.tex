
\section{Contexte}

L'adoption des outils collaboratifs par le grand public n'a eu de cesse
d'augmenter ces dernières décénies. En particulier, les éditeurs
collaboratifs~\cite{ellis1991groupware} remportent un vif succès puisqu'ils
permettent de répartir la charge d'écriture d'un document entre différents
auteurs. Ainsi, quelle que soit l'heure, quel que soit le lieu, un utilisateur
peut écrire son document numérique en association avec ses collaborateurs dont
il ferat bénéficier son expertise. La combinaison de ces expertises permet de
produire des documents de meilleure qualité. Ainsi, l'encyclopédie
\emph{Wikipédia} est le résultat de la collaboration de millions d'auteurs dont
la version anglaise possède une fiabilité combarable à l'\emph{Encyclop\ae{}dia
Britannica}~\cite{giles2005internet}.

Bien qu'étant d'une utilité indéniable, les éditeurs collaboratifs actuels ne
sont pas sans défaut. Les éditeurs basés sur serveur central, tel que Google
Docs~\cite{nichols1995high}, posent des problèmes liés à la confidentialité et
au passage à l'échelle. L'inquiètude du public vis-à-vis des fournisseurs de
services s'est avérée légitime~\cite{gellman2013us}. Les éditeurs débarassés de
serveur central résolvent les problèmes de confidentialité. Toutefois, les
problèmes de passage à l'échelle deumeurent.

%% Les applications pair-à-pairs place chaque utilisateur dans le rôle de client
%% et serveur. Ainsi, non seulement les utilisateurs profitent de l'application
%% normalement, mais participent au bon fonctionnement de celle-ci. Dès lors, le
%% serveur central n'est plus nécessaire.

Afin de garantir des documents accessibles et réactifs, les éditeurs
collaboratifs utilisent la réplication optimiste 

Les éditeurs collaboratifs décentralisés utilisent la réplication
optimiste~\cite{saito2005optimistic} afin de garantir accessibilité et
réactivité des documents partagés. Dans ce type de réplication, chaque
collaborateur possède une réplique locale du document et exécute directement ses
modifications dessus, qui seront partagés dans un second temps à l'ensemble des
participants. Les répliques du documents peuvent diverger temporairement, mais
convergent inéluctablement vers un état identique.

Dans ce manuscrit, nous nous intéressons aux structures de données répliquées
sans résolution de conflits~\cite{shapiro2011comprehensive, shapiro2011conflict}
(CRDTs) appartenant au paradigme de réplication optimiste. Les CRDTs pour
séquences~\cite{ahmed2011evaluating, conway2014language, grishchenko2010deep,
  oster2006data, preguica2009commutative, roh2011replicated, weiss2007wooki,
  wu2010partial, Yu2012stringwise, andre2013supporting, weiss2009logoot}
(structure la plus proche du document) fournissent deux opérations pour modifier
la séquence : l'insertion et la suppression d'un élément. Dans le cadre d'un
document et selon la granularité choisie, ces éléments peuvent être des
caractères, des lignes, des paragraphes etc. Ces deux opérations sont
commutatives, i.e., l'ordre d'intégration de ces opérations n'importe pas. Lors
d'une insertion, le CRDT génère un identificateur unique et immuable qui lui
servira à ordonner la séquence. L'une des familles de CRDTs pour séquence
utilise des identifiants dont la taille est
variable~\cite{preguica2009commutative,
  andre2013supporting,weiss2009logoot}. Dans ces approches, la principale
difficulté consiste à conserver des identifiants de petite taille. La première
contribution présentée dans ce manuscrit concerne une stratégie d'allocation
d'identifiants dont la taille est polylogarithmique par rapport au nombre
d'insertions effectuées sur la séquence~\cite{nedelec2013lseq,
  nedelec2013concurrency}.

Les CRDTs garantissent la cohérence à terme~\cite{bailis2013eventual} sous
l'hypothèse que les opérations arrivent à tous les participants de manière
inéluctable. Les CRDTs nécessitent donc un mécanisme de diffusion des
messages. La dissémination épidémique~\cite{demers1987epidemic,
  eugster2003lightweight, birman1999bimodal} (ou rumeur) est un moyen efficace
d'y parvenir. Chaque pair choisit une liste de pairs et envoie le message. Lors
de la réception d'un tel message, le pair peut choisir d'arrêter la diffusion du
message, ou de le transmettre à son tour à une liste de pairs. Ainsi, selon
toute probabilité, tous les pairs reçoivent le message. Pour obtenir les liste
de pairs, le mécanisme de dissémination peut s'appuyer sur les protocoles
d'échantillonnage aléatoire de pairs~\cite{jelasity2007gossip,
  voulgaris2005cyclon, ganesh2001scamp, tolgyeski2009adaptive,
  eugster2003lightweight}. Ces derniers maintiennent chez chaque pair une liste
de voisins comme vue partielle du réseau entier. Les réseaux générés partagent
de nombreuses similarités avec les graphes aléatoires~\cite{erdos1959random}. En
particulier, ils permettent de maintenir efficacement la connectivité du réseau,
la dissémination d'information, la robustesse aux défaillances etc. La seconde
contribution présentée dans ce manuscrit concerne un protocole d'échantillonnage
aléatoire dont les vues partielles s'adaptent automatiquement à la taille du
réseau, convergeant en temps exponentiel vers une topologie montrant des
similarités avec les graphes aléatoires, et utilisant seulement des interactions
de proche en proche.

Comparé aux approches décentralisées, les éditeurs collaboratifs centralisés ont
l'avantage d'être facile d'accès pour l'utilisateur. Par exemple, Google Docs
est accessible depuis un navigateur web quelconque. Partager un document est
aisé puisqu'il s'agit simplement de donner un lien que le collaborateur puisse
adresser. Toutefois, la récente technologie
WebRTC\footnote{\url{http://www.webrtc.org}} comble ce fossé en permettant la
communication de navigateur à navigateur, et ce, même avec des configurations
réseau complexes impliquant firewall, proxy, ou NAT (Network Address
Translation). En particulier, WebRTC étend le champs d'utilisateurs aux
dispositifs limités en ressource comme les smartphones, tablettes etc. Dans la
troisième partie de ce manuscrit est présenté CRATE (CollaboRATive Editor), un
éditeur collaboratif décentralisé et réparti dont le coeur est composé de LSEQ
(la stratégie d'allocation présenté en première partie) et Spray (le protocole
d'échantillonnage présenté en seconde partie) et accessible depuis un navigateur
web.

%%% Local Variables:
%%% mode: plain-tex
%%% TeX-master: "../../paper"
%%% End:
