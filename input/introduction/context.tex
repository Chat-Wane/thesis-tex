
\section{Contexte}

L'adoption des outils collaboratifs par le grand public n'a eu de cesse
d'augmenter ces dernières décénies. En particulier, les éditeurs
collaboratifs~\cite{ellis1991groupware} remportent un vif succès car ils
permettent de répartir la charge d'écriture d'un document entre différents
auteurs. Ainsi, quelle que soit l'heure, quel que soit le lieu, un utilisateur
peut rédiger son document numérique en association avec ses collaborateurs dont
les expertises combinées permettront de produire des documents de meilleure
qualité. Par exemple, l'encyclopédie \emph{Wikipédia} est le résultat de la
collaboration de millions d'auteurs dont la version anglaise possède une
fiabilité combarable à celle de l'\emph{Encyclop\ae{}dia
Britannica}~\cite{giles2005internet}.

Bien qu'étant d'une utilité indéniable, les éditeurs collaboratifs actuels ne
sont pas sans défaut. Les éditeurs basés sur serveur central, tel que Google
Docs~\cite{nichols1995high}, posent des problèmes liés à la confidentialité et
au passage à l'échelle. L'inquiètude du public vis-à-vis des fournisseurs de
services s'est par ailleurs avérée légitime~\cite{gellman2013us}. D'un autre
coté, les éditeurs débarassés de serveur central résolvent les problèmes de
confidentialité. Toutefois, les problèmes de passage à l'échelle deumeurent.

%% Les applications pair-à-pairs place chaque utilisateur dans le rôle de client
%% et serveur. Ainsi, non seulement les utilisateurs profitent de l'application
%% normalement, mais participent au bon fonctionnement de celle-ci. Dès lors, le
%% serveur central n'est plus nécessaire.

Afin de garantir des documents accessibles et réactifs aux changements, les
éditeurs collaboratifs utilisent la réplication
optimiste~\cite{saito2005optimistic} qui consiste à copier le document
localement afin de le modifier directement. Dans un second temps, les
modifications sont envoyées à tous les participants qui les appliques sur leur
propre copie.  Ces dernières peuvent diverger temporairement, mais convergent
inéluctablement vers un état identique~\cite{bailis2013eventual}.

Dans ce manuscrit, nous nous intéressons tout d'abord aux structures de données
répliquées dîtes \emph{sans résolution de conflits
(CRDTs)}~\cite{shapiro2011comprehensive, shapiro2011conflict}. Ce genre de
structure~\cite{ahmed2011evaluating, andre2013supporting, conway2014language,
grishchenko2010deep, oster2006data, preguica2009commutative, roh2011replicated,
weiss2007wooki, weiss2009logoot, wu2010partial, yu2012stringwise} fournit deux
opérations élémentaires pour éditer le document : l'insertion et la suppression
d'un élément pouvant, selon la granularité choisie, être un caractère, une
ligne, ou un paragraphe etc. La particularité de ces structures réside dans la
commutativité de ces opérations qui allège le coût de celles-ci grâce à plus de
fléxibilité.

Ces structures de données répliquées génèrent un identificateur par caractère
inséré dans le document. Ces identificateurs permettent d'ordonner les
caractères quel que soit l'ordre d'intégration des modifications. Toutefois ils
représentent un surcoût par rapport au simple document. Plusieurs approches
proposant différents compromis existent. Cette thèse s'interesse plus
précisément à la famille dont les identifiants ont une taille variable à la
génération~\cite{andre2013supporting, preguica2009commutative, weiss2009logoot}.
Il est crucial de maintenir cette taille bornée afin que la structure repliquée
représentant le document reste d'acceptable proportion.

Dans ce manuscrit, nous nous intéressons dans un second temps à la dissémination
des modifications aux participants. En effet, afin de garantir la cohérence à
terme~\cite{bailis2013eventual}, les structures de données répliquées émettent
l'hypothèse que tous les participants reçoivent toutes les modifications
inéluctablement. La dissémination épidémique~\cite{birman1999bimodal,
demers1987epidemic, eugster2003lightweight} (ou rumeur) est un moyen efficace
d'y parvenir. Pour cela, chaque participant possède une liste de voisins membre
de la session d'édition avec lesquels communiquer. Chaque modification effectuée
sur le document est transmise au voisinage qui la transmet à son tour à son
voisinage etc. Par transitivité, tous les participants reçoivent les
modifications.

Pour obtenir les liste de membres, le mécanisme de dissémination peut s'appuyer
sur les protocoles d'échantillonnage aléatoire de
pairs~\cite{eugster2003lightweight, ganesh2001scamp, jelasity2007gossip,
tolgyeski2009adaptive, voulgaris2005cyclon}. Les réseaux générés partagent de
nombreuses similarités avec les graphes aléatoires~\cite{erdos1959random} tels
la robustesse aux défaillances, un faible diamêtre, etc. Hélas, la plupart de
ces approches n'adaptent pas leur fonctionnement à la taille réelle du réseau.
\TODO{Moar about problem.}


% Comparé aux approches décentralisées, les éditeurs collaboratifs centralisés ont
% l'avantage d'être facile d'accès pour l'utilisateur. Par exemple, Google Docs
% est accessible depuis un navigateur web quelconque. Partager un document est
% aisé puisqu'il s'agit simplement de donner un lien que le collaborateur puisse
% adresser. Toutefois, la récente technologie
% WebRTC\footnote{\url{http://www.webrtc.org}} comble ce fossé en permettant la
% communication de navigateur à navigateur, et ce, même avec des configurations
% réseau complexes impliquant firewall, proxy, ou NAT (Network Address
% Translation). En particulier, WebRTC étend le champs d'utilisateurs aux
% dispositifs limités en ressource comme les smartphones, tablettes etc. Dans la
% troisième partie de ce manuscrit est présenté CRATE (CollaboRATive Editor), un
% éditeur collaboratif décentralisé et réparti dont le coeur est composé de LSEQ
% (la stratégie d'allocation présenté en première partie) et Spray (le protocole
% d'échantillonnage présenté en seconde partie) et accessible depuis un navigateur
% web.

%%% Local Variables:
%%% mode: plain-tex
%%% TeX-master: "../../paper"
%%% End:
