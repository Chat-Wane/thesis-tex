
\section{Contexte}

\lettrine{L}'édition collaborative concerne toutes les activités effectuées en
groupe dans le but de produire un document~\cite{ellis1989concurrency,
  johansen1988groupware}. Grâce à un effort collectif, les rédacteurs
bénificient de multiples points de vues et deviennent plus impliqués dans
l'écriture~\cite{noel2004empirical}. Les documents en résultant sont de
meilleure qualité. Ainsi l'encyclopédie \emph{Wikipédia} compte des millions
d'articles rédigés par des millions d'auteurs et sa version anglaise possède une
fiabilité combarable à celle de l'\emph{Encyclopædia
  Britannica}~\cite{giles2005internet}.

Les éditeurs collaboratifs de textes sont simplement des outils facilitant
l'écriture de documents par plusieurs auteurs. Ils présentent un texte dont le
contenu est modifiable par l'utilisateur. Les modifications sont transmises aux
autres participants qui peuvent alors les voir et écrire en conséquence. Grâce à
ces outils, la tâche d'écriture peut être répartie selon le temps, l'espace, et
les organisations. Ainsi, il n'est plus impératif de se retrouver dans la même
pièce ni même de se retrouver au même moment pour participer à la rédaction du
document~\cite{johansen1988groupware}.

Depuis la présentation du premier éditeur de ce type par Engelbart en
1968~\cite{engelbart1968research}, les éditeurs collaboratifs ont
fleuri. Toutefois, leur adoption massive par le public n'est que récente et
provient principalement des éditeurs web~\cite{mogan2010impact,
  perkel2014scientific} tels que \emph{Google Docs}~\cite{googledocs} ou
\emph{ShareLatex}~\cite{sharelatex}. Grâce au web, je peux créer, et éditer
facilement mon document depuis mon ordinateur ou mon portable. Un simple lien me
permet de le partager avec mes amis ou mes collègues. L'édition collaborative
devient à la portée de tous. Hélas, l'utilisation du web comme support à la
collaboration apporte aussi son lot de défauts. En effet, son organisation
éminemment centralisée pose les problèmes suivants :

% Pour se faire, un utilisateur crée le document et le partage avec ses
% collaborateurs. Ces derniers sont alors en mesure de visionner et de modifier le
% document. L'édition \emph{asynchrone}~\cite{johansen1988groupware} divise la
% tâche entre l'édition et la synchronisation des modifications. L'échange de
% documents par courriels ou les gestionnaires de versions \emph{Git}~\cite{git}
% ou \emph{Subversion}~\cite{svn} font parties de cette catégorie. L'édition
% \emph{temps réel}~\cite{ellis1991groupware, johansen1988groupware} permet de
% visionner et partager les changements immédiatement. Ainsi, l'ajout d'une lettre
% dans le document par un participant se répercute directement chez tous le monde
% (modulo la latence sur les canaux permettant de transmettre les
% changements). L'éditeur web \emph{Google Docs}~\cite{googledocs} appartient à
% cette catégorie.  Dans ce manuscrit, nous nous intéressons à ces deux modes
% d'édition représentant respectivement les modes d'édition ``hors ligne'' et ``en
% ligne''.

% En dépit des indéniables améliorations apportées par l'édition
% collaborative~\cite{noel2004empirical}, son adoption massive par le public
% demeure récente~\cite{perkel2014scientific}. L'intervention du web participe
% pour beaucoup à la popularité des éditeurs
% collaboratifs~\cite{mogan2010impact}. Les utilisateurs accèdent facilement à
% leurs documents via le navigateur web de leur ordinateur ou portable.  Les
% utilisateurs partagent leurs documents aisément grâce à de simple URL
% (\emph{Uniform Resource Locator}).



\paragraph{Confidentialité :} Chaque lettre tappée transite par un serveur
appartenant à un fournisseur de services. Ce dernier est en mesure de lire le
document et d'en vendre ou d'en concéder le contenu à des tiers tels que des
entreprises publicitaires ou des entités du gouvernement~\cite{gellman2013us};

\paragraph{Censure :} Bien souvent, la propriété du document doit être concédée
au fournisseur de service. À tout moment, celui-ci peut se voir supprimé, au bon
vouloir du fournisseur, si son contenu est jugé contradictoire aux termes du
service;

\paragraph{Passage à l'échelle :} Un serveur est en charge du document. Si le
nombre d'utilisateurs grimpe, le serveur risque de ne pouvoir le supporter. Les
utilisateurs souffrent alors de baisses en qualité de service. Par exemple, leur
document peut devenir temporairement inaccessible;

\paragraph{Point individuel de défaillance :} Puisqu'un seul serveur est en
charge du document, si celui-ci tombe en panne, l'accès aux documents qu'il
héberge est compromis.
  
%% penser a parler du coût réduit en terme d'infrastructure

La décentralisation permet de résoudres les problèmes liés à la confidentialité,
à la censure, et aux défaillances. Un document appartient réellement à ceux qui
l'éditent. Toutefois, les problèmes de passage à l'échelle demeurent auxquels
s'ajoutent les difficultés d'utilisations autrefois résolues grâce au web et au
Nuage~\cite{mell2011national}. Heureusement, la récente technologie
WebRTC~\cite{webrtc} permet d'établir des canaux de communication d'un
navigateur web à l'autre. Cette technologie offre la possibilité de développer
des applications aussi faciles d'accès qu'un service web tout en se passant de
leur fournisseur. Grâce à cela, un nouvel écosystème d'applications
décentralisées directement intégrées au web est en train de
fleurir~\cite{webtorrent}.

\textbf{L'édition collaborative temps réel est-elle possible sur le web, sans
  l'intervention de tiers et sans limites quant aux dimensions du système?}


% \lettrine{L}'émergence de l'outil textuel comme relai de la parole a permis de
% nombreuses avancées sur le plan scientifique, juridique, culturel
% etc. Récemment, l'informatique a étendu l'édition textuelle à plusieurs auteurs
% en facilitant les interactions distantes. Si l'objectif de collaboration est
% demeuré inchangé depuis les premiers éditeurs~\cite{engelbart1968research}, les
% dimensions du système, quant à elles, ont crû.


% En effet, l'adoption des éditeurs collaboratifs de
% textes~\cite{ellis1991groupware} n'a eu de cesse d'augmenter ces dernières
% décénies.
% % Tout le monde écrit, et tout le monde souhaite partager ses écrits.
% Quelle que soit l'heure, quel que soit le lieu, un auteur peut rédiger son
% document numérique avec ses collaborateurs dont les expertises combinées en
% amélioreront la qualité. Ainsi, l'encyclopédie \emph{Wikipédia} compte des
% millions d'articles rédigés par des millions d'auteurs et sa version anglaise
% possède une fiabilité combarable à celle de l'\emph{Encyclopædia
%   Britannica}~\cite{giles2005internet}.

% Améliorant à la fois efficacité et implication des
% utilisateurs~\cite{noel2004empirical}, les éditeurs collaboratifs sont d'une
% indéniable utilité. Toutefois, les outils proposés actuellement ne sont pas sans
% défaut. Les éditeurs basés sur serveur central, tel que \emph{Google
%   Docs}~\cite{googledocs}, posent des problèmes de
% confidentialité~\cite{gellman2013us}, de passage à l'échelle, et de tolérence
% aux pannes.
% % L'inquiètude du public vis-à-vis des fournisseurs de services s'est par
% %ailleurs avérée légitime lors des fuites sur
% %\emph{PRISM}~\cite{gellman2013us}.
% Afin de se réconcilier avec le principe de l'internet sans autorité centrale
% opérant de manière complètement répartie, les éditeurs doivent se débarasser de
% ces serveurs. Grâce à cela, les problèmes de confidentialité s'en trouvent
% résolus. Toutefois, les problèmes de passage à l'échelle subsitent auxquels
% s'ajoutent les difficultés d'utilisation autrefois résolues par le
% Nuage~\cite{mell2011national}.


%%% Local Variables:
%%% mode: latex
%%% TeX-master: "../../paper"
%%% End:
