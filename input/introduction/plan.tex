
\section{Plan du manuscrit}

Le reste du manuscrit est découpé en 4 chapitres :

\paragraph{Le chapitre~\ref{repl:chap:lseq}} décrit la structure de données
répliquée représentant le document partagé. Il présente les approches
appartenant à la réplication dite optimiste~\cite{demers1987epidemic,
  saito2005optimistic} : les approches à transformées
opérationnelles~\cite{sun1998achieving, sun2009contextbased} et les approches à
structures de données sans conflits~\cite{burckhardt2014replicated,
  shapiro2011conflict}. Le chapitre définit le problème scientifique à
résoudre. Ensuite, il détaille \LSEQ~\cite{nedelec2013concurrency,
  nedelec2013lseq}, une fonction d'allocation d'identifiants conçue pour le
second type d'approches et dont la taille des identifiants est polylogarithmique
par rapport au nombre d'insertions effectuées dans le document. Le reste du
chapitre s'attache à valider cette propriété.

\paragraph{Le chapitre~\ref{net:chap:spray}} commence par décrire les protocoles
qui nous permettent de former les sessions d'édition : les protocoles
d'échantillonnage aléatoire de pairs~\cite{jelasity2004peer,
  jelasity2007gossip}. Ceux-ci fournissent une vue partielle du réseau à chaque
membre. Cette vue peut être utilisée afin de communiquer avec l'ensemble du
réseau. L'état de l'art se décompose en deux familles :
\begin{inparaenum}[(i)]
\item Les vues partielles sont de taille constante configurée \emph{a
    priori}~\cite{eugster2003lightweight, jelasity2007gossip,
    leitao2007dependable, tolgyeski2009adaptive, voulgaris2005cyclon};
\item Les vues partielles s'adaptent à la volée aux dimensions du
  réseau~\cite{ganesh2001scamp, ganesh2003peer}.
\end{inparaenum}
Le chapitre définit le problème scientifique à résoudre. Ensuite, il décrit le
fonctionnement de \SPRAY~\cite{nedelec2015spray} appartenant à la seconde
famille d'approches. Le cycle de vie d'un pair est décomposé entre le moment où
il rejoint, le temps où il est membre, et le moment où il se retire. Tout au
long de ce cycle, le système dans sa globalité doit rester cohérent sur la
taille des vues partielles ainsi que sur leur contenu.  Le chapitre décrit les
propriétés assurées par \SPRAY. Le chapitre montre également les avantages
obtenus par un système de diffusion de messages profitant de \SPRAY.

\paragraph{Le chapitre~\ref{editor:chap:crate}} décrit
\CRATE~\cite{nedelec2016crate}, un éditeur collaboratif temps réel décentralisé
fonctionnant dans les navigateurs Web.  Tout d'abord,
le chapitre décrit les composants employés par \CRATE.  Profitant des propriétés
de \SPRAY et de \LSEQ, \CRATE parvient à adapter le trafic généré par l'édition
à la taille de la session et à la taille du document. Ensuite, des
expérimentations à large échelle confirment cette analyse.

\paragraph{Le chapitre~\ref{conclu:chap:conclusion}} clôt ce manuscrit et décrit
quelques perspectives ouvertes par ce type d'application ainsi que les défis
qu'il reste à relever.


%%% Local Variables:
%%% mode: latex
%%% TeX-master: "../../paper"
%%% End:
