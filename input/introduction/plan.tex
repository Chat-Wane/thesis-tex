
\section{Plan du manuscrit}

Le reste du manuscrit est découpé en 3 parties :

\begin{itemize}
\item [\textbf{La première partie :}] décrit le protocole d'échantillonnage
  aléatoire de pairs \SPRAY.
\item [\textbf{La seconde partie :}] décrit la structure de données répliquée
  représentant le document partagé. Cette partie est composée des
  chapitres~\ref{repl:chap:replication},
  ~\ref{repl:chap:crdts},~\ref{repl:chap:sequences},~\ref{repl:chap:lseq}. Le
  chapitre~\ref{repl:chap:replication} présente les deux méthodes de réplication
  de données. Le chapitre~\ref{repl:chap:crdts} présente un type de données
  appartenant à la réplication dîte optimiste. Ce type de données à pour but
  d'éviter la difficile tâche de résolution de conflits apparaissant lors
  d'opérations concurrentes. Il en existe pour les compteurs, les ensembles, les
  arbres etc. Le chapitre~\ref{repl:chap:sequences} s'intéresse particulièrement
  à ce type de données pour les séquences. Il en distingue deux familles : les
  approches utilisant des 'pierres tombales' et les approches dont les
  métadonnées sont de taille variable à la génération. Le
  chapitre~\ref{repl:chap:lseq} s'attache à décrire \LSEQ, une stratégie
  d'allocation de métadonnées dont la taille reste sous-linéaire par rapport à
  la taille du document.
\item [\textbf{La troisième partie :}] conclue ce manuscrit de thèse en
  présentant \CRATE, un éditeur collaboratif décentralisé tournant dans les
  navigateurs web. 
\end{itemize}


%% La partie ainsi que son sous-découpage.

%%% Local Variables:
%%% mode: latex
%%% TeX-master: "../../paper"
%%% End:
