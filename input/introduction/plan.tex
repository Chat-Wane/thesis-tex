
\section{Plan du manuscrit}

Le reste du manuscrit est découpé en 4 chapitres :

\paragraph{Le chapitre~\ref{repl:chap:lseq}} décrit la structure de données
répliquée représentant le document partagé. Il commence par la présentation des
deux grands schémas de réplication avant de passer en revue les approches
appartenant à la réplication dite optimiste~\cite{demers1987epidemic,
  saito2005optimistic}. En particulier, les approches à transformées
opérationnelles et les approches à structure de données sans résolution de
conflits~\cite{burckhardt2014replicated, shapiro2011conflict}. Ensuite, le
chapitre détaille \LSEQ~\cite{nedelec2013concurrency, nedelec2013lseq}, une
fonction d'allocation d'identifiants conçue pour ces dernières et dont la taille
des identifiants est bornée supérieurement selon le polylogarithme de la taille
du document. Le reste du chapitre s'attache à valider cette propriété.

\paragraph{Le chapitre~\ref{net:chap:spray}} commence par décrire les protocoles
réseau qui nous permettront de construire les sessions d'édition : les
protocoles d'échantillonnage aléatoire de pairs~\cite{jelasity2004peer,
  jelasity2007gossip}. Ceux-ci fournissent une vue partielle du réseau à chaque
membre. Cette vue peut être utilisée afin de communiquer avec l'ensemble du
réseau. L'état de l'art se décompose en deux familles :
\begin{inparaenum}[(i)]
\item Les vues partielles sont de taille constante configurée \emph{a
    priori}~\cite{eugster2003lightweight, jelasity2007gossip,
    leitao2007dependable, tolgyeski2009adaptive, voulgaris2005cyclon};
\item Les vues partielles s'adaptent aux dimensions du réseau à la
  volée~\cite{ganesh2001scamp, ganesh2003peer}.
\end{inparaenum}
Ensuite, le chapitre décrit le fonctionnement de \SPRAY~\cite{nedelec2015spray}
appartenant à cette seconde famille. Le cycle de vie d'un pair est décomposé
entre le moment où il rejoint, le temps où il reste membre du réseau, et le
moment où il le quitte. Tout au long de ce processus, le système dans sa
globalité doit rester cohérent sur la taille des vues partielles ainsi que sur
leur contenu.  Le chapitre décrit les propriétés assurées par \SPRAY et les
compare à un représentant de l'état de l'art nommé
\CYCLON~\cite{voulgaris2005cyclon}.

\paragraph{Le chapitre~\ref{editor:chap:crate}} décrit
\CRATE~\cite{nedelec2016crate}, un éditeur collaboratif temps réel décentralisé
fonctionnant dans les navigateurs web.  Tout d'abord, le chapitre décrit la
traditionnelle architecture client-serveur employée des éditeurs web bien connus
tels que Google Docs~\cite{googledocs} ou Etherpad~\cite{etherpad}.  Ensuite, il
présente la récente technologie WebRTC~\cite{webrtc} ayant ouvert la voie aux
applications décentralisées sur navigateurs. Le chapitre décrit les composants
employés par \CRATE.  Profitant des propriétés de \SPRAY et de \LSEQ, \CRATE
parvient à adapter le trafic généré par l'édition à la taille de la session et à
la taille du document. Des expérimentations à large échelle confirment cette
analyse.

\paragraph{Le chapitre~\ref{conclu:chap:conclusion}} clôt ce manuscrit et décrit
les perspectives ouvertes par ce type d'applications ainsi que les défis qu'il
reste à relever.


%%% Local Variables:
%%% mode: latex
%%% TeX-master: "../../paper"
%%% End:
