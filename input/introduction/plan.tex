
\section{Plan du manuscrit}

Le reste du manuscrit est découpé en 3 parties :

\begin{itemize}
\item [\textbf{La première partie :}] décrit le protocole réseau permettant de
  créer une session d'édition. Elle est composée des
  chapitres~\ref{diff:chap:rps},~\ref{diff:chap:spray}. Le
  chapitre~\ref{diff:chap:rps} présente l'état de l'art de ces protocoles
  d'échantillonnage aléatoire de pairs. Ceux-ci fournissent une vue partielle du
  réseau à chaque membre laquelle peut être utilisée afin de communiquer à
  l'ensemble du réseau. Nous en distinguons deux familles : La première fournit
  des vues dont la taille est constante quelle que soit la taille du réseau, en
  opposition à la seconde famille qui adapte la taille des vues en fonction du
  réseau sans toutefois avoir de connaissances globales sur ce dernier. Le
  chapitre~\ref{diff:chap:spray} décrit \SPRAY, un protocole d'échantillonnage
  aléatoire de pairs appartenant à cette seconde famille. Les vues s'ajustent
  automatiquement au logarithme de la taille du réseau.
\item [\textbf{La seconde partie :}] décrit la structure de données répliquée
  représentant le document partagé. Cette partie est composée des
  chapitres~\ref{repl:chap:replication},
  ~\ref{repl:chap:crdts},~\ref{repl:chap:sequences},~\ref{repl:chap:lseq}. Le
  chapitre~\ref{repl:chap:replication} présente les deux méthodes de réplication
  de données. Le chapitre~\ref{repl:chap:crdts} présente un type de données
  appartenant à la réplication dîte optimiste. Ce type de données a pour
  objectif d'éviter la difficile tâche de résolution de conflits apparaissant
  lors d'opérations concurrentes. Il en existe pour les compteurs, les
  ensembles, les arbres etc. Le chapitre~\ref{repl:chap:sequences} s'intéresse
  particulièrement à ce type de données pour les séquences. Il en distingue deux
  familles : les approches utilisant des 'pierres tombales' et les approches
  dont les métadonnées sont de taille variable à la génération. Le
  chapitre~\ref{repl:chap:lseq} s'attache à décrire \LSEQ, une stratégie
  d'allocation de métadonnées dont la taille reste sous-linéaire par rapport à
  la taille du document.
\item [\textbf{La troisième partie :}] conclue ce manuscrit de thèse. Elle est
  composé des
  chapitres~\ref{conclu:chap:crate},~\ref{conclu:chap:conclusion}. Le
  chapitre~\ref{conclu:chap:crate} présente \CRATE, un éditeur collaboratif
  décentralisé tournant dans les navigateurs web. Profitant des propriétés de
  \SPRAY et \LSEQ, \CRATE parvient à adapter le traffic généré par l'édition à
  la taille de la session et à la taille du document. Le
  chapitre~\ref{conclu:chap:conclusion} clot se manuscrit en décrivant les
  perspectives ouvertes par ce type d'application ainsi que les défis qu'il
  reste à relever.
\end{itemize}


%% La partie ainsi que son sous-découpage.

%%% Local Variables:
%%% mode: latex
%%% TeX-master: "../../paper"
%%% End:
