
\section{Plan du manuscrit}

Le reste du manuscrit est découpé en 4 chapitres :


\textbf{Le chapitre~\ref{net:chap:spray}} commence par décrire les protocoles
réseau qui nous permettront de construire les sessions d'édition : les
protocoles d'échantillonnage aléatoire de pairs. Ceux-ci fournissent une vue
partielle du réseau à chaque membre. Cette vue peut être utilisée afin de
communiquer à l'ensemble du réseau. L'état de l'art se décompose en deux
familles :
\begin{inparaenum}[(i)]
\item Les vues partielles sont de taille constante configurée \emph{a priori};
\item Les vues partielles s'adaptent aux dimensions du réseau à la volée.
\end{inparaenum}
Ensuite, le chapitre décrit le fonctionnement de \SPRAY appartenant à cette
seconde famille. Le cycle de vie d'un pair est décomposé entre le moment où il
rejoint, le temps où il reste membre du réseau, et le moment où il le
quitte. Tout au long de ce processus, le système dans sa globalité doit rester
consistant sur la taille des vues partielles ainsi que sur leur contenu.  Le
chapitre décrit les propriétés assurées par \SPRAY et les compares à un
représentant de l'état de l'art nommé \CYCLON.

\textbf{Le chapitre~\ref{repl:chap:lseq}} décrit la structure de données
répliquée représentant le document partagé. Il commence par la présentation des
deux grands schémas de réplication avant de passer en revue les approches
appartenant à la réplication dîte optimiste. En particulier, les approches à
transformés opérationnels et les approches à structure de données sans
résolution de conflits. Ensuite, le chapitre détaille \LSEQ, une fonction
d'allocation d'identifiants conçue pour ces dernières et dont la taille est
bornée supérieurement selon le polylogarithme de la taille du document. Le reste
du chapitre s'attache à valider cette propriété.

\textbf{Le chapitre~\ref{editor:chap:crate}} décrit \CRATE, un éditeur
collaboratif réparti et décentralisé fonctionnant dans les navigateurs web.
Tout d'abord, le chapitre décrit la récente technologie WebRTC ayant ouvert la
voie aux applications décentralisée sur navigateurs. Il décrit mode opératoire
afin d'entrer dans une session d'édition. Ensuite, le chapitre décrit les
fonctionnalités offertes par \CRATE.  Profitant des propriétés de \SPRAY et
\LSEQ, \CRATE parvient à adapter le traffic généré par l'édition à la taille de
la session et à la taille du document.

\textbf{Le chapitre~\ref{conclu:chap:conclusion}} clôt ce manuscrit et décrit
les perspectives ouvertes par ce type d'application ainsi que les défis qu'il
reste à relever.


%%% Local Variables:
%%% mode: latex
%%% TeX-master: "../../paper"
%%% End:
