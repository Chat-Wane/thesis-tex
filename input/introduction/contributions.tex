
\section{Contributions}

La première question de recherche nous place dans le contexte de la réplication
optimiste~\cite{demers1987epidemic, saito2005optimistic}, i.e., un schéma de
réplication où le document partagé est copié chez chaque utilisateur afin de
fournir réactivité et disponibilité. La structure de séquence constitue une
abstraction proche du document. Par conséquent, nous nous intéressons
particulièrement à un type de données pour séquences dont les opérations
commutent par nature~\cite{burckhardt2014replicated, shapiro2011comprehensive,
  shapiro2011conflict, zawirski2015dependable}. Ce type présente l'avantage de
supporter efficacement les opérations concurrentes~\cite{ahmed2015evaluation,
  ahmed2011evaluating}. Ces structures de données associent un identifiant
unique et immuable à chaque élément de la séquence. Nous proposons
\LSEQ~\cite{nedelec2013concurrency, nedelec2013lseq}, une fonction d'allocation
d'identifiants. Les identifiants générés ont une taille bornée
polylogarithmiquement par rapport au nombre d'insertions effectuées dans la
séquence. Nous définissons les conditions sous lesquelles s'applique cette borne
et en fournissons la preuve. Au travers de simulations, nous validons \LSEQ et
sa complexité.

Afin de répondre à la seconde question de recherche, nous nous sommes interessés
aux protocoles d'appartenance aux réseaux (\emph{membership}). Plus
particulièrement à ceux dont la topologie résultante possède des propriétés
similaires à celles des graphes aléatoires~\cite{erdos1959random}. Parmi ces
propriétés se trouve la capacité à gérer les connexions et déconnexions
fréquentes. Ces protocoles sont regroupés sous l'appellation de \og protocoles
d'échantillonnage aléatoire de pairs \fg~\cite{jelasity2004peer,
  jelasity2007gossip} et servent de base à de nombreux protocoles
répartis~\cite{dabek2004vivaldi, folz2016cyclades, montresor2005chord}. Nous
proposons une nouvelle approche nommée \SPRAY~\cite{nedelec2015spray}
appartenant à cette famille. Comparé à l'état de
l'art~\cite{eugster2003lightweight, ganesh2001scamp, jelasity2007gossip,
  leitao2007dependable, tolgyeski2009adaptive, voulgaris2005cyclon}, \SPRAY
s'adapte automatiquement à la taille du réseau en suivant une progression
logarithmique comparativement à la taille globale du réseau tout en supportant
les contraintes imposées par le processus complexe d'établissement de connexion
dans les navigateurs Web. Nous mettons cela en évidence grâce à des simulations.
Les protocoles de diffusion épidémique~\cite{birman1999bimodal} de messages
(\emph{gossip}) qui en dépendent bénéficient à leur tour de cette capacité à
s'adapter. Ainsi, le trafic généré peut s'adapter à la taille réelle du réseau.

Afin de répondre à la problématique générale de ce manuscrit, nous avons réuni
les deux approches \LSEQ et \SPRAY dans un éditeur collaboratif décentralisé
nommé \CRATE~\cite{nedelec2016crate}. Ce dernier fonctionne directement dans les
navigateurs Web. Pour peu que l'utilisateur ait un accès à l'internet, \CRATE
lui permet l'édition de documents n'importe quand, n'importe où, avec autant de
collaborateurs qu'il le souhaite, sans fournisseur tiers. Comparé à l'état de
l'art~\cite{etherpad, googledocs, hivejs, lautamaki2012cored,
  nicolaescu2015yjs},
\begin{inparaenum}[(i)]
\item \CRATE constitue un premier pas vers une solution aux problèmes liés à la
  confidentialité;
\item \CRATE résout les problèmes de passage à l'échelle;
\item \CRATE conserve la facilité d'utilisation propre aux éditeurs Web.
\end {inparaenum}
Nous validons \CRATE aux travers d'expériences réunissant jusqu'à 600 éditeurs
sur le banc d'essai Grid'5000. Les résultats confirment le facteur logarithmique
de \SPRAY et la taille polylogarithmique des identifiants générés par \LSEQ.



%%% Local Variables:
%%% mode: latex
%%% TeX-master: "../../paper"
%%% End:
