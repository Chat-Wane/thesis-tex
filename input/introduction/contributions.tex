
\section{Contributions}


Afin de répondre à la question de recherche \textbf{QR A.}, nous nous sommes
interessé dans un premier temps aux protocoles d'appartenance
(\emph{membership}) aux réseaux. Plus particulièrement à ceux dont la topologie
résultante possède des propriétés similaires à celles des graphes
aléatoires. Parmi ces propriétés se trouve la capacité à gérer les connexions et
deconnexions fréquentes. Ces protocoles sont regroupés sous l'appellation de
protocoles d'échantillonnage aléatoire de pairs et servent de base à de nombreux
protocoles répartis (\REF). Nous proposons une nouvelle approche nommée \SPRAY
appartenant à cette famille. Comparé à l'état de l'art, \SPRAY s'adapte à la
taille du réseau automatiquement en suivant une progression logarithmique
comparativement à la taille globale du réseau. Nous mettons cela en évidence
grâce à des simulations.  Les protocoles de diffusion épidémique de messages
(\emph{gossip}) qui en dépendent bénéficient à leur tour de cette
adaptation. Par conséquent, le trafic généré peut s'adapter à la taille réelle
du réseau.

Afin de répondre à la question de recherche \textbf{QR B.}, nous nous sommes
interessé aux approches appartenant à la réplication optimiste, i.e., un schema
de réplication où le document partagé est copié chez chaque utilisateur afin de
fournir réactivité et disponibilité. Nous nous intéressons particulièrement à un
type de données pour séquences dont les opérations commutent par nature. Elles
présentent l'avantage de supporter efficacement les opérations concurrentes. Ces
approches associent un identifiant unique et immuable à chacun de leurs
éléments. Nous proposons \LSEQ dont la taille des identifiants est bornée
polylogarithmiquement par rapport au nombre d'insertions effectués dans le
document. Nous définissons les conditions sous lesquelles s'appliquent cette
borne et en fournissons la preuve. Au travers de simulations, nous validons
\LSEQ et ses complexités.

Afin de répondre à la question de recherche \textbf{QR.} dans son ensemble, nous
avons réunis les deux approches susmentionnées dans un éditeur collaboratif
décentralisé nommé \CRATE. Ce dernier fonctionne directement dans les
navigateurs web. Il permet l'édition de documents n'importe quand, n'importe où,
quel que soit le nombre d'auteurs, sans fournisseurs tiers. Nous validons \CRATE
aux travers d' expériences réunissant jusqu'à 600 éditeurs sur le banc d'essai
Grid'5000. Les résultats confirment le facteur multiplicatif logarithmique de
\SPRAY et les messages de taille polylogarithmique grâce à \LSEQ.



%%% Local Variables:
%%% mode: latex
%%% TeX-master: "../../paper"
%%% End:
