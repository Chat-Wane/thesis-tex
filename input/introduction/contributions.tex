
\section{Contributions}

Afin de répondre à la première question de recherche, nous nous sommes interessés
à la réplication optimiste~\cite{demers1987epidemic, saito2005optimistic}, i.e.,
un schéma de réplication où le document partagé est copié chez chaque
utilisateur afin de fournir réactivité et disponibilité. Nous nous intéressons
particulièrement à un type de données pour séquences dont les opérations
commutent par nature~\cite{burckhardt2014replicated, shapiro2011comprehensive,
  shapiro2011conflict, zawirski2015dependable}. Elles présentent l'avantage de
supporter efficacement les opérations concurrentes~\cite{ahmed2015evaluation,
  ahmed2011evaluating}. Ces approches associent un identifiant unique et
immuable à chacun de leurs éléments. Nous proposons
\LSEQ~\cite{nedelec2013concurrency, nedelec2013lseq} dont la taille des
identifiants est bornée polylogarithmiquement par rapport au nombre d'insertions
effectuées dans le document. Nous définissons les conditions sous lesquelles
s'appliquent cette borne et en fournissons la preuve. Au travers de simulations,
nous validons \LSEQ et sa complexité.

Afin de répondre à la seconde question de recherche, nous nous sommes
interessés dans un premier temps aux protocoles d'appartenance
(\emph{membership}) aux réseaux. Plus particulièrement à ceux dont la topologie
résultante possède des propriétés similaires à celles des graphes
aléatoires~\cite{erdos1959random}. Parmi ces propriétés se trouve la capacité à
gérer les connexions et déconnexions fréquentes. Ces protocoles sont regroupés
sous l'appellation de ``protocoles d'échantillonnage aléatoire de
pairs''~\cite{jelasity2004peer, jelasity2007gossip} et servent de base à de
nombreux protocoles répartis~\cite{dabek2004vivaldi, folz2016cyclades,
  montresor2005chord}. Nous proposons une nouvelle approche nommée
\SPRAY~\cite{nedelec2015spray} appartenant à cette famille. Comparé à l'état de
l'art~\cite{eugster2003lightweight, ganesh2001scamp, jelasity2007gossip,
  leitao2007dependable, tolgyeski2009adaptive, voulgaris2005cyclon}, \SPRAY
s'adapte à la taille du réseau automatiquement en suivant une progression
logarithmique comparativement à la taille globale du réseau. Nous mettons cela
en évidence grâce à des simulations.  Les protocoles de diffusion
épidémique~\cite{birman1999bimodal} de messages (\emph{gossip}) qui en dépendent
bénéficient à leur tour de cette capacité à s'adapter. Entre autres, le trafic
généré peut s'adapter à la taille réelle du réseau.

Afin de répondre à la problématique générale de ce manuscrit, nous avons réuni
les deux approches susmentionnées dans un éditeur collaboratif décentralisé
nommé \CRATE~\cite{nedelec2016crate}. Ce dernier fonctionne directement dans les
navigateurs web. Pour peu que l'utilisateur ait un accès à l'internet, \CRATE
lui permet l'édition de documents n'importe quand, n'importe où, avec autant de
collaborateurs qu'il le souhaite, sans fournisseurs tiers. Comparé à l'état de
l'art~\cite{etherpad, googledocs, hivejs, lautamaki2012cored,
  nicolaescu2015yjs}, \CRATE résout les problèmes de confidentialité et de
passage à l'échelle tout en conservant une facilité d'accès. Nous validons
\CRATE aux travers d'expériences réunissant jusqu'à 600 éditeurs sur le banc
d'essai Grid'5000. Les résultats confirment le facteur logarithmique de \SPRAY
et le facteur polylogarithmique des messages échangés par \LSEQ.



%%% Local Variables:
%%% mode: latex
%%% TeX-master: "../../paper"
%%% End:
