
\section{Contributions}

\begin{itemize}
\item Afin de répondre à la question de recherche \textbf{QR A.}, nous nous
  sommes interessé dans un premier temps aux protocoles d'appartenance
  (\emph{membership}) aux réseaux. Plus particulièrement à ceux dont la
  topologie résultante possède des propriétés similaires à celles des graphes
  aléatoires, entre autres, la capacité à gérer les connexions et deconnexions
  fréquentes. Ces protocoles sont regroupés sous l'appellation de protocoles
  d'échantillonnage aléatoire de pairs et servent de base à de nombreux
  protocoles répartis (\REF). Nous proposons une nouvelle approche nommée \SPRAY
  appartenant à cette famille. Comparé à l'état de l'art, \SPRAY s'adapte à la
  taille du réseau automatiquement en suivant une progression logarithmique
  comparativement à la taille globale du réseau. Nous mettons cela en évidence
  grâce à des simulations.

  Les protocoles de diffusion épidémique de messages (\emph{gossip}) qui en
  dépendent bénéficient à leur tour de cette adaptation. \TODO{Moar.}

\item Afin de répondre à la question de recherche \textbf{QR B.}, nous nous
  sommes interessé aux approches appartenant à la réplication optimiste. Nous
  proposons \LSEQ qui \TODO{sucharge} les messages avec des metadonnées dont la
  taille est bornée polylogarithmiquement par rapport à la taille du
  document. \TODO{Moar.} Nous définissons les conditions sous lesquelles
  s'appliquent cette borne et en fournissons la preuve.
\end{itemize}

%%% Local Variables:
%%% mode: latex
%%% TeX-master: "../../paper"
%%% End:
