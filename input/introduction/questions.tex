
\section{Questions de recherche}

Un éditeur collaboratif de textes permet à plusieurs utilisateurs d'éditer un
même document~\cite{ellis1989concurrency}. Pour se faire, un utilisateur crée le
document et le partage avec ses collaborateurs. Ces derniers sont alors en
mesure de visionner et de modifier le document. L'édition
\emph{asynchrone}~\cite{johansen1988groupware} divise la tâche entre l'édition
et la synchronisation des modifications. L'échange de documents par courriels ou
les gestionnaires de versions \emph{Git}~\cite{git} ou
\emph{Subversion}~\cite{svn} font parties de cette catégorie. L'édition
\emph{temps réel}~\cite{ellis1991groupware, johansen1988groupware} permet de
visionner et partager les changements immédiatement. Ainsi, l'ajout d'une lettre
dans le document par un participant se répercute directement chez tous le monde
(modulo la latence sur les canaux permettant de transmettre les
changements). L'éditeur web \emph{Google Docs}~\cite{googledocs} appartient à
cette catégorie.  Dans ce manuscrit, nous nous intéressons à ces deux modes
d'édition représentant respectivement les modes d'édition ``hors ligne'' et ``en
ligne''.

En dépit des indéniables améliorations apportées par l'édition
collaborative~\cite{noel2004empirical}, son adoption massive par le public
demeure récente~\cite{perkel2014scientific}. L'intervention du web participe
pour beaucoup à la popularité des éditeurs
collaboratifs~\cite{mogan2010impact}. Les utilisateurs accèdent facilement à
leurs documents via le navigateur web de leur ordinateur ou portable.  Les
utilisateurs partagent leurs documents aisément grâce à de simple URL
(\emph{Uniform Resource Locator}). Hélas, l'utilisation du web comme support à
la collaboration apporte aussi de nombreux défauts. En effet, son organisation
éminemment centralisée pose les problèmes suivants :

\paragraph{Confidentialité :} Chaque lettre tappée transite par un serveur
appartenant à un fournisseur de services. Ce dernier est en mesure de lire le
document et d'en vendre ou d'en concéder le contenu à des tiers tels que des
entreprises publicitaires ou des entités du gouvernement~\cite{gellman2013us};

\paragraph{Censure :} Bien souvent, la propriété du document doit être concédée
au fournisseur de service. À tout moment, celui-ci peut se voir supprimé, au bon
vouloir du fournisseur, si son contenu est jugé contradictoire aux termes du
service;

\paragraph{Passage à l'échelle :} Un serveur est en charge du document. Si le
nombre d'utilisateurs grimpe, le serveur risque de ne pouvoir le supporter. Les
utilisateurs souffrent alors de baisses en qualité de service. Par exemple, leur
document peut devenir temporairement inaccessible;

\paragraph{Point individuel de défaillance :} Puisqu'un seul serveur est en
charge du document, si celui-ci tombe en panne, l'accès aux documents qu'il
héberge est compromis.
  
La décentralisation permet de résoudres les problèmes liés à la confidentialité,
à la censure, et aux défaillances. Un document appartient réellement à ceux qui
l'éditent. Toutefois, les problèmes de passage à l'échelle demeurent auxquels
s'ajoutent les difficultés d'utilisations autrefois résolues grâce au web et au
Nuage~\cite{mell2011national}. Heureusement, la récente technologie
WebRTC~\cite{webrtc} permet d'établir des canaux de communication d'un
navigateur web à l'autre. Cette technologie offre la possibilité de développer
des applications aussi faciles d'accès qu'un service web tout en se passant de
leur fournisseur. Grâce à cela, un nouvel écosystème d'applications
décentralisées directement intégrées au web est en train de
fleurir~\cite{webtorrent}.

\begin{itemize}
\item [\textbf{QR.}] \textbf{L'édition collaborative temps réel est-elle
    possible sur le web, sans l'intervention de tiers et sans limites quant aux
    dimensions du système?}
%%\end{itemize}

\begin{figure}
  \begin{center}
    \begin{tikzpicture}[scale=1.2]

\newcommand\X{25pt}
\newcommand\Y{20pt}

\newcommand\LIGHTGRAY{gray!20}
\newcommand\MEDIUMGRAY{gray!40}

\small
%% communication
\draw[rounded corners=2mm, color=\MEDIUMGRAY, fill=white](0pt, 0pt)+(-4*\X,-\Y)rectangle+(4*\X,\Y);
\draw(4*\X, \Y)node[anchor=north east]{\textbf{\DARKBLUE{communication}}};

\draw ( -3*\X, -0*\Y ) node[anchor=west]{\emph{send and receive changes}};

\small
%% causality
\draw[rounded corners=2mm, color=\MEDIUMGRAY, fill=\LIGHTGRAY](0pt, -2*\Y)+(-4*\X,-\Y)rectangle+(4*\X,\Y);
\draw(4*\X, -\Y)node[anchor=north east]{\textbf{causality}};

\draw( -3*\X, -2*\Y) node[anchor=west]{\emph{track causal relations between changes}};

\small
%% sequence structure
\draw[rounded corners=2mm, color=\MEDIUMGRAY, fill=white](0pt, -4*\Y)+(-4*\X,-\Y)rectangle+(4*\X,\Y);
\draw(4*\X, -3*\Y)node[anchor=north east, align=right]
{\DARKBLUE{\textbf{sequence}}\\\DARKBLUE{\textbf{structure}}};

\draw( -3*\X, -4*\Y) node[anchor=west]{\emph{maintain consistent documents}};

\small
%% gui
\draw[rounded corners=2mm, color=\MEDIUMGRAY, fill=\LIGHTGRAY](0pt, -6*\Y)+(-4*\X,-\Y)rectangle+(4*\X,\Y);
\draw(4*\X, -5*\Y)node[anchor=north east, align=right]
{\textbf{graphical}\\\textbf{user}\\\textbf{interface}};

\draw( -3*\X, -6*\Y) node[anchor=west]{\emph{provide visual cue for interactions}};

\end{tikzpicture}
    \caption[Architecture des éditeurs collaboratifs]
    {\label{intro:fig:architecture} Architecture en 4 couches des éditeurs
      collaboratifs.}
  \end{center}
\end{figure}

L'architecture d'un éditeur collaboratif est constituée de quatres parties
(cf. la figure~\ref{intro:fig:architecture}) :
\begin{inparaenum}[(i)]
\item l'interface homme-machine qui constitue le lien entre l'utilisateur et son
  document, 
\item la structure de données représentant le document partagé,
\item la structure de causalité qui permet d'ordonner les changements effectués
  sur le document,
\item la couche réseau qui permet d'envoyer et recevoir les changements
  effectués sur le document.
\end{inparaenum}

Aucune de ces couches ne doit empêcher le passage à l'échelle de l'application.
Cela concerne aussi bien le nombre d'utilisateurs, que la taille des documents,
que la fréquence des changements ainsi que la durée avant laquelle ils sont
intégrés.

% \begin{itemize}
% \item [\textbf{QR.}] \textbf{Quelle combinaison d'éléments permettent la
%     construction d'un éditeur collaboratif temps réel dont les performances
%     passent à l'échelle et dont le déploiement soit aussi simple que sur le
%     Nuage?}
  
  % Malheureusement, la couche réseau et la couche structure font actuellement
  % obstacle au passage à l'échelle. En effet, les solutions
  % centralisées~\cite{nichols1995high}, où un serveur fait office d'intermédiaire
  % entre les clients, posent des problèmes
  % \begin{inparaenum}[(i)]
  % \item de confidentialité
  % \item de censure,
  % \item de passage à l'échelle,
  % \item de point unique de défaillance.
  % \end{inparaenum}
  %%Seule la décentralisation permet résoudre ces problèmes. Un client est aussi

Pour un éditeur décentralisé, chaque client est aussi un serveur et participe
activement au bon fonctionnement du système. Ainsi, il contribue à la
disséminations des changements effectués sur le document. Pour ce faire, chaque
éditeur possède un ensemble d'éditeurs avec lequel communiquer. Lorsqu'un
éditeur reçoit un changement, il le propage à son propre ensemble d'éditeurs
connus. Les changements atteignent tous les éditeurs par
transitivité~\cite{birman1999bimodal}. La taille de cet ensemble est
déterminante. En effet, lorsqu'elle augmente, le trafic généré augmente au
risque de dépasser les capacités de l'appareil émetteur; lorsqu'elle diminue, le
réseau risque de se fragmenter en îlots, auquel cas les changements ne
parviennent plus à tous les éditeurs.

Afin de configurer la taille de cet ensemble idéalement~\cite{erdos1959random},
le développeur doit connaître la taille des réseaux qu'il souhaite
supporter. Malheureusement, celle-ci varie, que ce soit pendant la durée de vie
du réseau, e.g., une cours de formation en ligne
(\emph{MOOC})~\cite{breslow2013studying} voit sa population fluctuer selon
l'interêt qu'il suscite; ou entre les réseaux d'une même application, e.g., un
document créé pour un événement de grande ampleur n'a pas la même audience que
celle d'un rapport de projet écrit par des étudiants.

  % La première partie de cette question met en lumière la force du Nuage à
  % fournir des services dont l'utilisation et la configuration sont
  % aisées. L'accès à ces services inclut bien souvent une interface web
  % accessible grâce aux navigateurs Internet. Le fait que ces derniers soient
  % présents dans quasiment tous les appareils communiquants modernes participe à
  % l'hégémonie actuelle du Nuage : malgré l'hétérogénéité des appareils utilisés,
  % une simple URL (\emph{Uniform Resource Locator}) permet d'accèder aux services
  % quel que soit la localisation. L'expérience utilisateur ne dépend plus alors
  % que de la conception d'un site web adaptatif (\emph{responsive}) aux
  % appareils.

  % Ce n'est que grâce à la récente introduction de WebRTC~\cite{webrtc} comme
  % protocole de communication de navigateur-à-navigateur que les applications web
  % réparties et décentralisées sont possibles. Dans le contexte de l'édition
  % collaborative, chaque auteur est en mesure d'apporter des modifications au
  % document qu'ils partagent. Chacune de ces modifications doit être envoyées aux
  % autres auteurs afin qu'ils voient le même document. \TODO{Moar?} Cela
  % introduit donc la sous-question suivante:
  \begin{itemize}
  % \item [\textbf{QR A.}] \textbf{Quel protocole de dissémination de messages
  %     utiliser dont les performances passent à l'échelle, dont le bon
  %     fonctionnement soit résistant aux connexions et déconnexions fréquentes
  %     (\emph{churn}) possibles dans ce contexte?}    
  % \item [\textbf{QR A.}] \textbf{Comment disséminer les messages de telle façon
  %    que le trafic s'adapte automatiquement aux fluctuations des réseaux?}
  \item [\textbf{QR A.}] \textbf{Afin d'ajuster le trafic à la session
      d'édition, comment adapter efficacement le voisinage de chaque éditeur
      aux fluctuations des réseaux?}
  \end{itemize}

  Supposons qu'un tel protocole de communication existe. Les messages
  transportés par celui-ci se doivent d'être léger, et pourtant suffisamment
  descriptifs pour reconstruire un document identique indépendemment de l'auteur
  recevant le message.

  Afin d'augmenter la disponibilité et la réactivité des documents, les éditeurs
  actuels copient les documents chez chaque client~\cite{saito2005optimistic},
  et, lorsqu'une modification est effectuée, elle est directement appliquée sur
  leur copie avant d'être envoyée au reste des éditeurs où elle est intégrée. Un
  système est correct si, et seulement si, lorsque les même changements ont été
  intégrés, les utilisateurs lisent le même
  document~\cite{burckhardt2014replicated, shapiro2011conflict}.

  L'utilisation des structures de données ``classiques'' telles que les listes
  n'est pas directement possible du fait de la latence entre les
  participants. Par exemple, si un auteur insère un caractère en début de
  document pendant qu'un autre supprime un caractère au même endroit, le premier
  auteur risque, à tort, de voir supprimé le caractère qu'il vient
  d'insérer. Afin d'éviter de telles situations, de nouvelles structures et de
  nouveaux algorithmes doivent être employés.

  Une première famille d'approches consiste à transformer les arguments de
  l'opérations afin qu'elle considère les opérations intégrées
  concurremment~\cite{sun1998operational}. Dans l'exemple précédent, lors de
  l'intégration de la suppression par le premier auteur, l'opération, se rendant
  compte qu'un caractère a été ajouté en tête, aurait modifié ses arguments afin
  que le second caractère soit supprimé, et non plus le
  premier. Malheureusement, la détection de ces cas nécessite de transporter des
  données avec chaque opération dont le coût est prohibitivement
  élevé~\cite{sun2009contextbased}.

  Une seconde famille d'approches consiste à utiliser des structures de données
  dont les opérations sont commutatives~\cite{shapiro2011conflict}. Pour que les
  opérations commutent, ces approches associent à chaque caractère un
  identifiant unique et immuable. Toutefois, ces identifiants sont
  \begin{inparaenum}[(i)]
  \item soit indestructibles~\cite{oster2006data}. L'opération de suppression se
    contente alors de cacher le caractère à l'utilisateur. Afin de supprimer ces
    caractères cachés et conserver de bonnes performances, un protocole de
    ramasse-miète~\cite{abdullahi1998garbage} réparti doit être exécuté
    régulièrement. Malheureusement, ce genre de protocoles s'avère extrêmement
    coûteux;
  \item soit des listes dont la taille, définie à la génération, peut grandir
    très vite~\cite{weiss2009logoot}. Cette croissance impacte négativement les
    performances du système. Pour y remédier, un protocole de relocalisation des
    identifiants doit être exécuté régulièrement. Malheureusement, ces
    protocoles s'apparentent à l'obtention d'un consensus dans un contexte
    réparti dont le coût s'avère également très
    élevé~\cite{mostefaoui2015signature}.
  \end{inparaenum}

  \begin{itemize}
  \item [\textbf{QR B.}] \textbf{Afin d'éviter tout protocole additionnel de
      relocalisation des identifiants, comment allouer ces identifiants pour que
      leur taille soit directement sous-linéaire?}
  \end{itemize}

  % En effet, les auteurs étant distants les uns des autres, les modifications ne
  % sont pas reçues et traitées instantanéments par tous. De ce fait, l'ordre
  % d'application des modifications peut différer parmis les participants. En
  % particulier les modifications concurrentes doivent être soigneusement
  % étudiées. Elles apparaissent lorsque deux auteurs effectuent une modification
  % sans avoir connaissance de celle de leur homologue. Puisqu'aucune ne précède
  % l'autre, elles peuvent être appliquées dans n'importe quel ordre. Toutefois,
  % les structures doivent toujours converger vers un état équivalent. Contraindre
  % l'ordre d'application des opérations s'avère hélas extrêmement coûteux.

\end{itemize}

%%% Local Variables:
%%% mode: latex
%%% TeX-master: "../../paper"
%%% End:
