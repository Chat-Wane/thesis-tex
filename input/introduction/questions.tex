
\section{Questions de recherche}

% \begin{itemize}
% \item [\textbf{QR.}] \textbf{Comment construire un éditeur collaboratif temps-réel
%     permettant l'écriture de documents qu'importe la taille de ceux-ci, leur
%     nombre d'auteurs, \TODO{passant à l'échelle?} \TODO{moar}}.

%   Écrire le cheminement qui pousse aux sous questions suivantes
%   \begin{itemize}
%   \item [\textbf{QR A.}] \textbf{Est-il possible d'obtenir une structure de
%       données répliquées sans résolution de conflits dont le compromis propose
%       des identifiants de taille sous-linéaire?}
%   \item [\textbf{QR B.}] \textbf{Est-il possible d'obtenir un protocole de
%       dissémination de modifications s'adaptant à la taille de la session
%       d'édition?}
%   \end{itemize}
% \end{itemize}

\begin{itemize}
\item [\textbf{QR.}] \textbf{Comment proposer un service aussi facile d'accès et
    de déployement qu'un service \emph{Cloud} sans cependant dépendre de
    ceux-ci ou en dégrader la qualité?}
  
  Cette question met en lumière la force des services \emph{Cloud}, à savoir
  leur facilité d'usage. \TODO{moar}.
  \begin{itemize}
  \item [\textbf{QR A.}] \textbf{Quelles structures de données et quelles
      algorithmes?}
  \item [\textbf{QR B.}] \textbf{Quels moyens de communications?}
  \end{itemize}
\end{itemize}