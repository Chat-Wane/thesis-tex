
\section{Publications}

\begin{itemize}
\item [\CRATE: Writing Stories Together in our Browsers]\ \\
  -- Brice Nédelec, Pascal Molli, and Achour Mostefaoui
\item [Demo paper -- Proceedings of the 25th International Conference on World Wide
  Web]
\item [\textbf{Abstract:}] {\small Real-time collaborative editors are common
    tools to distribute work across space, time, and
    organizations. Unfortunately, mainstream editors such as Google Docs rely on
    central servers and raises privacy and scalability issues.  \CRATE is a
    real-time decentralized collaborative editor that runs directly in web
    browsers thanks to WebRTC. Compared to state-of-the-art, \CRATE is the first
    real-time editor that only requires browsers in order to support
    collaborative editing and to transparently handle from small to large groups
    of users. Consequently, \CRATE can also be used in massive online lectures,
    TV shows or large conferences to allow users to share their notes. \CRATE's
    properties rely on two scientific results:
  \begin{inparaenum}[(i)]
  \item a replicated sequence structure with sub-linear upper bound on space
    complexity; this prevents the editor from running costly distributed garbage
    collectors,
  \item an adaptive peer sampling protocol; this prevent the editor from
    oversizing routing tables, hence from letting small networks pay the price
    of large networks.
  \end{inparaenum}
  This paper describes \CRATE, its properties and its usage.}
\end{itemize}

\begin{itemize}
\item [\LSEQ: an Adaptive Structure for Sequences in Distributed Collaborative
  Editing]\ \\-- Brice Nédelec, Pascal Molli, Achour Mostefaoui, and Emmanuel
  Desmontils
\item [Research paper -- Proceedings of the 2013 ACM Symposium on Document
  Engineering]
\item [\textbf{Abstract:}] {\small Distributed collaborative editing systems
    allow users to work distributed in time, space and across
    organizations. Trending distributed collaborative editors such as Google
    Docs, Etherpad or Git have grown in popularity over the years. A new kind of
    distributed editors based on a family of distributed data structure
    replicated on several sites called Conflict-free Replicated Data Type (CRDT
    for short) appeared recently. This paper considers a CRDT that represents a
    distributed sequence of basic elements that can be lines, words or
    characters (sequence CRDT). The possible operations on this sequence are the
    insertion and the deletion of elements. Compared to the state of the art,
    this approach is more decentralized and better scales in terms of the number
    of participants. However, its space complexity is linear with respect to the
    total number of inserts and the insertion points in the document. This makes
    the overall performance of such editors dependent on the editing behaviour
    of users. This paper proposes and models \LSEQ, an adaptive allocation
    strategy for a sequence CRDT. \LSEQ achieves in the average a sub-linear
    spatial complexity whatever is the editing behaviour. Series of experiments
    validate \LSEQ showing that it outperforms existing approaches.}
\end{itemize}

\begin{itemize}
\item [Concurrency Effects Over Variable-size Identifiers in Distributed 
  Collaborative Editing] \ \\
  -- Brice Nédelec, Pascal Molli, Achour Mostefaoui, and Emmanuel Desmontils
\item [Workshop -- Document Changes'13: Modeling, Detection, Storage and Visualization]
\item [\textbf{Abstract:}] {\small Distributed collaborative editors such as
    Google Docs or Etherpad allow to distribute the work across time, space and
    organizations. In this paper, we focus on distributed collaborative editors
    based on the Conflict-free Replicated Data Type approach (CRDT). CRDTs
    encompass a set of well-known data types (sets, graphs, sequences,
    etc.). CRDTs for sequences model a document as a set of elements (character,
    line, paragraph, etc.) with unique identifiers, providing two commutative
    update operations: insert and delete. The identifiers of elements can be
    either of fixed-size or variable-size. Recently, a strategy for assigning
    variable-size identifiers called \LSEQ has been proposed for CRDTs for
    sequences. \LSEQ lowers the space complexity of variable-size identifiers
    CRDTs from linear to sub-linear. While experiments show that it works
    locally, it fails to provide this bound with multiple users and latency. In
    this paper, we propose $h$-\LSEQ, an improvement of \LSEQ that preserves its
    space complexity among all collaborators, regardless of the
    latency. Ultimately, this improvement allows to safely build distributed
    collaborative editors based on CRDTs. We validate our approach with
    simulations involving latency and multiple users.}
\end{itemize}