
\subsection{WOOT}

WOOT~\cite{oster2006data} est historiquement reconnu comme étant le premier
type de données répliqué sans résolution de conflits. Cette approche associe un
identifiant unique et non-mutable à chaque element de la sequence partagée. Les
identifiants font réference aux deux élements entourant l'endroit de
l'insertion. Par exemple, en considérant la chaine de caractères ``thse'',
insérer le caractère ``è'' en troisième position lui allouera un identifiant
référençant les identifiants du ``h'' et du ``s''. De plus, un identifiant de
site et un compteur local à ce site sont ajoutés afin de lever toute ambiguité
pouvant être induite par la concurrence. Finalement, on obtient
$\mathcal{I}: \mathcal{I} \times \mathcal{D} \times \mathcal{A} \times
\mathcal{I}$.

Le principal problème de cette approche concerne les operations de suppression.
En effet, l'ordre des élements est garanti par leurs identifiants. En
particulier, cet ordonancement se base dans un premier temps sur les références
contenues dans les identifiants. Ainsi, on ne peut placer un élement et son
identifiant dans la sequence si l'une de ses références vient à manquer. C'est
pour cette raison qu'une suppression cache seulement cet identifiant à
l'utilisateur, alors qu'il se trouve toujours dans le modèle. En d'autres
termes, toutes les operations impactent le modèle de manière définitive. On
peut alors se retrouver avec une séquence apparement vide, alors qu'elle
contient un très grand nombre d'élements cachés.

Le second problème concerne la complexité temporelle de l'algorithme permettant
d'ordonner la séquence. Tout d'abord, elle dépend de la totalité des operations
effectuées sur la sequence. Ensuite, l'algorithme récursif, bien qu'élégant,
possède une complexité cubique.

\begin{algorithm}
  
\small
\algrenewcommand{\algorithmiccomment}[1]{\hskip2em$\rhd$ #1}

\newcommand{\comment}[1]{$\rhd$ #1}


\algblockdefx[initially]{initially}{endInitially}
  [0] {\textbf{INITIALLY:}} 

\algblockdefx[local]{local}{endLocal}
  [0] {\textbf{LOCAL UPDATE:}}

\algblockdefx[received]{received}{endReceived}
  [0] {\textbf{RECEIVED UPDATE:}}

\algblockdefx[onInsert]{onLocal}{endOnLocal}
  [0] {\textbf{on} insert ($p \in \mathcal{I},\,\alpha \in \mathcal{A},\,
   q\in\mathcal{I}$):}
  [0] {\textbf{on} delete ($i \in \mathcal{I}$):} 

\algblockdefx[onRemote]{onRemote}{endOnRemote}
  [0] {\textbf{on} insert ($i\in\mathcal{I}$):\hfill\comment{once per 
  distinct triple in $\mathcal{I}$}}
  [0] {\textbf{on} delete ($i\in\mathcal{I}$):\hfill\comment{after the 
  remote $insert(i)$ is done}} 

\newcommand{\LINEFOR}[2]{%
  \algorithmicfor\ {#1}\ \algorithmicdo\ {#2} %
  }

\newcommand{\LINEIFTHEN}[2]{%
  \algorithmicif\ {#1}\ \algorithmicthen\ {#2} %
  }

\newcommand{\INDSTATE}[1][1]{\State\hspace{\algorithmicindent}}

\begin{algorithmic}[1]
  \Statex
  \initially
    \State $\mathcal{S} \leftarrow [\,]$; \hfill \comment{Replicated sequence}
    \State $\mathcal{D} \leftarrow \langle siteId,\, counter \rangle$;
    \hfill \comment{Disambiguator}
  \endInitially
  
  \local
    \onLocal
    \State $broadcast('insert',\, \langle p,\, inc(\mathcal{D}),\, \alpha,\,
    q \rangle)$;
    \endOnLocal
    \INDSTATE $broadcast('delete',\,i)$;
  \endLocal
  
  \received
    \onRemote
    \State \textbf{let} $p,\, \delta,\, \alpha,\, q \leftarrow i$;
    \hfill \comment{we assume that $p$ and $q$ arrived}
    \State \textbf{let} $index \leftarrow getIndex(p,\, i,\, q)$;
    \State $insert(\mathcal{S},\,index,\,i)$;
    \endOnRemote
    \INDSTATE $hide(\mathcal{S},\, i)$; \hfill \comment{Search the object and
      hide it}
  \endReceived
  
\end{algorithmic}

  \caption{\label{algo:woot}WOOT.}
\end{algorithm}

\begin{algorithm}
  
\small
\algrenewcommand{\algorithmiccomment}[1]{\hskip2em$\rhd$ #1}

\newcommand{\comment}[1]{$\rhd$ #1}

\newcommand{\LINEFOR}[2]{%
  \algorithmicfor\ {#1}\ \algorithmicdo\ {#2} %
  }

\newcommand{\LINEWHILE}[2]{%
  \algorithmicwhile\ {#1}\ \algorithmicdo\ {#2} %
  }

\newcommand{\LINEIFTHEN}[2]{%
  \algorithmicif\ {#1}\ \algorithmicthen\ {#2} %
  }

\newcommand{\INDSTATE}[1][1]{\State\hspace{\algorithmicindent}}

\begin{algorithmic}[1]
  \Function{getIndex}{$p\in\mathcal{I},\, i\in\mathcal{I},\, q\in\mathcal{I}$}
  \State \textbf{let} $result \leftarrow -1$;
  \State \textbf{let} $\mathcal{S}' \leftarrow subseq(\mathcal{S},\,p,\,q)$;
  \hfill \comment{get the elements between $p$ and $q$ in $\mathcal{S}$}
  \If{$|\mathcal{S}'| = 0$}
  \State $result \leftarrow indexOf(\mathcal{S},\,q)$;
  \Else
  \State \textbf{let} $tempP,\,tempQ \leftarrow p,\,q$;
  \State \textbf{let} $found,\, i \leftarrow \bot,\,0$;
  \While{$(i<|\mathcal{S'}| \wedge \neg found)$}
  \State \textbf{let} $p',\,d',\,\alpha',\,q' \leftarrow \mathcal{S}'[i]$;
  \State\LINEIFTHEN{$(p' \leq_{\mathcal{S}} p \wedge q \leq_{\mathcal{S}} q')$}
  {$tempP,\, tempQ \leftarrow tempQ,\, \mathcal{S}'[i]$};
  \State\LINEIFTHEN{$(tempQ <_{\mathcal{I}} i)$}
  {$found,\, result \leftarrow \top,\, getIndex(tempP,\,i,\,tempQ)$}
  \EndWhile
  \EndIf
  \State \Return $result$;
  \EndFunction
\end{algorithmic}

  \caption{\label{algo:wootrecurs}Cherche l'indice où insérer le nouvel 
    élement dans WOOT.}
\end{algorithm}

%%% Local Variables:
%%% mode: latex
%%% TeX-master: "../paper"
%%% End:
