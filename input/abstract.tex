
\begin{resume}
  Un éditeur collaboratif réparti ne devrait admettre aucune limite sur le
  nombre d'auteurs, sur le lieu ou l'heure, ou de tierces parties. Cette thèse
  s'intéresse aux approches décentralisées dont la topologie elle-même permet de
  résoudre les problèmes de confidentialité. Toutefois, restent les problèmes de
  passages à l'échelle. Nos contributions ciblent deux niveaux de l'application
  répartie.  Tout d'abord, nous proposons une fonction d'allocation
  d'identifiants uniques, immuables, et totalement ordonnés garantissant que
  chaque utilisateur voit une copie identique du document. À chaque caractère
  est associé son identifiant. La taille de ces derniers est bornée de manière
  polylogarithmique par rapport à la taille du document. Ce faible surcoût
  induit par la structure de données et dépendant uniquement de la taille du
  document rend l'approche adéquate pour l'édition collaborative de grands
  documents. Dans un second temps, et afin de guarantir une propagation des
  modifications efficaces, nous proposons un protocole d'échantillonnage
  aléatoire de pairs fournissant une vue partielle à chaque pair. La taille de
  cette dernière grandit et rapetisse logarithmiquement par rapport à la taille
  du réseau. Sans connaissance globale, celui-ci converge rapidement vers une
  topologie possédant des propriétés similaires à celles des graphes
  aléatoires. Le surcoût de cette approche est une redondance dans les vues
  locales dont la proportion devient négligeable lorsque le réseau
  grandit. Puisque la diffusion des modifications hérite de l'adaptivité de ce
  protocole d'échantillonnage, le traffic généré lors de l'édition reflète la
  taille du réseau et est équitablement réparti entre les membres.
\end{resume}

\begin{motscles}
  Édition collaborative, décentralisé, temps réel, passage à l'échelle,
  structure répartie de séquence, échantillonnage aléatoire de pairs.
\end{motscles}

\begin{abstract}
  Collaborative editors should not limit the number of authors, place or time,
  or involve third parties. This thesis focus on decentralized approaches the
  topology of which solves privacy concerns. Still, scalability issues
  remain. Our contributions target two level of the distributed
  application. First, to guarantee scalable consistent documents, we provide an
  algorithm that allocates unique and immutable identifiers to each
  character. The size of identifiers is sublinearly upper-bounded by the
  document size. Ordering the identifiers using a lexicographical order allows
  retrieving identical documents. The small overhead induced by this data
  structure only depends of the document which makes it suitable for
  collaborative authoring of large documents. Second, to guarantee scalable
  propagation of changes, we propose a random peer sampling protocol that
  provides each peer with a local view logarithmically scaling to the network
  size. Without any global knowledge, the network quickly converges to a
  topology exposing properties similar to those of random graphs. It costs
  redundancy in views which becomes negligible as the network grows. This
  protocol is suitable for large-scale applications requiring broadcast since
  the latter inherits the scalability of the former.
\end{abstract}

\begin{keywords}
  Collaborative editing, decentralized, real-time, scalable, distributed
  sequence structure, random peer sampling.
\end{keywords}

%%% Local Variables:
%%% mode: latex
%%% TeX-master: "../paper"
%%% End:
