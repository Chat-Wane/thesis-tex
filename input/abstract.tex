
\begin{resume}
  La récente apparition de la communication de navigateur-à-navigateur a
  transformé le programme le plus largement répendu en récéptacle à applications
  web réparties. Chaque navigateur web devient un candidat pour le Fog Computing
  mélant les avantages du Cloud et du Edge. Cette thèse se concentre sur
  l'édition collaborative temps réel. Nos contributions comprennent
  \begin{inparaenum}[(i)]
  \item un protocole d'échantillonnage aléatoire de pairs qui adapte son
    fonctionnement à la taille du réseau sans utiliser de connaissances
    globales;
  \item une structure de données répartie dont la compléxité spatiale est bornée
    sous-linéairement comparé à la taille de la séquence.
  \end{inparaenum}
\end{resume}

\begin{motscles}
  Fog computing, édition collaborative, décentralisé, temps réel, passage à
  l'échelle, structure répartie de séquence, échantillonnage aléatoire de pairs.
\end{motscles}

\begin{abstract}
  Enabling browser-to-browser communication transformed the most widely spread
  program into a receptacle for distributed web applications. Each browser
  becomes an edge-of-the-network candidate for Fog Computing bringing the best
  of both Cloud and Edge. This thesis focuses on real-time collaborative editing. Our
  contributions include
  \begin{inparaenum}[(i)]
  \item a random peer sampling protocol that adapts its operation to the network
    size without any global knowledge. 
  \item a distributed data structure for sequences that enjoys a sub-linear
    upper bound on its space complexity regarding the sequence size.
  \end{inparaenum}
\end{abstract}

\begin{keywords}
  Fog computing, collaborative editing, decentralized, real-time, scalable,
  distributed sequence structure, random peer sampling.
\end{keywords}

%%% Local Variables:
%%% mode: latex
%%% TeX-master: "../paper"
%%% End:
