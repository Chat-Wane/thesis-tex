
\begin{resume}
  Par défaut, mais aussi par intérêt, les éditeurs collaboratifs temps réel sur
  le web sont centralisés. Un serveur du fournisseur de service gère une session
  d'édition. Cela pose des problèmes de confidentialité et de passage à
  l'échelle.  Seulement récemment l'opportunité nous a été offerte d'établir des
  canaux de communication d'un navigateur web à l'autre. Cette possibilité ouvre
  les portes aux applications décentralisées directement accessibles dans les
  navigateurs. Cette thèse comporte trois contributions :
  \begin{inparaenum}[(i)]
  \item Un protocole d'échantillonnage aléatoire de pairs s'auto-ajustant au
    variation de taille du réseau sans connaissances globales;
  \item Une structure de données répliquée dont la taille des métadonnées passe
    à l'échelle;
  \item Un éditeur collaboratif temps réel fonctionnant dans les navigateurs web
    de manière décentralisée utilisant les deux approches ci-dessus.
  \end{inparaenum}
\end{resume}

\begin{motscles}
  Édition collaborative, décentralisé, temps réel, passage à l'échelle,
  structure de donnée répartie pour séquences, échantillonnage aléatoire de
  pairs.
\end{motscles}

\begin{abstract}
  Real-time collaborative editors on the web are centralized. A service
  provider's server hosts an editing session. It raises privacy and scalability
  issues. Only recently the opportunity to establish browser-to-browser
  communication channels has been enabled. This opens the way to decentralized
  application running directly in web browsers. Contributions of this thesis are
  threefold: 
  \begin{inparaenum}[(i)]
  \item A random peer sampling protocol that auto-adjust its functioning to
    the variations in membership of networks, without global knowledge;
  \item A replicated data structure for sequences using metadata the size of
    which scales;
  \item A real-time collaborative editor running in web browsers in a
    decentralized fashion and using the two aforementioned approaches.
  \end{inparaenum}
\end{abstract}

\begin{keywords}
  Collaborative editing, decentralized, real-time, scalable,
  distributed structure for sequences, random peer sampling.
\end{keywords}


  % La récente apparition de la communication de navigateur-à-navigateur a
  % transformé le programme le plus largement répendu en récéptacle à applications
  % web réparties. Chaque navigateur web devient un candidat pour le Fog Computing
  % mélant les avantages du Cloud et du Edge. Cette thèse se concentre sur
  % l'édition collaborative temps réel. Nos contributions comprennent
  % \begin{inparaenum}[(i)]
  % \item un protocole d'échantillonnage aléatoire de pairs qui adapte son
  %   fonctionnement à la taille du réseau sans utiliser de connaissances
  %   globales;
  % \item une structure de données répartie dont la compléxité spatiale est bornée
  %   sous-linéairement comparé à la taille de la séquence.
  % \end{inparaenum}



  % Enabling browser-to-browser communication transformed the most widely spread
  % program into a receptacle for distributed web applications. Each browser
  % becomes an edge-of-the-network candidate for Fog Computing bringing the best
  % of both Cloud and Edge. This thesis focuses on real-time collaborative editing. Our
  % contributions include
  % \begin{inparaenum}[(i)]
  % \item a random peer sampling protocol that adapts its operation to the network
  %   size without any global knowledge. 
  % \item a distributed data structure for sequences that enjoys a sub-linear
  %   upper bound on its space complexity regarding the sequence size.
  % \end{inparaenum}

%%% Local Variables:
%%% mode: latex
%%% TeX-master: "../paper"
%%% End:
