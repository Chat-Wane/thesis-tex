
\begin{resume}
  Un éditeur collaboratif permet de répartir la tâche de rédaction d'un document
  à travers le temps et l'espace. Par leur simplicité d'utilisation, les
  éditeurs collaboratifs temps réel du Web ont contribué à l'adoption massive
  de ces outils par le grand public. Cependant, les éditeurs actuels sont
  centralisés. Le serveur d'un fournisseur de services gère une session
  d'édition. Des problèmes de confidentialité et de passage à l'échelle en
  résultent.

  \noindent Récemment, la possibilité d'établir des communications d'un
  navigateur Web à l'autre a ouvert de nouvelles opportunités en faveur d'un Web
  décentralisé. Un éditeur collaboratif temps réel décentralisé fonctionnant
  dans les navigateurs Web doit gérer efficacement des groupes de taille
  variable et hautement dynamiques.

  \noindent Cette thèse comporte trois contributions :
  \begin{inparaenum}[(i)]
  \item Pour représenter le document, nous proposons une structure de données
    répliquée dont la taille des métadonnées croît de manière sous-linéaire par
    rapport au nombre de caractères insérés dans le document.
  \item Pour propager efficacement les changements à tous les éditeurs
    participant à l'édition, nous proposons un protocole d'échantillonnage
    aléatoire de pairs adapté aux contraintes des navigateurs Web et s'ajustant
    automatiquement au logarithme de la taille de la session d'édition.
  \item Pour démontrer la faisabilité d'un éditeur collaboratif temps réel
    décentralisé fonctionnant dans les navigateurs Web, nous proposons un
    éditeur réunissant (i) et (ii), et dont les performances passent à
    l'échelle.
  \end{inparaenum}
\end{resume}

\begin{motscles}
  Édition collaborative, décentralisé, temps réel, passage à l'échelle,
  structure de donnée répartie pour séquences, échantillonnage aléatoire de
  pairs.
\end{motscles}

\begin{abstract}
  Real-time collaborative editors on the Web are centralized. A service
  provider's server hosts an editing session. It raises privacy and scalability
  issues. Only recently the opportunity to establish browser-to-browser
  communication channels has been enabled. This opens the way to decentralized
  application running directly in web browsers. Contributions of this thesis are
  threefold: 
  \begin{inparaenum}[(i)]
  \item A replicated data structure for sequences using metadata the size of
    which scales;
  \item A random peer sampling protocol that self-adjust its functioning to
    the variations in membership of networks, without global knowledge;
  \item A real-time collaborative editor running in web browsers in a
    decentralized fashion and using the two aforementioned approaches.
  \end{inparaenum}
\end{abstract}

\begin{keywords}
  Collaborative editing, decentralized, real-time, scalable,
  distributed structure for sequences, random peer sampling.
\end{keywords}


  % La récente apparition de la communication de navigateur-à-navigateur a
  % transformé le programme le plus largement répendu en récéptacle à applications
  % web réparties. Chaque navigateur web devient un candidat pour le Fog Computing
  % mélant les avantages du Cloud et du Edge. Cette thèse se concentre sur
  % l'édition collaborative temps réel. Nos contributions comprennent
  % \begin{inparaenum}[(i)]
  % \item un protocole d'échantillonnage aléatoire de pairs qui adapte son
  %   fonctionnement à la taille du réseau sans utiliser de connaissances
  %   globales;
  % \item une structure de données répartie dont la compléxité spatiale est bornée
  %   sous-linéairement comparé à la taille de la séquence.
  % \end{inparaenum}



  % Enabling browser-to-browser communication transformed the most widely spread
  % program into a receptacle for distributed web applications. Each browser
  % becomes an edge-of-the-network candidate for Fog Computing bringing the best
  % of both Cloud and Edge. This thesis focuses on real-time collaborative editing. Our
  % contributions include
  % \begin{inparaenum}[(i)]
  % \item a random peer sampling protocol that adapts its operation to the network
  %   size without any global knowledge. 
  % \item a distributed data structure for sequences that enjoys a sub-linear
  %   upper bound on its space complexity regarding the sequence size.
  % \end{inparaenum}

%%% Local Variables:
%%% mode: latex
%%% TeX-master: "../paper"
%%% End:
