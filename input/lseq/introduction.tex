
Les outils d'édition collaborative répartis sont des applications qui
permettent de répartir le travail de rédaction d'un document selon les trois
dimensions suivantes: le temps, l'espace, et les organismes (REF). Par exemple,
un Français, un Américain et un Australien peuvent travailler sur un même
document, tous ensemble ou a des intervals de temps différents.  Le document
résultant de cette collaboration est le produit de l'expertise de chacun des
collaborateurs. Ainsi, (REF) montre que la qualité des articles Wikipédia est
similaire à la qualité des articles de English (MACHIN TRUC). Toutefois, les
outils actuels ne sont pas entièrement satisfaisants sur des problématiques
\begin{inparaenum}[(i)]
\item liées à l'éthique (e.g. censure, vie privée, intelligence economique
  etc.) et
\item liées au passage à l'échelle (e.g. nombre de collaborateurs, taille des
  documents etc.).
\end{inparaenum}
Ce chapitre a pour mission de détailler les structures et algorithmes
représentant un document partagé.
