
\section{Introduction}

\lettrine{L}a réplication optimiste permet de garantir accessibilité et
réactivité en copiant les données partagées directement chez
l'utilisateur. Ainsi, chaque utilisateur applique ses opérations sur sa copie et
les communique aux autres détenteurs de copies afin qu'ils ajustent l'état de
leur propre copie.

Dans ce chapitre, nous nous intéressons aux séquences répliquées \TODO{(Eventual
  consistency, decentralized?)} permettant, par exemple, de représenter des
documents. Les plus anciennes approches se nomment les \emph{transformés
  opérationelles}. Comme leur nom l'indique, celles-ci modifient les opérations
réçues par rapport aux opérations effectuées en concurrence à
celles-ci. Malheureusement ce procédé peut s'avérer coûteux aussi bien en temps
qu'en espace.  Afin de réduire ce coût, des structures de données répliquées
furent proposées. Elles s'affranchissent des vérifications de concurrence grâce
à des opérations dont les résultats sont commutatifs entre eux. Ainsi, quel que
soit l'ordre d'intégration des opérations, les répliques convergent vers un
état équivalent.

Nous distinguons deux familles de structure. La première utilise des
\emph{pierres tombales} afin d'indiquer qu'un élément de la séquence a été
supprimé. Elles sont nécéssaires car la position de l'élément, bien que
supprimé, reste importante pour la suite de l'exécution. Malheureusement,
supprimer ces pierres tombales requière l'utilisation d'un ramasse-miète réparti
dont le coût est encore plus prohibitif que ces premières. La seconde famille
utilise des identifiants dont la taille varie à la génération. Hélas, cette
taille dépend des positions où les éléments sont insérés. \TODO{more}

La section~\ref{lseq:sec:stateoftheart} de ce chapitre présente l'état de l'art
des approches appartenant à la réplication optimiste de séquences. La
section~\ref{lseq:sec:proposal} détaille \LSEQ, une stratégie d'allocation dont
les identifiants jouissent d'une complexité spatiale sous-linéaire comparée à la
taille de la séquence. La section~\ref{lseq:sec:experiments} valide notre
approche au travers d'expérimentations \TODO{large échelle}. Enfin, nous
\TODO{concluons et discutons} en section~\ref{lseq:sec:conclusion}.

%%% Local Variables:
%%% mode: latex
%%% TeX-master: "../../paper"
%%% End:
