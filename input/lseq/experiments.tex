
\section{Experimentations}

\begin{figure}
  \centering
  \includegraphics{./img/logoot.eps}
  \caption{\label{fig:logoot}Logoot comme référence de stratégie d'allocation.}
\end{figure}

\begin{figure}
  \centering
  \includegraphics{./img/robin.eps}
  \caption{\label{fig:logoot}Alternance de stratégie d'allocation antagonistes.}
\end{figure}

\begin{figure}
  \centering
  \includegraphics{./img/double.eps}
  \caption{\label{fig:logoot}Augmentation de l'espace d'allocation en fonction de 
    la profondeur de l'arbre.}
\end{figure}

\begin{figure}
  \centering
  \includegraphics{./img/lseq.eps}
  \caption{\label{fig:logoot}LSEQ en tant que combinaison de l'alternance et de
    l'augmentation.}
\end{figure}

\begin{figure*}
  \centering
  \subfloat[Comportement d'édition attendu]
  [\label{fig:compliant}Le comportement d'édition correspond aux attentes
  de la stratégie d'allocation]
  {\includegraphics[width=0.48\textwidth]{./img/poste.eps}}
  \hspace{10pt}
  \subfloat[Comportement d'édition inattendu]
  [\label{fig:motivating}Le comportement d'édition va à l'encontre des attentes
  de la stratégie d'allocation]
  {\includegraphics[width=0.48\textwidth]{./img/didyouknow.eps}}
  \caption{\label{fig:allocation}Spectre de document Wikipedia sous différent
    comportements d'édition antagonistes. La figure du haut représente la
    révision à laquelle la ligne a été inséré, i.e., sa date de naissance.  La
    figure du bas représente la taille de l'identifiant associé à chaque ligne.}
\end{figure*}

%%% Local Variables:
%%% mode: latex
%%% TeX-master: "../../paper"
%%% End:
