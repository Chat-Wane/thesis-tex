
\section{LSEQ : une stratégie d'allocation polylogarithmique}

\LSEQ (abréviation pour \emph{polyLogarithimic SEQuence}) est une stratégie
d'allocation d'identifiants dont la taille est variable à la génération. Pour
générer ses identifiants immuables et uniques, \LSEQ utilise un arbre
exponentiel comme structure de données, deux sous-stratégies d'allocations avec
objectifs antagonistes, et un composant permettant de choisir la stratégie.

\subsection{Principe général}

Le principe général de \LSEQ consiste à amortir les mauvais choix d'identifiants
par un ensemble suffisant d'identifiants dont la taille est
satisfaisante. Puisqu'il n'existe pas de stratégie d'allocation parfaite sans
connaissance préalable de la séquence d'édition, \LSEQ cherche a être assez
général pour gérer la plupart des comportements d'édition.


\subsection{Arbre exponentiel}

Un arbre exponentiel est une structure d'arbre dont chaque élément de l'arbre
possède k-fois plus de fils que son parent. Par exemple, si l'on fixe $k$ à $2$,
un élément dont le parent possède $4$ fils en possède $8$. Dans ce cas, il y a
une augmentation quadratique du nombre de fils en fonction de la profondeur qui
s'ajoute à l'augmentation commune de l'arbre.

\TODO{figure}

Un identifiant \LSEQ est un suite d'entiers représentant le chemin de l'élément
dans l'arbre. Pour encoder cette suite d'entiers


\subsection{Sous-stratégies d'allocation}

\subsection{Choix de stratégie}

%%% Local Variables:
%%% mode: latex
%%% TeX-master: "../../paper"
%%% End:
