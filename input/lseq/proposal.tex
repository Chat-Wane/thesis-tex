
\section{LSEQ : une stratégie d'allocation polylogarithmique}

\LSEQ (abréviation de \emph{polyLogarithimic SEQuence}) est une stratégie
d'allocation d'identifiants dont la taille est variable à la génération. Elle
alloue des identifiants avec une complexité en espace dont la borne supérieur
est de l'ordre du polylogarithme comparé au nombre d'insertions dans la
séquence.  Pour générer ses identifiants immuables et uniques, \LSEQ utilise
\begin{inparaenum}[(i)]
\item un arbre exponentiel comme structure de données,
\item deux sous-stratégies d'allocations avec objectifs antagonistes,
\item et un composant permettant de choisir la stratégie.
\end{inparaenum}

Puisqu'il est complexe de prédire le comportement d'édition des utilisateurs sur
un document, il est possible que \LSEQ alloue une série d'identifiants dont la
taille ne serait pas satisfaisante dans l'immédiat. Néanmoins, ce sacrifice
permet d'atteindre un niveau de l'arbre \LSEQ où les identifiants seront alloués 
efficacement jusqu'à compenser les pertes des identifiants alloués précédemment.

\subsection{Arbre exponentiel}

Un arbre est un graphe orienté, acyclique, et connexe. En d'autres termes, il
s'agit d'une structure dont les éléments sont liés entre eux par une relation de
parenté. Généralement, l'arité d'un arbre (i.e. le nombre de fils qu'un élément
peut posséder) est limitée par une constante. Par exemple, la
figure~\ref{fig:lseq:allocpathexample} présente des arbres d'arité
décimale. Ainsi, chaque élément possède au maximum 10 fils.

Un arbre exponentiel est un arbre dont l'arité augmente avec sa
profondeur. Ainsi, chaque élément peut posséder $c$-fois plus de fils que son
parent. Par exemple, avec un arbre exponentiel dont le facteur de croissance est
2, un élément dont l'arité maximum est $4$ voit chacun de ces fils posséder une
arité maximum de $8$, etc.  Un tel arbre de profondeur $k$ peut alors accueillir
:
\begin{equation}
  n = \sum\limits_{i=0}^{k} 2^{(i^2-i)/2}\,\,\, identifiants.
\end{equation}

Chaque lien de parenté est étiqueté et la série de ces étiquettes crée le chemin
vers un élément de la séquence. La représentation en mémoire de cette série
dépend de l'arité de l'arbre. Par exemple, les arbres d'arité décimale présents
dans la figure~\ref{fig:lseq:allocpathexample} ont besoin de $k*\log_2(10)$ bits
afin de représenter un chemin de profondeur $k$.  En revanche, les chemins \LSEQ
ont besoin d'un bit additionnel par niveau afin de refléter l'augmentation
d'arité de l'arbre. Pour représenter en mémoire un chemin de profondeur $k$, un
arbre exponentiel de facteur d'augmentation $2$ et dont la première profondeur
nécessite $b$ bits a besoin de :
\begin{equation}
  \sum\limits_{i=1}^{k}b+i = kb + k(k+1)/2 = O(k^2) \,\,\, bits.
\end{equation}

\begin{table}
  \centering
  \begin{tabular}{@{}lccc@{}}
    \toprule
    & Aléatoire & Monotone & Pire cas \\ \cmidrule{2-4}
    taille du chemin & $O(\sqrt{log\,n})$ & $O(log\,n)$ & $O(n)$ \\ \midrule
    taille binaire du chemin & \multicolumn{3}{c}{ $\sum\limits_{i=1}^{k}b+i =
      kb + k(k+1)/2 = O(k^2)$} \\ \midrule
    Complexité en espace & $O(log\,n)$ & $O((log\,n)^2)$ &
    $O(n^2)$ \\ \bottomrule
  \end{tabular}
  \caption{Complexité spatiale de \LSEQ avec $n$ le nombre d'opérations
    d'insertion effectuées sur la séquence; $k$ la profondeur d'un chemin;
    $b$ la taille binaire nécessaire pour représenter le chemin de profondeur 1.}
\end{table}

Une telle structure a l'avantage d'adapter sa consommation en mémoire à la
taille du document. En effet, commencer l'édition avec une petite arité reflète
la petite taille du document. Au contraire, lorsque le document devient plus
imposant, la profondeur de l'arbre \LSEQ augmente. Par conséquent, l'arité à
cette profondeur augmente elle aussi.

Nous nous intéressons maintenant à l'analyse en complexité spatiale lors de 3
types d'édition différents (pas nécessairement réaliste) afin de mettre en
évidence le comportement d'allocation de \LSEQ, à savoir:
\begin{asparadesc}
\item [Édition aléatoire:] insertions d'éléments à des positions aléatoires dans
  la séquence.
\item [Édition monotone:] insertions d'éléments à des positions adjacentes
  successivement. Par exemple, la figure~\ref{fig:lseq:allocpathexample} présente
  deux types d'édition monotone.
\item [Édition pire cas:] insertions d'éléments aux pires positions possibles
  résultant en une augmentation de la profondeur de l'arbre à chaque insertion.
\end{asparadesc}

\subsection{Sous-stratégies d'allocation}

\LSEQ utilise deux sous-stratégies d'allocation d'identifiants dont la finalité
est antagoniste. En effet, l'une de ces stratégies est adaptée à l'édition
monotone effectuée de gauche à droite, tandis que l'autre est adaptée à
l'édition monotone effectuée de droite à gauche.

\begin{figure*}
  \centering
  \subfloat[Sous-stratégie adaptée à l'édition monotone de gauche à droite]
  [Sous-stratégie d'allocation adaptée à l'édition monotone de gauche à droite]
  {\begin{tikzpicture}[scale=1]
  \small
  \draw [<-] (2.625,-16pt)--( 2.625, -26pt)  node[anchor=north] {insertion};

  \draw  (1.25,60pt)-- ( 3.75, 60pt);
  \draw  (0,0)-- (5, 0);
  
  \draw [dashed] (2.5,3pt) to[out=90,in=280] (1.25,55pt);
  \draw [->,ultra thick,dashed] (2.5,3pt) to[out=90,in=280] (1.6,55pt);
  \draw [dashed] (2.85,3pt) to[out=90,in=280] (3.75,55pt);  
  
  \draw [->,thick, color=black] (1.25,77pt) -- node[anchor=south]{$+20$}
  (1.9,77pt);
  \draw [dashed, color=black] (1.9,77pt) -- (1.9,57pt);


  \draw (0,1pt) -- (0,-3pt) node[anchor=north] {0};
  \draw (5,1pt) -- (5,-3pt) node[anchor=north] {$100$};
  \draw (1.25,57pt) -- (1.25,61pt) node[anchor=east] {0};
  \draw (1.6,57pt) -- (1.6,61pt) node[anchor=south] {\textbf{11}};
  \draw (3.75,57pt) -- (3.75,61pt) node[anchor=west] {$100$};
  \draw (2.5,1pt) -- (2.5,-3pt) node[anchor=north] {\textbf{50}};
  \draw (2.85,1pt) -- (2.85,-3pt) node[anchor=north] {$51$};

\end{tikzpicture}}
  \hspace{20pt}
  \subfloat[Sous-stratégie adaptée à l'édition monotone de droite à gauche]
  [Sous-stratégie d'allocation adaptée à l'édition monotone de droite à gauche]
  {\begin{tikzpicture}[scale=1]
  \small
  \draw [<-] (2.625,-16pt)--( 2.625, -26pt)  node[anchor=north] {insertion};

  \draw  (1.25,60pt)-- ( 3.75, 60pt);
  \draw  (0,0)-- (5, 0);
  
  \draw [dashed] (2.5,3pt) to[out=90,in=280] (1.25,55pt);
  \draw [->,ultra thick,dashed] (2.5,3pt) to[out=90,in=280] (3.4,55pt);
  \draw [dashed] (2.85,3pt) to[out=90,in=280] (3.75,55pt);  

  \draw [->,thick, color=black] (3.75,77pt) -- node[anchor=south]{$-20$}
  (3.1,77pt);
  \draw [dashed, color=black] (3.1,77pt) -- (3.1,57pt);
  
  \draw (0,1pt) -- (0,-3pt) node[anchor=north] {0};
  \draw (5,1pt) -- (5,-3pt) node[anchor=north] {$100$};
  \draw (1.25,57pt) -- (1.25,61pt) node[anchor=east] {0};
  \draw (3.4,57pt) -- (3.4,61pt) node[anchor=south] {\textbf{89}};
  \draw (3.75,57pt) -- (3.75,61pt) node[anchor=west] {$100$};
  \draw (2.5,1pt) -- (2.5,-3pt) node[anchor=north] {\textbf{50}};
  \draw (2.85,1pt) -- (2.85,-3pt) node[anchor=north] {$51$};

\end{tikzpicture}}
  \caption{\label{fig:lseq:substrategiesexample}Les sous-stratégies employées
    par \LSEQ.}
\end{figure*}

\subsection{Choix de stratégie}

\begin{algorithm}
  
\small
\algrenewcommand{\algorithmiccomment}[1]{\hskip2em$\rhd$ #1}
\newcommand*{\comment}[1]{$\rhd$ #1}

  \begin{algorithmic}[1]
  \State \textbf{let} $boundary \leftarrow 10$; \Comment{Any constant} 
  \State \textbf{let} $h:\mathbb{N} \rightarrow (\mathcal{P}\times
  \mathcal{P}\rightarrow \mathcal{P})$; \hfill \comment{get sub-allocation
    function}
  \Statex
    \Function{allocPath}{$p,\, q \in \mathcal{P}$}
    $\rightarrow \mathcal{P}$
    \State \textbf{let} $depth,\,\_ \leftarrow getDepthInterval(p,\,q)$;
    \State \textbf{return} $h(depth)(p,\,q)$; \Comment{Defers the call}
    \EndFunction
    \Statex
    \Function{endEditing}{$p,\,q \in \mathcal{P}$}
    $\rightarrow \mathcal{P}$
    \Statex \comment{\#1 Get the depth of the new path}
    \State \textbf{let} $depth,\,interval \leftarrow getDepthInterval(p,q);$
    \Statex \comment{\#2 Process a maximal space between two identifiers}
    \State \textbf{let} $step \leftarrow min(boundary,interval)$;
    \Statex \comment{\#3 Create the new path}
    \State \textbf{return} $subPath(p,depth) + rand(0,step)$;
    \EndFunction
    \Statex
    \Function{frontEditing}{$p,\,q \in \mathcal{P}$}
    $\rightarrow \mathcal{P}$
    \State \textbf{let} $depth,\, interval \leftarrow getDepthInterval(p,q);$
    \hfill \comment{\#1}
    \State \textbf{let} $step \leftarrow min(boundary,interval)$;
    \hfill \comment{\#2}
    \State \textbf{return} $subPath(q,depth) - rand(0,step)$;
    \hfill \comment{\#3}
    \EndFunction

    \Statex 
    \Statex 
    \comment{Which depth has enough space for 1 path}
    \Function{getDepthInterval}{$p,\,q \in \mathcal{P}$} $\rightarrow \mathbb{N} \times \mathbb{N}$
      \State \textbf{let} $depth \leftarrow 0$; $interval \leftarrow 0$;
      \While{$(interval < 2)$}
        \State $depth \leftarrow depth + 1$;
        \State $interval \leftarrow subPath(q,depth) - subPath(p,depth)$;
      \EndWhile
      \State \textbf{return} $\langle depth,\, interval\rangle$;
    \EndFunction

  \end{algorithmic}


  \caption{\label{algo:lseq:allocpath}Allocation of the path.}
\end{algorithm}

\begin{figure*}
  \centering
  \subfloat[\LSEQ sur comportement d'édition de gauche à droite]
  [\LSEQ sur un comportement d'édition de gauche à droite]
  {\input{input/lseq/lseqtreeexampleA.tex}}
  \hspace{10pt}
  \subfloat[\LSEQ sur comportement d'édition de droite à gauche]
  [\LSEQ sur un comportement d'édition de droite à gauche]
  {\input{input/lseq/lseqtreeexampleB.tex}}
  \caption{\label{fig:lseqtreeexample} Arbre exponentiel de \LSEQ sur deux
    comportements d'édition antagonistes créant la séquence QWERTY. Dans cet
    exemple, La fonction $h$ de hachage défère l'allocation du chemin aux
    sous-stratégies d'allocation prévus pour l'édition en tête et l'édition en
    fin au premier et second niveau respectivement. De plus, l'arité de l'arbre
    est doublée à chaque niveau. Contrairement aux exemples de la
    figure~\ref{fig:allocpathexample}, la profondeur de ces arbres n'augmente
    pas linéairement.}
\end{figure*}

\begin{algorithm}
  
\small
\algrenewcommand{\algorithmiccomment}[1]{\hskip2em$\rhd$ #1}
\newcommand*{\comment}[1]{$\rhd$ #1}

\newcommand{\LINEIFTHEN}[2]{%
  \algorithmicif\ {#1}\ \algorithmicthen\ {#2} %
  }

  \begin{algorithmic}[1]
    \State \textbf{let} $site$ \hfill \comment{the unique site identifier}
    \State \textbf{let} $counter \leftarrow 0$ \hfill \comment{a local counter}
    \Statex
    \Function{allocDis}{$p \in \mathcal{I},\, path\in\mathcal{P},\, q \in
      \mathcal{I}$} $\rightarrow \mathcal{D}$
    \State \textbf{let} $dis \leftarrow [\,]$;
    \State $counter \leftarrow counter + 1$;
    \For{$i$ \textbf{from} $1$ \textbf{to} $|path|$}
    \State $dis[i] \leftarrow \langle site,\, counter \rangle$;
    \State \LINEIFTHEN {$path[i]=q.P[i]$} {$dis[i] \leftarrow q.D[i];$}
    \State \LINEIFTHEN {$path[i]=p.P[i]$} {$dis[i] \leftarrow p.D[i];$}
    \EndFor
    \State \textbf{return} $dis$;
    \EndFunction
  \end{algorithmic}

  \caption{\label{algo:allocdes}Allocation of the disambiguator.}
\end{algorithm}


%%% Local Variables:
%%% mode: latex
%%% TeX-master: "../../paper"
%%% End:
