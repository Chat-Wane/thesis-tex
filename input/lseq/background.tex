
\section{État de l'art}

\subsection{Réplication optimiste}

\subsection{Transformés opérationnels}

Les outils d'édition collaboratifs répartis peuvent être divisés en deux
classes. Les approches à \emph{transformés
  opérationnels}~\cite{sun2009contextbased, sun1998operational} (OT) sont les
plus anciennes et s'appliquent à un large champs d'applications tels que
l'édition de texte, l'édition d'images etc. Dans le cadre de l'édition de texte,
OT, en plus des usuels opérations \emph{insert} et \emph{delete}, apportent des
opérations ciblant les chaînes de caractères, à savoir \emph{move}, \emph{cut --
  paste}, etc. Toutefois, l'analyse de correction nécessite d'examiner chaque
paire d'opérations ainsi que leurs paramètres. En conséquence, lors de
l'écriture du papier \cite{imine2003proving}, peu d'approches étaient réellement
correctes. De plus, cette classe d'approches peut être divisé en deux
sous-classes: les approches centralisées et les approches décentralisées. Les
premières réarrangent les opérations sur un serveur
central~\cite{nichols1995high} afin de faciliter la convergence. Toutefois, la
topologie elle-même implique un point individuel de défaillance, des problèmes
de respect de la vie privée, des problèmes d'intelligence économique, des
problèmes de censure, et enfin, de passage à l'échelle. Les approches
décentralisées~\cite{sun2009contextbased}, quant à elle, nécessitent un vecteur
de version afin d'identifier les contextes de génération des opérations
reçues. Elles transforment les arguments de l'opération reçue par rapport aux
opérations concurrentes dans le but d'exécuter de manière cohérente l'opération
sans avoir à défaire et réexecuter ces opérations. De ce fait, bien que
l'exécution locale d'une opération soit très efficace, l'exécution des
opérations reçues est très coûteuse en cas de concurrence. Ainsi, confiné aux
environnements maîtrisés, OT reste efficace.

\subsection{Structures de données répliquées sans résolution de conflits}

Plus récemment, les approches à base de \emph{structures de données répliqués
  sans résolution de conflits}~\cite{shapiro2011comprehensive,
  shapiro2011conflict} (CRDTs) commencèrent à émerger. Ces approches partagent
le coût de calcul entre la partie locale et la partie distante de la réplication
optimiste. Tandis que les CRDTs améliorent de manière significative la
complexité temporelle des opérations comparé aux approches OT décentralisées,
ils cachent leur complexité dans l'occupation mémoire. Les CRDTs pour séquences
peuvent être divisés en deux.

\subsubsection{Pierres tombales}

Les approches utilisant des pierres tombales~\cite{ahmed2011evaluating,
  conway2014language, grishchenko2010deep, oster2006data,
  preguica2009commutative, roh2011replicated, weiss2007wooki, wu2010partial,
  Yu2012stringwise} qui marquent simplement les éléments supprimés afin de les
cacher à l'utilisateur. Bien que supprimé, ces pierres tombales existent
toujours dans la structure sous-jacente et continuent d'avoir un impact sur les
performances. Ainsi, la complexité dépend de l'ensemble des opérations
\emph{insert} et \emph{delete} ayant été effectué sur le document. Cela devient
problématique dans certains documents particulièrement assujettis à vandalisme
(e.g. Wikipedia). La conséquence étant qu'un document n'ayant en apparence que
peu de contenu consomme beaucoup de mémoire du fait des nombreuses pierres
tombales cachées.

\subsubsection{Identifiants de taille variable}

Les approches utilisant des identifiants de taille
variable~\cite{preguica2009commutative, andre2013supporting,
  weiss2009logoot}. Contrairement aux approches basées sur les pierres tombales,
les éléments supprimés disparaissent complètement de la structure. En
contrepartie, les identifiants encodent un ordre total permettant de les placer
dans le document indépendamment des autres éléments. La complexité spatiale de
ces identifiants est cruciale et dépend seulement des opérations d'insertion.

\subsection{Stratégie d'allocation d'identifiants}

Dans la littérature, deux genres de stratégies d'allocations d'identifiants
existent. Tout d'abord, considérons l'insertion de l'élément $b$ entre les
éléments $a$ et $c$ dont les identifiants respectifs sont $id_a$ et $id_c$ avec
$id_a<id_c$. Dans tous les cas, une fonction d'allocation cherche à allouer le
plus petit identifiant possible. Ainsi, l'identifiant généré à une taille
maximum de $min(|id_a|,\, |id_c|)+1$. La première stratégie consiste à allouer
une position aléatoire entre les deux identifiants afin d'éviter au maximum les
conflits dûs à la concurrence. Toutefois, cette stratégie consume en moyenne la
moitié de l'espace entre deux identifiants. La seconde stratégie provient
d'observations faites sur un corpus de textes favorable à l'édition de gauche à
droite. Dans ce genre de cas, la stratégie consiste à rapprocher les
identifiants générés du précédent identifiant (e.g. $id_b$ proche de
$id_a$). Toutefois, lorsque le comportement d'édition ne suit pas celui attendu,
la taille des identifiants croît très rapidement.

%%% Local Variables:
%%% mode: latex
%%% TeX-master: "../../paper"
%%% End:
