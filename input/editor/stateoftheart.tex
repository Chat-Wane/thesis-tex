
\section{État de l'art}

Les éditeurs collaboratifs répartis autorisant l'édition temps réel dans les
navigateurs web sont centralisés. Entre autres, les éditeurs tels que Google
Docs~\cite{googledocs} ou Etherpad~\cite{etherpad} ont rendu l'édition
collaborative aisée pour les millions d'utilisateurs à travers le monde.

\begin{figure}
  \begin{center}
    \begin{tikzpicture}[scale=1.2]
  
  \newcommand\X{40pt}
  \newcommand\Y{-50pt}
  
  \draw[->, very thick, color=darkblue](-2.5-2*\X, 5+1*\Y)--(-2.5-1.5*\X, -10pt);
  \draw[<-, dashed](2.5-2*\X, 5+1*\Y)--(2.5-1.5*\X, -10pt);

  \draw[->, dashed](-2.5-1*\X, 5+1*\Y)--(-2.5-1*\X, -10pt);
  \draw[<-, very thick, color=darkblue](2.5-1*\X, 5+1*\Y)--(2.5-1*\X, -10pt);

  \draw[->, dashed](-2.5+0*\X, 5+1*\Y)--(-2.5+0*\X, -10pt);
  \draw[<-, very thick, color=darkblue](2.5+0*\X, 5+1*\Y)--(2.5+0*\X, -10pt);
    
  \draw[->, dashed](-2.5+1*\X, 10+1*\Y)--(-2.5+0.5*\X, -10pt);
  \draw[<-, very thick, color=darkblue](2.5+1*\X, 10+1*\Y)--(2.5+0.5*\X, -10pt);


  \draw[fill=white](-0.5*\X, 0*\Y) node{Service provider's document server}
  +(-70pt,-10pt )rectangle+(70pt,10pt);

  \draw[fill=white](-2*\X, 1*\Y) node{$e_1$}
  +(5pt, 5pt) rectangle +(-5pt, -5pt) rectangle +(5pt, 5pt);
  \draw[fill=white](-1*\X, 1*\Y) node{$e_2$}
  +(5pt, 5pt) rectangle +(-5pt, -5pt) rectangle +(5pt, 5pt);
  \draw[fill=white]( 0*\X, 1*\Y) node{$e_3$}
  +(5pt, 5pt) rectangle +(-5pt, -5pt) rectangle +(5pt, 5pt);



  \draw[->,dashed](-2.5+1.5*\X, 5+2*\Y)--(-2.5+1.5*\X, -10+1*\Y);
  \draw[<-,very thick, color=darkblue]( 2.5+1.5*\X, 5+2*\Y)--( 2.5+1.5*\X, -10+1*\Y);

  \draw[->,dashed](-2.5+2.5*\X, 5+2*\Y)--(-2.5+2.5*\X, -10+1*\Y);
  \draw[<-,very thick, color=darkblue]( 2.5+2.5*\X, 5+2*\Y)--( 2.5+2.5*\X, -10+1*\Y);

  \draw[->,dashed](-2.5+3.5*\X, 5+2*\Y)--(-2.5+3.5*\X, -10+1*\Y);
  \draw[<-,very thick, color=darkblue]( 2.5+3.5*\X, 5+2*\Y)--( 2.5+3.5*\X, -10+1*\Y);

  \draw[->,dashed](-2.5+4.5*\X, 5+2*\Y)--(-2.5+4.5*\X, -10+1*\Y);
  \draw[<-,very thick, color=darkblue]( 2.5+4.5*\X, 5+2*\Y)--( 2.5+4.5*\X, -10+1*\Y);

  \draw[fill=white](3*\X, 1*\Y) node{Service provider's transmission server}
  +(-100pt,-10pt )rectangle+(100pt,10pt);

  \draw[fill=white]( 1.5*\X, 2*\Y) node{$e_4$}
  +(5pt, 5pt) rectangle +(-5pt, -5pt) rectangle +(5pt, 5pt);
  \draw[fill=white]( 2.5*\X, 2*\Y) node{$e_5$}
  +(5pt, 5pt) rectangle +(-5pt, -5pt) rectangle +(5pt, 5pt);
  \draw[fill=white]( 3.5*\X, 2*\Y) node{$e_6$}
  +(5pt, 5pt) rectangle +(-5pt, -5pt) rectangle +(5pt, 5pt);
  \draw[fill=white]( 4.5*\X, 2*\Y) node{$e_7$}
  +(5pt, 5pt) rectangle +(-5pt, -5pt) rectangle +(5pt, 5pt);

\end{tikzpicture}
    \caption{\label{editor:fig:serviceprovider} Les transmissions aux clients
      connectés et les transformations d'opérations sont à la charge du 
      fournisseur de services.}
  \end{center}
\end{figure}

La figure~\ref{editor:fig:serviceprovider} montre le fonctionnement de ces
éditeurs. Un serveur central possède le document. Les participants se connectent
à ce serveur (i.e. $e_1$, $e_2$, et $e_3$) ou à un serveur associé (i.e. $e_4$,
$e_5$, $e_6$, $e_7$) mis à disposition de manière élastique afin d'alléger la
charge de diffusion des messages du premier. Lorsque le collaborateur $e_1$
effectue une action telle que l'insertion d'un caractère, l'opération est émise
au serveur possédant le document. Celui-ci, fonctionnant avec une approche de
transformés opérationnels, transforme l'opération reçue vis-à-vis de toutes
celles que l'utilisateur n'avait pas encore reçu lors de la création de son
opération. La transformation s'avère toutefois moins coûteuse puisque seul le
serveur doit l'effectuer. En particulier, le vecteur de contexte transporté dans
les messages des approches OT décentralisées n'existe pas dans les versions
centralisée puisque seule importe la paire de versions de l'utilisateur et du
serveur. En revanche, le serveur est en charge de toutes les transformations. De
plus, il doit diffuser l'ensemble des changements aux participants. Le
fournisseur de service prend en charge la presque intégralité du coût de la
session d'édition. Pour les utilisateurs, se posent toujours les problèmes de
confidentialité : Le fournisseur de service voit les documents et peut en
utiliser le contenu à ses propres fins. Pire encore, la propriété du document
doit bien souvent être accordée au fournisseur de service. À cela s'ajoute le
problème du point individuel de défaillance : Lorsque le serveur hébergeant le
document tombe en panne, le document n'est plus accessible.

Malgré tous ces défauts, ces approches centralisées sont les plus populaires.
Jusqu'à présent, aucun éditeur décentralisé dans le navigateur web n'avait été
déployé. La récente technologie WebRTC permet d'établir des canaux de
communication d'un navigateur à l'autre. En d'autres termes, les applications
réellement décentralisée aussi facile d'accès que les applications web du Nuage
deviennent possibles.

\paragraph{WebRTC~\cite{webrtc}.} Acronyme de \emph{Web Real-Time
  Communication}.  Cette technologie permet l'établissement de canaux de
communication d'un navigateur web à l'autre. Établir une connexion WebRTC
requière une négociation où le nœud d'origine et le nœud de réception s'envoient
mutuellement des moyens d'accès distants dans l'ordre définit du plus aisé au
plus ardu. Par exemple, les échanges vont d'abord concerner la boucle locale
(\emph{localhost}), puis le réseau local (e.g. $192.168.255.255$), puis
l'internet\ldots Aussitôt que la négociation s'achève avec succès, un canal de
communication bidirectionnel est établi. Les nœuds peuvent alors communiquer
entre eux.

\noindent Comparé aux méthodes plus traditionnelles d'établissement de
connexions (\TODO{Spécifier}), les connexions WebRTC sont nettement plus
coûteuse à mettre en place et à entretenir. À ce titre, elles ne doivent pas
être établies à la légère.

%%% Local Variables:
%%% mode: latex
%%% TeX-master: "../../paper"
%%% End:
