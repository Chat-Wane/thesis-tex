
% \section{État de l'art}
% \label{editor:sec:stateoftheart}


% \begin{figure}
%   \begin{center}
%     \begin{tikzpicture}[scale=1.2]
  
  \newcommand\X{40pt}
  \newcommand\Y{-50pt}
  
  \draw[->, very thick, color=darkblue](-2.5-2*\X, 5+1*\Y)--(-2.5-1.5*\X, -10pt);
  \draw[<-, dashed](2.5-2*\X, 5+1*\Y)--(2.5-1.5*\X, -10pt);

  \draw[->, dashed](-2.5-1*\X, 5+1*\Y)--(-2.5-1*\X, -10pt);
  \draw[<-, very thick, color=darkblue](2.5-1*\X, 5+1*\Y)--(2.5-1*\X, -10pt);

  \draw[->, dashed](-2.5+0*\X, 5+1*\Y)--(-2.5+0*\X, -10pt);
  \draw[<-, very thick, color=darkblue](2.5+0*\X, 5+1*\Y)--(2.5+0*\X, -10pt);
    
  \draw[->, dashed](-2.5+1*\X, 10+1*\Y)--(-2.5+0.5*\X, -10pt);
  \draw[<-, very thick, color=darkblue](2.5+1*\X, 10+1*\Y)--(2.5+0.5*\X, -10pt);


  \draw[fill=white](-0.5*\X, 0*\Y) node{Service provider's document server}
  +(-70pt,-10pt )rectangle+(70pt,10pt);

  \draw[fill=white](-2*\X, 1*\Y) node{$e_1$}
  +(5pt, 5pt) rectangle +(-5pt, -5pt) rectangle +(5pt, 5pt);
  \draw[fill=white](-1*\X, 1*\Y) node{$e_2$}
  +(5pt, 5pt) rectangle +(-5pt, -5pt) rectangle +(5pt, 5pt);
  \draw[fill=white]( 0*\X, 1*\Y) node{$e_3$}
  +(5pt, 5pt) rectangle +(-5pt, -5pt) rectangle +(5pt, 5pt);



  \draw[->,dashed](-2.5+1.5*\X, 5+2*\Y)--(-2.5+1.5*\X, -10+1*\Y);
  \draw[<-,very thick, color=darkblue]( 2.5+1.5*\X, 5+2*\Y)--( 2.5+1.5*\X, -10+1*\Y);

  \draw[->,dashed](-2.5+2.5*\X, 5+2*\Y)--(-2.5+2.5*\X, -10+1*\Y);
  \draw[<-,very thick, color=darkblue]( 2.5+2.5*\X, 5+2*\Y)--( 2.5+2.5*\X, -10+1*\Y);

  \draw[->,dashed](-2.5+3.5*\X, 5+2*\Y)--(-2.5+3.5*\X, -10+1*\Y);
  \draw[<-,very thick, color=darkblue]( 2.5+3.5*\X, 5+2*\Y)--( 2.5+3.5*\X, -10+1*\Y);

  \draw[->,dashed](-2.5+4.5*\X, 5+2*\Y)--(-2.5+4.5*\X, -10+1*\Y);
  \draw[<-,very thick, color=darkblue]( 2.5+4.5*\X, 5+2*\Y)--( 2.5+4.5*\X, -10+1*\Y);

  \draw[fill=white](3*\X, 1*\Y) node{Service provider's transmission server}
  +(-100pt,-10pt )rectangle+(100pt,10pt);

  \draw[fill=white]( 1.5*\X, 2*\Y) node{$e_4$}
  +(5pt, 5pt) rectangle +(-5pt, -5pt) rectangle +(5pt, 5pt);
  \draw[fill=white]( 2.5*\X, 2*\Y) node{$e_5$}
  +(5pt, 5pt) rectangle +(-5pt, -5pt) rectangle +(5pt, 5pt);
  \draw[fill=white]( 3.5*\X, 2*\Y) node{$e_6$}
  +(5pt, 5pt) rectangle +(-5pt, -5pt) rectangle +(5pt, 5pt);
  \draw[fill=white]( 4.5*\X, 2*\Y) node{$e_7$}
  +(5pt, 5pt) rectangle +(-5pt, -5pt) rectangle +(5pt, 5pt);

\end{tikzpicture}
%     \caption[Fonctionnement des éditeurs centralisés]
%     {\label{editor:fig:serviceprovider} Les transmissions aux clients connectés
%       et les transformations d'opérations sont à la charge du fournisseur de
%       services.}
%   \end{center}
% \end{figure}

% La figure~\ref{editor:fig:serviceprovider} montre le fonctionnement de ces
% éditeurs. Un serveur central possède le document. Les participants se connectent
% à ce serveur (i.e. $e_1$, $e_2$, et $e_3$) ou à un serveur associé (i.e. $e_4$,
% $e_5$, $e_6$, $e_7$) mis à disposition de manière élastique afin d'alléger la
% charge de diffusion des messages du premier. Lorsque le collaborateur $e_1$
% effectue une action telle que l'insertion d'un caractère, l'opération est émise
% au serveur possédant le document. Celui-ci, fonctionnant avec une approche de
% transformées opérationnelles~\cite{nichols1995high}, transforme l'opération
% reçue vis-à-vis de toutes celles que l'utilisateur n'avait pas encore reçu lors
% de la création de son opération. La transformation s'avère toutefois moins
% coûteuse puisque seul le serveur doit l'effectuer. En particulier, le vecteur
% d'horloges ou de contexte transporté dans les messages des approches OT
% décentralisées n'existe pas dans les versions centralisée puisque seule importe
% la paire de versions de l'utilisateur et du serveur. En revanche, le serveur est
% en charge de toutes les transformations. De plus, il doit diffuser l'ensemble
% des changements aux participants. Le fournisseur de service prend en charge la
% presque intégralité du coût de la session d'édition. Pour les utilisateurs, se
% posent toujours les problèmes de confidentialité : le fournisseur de service
% voit les documents et peut en utiliser le contenu à ses propres fins. Pire
% encore, la propriété du document doit bien souvent être accordée au fournisseur
% de service. À cela s'ajoute le problème du point individuel de défaillance : si
% le serveur hébergeant le document tombe en panne, le document n'est plus
% accessible.

% Malgré tous ces défauts, ces approches centralisées sont les plus populaires.
% Jusqu'à présent, aucun éditeur décentralisé dans le navigateur Web n'avait été
% déployé. La récente technologie WebRTC~\cite{webrtc} permet d'établir des canaux
% de communication d'un navigateur à l'autre. En d'autres termes, les applications
% réellement décentralisées aussi facile d'accès que les applications Web
% hébergées sur le Nuage deviennent possibles.


 

% Utiliser les services de signalement et les connections WebRTC existantes permet
% de déployer facilement les protocoles d'échantillonnage aléatoire de
% pairs~\cite{jelasity2007gossip}. Ces derniers étant présent dans les navigateurs
% modernes disponibles sur les smartphones, les tablettes, etc. Dans ce contexte,
% il est impératif de conserver autant que possible un petit nombre de connections
% afin de réduire le trafic réseau et limiter la consommation de ressources.


%%% Local Variables:
%%% mode: latex
%%% TeX-master: "../../paper"
%%% End:
