
\section{État de l'art}
\label{editor:sec:stateoftheart}

\begin{figure}
  \begin{center}
    
\begin{tikzpicture}[scale=1.2]

  \newcommand\X{125pt}
  \newcommand\Y{100pt}
  
  \newcommand\LIGHTGRAY{gray!20}
  \newcommand\MEDIUMGRAY{gray!40}
  
  \draw(0,0) +(-\X, -\Y) rectangle +( \X, \Y);
  
  \draw[fill=\LIGHTGRAY](-0.5*\X,-0.5*\Y) +(-0.5*\X,-0.5*\Y) rectangle +(0.5*\X,0.5*\Y);
  \draw(-0.5*\X, 0.5*\Y) +(-0.5*\X,-0.5*\Y) rectangle +(0.5*\X,0.5*\Y);
  \draw(0.5*\X,-0.5*\Y) +(-0.5*\X,-0.5*\Y) rectangle +(0.5*\X,0.5*\Y);
  \draw[fill=\LIGHTGRAY](0.5*\X, 0.5*\Y) +(-0.5*\X,-0.5*\Y) rectangle +(0.5*\X,0.5*\Y);
  
  \draw(-0.5*\X,\Y) node[anchor=south]{\textsc{same time}};
  \draw(0.5*\X,\Y) node[anchor=south]{\textsc{different time}};
  \draw(-\X,0.5*\Y) node[anchor=east]{\textsc{same place}};
  \draw(-\X,-0.5*\Y) node[anchor=east]{\textsc{different place}};

  \small

  \draw (-0.5*\X, 0.5*\Y) node[align=center]{\textbf{Face to face interactions}\\
    (e.g. decision rooms, single\\ display groupware, shared\\ table, wall displays,
    roomware)};

  \draw ( 0.5*\X, 0.5*\Y) node[align=center]{\textbf{Continuous task}\\ (e.g. team rooms,
    large public\\ displays, shift work, groupware,\\ project management)};

  \draw (-0.5*\X,-0.5*\Y) node[align=center]{\DARKBLUE{\textbf{Remote
        interactions}}\\
    (e.g. video,
    conferencing,\\ instant messaging,\\ chats/MUDs/virtual worlds,\\ shared screens,
    multi-user\\ editors)};

  \draw ( 0.5*\X,-0.5*\Y) node[align=center]{
    \textbf{Communication + coordination}\\
    (e.g. email, bulletin boards,\\ blogs, asynchronous\\ conferencing, group
    calendars,\\ workflow, version control,\\ wikis)};

\end{tikzpicture}
    \caption[Matrice de distribution des systèmes collaboratifs] {
      \label{editor:fig:groupware} Matrice de distribution des systèmes
      collaboratifs~\cite{johansen1988groupware}.}
  \end{center}
\end{figure}


L'édition collaborative répartit l'écriture selon deux dimensions : le temps et
l'espace~\cite{desanctis1987foundation, grudin1994computersupported,
  johansen1988groupware}.  La figure~\ref{editor:fig:groupware} résume cette
répartition. Dans ce chapitre, nous nous intéressons particulièrement :
\begin{itemize}
\item aux collaborateurs connectés en \textbf{même temps} en des \textbf{lieux
    différents}, ce qui correspond à la problématique du temps réel. Dans ce
  contexte, \emph{temps réel} signifie que les modifications doivent être
  perçues au plus tôt~\cite{ellis1989concurrency}. La tolérance d'un utilisateur
  varie selon les outils collaboratifs à sa disposition. Une modification locale
  doit être observée immédiatement; une modification distante peut prendre plus
  de temps;
\item aux collaborateurs connectés à \textbf{différents moments} en des
  \textbf{lieux différents}, ce qui correspond au travail en mode hors
  ligne. Les documents sont synchronisés lorsqu'ils en ont l'occasion.
\end{itemize}

Depuis la première démonstration d'éditeur collaboratif temps réel en
1968~\cite{engelbart1968research}, les technologies ont bien changé. Entre
autres, le web est devenu un terrain fertile pour les nouvelles
applications~\cite{lautamaki2013development}. Elles sont bien souvent préférées
par les utilisateurs aux applications natives ou aux
extensions~\cite{mogan2010impact}; elles sont bien souvent préférées par les
entreprises qui s'évitent le développement d'applications spécifiques aux
systèmes d'exploitation et aux matériels informatiques~\cite{mogan2010impact}.

Entre autres, les éditeurs web tels que \emph{Google Docs}~\cite{googledocs} ou
\emph{Etherpad}~\cite{etherpad} ont rendu l'édition collaborative aisée pour les
millions d'utilisateurs à travers le monde. Un simple lien permet à plusieurs
collaborateurs d'accéder, puis lire et modifier en temps réel un document. La
finalité des éditeurs est variée. Par exemple, certains éditeurs permettent la
rédaction de documents génériques~\cite{etherpad, googledocs, googlewave,
  hivejs}, tandis que d'autres se concentrent sur la rédaction de documents
scientifiques~\cite{authorea, overleaf, sharelatex, fidus}, ou sur l'écriture de
code~\cite{lautamaki2012cored, hyperdev}.  Cependant, les éditeurs collaboratifs
répartis dans les navigateurs web se reposent sur une topologie centralisée où
quelques serveurs prennent en charge un nombre important de clients. En résulte
des problèmes de confidentialité, de censure, de passage à l'échelle et de point
unique de défaillance.



\begin{figure}
  \begin{center}
    \begin{tikzpicture}[scale=1.2]
  
  \newcommand\X{40pt}
  \newcommand\Y{-50pt}
  
  \draw[->, very thick, color=darkblue](-2.5-2*\X, 5+1*\Y)--(-2.5-1.5*\X, -10pt);
  \draw[<-, dashed](2.5-2*\X, 5+1*\Y)--(2.5-1.5*\X, -10pt);

  \draw[->, dashed](-2.5-1*\X, 5+1*\Y)--(-2.5-1*\X, -10pt);
  \draw[<-, very thick, color=darkblue](2.5-1*\X, 5+1*\Y)--(2.5-1*\X, -10pt);

  \draw[->, dashed](-2.5+0*\X, 5+1*\Y)--(-2.5+0*\X, -10pt);
  \draw[<-, very thick, color=darkblue](2.5+0*\X, 5+1*\Y)--(2.5+0*\X, -10pt);
    
  \draw[->, dashed](-2.5+1*\X, 10+1*\Y)--(-2.5+0.5*\X, -10pt);
  \draw[<-, very thick, color=darkblue](2.5+1*\X, 10+1*\Y)--(2.5+0.5*\X, -10pt);


  \draw[fill=white](-0.5*\X, 0*\Y) node{Service provider's document server}
  +(-70pt,-10pt )rectangle+(70pt,10pt);

  \draw[fill=white](-2*\X, 1*\Y) node{$e_1$}
  +(5pt, 5pt) rectangle +(-5pt, -5pt) rectangle +(5pt, 5pt);
  \draw[fill=white](-1*\X, 1*\Y) node{$e_2$}
  +(5pt, 5pt) rectangle +(-5pt, -5pt) rectangle +(5pt, 5pt);
  \draw[fill=white]( 0*\X, 1*\Y) node{$e_3$}
  +(5pt, 5pt) rectangle +(-5pt, -5pt) rectangle +(5pt, 5pt);



  \draw[->,dashed](-2.5+1.5*\X, 5+2*\Y)--(-2.5+1.5*\X, -10+1*\Y);
  \draw[<-,very thick, color=darkblue]( 2.5+1.5*\X, 5+2*\Y)--( 2.5+1.5*\X, -10+1*\Y);

  \draw[->,dashed](-2.5+2.5*\X, 5+2*\Y)--(-2.5+2.5*\X, -10+1*\Y);
  \draw[<-,very thick, color=darkblue]( 2.5+2.5*\X, 5+2*\Y)--( 2.5+2.5*\X, -10+1*\Y);

  \draw[->,dashed](-2.5+3.5*\X, 5+2*\Y)--(-2.5+3.5*\X, -10+1*\Y);
  \draw[<-,very thick, color=darkblue]( 2.5+3.5*\X, 5+2*\Y)--( 2.5+3.5*\X, -10+1*\Y);

  \draw[->,dashed](-2.5+4.5*\X, 5+2*\Y)--(-2.5+4.5*\X, -10+1*\Y);
  \draw[<-,very thick, color=darkblue]( 2.5+4.5*\X, 5+2*\Y)--( 2.5+4.5*\X, -10+1*\Y);

  \draw[fill=white](3*\X, 1*\Y) node{Service provider's transmission server}
  +(-100pt,-10pt )rectangle+(100pt,10pt);

  \draw[fill=white]( 1.5*\X, 2*\Y) node{$e_4$}
  +(5pt, 5pt) rectangle +(-5pt, -5pt) rectangle +(5pt, 5pt);
  \draw[fill=white]( 2.5*\X, 2*\Y) node{$e_5$}
  +(5pt, 5pt) rectangle +(-5pt, -5pt) rectangle +(5pt, 5pt);
  \draw[fill=white]( 3.5*\X, 2*\Y) node{$e_6$}
  +(5pt, 5pt) rectangle +(-5pt, -5pt) rectangle +(5pt, 5pt);
  \draw[fill=white]( 4.5*\X, 2*\Y) node{$e_7$}
  +(5pt, 5pt) rectangle +(-5pt, -5pt) rectangle +(5pt, 5pt);

\end{tikzpicture}
    \caption[Fonctionnement des éditeurs centralisés]
    {\label{editor:fig:serviceprovider} Les transmissions aux clients connectés
      et les transformations d'opérations sont à la charge du fournisseur de
      services.}
  \end{center}
\end{figure}

La figure~\ref{editor:fig:serviceprovider} montre le fonctionnement de ces
éditeurs. Un serveur central possède le document. Les participants se connectent
à ce serveur (i.e. $e_1$, $e_2$, et $e_3$) ou à un serveur associé (i.e. $e_4$,
$e_5$, $e_6$, $e_7$) mis à disposition de manière élastique afin d'alléger la
charge de diffusion des messages du premier. Lorsque le collaborateur $e_1$
effectue une action telle que l'insertion d'un caractère, l'opération est émise
au serveur possédant le document. Celui-ci, fonctionnant avec une approche de
transformées opérationnelles~\cite{nichols1995high}, transforme l'opération
reçue vis-à-vis de toutes celles que l'utilisateur n'avait pas encore reçu lors
de la création de son opération. La transformation s'avère toutefois moins
coûteuse puisque seul le serveur doit l'effectuer. En particulier, le vecteur
d'horloges ou de contexte transporté dans les messages des approches OT
décentralisées n'existe pas dans les versions centralisée puisque seule importe
la paire de versions de l'utilisateur et du serveur. En revanche, le serveur est
en charge de toutes les transformations. De plus, il doit diffuser l'ensemble
des changements aux participants. Le fournisseur de service prend en charge la
presque intégralité du coût de la session d'édition. Pour les utilisateurs, se
posent toujours les problèmes de confidentialité : le fournisseur de service
voit les documents et peut en utiliser le contenu à ses propres fins. Pire
encore, la propriété du document doit bien souvent être accordée au fournisseur
de service. À cela s'ajoute le problème du point individuel de défaillance : si
le serveur hébergeant le document tombe en panne, le document n'est plus
accessible.

Malgré tous ces défauts, ces approches centralisées sont les plus populaires.
Jusqu'à présent, aucun éditeur décentralisé dans le navigateur web n'avait été
déployé. La récente technologie WebRTC~\cite{webrtc} permet d'établir des canaux
de communication d'un navigateur à l'autre. En d'autres termes, les applications
réellement décentralisée aussi facile d'accès que les applications web hébergées
sur le Nuage deviennent possibles.


 

% Utiliser les services de signalement et les connections WebRTC existantes permet
% de déployer facilement les protocoles d'échantillonnage aléatoire de
% pairs~\cite{jelasity2007gossip}. Ces derniers étant présent dans les navigateurs
% modernes disponibles sur les smartphones, les tablettes, etc. Dans ce contexte,
% il est impératif de conserver autant que possible un petit nombre de connections
% afin de réduire le trafic réseau et limiter la consommation de ressources.


%%% Local Variables:
%%% mode: latex
%%% TeX-master: "../../paper"
%%% End:
