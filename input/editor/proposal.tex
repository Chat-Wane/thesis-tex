
\section{\CRATE : Un éditeur décentralisé dans les navigateurs}
\label{editor:sec:crate}

\CRATE est un éditeur décentralisé permettant l'édition en temps réel de
documents directement dans les navigateurs web. Cette section décrit son
architecture avant de détailler chacun des composants constituant cette
architecture.


\begin{figure}
  \begin{center}
    \input{input/editor/figarchitecture.tex}
    \caption{\label{editor:fig:architecture}Architecture en 4 couches de \CRATE.}
  \end{center}
\end{figure}

La figure~\ref{editor:fig:architecture} montre l'architecture en 4 couches de
\CRATE. Chacune de ces couches peut devenir un obstacle au passage à l'échelle
de l'éditeur :
\begin{inparaenum}[(i)]
\item la couche de communication comprend le méchanisme d'appartenance au réseau
  et la propagation des messages dans ce réseau;
\item la couche de causalité comprend la structure permettant de lier les
  opérations entre elles afin qu'elles soient intégrées dans un ordre reflétant
  une forme de causalité, e.g., elle assure que les opérations de suppression
  suivent toujours les opérations d'insertion de l'élément correspondant;
\item la couche de structure pour séquences dont les opérations d'insertions et
  de suppression doivent garantir des répliques convergeantes du document;
\item la couche d'interaction homme-machine fournissant les outils avec lesquels
  l'utilisateur peut intéragir.
\end{inparaenum}

La partie gauche de la figure montre le processus le plus courant : Lorsqu'un
participant effectue une opération sur le document, l'opération est appliquée à
la séquence répartie. L'opération est ensuite décorée avec des métadonnées
correspondant à la causalité. Enfin, l'éditeur propage l'opération en utilisant
le voisinage de l'éditeur fournit par le protocol d'appartenance au réseau.  À
l'inverse, lorsque l'éditeur reçoit une opération, il vérifie si cette dernière
est prête à être intégrée. Lorsque la condition est vérifiée, l'éditeur intégre
l'opération à la réplique de la séquence. Enfin l'interface utilisateur est
notifiée du changement.

La partie droite de la figure correspond à la stratégie de rattrapage où un
participant a peut-être manqué quelques opérations à cause de pertes de messages
dans le réseau, ou simplement car il était hors-ligne pendant un moment. Ainsi,
dès que l'éditeur est en ligne, il vérifie régulièrement auprès de ses voisins
s'il lui manque des opérations.


\subsection{Communications}

Les sessions d'édition peuvent rassembler de petits groupes comme de larges
groupes pendant leur durée de vie. Par exemple, un cours de formation en ligne
ouverte à tous (\emph{MOOC}) peut commencer avec un grand nombre d'étudiants
dont le nombre peut s'amoindrir très rapidement par de manque
d'intérêt~\cite{breslow2013studying}. De plus, les sessions d'édition varient en
taille selon la portée du document. Par exemple, un document décrivant un projet
personnel et dont la visibilité est limitée aux amis rassemble significativement
moins de monde qu'un document décrivant un grand événement, tel que le rapport
collaboratif d'une conférence. La couche de communication doit être en mesure de
gérer de manière transparente toute session d'édition, quelle que soit sa
taille, tout en passant à l'échelle.

\CRATE utilise \SPRAY (cf. chapitre~\ref{net:chap:spray}) afin
d'auto-ajuster son fonctionnement à la session d'édition. Ainsi, chaque éditeur
possède une vue partielle avec laquelle communiquer. La taille de cette dernière
croît et décroît logarithmiquement par rapport à la taille du réseau. Si une
session d'éditoin démarre avec 10 auteurs, ils auront 2.3 voisins en moyenne. Si
la session d'édition grandit pour atteindre le millier de participants, ceux-ci
auront 6.9 voisins en moyenne. Enfin, si l'édition se retrouve avec 10
participants à nouveau, ils auront à nouveau 2.3 voisins en moyenne.

La diffusion des messages fait un usage intensif des voisinnages. Lorsqu'un
utilisateur procède à des changements dans le document, l'éditeur l'envoie à son
voisinnage. Chacun des voisins ayant reçu le message le transmet à son tour à
son voisinnage. Les modifications sur le document atteignent rapidement tous les
éditeurs par transitivité. La complexité en communication chez chaque éditeur
est bornée par $\mathcal{O}(M\ln(R))$, où $M$ est la taille du message, et
$\mathcal{R}$ est le nombre de répliques connectées au moment de la propagation.

\begin{figure}
  \begin{center}
    \input{input/editor/figenteringsession.tex}
    \caption{\label{editor:fig:entering}Fonctionnement d'une session d'édition.}
  \end{center}
\end{figure}

La figure~\ref{editor:fig:entering} décrit le processus d'entrée dans le réseau.
Une session d'édition, composée de 8 éditeurs existe. Celle-ci est inaccessible
depuis l'exterieur. Pour rejoindre le réseau, au moins un des éditeurs doit
partager son accès. Pour ce faire, il crée une connexion avec un serveur de
signalisation facilement accessible via \emph{websocket}. L'éditeur ayant
partagé son accès donne à son collègue une adresse URL contenant suffisamment
d'informations pour qu'il retrouve le médiateur et demande à rejoindre la
session d'édition. Lorsqu'il clique sur l'adresse, une connexion est établie
avec le serveur de signalisation qui va jouer le rôle de médiateur afin de créer
le premier canal de communication WebRTC entre le nouveau membre et l'éditeur
partageant l'accès. Une fois celui-ci établi, le nouvel arrivant se déconnecte
du médiateur et applique le protocole \SPRAY en utilisant le canal
WebRTC. Ensuite, les éditeurs eux-même deviennent des médiateurs et permettent
d'établir les connexions WebRTC d'un voisinnage à l'autre.

\subsection{Détection de causalité}

Pour préserver des répliques cohérentes, le même résultat doit advenir lors de
la création de l'opération et de son intégration. Souvent, le comportement d'une
opération dépend d'autres opérations précédemment intégrées.
\TODO{Moar.}

\begin{figure}
  \begin{center}
    \begin{tikzpicture}[scale=1.35]
  
  \newcommand\X{ 20pt}
  \newcommand\Y{-30pt}

  \draw[->] (0pt,0pt) node[anchor=east]{$\pmb{c_1}$} -- (11*\X,0pt);
  \draw[->] (0pt,-30pt) node[anchor=east]{$\pmb{c_2}$} -- (11*\X,\Y);
  \draw[->] (0pt,-60pt) node[anchor=east]{$\pmb{c_3}$} -- (11*\X,2*\Y);

  \scriptsize
  \draw (0.5*\X,2+0*\Y) node[anchor=south]{[ ]} -- (0.5*\X,-2+0*\Y);
  \draw (0.5*\X,2+1*\Y) node[anchor=south]{[ ]} -- (0.5*\X,-2+1*\Y);
  \draw (0.5*\X,2+2*\Y) node[anchor=south]{[ ]} -- (0.5*\X,-2+2*\Y);

  \draw[->,densely dashdotted] (1.5*\X,0pt) -- (2.5*\X,2*\Y);
  \draw[->,densely dashdotted] (2.5*\X,0pt) -- (3.5*\X,1*\Y);
  \draw[->,densely dashdotted] (2.5*\X,0pt) -- (3.5*\X,2*\Y);


  \draw (5*\X,2+1*\Y) node[anchor=south]{[$\langle c_1,\,2,\,\DARKBLUE{\pmb{\{1\}}}\rangle$]}
  -- (5*\X,-2+1*\Y);
  \draw (5*\X,2+2*\Y) node[anchor=south]{[$\langle c_1,\,2,\,\varnothing\rangle$]}
  -- (5*\X,-2+2*\Y);

  \draw [->,densely dashdotted, very thick, color=darkblue]
  (1.5*\X,0pt) to[out=35,in=135] node[anchor=north]{\DARKBLUE{\textbf{lent}}}
  (6.65*\X,0pt) to[out=-45,in=110] (7.5*\X,\Y);

  \draw[fill=white] (1.5*\X,0pt) circle (2pt);
  \draw[fill=white] (2.5*\X,0pt) circle (2pt);

  \draw[->,densely dashdotted] (6*\X,2*\Y)--(6.5*\X,1*\Y);
  \draw[->,densely dashdotted] (6*\X,2*\Y)--(6.5*\X,0pt);

  \draw[fill=black] (6*\X,2*\Y) circle (2pt);
  
  \draw[->,densely dashdotted, very thick, color=darkblue]
  (6.5*\X,1*\Y)to[out=-35,in=-145]
  node[anchor=north]{\DARKBLUE{\textbf{attend}}}(8.5*\X,1*\Y);


  \draw (9.5*\X,2pt) node[anchor=south]
  {[$\langle c_1,\,2,\,\varnothing\rangle \langle c_3,\,1,\,\varnothing\rangle$]}
  -- (9.5*\X,-2pt);
  \draw (9.5*\X,2+1*\Y) node[anchor=south]
  {[$\langle c_1,\,2,\,\varnothing\rangle \langle c_3,\,1,\,\varnothing\rangle$]}
  -- (9.5*\X,-2+1*\Y);
  \draw (9.5*\X,2+2*\Y) node[anchor=south]
  {[$\langle c_1,\,2,\,\varnothing\rangle \langle c_3,\,1,\,\varnothing\rangle$]}
  -- (9.5*\X,-2+2*\Y);
  
  \begin{scope}[shift={(2*\X, 2.5*\Y)}]
  \draw[->,densely dashdotted] (0pt,-1pt) -- (10pt,-1pt)
  node[anchor=west]{messages};
  \draw[fill=white] (55pt,0pt)node[anchor=west]{$\,\textsc{insert}$}circle (2pt);
  \draw[fill=black] (92.5pt,0pt)
  node[anchor=west]{$\,\textsc{delete}(\langle c_1,\,1\rangle)$} circle (2pt);
  \end{scope}

\end{tikzpicture}

    \caption{\label{editor:fig:timeline}Exemple de détection de relations
      causales. Les triples dans les vecteurs sont composés de
      $\langle origin,\, max,\, exceptions\rangle$.}
  \end{center}
\end{figure}

\subsection{Séquence répliquée}


\subsection{Interface utilisateur}

\begin{figure}
  \begin{center}
    \includegraphics[scale=0.6]{img/editor/cratescreenshot.png}
    \caption{\label{editor:img:screenshot}Capture d'écran de \CRATE.}
  \end{center}
\end{figure}

La capture d'écran en figure~\ref{editor:img:screenshot} montre l'interface
visible par l'utilisateur. Dans cet exemple, au moins trois participants sont
impliqués dans la session d'édition. En effet, 3 curseurs sont affichés. Le
premier curseur appartient à \emph{Anonymous Mole} et semble être à l'origine du
texte \emph{Hello}. Le second curseur appartient à \emph{Anonymous Penguin} et
semble être à l'origine du texte \emph{World}. Le troisième curseur est celui de
l'utilisateur ayant prit la capture d'écran.

La barre d'état nous indique que
\begin{inparaenum}[(i)]
\item l'éditeur est en train de partager l'accès à la session d'édition via le
  cercle bleue en rotation. L'utilisateur peut alors donner l'URL en bas de page
  à d'autres collaborateurs afin qu'ils le rejoignent dans l'écriture du
  document, en un simple clique;
\item que l'utilisateur est bien connecté à d'autres collaborateurs via la
  planète verte.
\end{inparaenum}

La barre d'état possède aussi des boutons tels que
\begin{inparaenum}[(i)]
\item la disquette qui enregistre sur le disque la réplique locale du
  document. En ouvrant ce fichier, l'éditeur est capable de se réconnecter à la
  session d'édition sans avoir à rattrapper toutes les opérations qui lui manque
  depuis son départ;
\item l'oeil sert à visualiser le texte écrit dans le langage
  Markdown~\cite{markdown}. Ainsi, le document n'est plus une simple suite de
  caractères mais est interprété afin de présenter un document structuré plus
  agréable à lire;
\item la chaîne sert à partager l'accès au document ou à le stopper;
\item les engrenages servent à la configuration.
\end{inparaenum}


%%% Local Variables:
%%% mode: latex
%%% TeX-master: "../../paper"
%%% End:
