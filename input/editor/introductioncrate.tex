
\section{Crate : Un éditeur décentralisé dans les navigateurs}
\label{editor:sec:crate}

\begin{figure}
  \begin{center}
    \includegraphics[scale=0.5]{img/editor/cratescreenshot.png}
    \caption{\label{editor:img:screenshot}Capture d'écran de \CRATE.}
  \end{center}
\end{figure}

La capture d'écran en figure~\ref{editor:img:screenshot} montre l'interface
visible par l'utilisateur. Dans cet exemple, au moins trois participants sont
impliqués dans la session d'édition. En effet, 3 curseurs sont affichés. Le
premier curseur appartient à \emph{Anonymous Mole} et semble être à l'origine du
texte \emph{Hello}. Le second curseur appartient à \emph{Anonymous Penguin} et
semble être à l'origine du texte \emph{World}. Le troisième curseur est celui de
l'utilisateur ayant prit la capture d'écran.

La barre d'état nous indique que
\begin{inparaenum}[(i)]
\item l'éditeur est en train de partager l'accès à la session d'édition via le
  cercle bleue en rotation. L'utilisateur peut alors donner l'URL en bas de page
  à d'autres collaborateurs afin qu'ils le rejoignent dans l'écriture du
  document, en un simple clique;
\item que l'utilisateur est bien connecté à d'autres collaborateurs via la
  planète verte.
\end{inparaenum}

La barre d'état possède aussi des boutons tels que
\begin{inparaenum}[(i)]
\item la disquette qui enregistre sur le disque la réplique locale du
  document. En ouvrant ce fichier, l'éditeur est capable de se réconnecter à la
  session d'édition sans avoir à rattrapper toutes les opérations qui lui manque
  depuis son départ;
\item l'oeil sert à visualiser le texte écrit dans le langage
  Markdown~\cite{markdown}. Ainsi, le document n'est plus une simple suite de
  caractères mais est interprété afin de présenter un document structuré plus
  agréable à lire;
\item la chaîne sert à partager l'accès au document ou à le stopper;
\item les engrenages servent à la configuration.
\end{inparaenum}


\subsection{Architecture}

\begin{figure}
  \begin{center}
    \input{input/editor/figurearchitecture.tex}
    \caption{\label{editor:fig:architecture}Architecture en 4 couches de \CRATE.}
  \end{center}
\end{figure}

La figure~\ref{editor:fig:architecture} montre l'architecture en 4 couches de
\CRATE. Chacune de ces couches peut devenir un obstacle au passage à l'échelle
de l'éditeur :
\begin{inparaenum}[(i)]
\item la couche de communication comprend le méchanisme d'appartenance au réseau
  et la propagation des messages dans ce réseau;
\item la couche de causalité comprend la structure permettant de lier les
  opérations entre elles afin qu'elles soient intégrées dans un ordre reflétant
  une forme de causalité, e.g., elle assure que les opérations de suppression
  suivent toujours les opérations d'insertion de l'élément correspondant;
\item la couche de structure pour séquences dont les opérations d'insertions et
  de suppression doivent garantir des répliques convergeantes du document;
\item la couche d'interaction homme-machine fournissant les outils avec lesquels
  l'utilisateur peut intéragir.
\end{inparaenum}

La partie gauche de la figure montre le processus le plus courant : Lorsqu'un
participant effectue une opération sur le document, l'opération est appliquée à
la séquence répartie. L'opération est ensuite décorée avec des métadonnées
correspondant à la causalité. Enfin, l'éditeur propage l'opération en utilisant
le voisinage de l'éditeur fournit par le protocol d'appartenance au réseau.  À
l'inverse, lorsque l'éditeur reçoit une opération, il vérifie si cette dernière
est prête à être intégrée. Lorsque la condition est vérifiée, l'éditeur intégre
l'opération à la réplique de la séquence. Enfin l'interface utilisateur est
notifiée du changement.

La partie droite de la figure correspond à la stratégie de rattrapage où un
participant a peut-être manqué quelques opérations à cause de pertes de messages
dans le réseau, ou simplement car il était hors-ligne pendant un moment. Ainsi,
dès que l'éditeur est en ligne, il vérifie régulièrement auprès de ses voisins
s'il lui manque des opérations.

\subsection{Processus}

\begin{figure}
  \begin{center}
    \input{input/editor/livefigureB.tex}
    \caption{\label{editor:fig:processus}Fonctionnement d'une session d'édition.}
  \end{center}
\end{figure}

\subsection{Anti-entropie}

%%% Local Variables:
%%% mode: latex
%%% TeX-master: "../../paper"
%%% End:
