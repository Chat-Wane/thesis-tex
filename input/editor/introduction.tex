
\lettrine{L}'édition collaborative~\cite{ellis1989concurrency,
  johansen1988groupware} permet de répartir la rédaction d'un document au cours
du temps et à travers l'espace~\cite{desanctis1987foundation,
  grudin1994computersupported, johansen1988groupware}. Elle améliore
l'implication des participants et la qualité des documents
produits~\cite{giles2005internet, noel2004empirical}. Les éditeurs collaboratifs
temps réel permettent à un utilisateur de visualiser, modifier, et partager un
document. Le partage du document avec un ou plusieurs collaborateurs permet à
ces derniers d'avoir à leur tour accès à la lecture et la rédaction du document.

Bien que le premier éditeur collaboratif remonte à
1968~\cite{engelbart1968research}, leur adoption demeure récente. En effet, les
éditeurs collaboratifs ont été largement popularisés grâce aux services Web tels
que \emph{Google Docs}~\cite{googledocs} ou \emph{Etherpad}~\cite{etherpad}. Ces
services, directement accessibles via les navigateurs Web, rendent aisée la
participation aux sessions d'édition temps réel~\cite{mogan2010impact}. L'accès
au document partagé s'effectue grâce à un simple lien. Chaque modification
apportée au document est visible des collaborateurs ayant ouvert le document.

Cette facilité d'accès est rendue possible par le biais d'un serveur centralisé
appartenant à un fournisseur de service hébergé sur le
Nuage~\cite{mell2011national}. Cette organisation, où quelques serveurs sont en
charge d'un grand nombre de clients, pose des problèmes de confidentialité,
de censure, de passage à l'échelle, et de point unique de défaillance. Seule la
décentralisation permet de résoudre ces problèmes. Toutefois, sa possibilité
d'intégration dans les navigateurs Web via WebRTC~\cite{webrtc} n'est que
récente.

%%% L'édition collaborative temps réel est-elle possible sur le web, sans
%%%  l'intervention d'un tier et sans limites quant aux dimensions du système

Ce chapitre présente \CRATE, le premier\footnote{à notre connaissance} éditeur
collaboratif temp réel complètement décentralisé et fonctionnant dans les
navigateurs Web. À ce titre, un document n'appartient qu'aux membres de la
session d'édition. Les collaborateurs, par leur présence, participent au bon
fonctionnement de la session d'édition. Pour passager à l'échelle, \CRATE
utilise à la fois \SPRAY (cf. §\ref{net:chap:spray}) pour que chaque
modification soit propagée efficacement à l'ensemble des éditeurs impliqués dans
la rédaction; et \LSEQ (cf. §\ref{repl:chap:lseq}) pour que les répliques du
document hébergé par chaque éditeur soient identiques. Le trafic généré par
l'application est logarithmique comparé à la taille de la session d'édition, et
polylogarithmique comparé au nombre d'insertions effectuées dans le document.

Le reste de ce chapitre s'organise de la façon suivante : tout d'abord, la
section~\ref{editor:sec:stateoftheart} présente l'état de l'art en de plus
amples détails. La section~\ref{editor:sec:crate} introduit \CRATE, son
architecture et ses composants internes, ainsi que son fonctionnement. La
section~\ref{editor:sec:experimentation} décrit les résultats obtenus lors d'une
expérimentation à large échelle impliquant jusqu'à 600 navigateurs. La
section~\ref{editor:sec:conclusion} conclut ce chapitre.


%%% Local Variables:
%%% mode: latex
%%% TeX-master: "../../paper"
%%% End:
