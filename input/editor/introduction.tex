
\lettrine{L}'édition collaborative~\cite{ellis1989concurrency,
  johansen1988groupware} permet de répartir la rédaction d'un document au cours
du temps et à travers l'espace~\cite{desanctis1987foundation,
  grudin1994computersupported, johansen1988groupware}. Elle améliore
l'implication des participants et la qualité des documents
produits~\cite{giles2005internet, noel2004empirical}. Les éditeurs collaboratifs
temps réel permettent à un utilisateur de visualiser, modifier, et partager un
document. Le partage du document avec un ou plusieurs collaborateurs permet à
ces derniers d'avoir à leur tour accès à la lecture et la rédaction du document.

Depuis la première démonstration d'éditeur collaboratif temps réel en
1968~\cite{engelbart1968research}, les technologies ont bien changé. Entre
autres, le Web est devenu un terrain fertile pour les nouvelles
applications~\cite{lautamaki2013development}. Ces dernières sont bien souvent
préférées par les utilisateurs aux applications natives ou aux extensions pour
leur facilité d'usages~\cite{mogan2010impact}; elles sont bien souvent préférées
par les entreprises qui s'évitent le développement d'applications spécifiques
aux systèmes d'exploitation et aux matériels
informatiques~\cite{mogan2010impact}.

Les éditeurs Web ont des fins variées.  Certains éditeurs Web permettent la
rédaction de documents génériques~\cite{etherpad, googledocs, googlewave,
  hivejs}, tandis que d'autres se concentrent sur la rédaction de documents
scientifiques~\cite{authorea, overleaf, sharelatex, fidus}, ou sur l'écriture de
code~\cite{lautamaki2012cored, hyperdev}.

Les éditeurs Web tels que Google Docs~\cite{googledocs} ou
Etherpad~\cite{etherpad} ont grandement contribué à rendre ces outils
populaires. L'édition collaborative est devenue aisée pour les millions
d'utilisateurs à travers le monde. Un simple lien permet à plusieurs
collaborateurs d'accéder, puis lire et modifier un document en temps réel.
Cette facilité d'accès est rendue possible par le biais d'un serveur centralisé
appartenant à un fournisseur de service hébergé sur le
Nuage~\cite{mell2011national}. Cette organisation, où quelques serveurs sont en
charge d'un grand nombre de clients, pose des problèmes de confidentialité, de
censure, de passage à l'échelle, et de point unique de défaillance. Seule la
décentralisation permet de résoudre ces problèmes. Toutefois, sa possibilité
d'intégration dans les navigateurs Web via WebRTC~\cite{webrtc} n'est que
récente.

% La finalité des éditeurs est variée. Par exemple, certains éditeurs permettent
% la rédaction de documents génériques~\cite{etherpad, googledocs, googlewave,
%   hivejs}, tandis que d'autres se concentrent sur la rédaction de documents
% scientifiques~\cite{authorea, overleaf, sharelatex, fidus}, ou sur l'écriture de
% code~\cite{lautamaki2012cored, hyperdev}.  Cependant, les éditeurs collaboratifs
% répartis dans les navigateurs Web se reposent sur une topologie centralisée où
% quelques serveurs prennent en charge un nombre important de clients. En résulte
% des problèmes de confidentialité, de censure, de passage à l'échelle et de point
% unique de défaillance.

% Bien que le premier éditeur collaboratif remonte à
% 1968~\cite{engelbart1968research}, leur adoption demeure récente. En effet, les
% éditeurs collaboratifs ont été largement popularisés grâce aux services Web tels
% que Google Docs~\cite{googledocs} ou Etherpad~\cite{etherpad}. Ces
% services, directement accessibles via les navigateurs Web, rendent aisée la
% participation aux sessions d'édition temps réel~\cite{mogan2010impact}. L'accès
% au document partagé s'effectue grâce à un simple lien. Chaque modification
% apportée au document est visible des collaborateurs ayant ouvert le document.


%%% L'édition collaborative temps réel est-elle possible sur le web, sans
%%%  l'intervention d'un tier et sans limites quant aux dimensions du système

Ce chapitre présente \CRATE, le premier\footnote{à notre connaissance} éditeur
collaboratif temps réel complètement décentralisé et fonctionnant dans les
navigateurs Web. À ce titre, un document n'appartient qu'aux membres de la
session d'édition. Les collaborateurs, par leur présence, participent au bon
fonctionnement de la session d'édition. Pour passager à l'échelle, \CRATE
utilise à la fois \SPRAY (cf. §\ref{net:chap:spray}) pour que chaque
modification soit propagée efficacement à l'ensemble des éditeurs impliqués dans
la rédaction; et \LSEQ (cf. §\ref{repl:chap:lseq}) pour que les répliques du
document hébergé par chaque éditeur soient identiques. Le trafic généré par
l'application est logarithmique comparé à la taille de la session d'édition, et
polylogarithmique comparé au nombre d'insertions effectuées dans le document.

Le reste de ce chapitre s'organise de la façon suivante : 
%tout d'abord, la
%section~\ref{editor:sec:stateoftheart} présente l'état de l'art en de plus
%amples détails. 
la section~\ref{editor:sec:crate} introduit \CRATE, son architecture et ses
composants internes, ainsi que son fonctionnement; la
section~\ref{editor:sec:experimentation} décrit les résultats obtenus lors d'une
expérimentation à large échelle impliquant jusqu'à 600 navigateurs; la
section~\ref{editor:sec:conclusion} conclut ce chapitre.


%%% Local Variables:
%%% mode: latex
%%% TeX-master: "../../paper"
%%% End:
