
\lettrine{L}es éditeurs collaboratifs ont été largement popularisés par les
services tels que Google Docs~\cite{googledocs} ou Etherpad~\cite{etherpad}. Ces
services, directement accessibles via les navigateurs web, rendent aisée la
participation aux sessions d'édition temps réel. Grâce à une simple URL
(\emph{Uniform Resource Locator}), le navigateur est en mesure de charger le
document partagé. Chaque modification apportée au document est visible des
collaborateurs ayant ouvert le document.

La facilité d'accès au document est possible grâce à un serveur centralisé mis à
disposition par un fournisseur de service. Malheureusement, cela pose des
problèmes de confidentialité, de passage à l'échelle, et de tolérence aux
pannes. La décentralisation permet de résourdre ces problèmes, hélas, au prix de
l'aisance d'accès.

Un éditeur collaboratif fonctionne grâce à quatre couches :
\begin{inparaenum}[(i)]
\item Une couche réseau afin de propager les opérations;
\item Une couche de détection de causalité afin que les opérations soient
  délivrées dans un certain ordre;
\item Une couche de structure de données afin d'obtenir un document identique
  partout;
\item Une couche d'interface graphique afin de fournir la possibilité à
  l'utilisateur de lire et modifier le document.
\end{inparaenum}
À cet égard, le problème est le suivant :

\begin{itemize}
\item[\textbf{QR.}] \textbf{Quelle combinaison d'éléments permettent la
    construction d'un éditeur collaboratif temps réel dont les performances
    passent à l'échelle et dont le déploiement soit aussi simple que sur le
    Nuage?}
\end{itemize}

La récente possibilité d'établir des canaux de communication d'un navigateur à
l'autre -- grâce à WebRTC -- ouvre la voie aux applications décentralisées
directement accessibles dans le navigateur web.

Ce chapitre présente \CRATE, un éditeur collaboratif décentralisé dans le
navigateur. À ce titre, un document n'appartient qu'aux membres de la session
d'édition. Au lieu d'avoir à déployer un serveur, les même membres participent
activement au bon fonctionnement de la session d'édition. \CRATE utilise \SPRAY
(cf. §\ref{net:chap:spray}) et \LSEQ (cf. §\ref{repl:chap:lseq}). Grâce à cela,
le trafic généré par l'application est logarithmique comparé à la taille de la
session d'édition, et polylogarithmique comparé au nombre d'insertions
effectuées dans le document. À ce titre, l'application passe à l'échelle.

Le reste de ce chapitre s'organise de la façon suivante : Tout d'abord, la
section~\ref{editor:sec:stateoftheart} présente l'état de l'art en de plus
amples détails. La section~\ref{editor:sec:crate} introduit \CRATE, son
architecture et ses composants internes, ainsi que son fonctionnement. La
section~\ref{editor:sec:experimentation} décrit les résultats obtenus lors d'une
expérimentation à large échelle impliquant jusqu'à 600 navigateurs. La
section~\ref{editor:sec:conclusion} conclut ce chapitre.



%%% Local Variables:
%%% mode: latex
%%% TeX-master: "../../paper"
%%% End:
