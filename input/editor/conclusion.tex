
\section{Conclusion}
\label{editor:sec:conclusion}

Ce chapitre a présenté \CRATE, un éditeur collaboratif décentralisé fonctionnant
directement dans les navigateurs Web. Un utilisateur peut créer, modifier en
temps réel, et partager son document. À l'instar des éditeurs centralisés tel
que Google Docs, le partage s'effectue très facilement grâce à un simple
lien. Lorsqu'un utilisateur clique sur celui-ci, il rejoint la session d'édition
et peut à son tour voir, modifier en temps réel et partager le document.

Ce que \CRATE ne gère pas : la persistance des sessions d'édition. De nos jours,
avec l'hégémonie du Nuage, il devient plus difficile de supprimer des données
que de les sauvegarder. À l'opposé, un document dans \CRATE n'appartient qu'à
ses rédacteurs. La session d'édition, et donc le document cessent d'exister
lorsque tous les collaborateurs ferment leur éditeur. Si ceux-ci souhaitent
sauvegarder le document, ils peuvent le faire localement. Si ceux-ci souhaitent
préserver la session d'édition temps réel, ils doivent s'assurer qu'au moins un
éditeur reste actif et accessible.

% demander au Nuage d'héberger un éditeur collaboratif. Ainsi, le document
% deviendrait persistant, et le poids de l'édition resterait réparti entre les
% collaborateurs de l'édition. Par conséquent, le Nuage serait soulagé d'une
% partie de sa charge.

\CRATE démontre que l'édition collaborative temps réel est possible dans les
navigateurs Web, sans l'intervention de tiers et sans limites quant aux
dimensions du système.

% Afin de passer à l'échelle, \CRATE utilise \SPRAY, un protocole
% d'échantillonnage aléatoire de pairs permettant d'adapter les vues partielles de
% chaque éditeur collaboratif à la taille de la session d'édition. De plus, pour
% ne pas surcharger les messages, \CRATE contraint au minimum l'ordre
% d'intégration des opérations. Enfin, \CRATE utilise \LSEQ, une fonction
% d'allocation d'identifiants dont les identifiants ont une borne supérieures
% sous-linéaire par rapport au nombre d'insertions effectuées dans le
% document. Grâce à ces composants internes, \CRATE ajuste son trafic généré
% logarithmiquement par rapport à la taille de la session d'édition, et
% polylogarithmiquement par rapport au nombre d'insertions effectuées sur le
% document. Par conséquent, \CRATE passe à l'échelle. Ces affirmations ont été
% validées au travers d'expérimentations impliquant jusqu'à 601 navigateurs
% écrivant en temps réel.

Le chapitre~\ref{conclu:chap:conclusion} revient sur les précédentes
contributions et présente les perspectives scientifiques ouvertes par celles-ci.

%%% Local Variables:
%%% mode: latex
%%% TeX-master: "../../paper"
%%% End:
