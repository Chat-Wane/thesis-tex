
\section{Conclusion}
\label{editor:sec:conclusion}

Ce chapitre a présenté \CRATE, un éditeur collaboratif décentralisé fonctionnant
directement dans les navigateurs web.

\begin{itemize}
\item [\textbf{QR.}] \textbf{Quelle combinaison d'éléments permettent la construction
  d'un éditeur collaboratif temps réel dont les performances passent à l'échelle
  et dont le deploiement soit aussi simple que sur le Nuage?}
\end{itemize}

Afin de passer à l'échelle, \CRATE utilise \SPRAY, un protocole
d'échantillonnage aléatoire de pairs permettant d'adapter les vues partielles de
chaque éditeur à la taille de la session d'édition. De plus, \CRATE contraint au
minimum l'ordre d'intégration des opérations. Cela lui permet de ne pas ajouter
de surcoût sur les messages. Enfin, \CRATE utilise \LSEQ, une fonction
d'allocation d'identifiants dont les identifiants ont une borne supérieures
sous-linéaire par rapport au nombre d'insertions effectuées dans le
document. Grâce à ces composants internes, \CRATE ajuste son trafic généré
logarithmiquement par rapport à la taille de la session d'édition, et
polylogarithmiquement par rapport au nombre d'insertions effectuées sur le
document. Par conséquent, \CRATE passe à l'échelle. Ces affirmations ont été
validées au travers d'expérimentations impliquant jusqu'à 601 navigateurs
écrivant en temps réel. Le trafic mesurée suit les attentes susmentionnées.

\CRATE est accessible via un navigateur web. Lorsqu'un utilisateur partage son
document, il autorise l'accès par son intermédiaire à la session d'édition. À
l'instar des solutions centralisée (e.g. Google Docs), le partage s'effectue
simplement avec une URL.

Le chapitre~\ref{conclu:chap:conclusion} revient sur les précédentes
contributions et présente les perspectives scientifiques ouvertes par celles-ci.

%%% Local Variables:
%%% mode: latex
%%% TeX-master: "../../paper"
%%% End:
