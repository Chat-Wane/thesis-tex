
\section{Expérimentation}
\label{editor:sec:experimentation}


\begin{figure}
  \begin{center}
    \includegraphics[scale=0.8\textwidth]{img/editor/communication.eps}
    \caption{\label{editor:img:communication} Trafic moyen par seconde généré
      par chaque éditeur pendant la session d'édition. L'axe des abscisses
      montre la progression de l'expérimentation. L'axe des ordonnées montre le
      trafic moyen sortant par éditeur en mégaoctets par seconde.}
  \end{center}
\end{figure}

\paragraph{Objectif :} Montrer que l'éditeur collaboratif décentralisé \CRATE
passe à l'échelle en termes de nombre d'utilisateurs et de taille de
documents.

\paragraph{Description :} Cette expérimentation construit des réseaux de 101 à
601 participants. Chaque session d'édition est en charge d'écrire un document
artificiel de plusieurs millions de caractères en insérant ceux-ci à la fin du
document. Les mesures concernent le trafic sortant moyen de chacun des
membres. Globalement, 100 opérations sont effectuées par seconde, uniformément
réparties parmi les participants. La session dure 7 heures.

\paragraph{Résultat :} La figure~\ref{editor:img:communication} montre le
résultat de cette expérimentation.
\begin{inparaenum}[(i)]
\item Le trafic généré croît de façon polylogarithmique comparativement à la
  taille du document : Plus l'expérience progresse, plus le nombre d'insertions
  effectuées sur le document est important, plus la croissance des identifiants
  diminue.
\item À cela s'ajoute un facteur multiplicatif logarithmique comparé au nombre
  de partipants : Plus la session d'édition comporte de membres, plus le trafic
  généré est important. Toutefois, l'écart entre les mesures diminues tandis que
  le nombre de participants est incrémenté linéairement. On note toutefois des
  exceptions à ces observations. Par exemple, les mesures sur 501 editeurs
  dominent les mesures sur 601 éditeurs.
\end{inparaenum}

\paragraph{Explication :} \CRATE utilise \LSEQ
(cf. chapitre~\ref{repl:chap:lseq}). Le comportement d'édition des participants
est monotone. \LSEQ possède une complexité polylogarithmique sur ses
identifiants, comparé au nombre d'insertions effectuées dans la séquence. Ainsi,
plus les expérimentations progressent, plus le nombre d'insertions dans la
séquence augmente. Les messages envoyées ne comportant que les identifiants de
\LSEQ, la croissance du trafic hérite de cette croissance
polylogarithmique. \CRATE utilise \SPRAY
(cf. chapitre~\ref{repl:chap:spray}). Lorsqu'un nouveau membre rejoins une
session d'édition, il apporte avec lui un nombre logarithmique de connexions par
rapport à la taille du réseau. Chacune ces connexions est activement utilisée
par le mécanisme de diffusion de messages afin que chacun des identifiants
parvienne à tous les éditeurs. Pour chaque opération locale effectuée, chaque
membre retransmet le message aux membres sa vue partielle exactement une
fois. D'où la progression logarithmique du trafic généré entre les sessions
d'édition. L'exception perçue entre certaines configurations de
l'expérimentation est dûe au coté aléatoire du protocole. Sur un grand nombre
d'exécution, cet aléatoire serait gommé.


