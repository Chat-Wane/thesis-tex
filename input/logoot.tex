
\subsection{Logoot}

Logoot~\cite{weiss2009logoot} appartient à la classe de données utilisant des
identifiants sous forme de liste. Ainsi, ces identifiants sont de taille
variable. Dans ce type d'approches, la difficulté consiste à réduire la taille
de ces identifiants.

Contrairement à WOOT, un identifiant ne fait pas référence aux voisins de la
position d'insertion. L'identifiant est un élément d'un ensemble dense en
lui-même\footnote{Pour deux éléments appartenant à l'ensemble, il existe
  toujours un élément situé entre ceux-ci.}. Logoot utilise donc les
identifiants voisins à l'insertion pour générer un nouvel identifiant qui est,
en utilisant un ordre lexicographique, positionné au milieu de ceux-ci. On
obtient
$\mathcal{I}: \langle \mathbb{N} \times \mathcal{D}\rangle^{+} \times
\mathcal{A}$.

Puisque les identifiants ne font pas référence aux voisins, ils sont
autonomes. Conséquemment, lors de suppressions, Logoot ne se contente pas de
cacher l'élément ciblé, mais le supprime réellement de la structure. Les
complexités spatiale et temporelle concernent donc exclusivement les opérations
d'insertion. De plus, le coût d'une opération est réparti plus équitablement
entre local et distant. Cela est extrêmement important lorsque l'on considère
que pour toute opération local, l'ensemble des utilisateurs doivent exécuter la
partie distante de l'opération.

\begin{algorithm}
  
\small
\algrenewcommand{\algorithmiccomment}[1]{\hskip2em$\rhd$ #1}

\newcommand{\comment}[1]{$\rhd$ #1}


\algblockdefx[initially]{initially}{endInitially}
  [0] {\textbf{INITIALLY:}} 

\algblockdefx[local]{local}{endLocal}
  [0] {\textbf{LOCAL UPDATE:}}

\algblockdefx[received]{received}{endReceived}
  [0] {\textbf{RECEIVED UPDATE:}}

\algblockdefx[onInsert]{onLocal}{endOnLocal}
  [0] {\textbf{on} insert ($p \in \mathcal{I},\,\alpha \in \mathcal{A},\,
   q\in\mathcal{I}$):}
  [0] {\textbf{on} delete ($i \in \mathcal{I}$):} 

\algblockdefx[onRemote]{onRemote}{endOnRemote}
  [0] {\textbf{on} insert ($i\in\mathcal{I}$):\hfill\comment{once per 
  distinct triple in $\mathcal{I}$}}
  [0] {\textbf{on} delete ($i\in\mathcal{I}$):\hfill\comment{after the 
  remote $insert(i)$ is done}} 

\newcommand{\LINEFOR}[2]{%
  \algorithmicfor\ {#1}\ \algorithmicdo\ {#2} %
  }

\newcommand{\LINEIFTHEN}[2]{%
  \algorithmicif\ {#1}\ \algorithmicthen\ {#2} %
  }

\newcommand{\INDSTATE}[1][1]{\State\hspace{\algorithmicindent}}

\begin{algorithmic}[1]
  \Statex
  \initially
    \State $\mathcal{S} \leftarrow \varnothing$;
    \hfill \comment{Replicated sequence as a sorted tree}
    \State $\mathcal{D} \leftarrow \langle siteId,\, counter \rangle$;
    \hfill \comment{Disambiguator}
  \endInitially
  
  \local
    \onLocal
    \State \textbf{let} $i \leftarrow allocate(p,\,\alpha,\,q)$;
    \State $broadcast('insert',\, i)$;
    \endOnLocal
    \INDSTATE $broadcast('delete',\,i)$;
  \endLocal
  
  \received
    \onRemote
    \State $insert(\mathcal{S},\,i)$;
    \hfill \comment{Similar to insertion in sorted tree}
    \endOnRemote
    \INDSTATE $remove(\mathcal{S},\, i)$;
    \hfill \comment{Similar to deletion in sorted tree}
  \endReceived
  
\end{algorithmic}

  \caption{\label{algo:logoot}Logoot.}
\end{algorithm}

\begin{algorithm}
  
\small
\algrenewcommand{\algorithmiccomment}[1]{\hskip2em$\rhd$ #1}

\newcommand{\comment}[1]{$\rhd$ #1}

\newcommand{\LINEFOR}[2]{%
  \algorithmicfor\ {#1}\ \algorithmicdo\ {#2} %
  }

\newcommand{\LINEWHILE}[2]{%
  \algorithmicwhile\ {#1}\ \algorithmicdo\ {#2} %
  }

\newcommand{\LINEIFTHEN}[2]{%
  \algorithmicif\ {#1}\ \algorithmicthen\ {#2} %
  }

\newcommand{\INDSTATE}[1][1]{\State\hspace{\algorithmicindent}}

\begin{algorithmic}[1]
  \Function{allocate}
  {$p\in\mathcal{I},\, \alpha\in\mathcal{A},\, q\in\mathcal{I}$}
  \State \textbf{let} $pi,p\alpha \leftarrow p$;
  \State \textbf{let} $qi,q\alpha \leftarrow q$;
  \State \textbf{let} $interval,\, depth \leftarrow getInterval(pi,\,qi)$;
  
  \State \textbf{let}
  $copy \leftarrow min(truncate(pi,\, depth),\, truncate(qi,\, depth))$;

  \State \textbf{let} $toAdd \leftarrow random(min(interval,\,boundary))+1$;

  \For{$j \leftarrow depth-1$ \textbf{to} $0$}
  \State \textbf{let} $toAdd, \, copy[j]  \leftarrow divmod(copy[j] + toAdd)$;
  \EndFor

  \State \Return $\langle copy,\, \alpha\rangle$;
  \EndFunction
\end{algorithmic}

  \caption{\label{algo:logootalloc}Allocation des identifiants de Logoot.}
\end{algorithm}

\begin{algorithm}
  
\small
\algrenewcommand{\algorithmiccomment}[1]{\hskip2em$\rhd$ #1}

\newcommand{\comment}[1]{$\rhd$ #1}

\newcommand{\LINEFOR}[2]{%
  \algorithmicfor\ {#1}\ \algorithmicdo\ {#2} %
  }

\newcommand{\LINEWHILE}[2]{%
  \algorithmicwhile\ {#1}\ \algorithmicdo\ {#2} %
  }

\newcommand{\LINEIFTHEN}[2]{%
  \algorithmicif\ {#1}\ \algorithmicthen\ {#2} %
  }

\newcommand{\INDSTATE}[1][1]{\State\hspace{\algorithmicindent}}

\begin{algorithmic}[1]
  \Function{getInterval}{$p\in\mathcal{I},\, q\in\mathcal{I}$}
  \State \textbf{let} $depth,\, interval \leftarrow 0,\, 0$;
  \State \textbf{let} $pIsLower \leftarrow true$;
  \While{$interval \leq 0$}
  \State \LINEIF{$(depth < |p| && !pIsLower)$}{};
  \State \LINEIF{$(depth < |q|)$}{}
  \EndWhile
  \EndFunction
\end{algorithmic}

  \caption{\label{algo:logootinterval}Process the interval (i.e. distance)
    between two identifiers.}
\end{algorithm}

%%% Local Variables:
%%% mode: latex
%%% TeX-master: "../paper"
%%% ispell-local-dictionary: "francais"
%%% End:
