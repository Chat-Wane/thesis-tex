
\subsection{WOOTO}

WOOTO~\cite{weiss2007wooki} est une amélioration de WOOT adressant le problème
de complexité temporelle à l'intégration des operations. Dans cette approche,
une notion de degré est ajoutée aux identifiants. Ce degré permet de filtrer
les identifiants non pertinants lors de l'insertion d'un élement. Ainsi, la
position d'insertion est retrouvée plus rapidement. Les identifiants sont de la
forme
$\mathcal{I}: \mathcal{I} \times \mathcal{D} \times \mathcal{A} \times
\mathbb{N} \times \mathcal{I}$.

La forme de l'algorithme~\ref{algo:woot} reste quasiment inchangée, i.e., la
création d'un identifiant ajoute le calcule du degré: le plus haut degré de ses
voisins plus 1. En revanche, l'algorithme recherchant la position d'insertion
diffère. La complexité temporelle de l'intégration d'une operation d'insertion
passe de cubique à quadratique pour WOOT et WOOTO respectivement.

WOOTO souffre du même problème que son ainé concernant les operations de
suppression: les élements sont simplement cachés et non réellement supprimés.
Par conséquent, les complexités des opérations dépendent de l'ensemble des 
opérations jamais effectuées sur la séquence.

\begin{algorithm}
  \input{./input/wootorecursalgo.tex}
  \caption{\label{algo:wootorecurs}Cherche l’indice où insérer le nouvel
    élement dans WOOTO.}
\end{algorithm}

WOOTH~\cite{ahmed2011evaluating} améliore légèrement la complexité temporelle
de WOOTO en utilisant table de hachage et liste chainée.


%%% Local Variables:
%%% mode: latex
%%% TeX-master: "../paper"
%%% End:
