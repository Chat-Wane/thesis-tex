
\chapter{Éditeurs collaboratifs répartis}

Les outils d'édition collaborative répartis sont des applications qui
permettent de répartir le travail de rédaction d'un document selon les trois
dimensions suivantes: le temps, l'espace, et les organismes (REF). Par exemple,
un Français, un Américain et un Australien peuvent travailler sur un même
document, tous ensemble ou a des intervals de temps différents.  Le document
résultant de cette collaboration est le produit de l'expertise de chacun des
collaborateurs. Ainsi, (REF) montre que la qualité des articles Wikipédia est
similaire à la qualité des articles de English (MACHIN TRUC). Toutefois, les
outils actuels ne sont pas entièrement satisfaisants sur des problématiques
\begin{inparaenum}[(i)]
\item liées à l'éthique (e.g. censure, vie privée, intelligence economique
  etc.) et
\item liées au passage à l'échelle (e.g. nombre de collaborateurs, taille des
  documents etc.).
\end{inparaenum}

Ce chapitre a pour mission de détailler les différents aspects liés à ces
outils. Un éditeurs collaboratif répartis peut être décomposé en plusieurs
couches qui constitueront les axes de ce chapitre.
\begin{inparaenum}[(1)]
\item \emph{La couche éditeur} composée:
  \begin{inparaenum}[(i)]
  \item d'un \emph{document} contenant des caractères, des lignes, des
    paragraphes etc. L'utilisateur peut directement intéragir avec celle-ci, et
    il s'attends à ce que les changements se propagent à la couche application
    des autres utilisateurs impliqués dans l'édition;
  \item d'un traqueur de relation de \emph{causalité} qui définit les
    contraintes d'ordre sur les modifications émises depuis le document. Un
    exemple de contrainte: tous les caractères doivent être insérés dans
    l'ordre où ils furent tappés;
  \item d'un protocole d'\emph{anti-entropy} qui assure que toutes les
    répliques d'un document partagé finiront par être identiques.
  \end{inparaenum}
\item \emph{La couche réseau} composée:
  \begin{inparaenum}[(i)]
  \item d'un protocole d'\emph{appartenance} qui lie les membres du réseau
    entre eux. La résultante des liens entre les membres peut être représentée
    sous forme de graphe. Il est alors possible de modeler ce graphe selon
    certaines attentes. Par exemple, selon plus court chemin moyen, etc.
  \item d'un protocole de \emph{diffusion} qui permet d'envoyer un message aux
    membres appartenant au réseau. L'efficacité de cette couche peut dépendre
    de la couche d'appartenance.
  \end{inparaenum}
\end{inparaenum}

